% Se pre-carga la información del estudiante sólo para poder emplear el macro de
% selección de versión (digital o impresa)
% ===============================================================================
% El estudiante debe llenar sus datos en esta sección para que la plantilla los 
% auto-importe y genere automáticamente las páginas de portada y de firmas 
% autorizadas.
% ===============================================================================
% Datos del estudiante:
% -------------------------------------------------------------------------------
% Nombre completo
\def \nombreestudiante {Erick Stiv Junior Guerra Muñoz}
% Carné
\def \uvgcarne {21781}
% Facultad
\def \uvgfacultad {Ingeniería}
% Carrera
\def \uvgcarrera {Ingeniería en Ciencias de la Computación y Tecnologías de la Información}

% Datos del trabajo:
% -------------------------------------------------------------------------------
% Título completo
\def \titulotesis {Ciudadano Digital: La Inteligencia Artifcial como herramienta de acompañamiento informal en educación sobre ciudadanía y valores morales.}
% Año de entrega
\def \anoentrega {2025}
% Asesor
\def \nombreasesor {MA. Luis Furlán}

% Datos del tribunal examinador:
% -------------------------------------------------------------------------------
% Nombre del primer examinador
\def \nombreprimerex {MSc. Douglas Barrios}
% Nombre del segundo examinador
% \def \nombresegundoex {PhD. Gabriel Barrientos}
% Año de aprobación
\def \anoaprobacion {2025}

% Capítulos pre-definidos
% -------------------------------------------------------------------------------
% Comentar las líneas de las secciones que desean omitirse, por defecto se 
% se incluyen todas.


% Formato y estilo de la plantilla
% -------------------------------------------------------------------------------
% Modo impresión: Puede des-comentar la siguiente línea para generar un documento pdf sin la portada, para cuando se desee imprimir el documento para encuadernación
%\def \printver {Versión del documento para impresión}

% Portada: Puede cambiarse la imagen en la portada al cambiar el nombre del 
% archivo siguiente. NOTA: debe tener la suficiente resolución para cubrir el área
% designada
\def \imagenportada {plantilla/portadacit.jpg}

% Referencias: Puede des-comentar la siguiente línea para utilizar el formato de referencias APA
%\def \usarAPA {Usar formato APA}

% Párrafo: Puede comentar la siguiente línea si desea emplear un formato de 
% párrafo distinto al establecido por defecto
\def \parpordefecto {Formato de párrafo por defecto}

% Capítulos y secciones: Puede des-comentar la siguiente línea para establecer el
% formato de los capítulos y secciones bajo el estándar original de UVG para
% trabajos de graduación. Este incluye: capítulos con numeración romana, secciones
% con letras mayúsculas, sub-secciones con números y sub-sub-secciones con letras
% minúsculas
%\def \capsecuvg {Formato UVG para capítulos y secciones}

\ifdefined\printver
    \documentclass[11pt, letterpaper, twoside, openright]{report}
\else
    \documentclass[11pt, letterpaper]{report}
\fi

% Eliminar la opción de twoside y openright si se desea generar la versión
% digital del documento en lugar de la versión impresa
%\documentclass[11pt, letterpaper, twoside, openright]{report}
\usepackage[spanish, es-nodecimaldot, es-noquoting]{babel}
% cambiar a spanish, mexico si se quiere emplear tabla en lugar de cuadro
\selectlanguage{spanish}
\usepackage[utf8]{inputenc}
\usepackage[T1]{fontenc}

% Utilizar pseudocódigo
\usepackage{algorithm}
\usepackage{algpseudocode}
\floatname{algorithm}{Algoritmo}

% ======= TRADUCCIÓN DE PROCEDURE / ENDPROCEDURE =======
% Crear textos para Procedure y EndProcedure (NO existen por defecto)
% \algnewcommand\algorithmicprocedure{\textbf{Procedimiento}}
\algrenewcommand\algorithmicprocedure{\textbf{Procedimiento}}
\algnewcommand\algorithmicendprocedure{\textbf{Fin del procedimiento}}

% Crear los entornos Procedure y EndProcedure
\algdef{SE}[PROCEDURE]{Procedure}{EndProcedure}%
  {\algorithmicprocedure\ }{\algorithmicendprocedure}

  % ===== TRADUCCIÓN DE PALABRAS RESERVADAS GENERALES =====
\algnewcommand\algorithmicelseif{\textbf{Sino si}}

\algnewcommand\algorithmicendwhile{\textbf{FinMientras}}

\algrenewcommand\algorithmicrepeat{\textbf{Repetir}}
\algrenewcommand\algorithmicuntil{\textbf{Hasta que}}
\algrenewcommand\algorithmicrequire{\textbf{Requiere:}}
\algrenewcommand\algorithmicensure{\textbf{Asegura:}}

\algrenewcommand\algorithmicif{\textbf{Si}}
\algrenewcommand\algorithmicthen{\textbf{entonces}}
\algrenewcommand\algorithmicelse{\textbf{Sino}}

\algrenewcommand\algorithmicfor{\textbf{Para}}
\algrenewcommand\algorithmicforall{\textbf{Para cada}}

\algrenewcommand\algorithmicwhile{\textbf{Mientras}}
\algrenewcommand\algorithmicreturn{\textbf{Retornar}}

% Traducciones especiales
\algrenewtext{EndIf}{\textbf{Fin Si}}
\algrenewtext{EndFor}{\textbf{Fin Para}}

\makeatletter
\algrenewcommand\algorithmicdo{\textbf{hacer}} % por si se usa en línea
\algrenewtext{Do}{\textbf{hacer}} % por si se usa como palabra independiente

\title{Plantilla para Trabajos de Graduación IE-MT 2019v3}
\author{MSc. Miguel Zea}
\date{\today}

% Información del estudiante en el archivo datos_estudiante.tex
% ===============================================================================
% El estudiante debe llenar sus datos en esta sección para que la plantilla los 
% auto-importe y genere automáticamente las páginas de portada y de firmas 
% autorizadas.
% ===============================================================================
% Datos del estudiante:
% -------------------------------------------------------------------------------
% Nombre completo
\def \nombreestudiante {Erick Stiv Junior Guerra Muñoz}
% Carné
\def \uvgcarne {21781}
% Facultad
\def \uvgfacultad {Ingeniería}
% Carrera
\def \uvgcarrera {Ingeniería en Ciencias de la Computación y Tecnologías de la Información}

% Datos del trabajo:
% -------------------------------------------------------------------------------
% Título completo
\def \titulotesis {Ciudadano Digital: La Inteligencia Artifcial como herramienta de acompañamiento informal en educación sobre ciudadanía y valores morales.}
% Año de entrega
\def \anoentrega {2025}
% Asesor
\def \nombreasesor {MA. Luis Furlán}

% Datos del tribunal examinador:
% -------------------------------------------------------------------------------
% Nombre del primer examinador
\def \nombreprimerex {MSc. Douglas Barrios}
% Nombre del segundo examinador
% \def \nombresegundoex {PhD. Gabriel Barrientos}
% Año de aprobación
\def \anoaprobacion {2025}

% Capítulos pre-definidos
% -------------------------------------------------------------------------------
% Comentar las líneas de las secciones que desean omitirse, por defecto se 
% se incluyen todas.


% Formato y estilo de la plantilla
% -------------------------------------------------------------------------------
% Modo impresión: Puede des-comentar la siguiente línea para generar un documento pdf sin la portada, para cuando se desee imprimir el documento para encuadernación
%\def \printver {Versión del documento para impresión}

% Portada: Puede cambiarse la imagen en la portada al cambiar el nombre del 
% archivo siguiente. NOTA: debe tener la suficiente resolución para cubrir el área
% designada
\def \imagenportada {plantilla/portadacit.jpg}

% Referencias: Puede des-comentar la siguiente línea para utilizar el formato de referencias APA
%\def \usarAPA {Usar formato APA}

% Párrafo: Puede comentar la siguiente línea si desea emplear un formato de 
% párrafo distinto al establecido por defecto
\def \parpordefecto {Formato de párrafo por defecto}

% Capítulos y secciones: Puede des-comentar la siguiente línea para establecer el
% formato de los capítulos y secciones bajo el estándar original de UVG para
% trabajos de graduación. Este incluye: capítulos con numeración romana, secciones
% con letras mayúsculas, sub-secciones con números y sub-sub-secciones con letras
% minúsculas
%\def \capsecuvg {Formato UVG para capítulos y secciones}
% ================================================================================
% En este archivo se colocan opciones adicionales para modificar el formato de la
% plantilla, para emplearse en otros tipos de documentos que no sean trabajos de
% graduación. Si usted está trabajando su tesis, NO modifique este archivo
% ================================================================================
% Capítulos pre-definidos
% --------------------------------------------------------------------------------
% Comentar las líneas de las secciones que desean omitirse, por defecto se 
% se incluyen todas.
\def \CAPportada {Portada}
\def \CAPcaratula {Caratula}
\def \CAPcaratula {Caratula}
\def \CAPfirmas {Hoja de firmas}

%Opcionales
% Dedicatoria / Reconocimiento (opcional)
% \def \CAPdedicatoria {Dedicatoria}
% \def \CAPprefacio {Prefacio}

\def \CAPresumen {Resumen}
% Abstract
\def \CAPabstract {Resumen}

%Tablas de contenido
\def \CAPindice {Índice general}
\def \CAPcuadros {Listado de tablas}
% \def \CAPfiguras {Listado de figuras}
% \def \CAPglosario {Glosario}

\def \CAPintroduccion {Introducción}

%Opcionales
\def \CAPjustificacion {Justificación}
% \def \CAPantecedentes {Antecedentes}

\def \CAPobjetivos {Objetivos}

%Opcionales
% \def \CAPalcance {Alcances}

\def \CAPmarcoteorico {Marco teórico}
% Metodología
\def \CAPmetodologia {Metodología}
% Resultados
\def \CAPresultados {Resultados}
% Discusión
\def \CAPdiscusion {Discusión}
\def \CAPconclusiones {Conclusiones}
\def \CAPrecomendaciones {Recomendaciones}
\def \CAPbibliografia {Bibliografía}

% Opcionales
% \def \CAPanexos {Anexos}
% Apéndice
% \def \CAPapendice {Apéndice}

% ==============================================================================
% DEFINICIÓN DE PAQUETES
% ==============================================================================
\usepackage{xcolor}
\usepackage{longtable}
\usepackage{booktabs}
\usepackage{amsfonts}
\usepackage{amsmath}
\usepackage{amssymb}
\usepackage{amsthm}
\usepackage{amsfonts}
\usepackage{mathtools}
\usepackage{graphicx}
\usepackage{xfrac}
\usepackage{float}
\usepackage{mathtools}
\usepackage[hypertexnames=false]{hyperref}
% \usepackage{bookmark}
\usepackage[font=small]{caption}
\usepackage{subcaption}
%\usepackage{csquotes}
\usepackage{xpatch}
\usepackage{emptypage}
\usepackage{hyphenat}
\usepackage{fancyhdr}
\ifdefined\usarAPA
  \usepackage[backend=biber, style=apa]{biblatex}
\else
  \usepackage[backend=biber, style=ieee, sorting=none, citestyle=numeric-comp]{biblatex}
\fi
\addbibresource{q-bibliografia.bib}

\usepackage[percent]{overpic}

\usepackage{chngcntr}

%\usepackage[toc]{glossaries}
\usepackage[numberedsection]{glossaries}
\makeglossaries

% \ifdefined\CAPglosario
%     \newglossaryentry{acid}
{
    name=ACID,
    description={Atomicity, Consistency, Isolation and Durability (Atomicidad, Consistencia, Aislamiento y Durabilidad)}
}
\newglossaryentry{api}
{
    name=API,
    description={Application Programming Interface (Interfaz de Programación de Aplicaciones)}
}
\newglossaryentry{aws}
{
    name=AWS,
    description={Amazon Web Services (Servicios Web de Amazon)}
}

\newglossaryentry{crud}
{
    name=CRUD,
    description={Create, Read, Update, Delete (Crear, Leer, Actualizar, Eliminar)}
}

\newglossaryentry{di}
{
    name=DI,
    description={Dependency Injection (Inyección de dependencias)}
}

\newglossaryentry{er}
{
    name=ER,
    description={Entidad-Relación}
}
\newglossaryentry{html}
{
    name=HTML,
    description={HyperText Markup Language (Lenguaje de Marcado de HiperTexto)}
}
\newglossaryentry{http}
{
    name=HTTP,
    description={HyperText Transfer Protocol (Protocolo de Transferencia de HiperTexto)}
}
\newglossaryentry{icmp}
{
    name=ICMP,
    description={Internet Control Message Protocol (Protocolo de Mensajes de Control de Internet)}
}
\newglossaryentry{ia}
{
    name=IA,
    description={Inteligencia Artificial}
}
\newglossaryentry{json}
{
    name=JSON,
    description={JavaScript Object Notation (Notación de Objetos de JavaScript)}
}

\newglossaryentry{jwt}
{
    name=JWT,
    description={JSON Web Token (Token Web en formato JSON)}
}

\newglossaryentry{llm}
{
    name=LLM,
    description={Large Language Model (Modelo Grande de Lenguaje)}
}

\newglossaryentry{mvc}
{
    name=MVC,
    description={Model-View-Controller (Modelo-Vista-Controlador)}
}

\newglossaryentry{mvp}
{
    name=MVP,
    description={Minimum Viable Product (Producto Mínimo Viable)}
}

\newglossaryentry{mvvm}
{
    name=MVVM,
    description={Model-View-ViewModel (Modelo-Vista-Modelo de Vista)}
}
\newglossaryentry{nfkd}
{
    name=NFKD,
    description={Normalization Form Compatibility Decomposition (Descomposición de Compatibilidad de Formas de Normalización)}
}

\newglossaryentry{nlp}
{
    name=NLP,
    description={Natural Language Processing (Procesamiento de Lenguaje Natural)}
}
\newglossaryentry{npm}
{
    name=npm,
    description={Node Package Manager (Gestor de Paquetes de Node)}
}
\newglossaryentry{ocr}
{
    name=OCR,
    description={Optical Character Recognition (Reconocimiento Óptico de Caracteres)}
}

\newglossaryentry{rag}
{
    name=RAG,
    description={Retrieval-Augmented Generation (Generación Mejorada por Recuperación)}
}

\newglossaryentry{rds}
{
    name=RDS,
    description={Relational Database Service (Servicio de Base de Datos Relacional)}
}
\newglossaryentry{s3}
{
    name=S3,
    description={Simple Storage Service (Servicio de Almacenamiento Sencillo)}
}

\newglossaryentry{sms}
{
    name=SMS,
    description={Short Message Service (Servicio de Mensajes Cortos)}
}

\newglossaryentry{ui}
{
    name=UI,
    description={User Interface (Interfaz de Usuario)}
}
\newglossaryentry{ux}
{
    name=UX,
    description={User Experience (Experiencia de Usuario)}
}
\newglossaryentry{xml}
{
    name=XML,
    description={Extensible Markup Language (Lenguaje de Marcado Extensible)}
}


% \fi

% ==============================================================================
% MÁRGENES Y FORMATO GENERALES
% ==============================================================================
\usepackage[top=1in, left=1.5in, right=1in, bottom=1in]{geometry}
%Options: Sonny, Lenny, Glenn, Conny, Rejne, Bjarne, Bjornstrup
\usepackage[Sonny]{fncychap}

% ==============================================================================
% DEFINICIONES DE LA PLANTILLA
% ==============================================================================
\graphicspath{ {figuras/} }
\definecolor{uvg-green}{RGB}{17,71,52}
\newcommand{\defaultparformat}[1]{
	{\setlength{\parskip}{2ex}
     \input{#1}}
}
\ifdefined\capsecuvg
	\renewcommand\thechapter{\Roman{chapter}}
    \renewcommand\thesection{\Alph{section}}
	\renewcommand\thesubsection{\arabic{subsection}}
    \renewcommand\thesubsubsection{\alph{subsubection}}
\fi
\counterwithout{figure}{chapter}
\counterwithout{table}{chapter}
\counterwithout{equation}{chapter}

\newcommand{\blankpage}{
\newpage
\thispagestyle{empty}
\mbox{}
\newpage
}
% ==============================================================================

% Comandos definidos por el usuario en el archivo comandos_usuario.tex
\input{2-paquetes_y_comandos_usuario}

% ==============================================================================
% CUERPO DEL TRABAJO
% ==============================================================================
\pagestyle{headings}
\begin{document}
\ifdefined\CAPglosario
    \newglossaryentry{acid}
{
    name=ACID,
    description={Atomicity, Consistency, Isolation and Durability (Atomicidad, Consistencia, Aislamiento y Durabilidad)}
}
\newglossaryentry{api}
{
    name=API,
    description={Application Programming Interface (Interfaz de Programación de Aplicaciones)}
}
\newglossaryentry{aws}
{
    name=AWS,
    description={Amazon Web Services (Servicios Web de Amazon)}
}

\newglossaryentry{crud}
{
    name=CRUD,
    description={Create, Read, Update, Delete (Crear, Leer, Actualizar, Eliminar)}
}

\newglossaryentry{di}
{
    name=DI,
    description={Dependency Injection (Inyección de dependencias)}
}

\newglossaryentry{er}
{
    name=ER,
    description={Entidad-Relación}
}
\newglossaryentry{html}
{
    name=HTML,
    description={HyperText Markup Language (Lenguaje de Marcado de HiperTexto)}
}
\newglossaryentry{http}
{
    name=HTTP,
    description={HyperText Transfer Protocol (Protocolo de Transferencia de HiperTexto)}
}
\newglossaryentry{icmp}
{
    name=ICMP,
    description={Internet Control Message Protocol (Protocolo de Mensajes de Control de Internet)}
}
\newglossaryentry{ia}
{
    name=IA,
    description={Inteligencia Artificial}
}
\newglossaryentry{json}
{
    name=JSON,
    description={JavaScript Object Notation (Notación de Objetos de JavaScript)}
}

\newglossaryentry{jwt}
{
    name=JWT,
    description={JSON Web Token (Token Web en formato JSON)}
}

\newglossaryentry{llm}
{
    name=LLM,
    description={Large Language Model (Modelo Grande de Lenguaje)}
}

\newglossaryentry{mvc}
{
    name=MVC,
    description={Model-View-Controller (Modelo-Vista-Controlador)}
}

\newglossaryentry{mvp}
{
    name=MVP,
    description={Minimum Viable Product (Producto Mínimo Viable)}
}

\newglossaryentry{mvvm}
{
    name=MVVM,
    description={Model-View-ViewModel (Modelo-Vista-Modelo de Vista)}
}
\newglossaryentry{nfkd}
{
    name=NFKD,
    description={Normalization Form Compatibility Decomposition (Descomposición de Compatibilidad de Formas de Normalización)}
}

\newglossaryentry{nlp}
{
    name=NLP,
    description={Natural Language Processing (Procesamiento de Lenguaje Natural)}
}
\newglossaryentry{npm}
{
    name=npm,
    description={Node Package Manager (Gestor de Paquetes de Node)}
}
\newglossaryentry{ocr}
{
    name=OCR,
    description={Optical Character Recognition (Reconocimiento Óptico de Caracteres)}
}

\newglossaryentry{rag}
{
    name=RAG,
    description={Retrieval-Augmented Generation (Generación Mejorada por Recuperación)}
}

\newglossaryentry{rds}
{
    name=RDS,
    description={Relational Database Service (Servicio de Base de Datos Relacional)}
}
\newglossaryentry{s3}
{
    name=S3,
    description={Simple Storage Service (Servicio de Almacenamiento Sencillo)}
}

\newglossaryentry{sms}
{
    name=SMS,
    description={Short Message Service (Servicio de Mensajes Cortos)}
}

\newglossaryentry{ui}
{
    name=UI,
    description={User Interface (Interfaz de Usuario)}
}
\newglossaryentry{ux}
{
    name=UX,
    description={User Experience (Experiencia de Usuario)}
}
\newglossaryentry{xml}
{
    name=XML,
    description={Extensible Markup Language (Lenguaje de Marcado Extensible)}
}


\fi

% ==============================================================================
% PORTADA
% ==============================================================================
\ifdefined\printver
	\let\CAPportada\undefined
\fi

\ifdefined\CAPportada
	\cleardoublepage\phantomsection
	% \pdfbookmark{Portada}{toc}
	\newgeometry{left=3cm, bottom=0in, top=1in, right=3cm}
	\pagecolor{uvg-green}
	\thispagestyle{empty}

	\color{white}
	\noindent \hrulefill \par
	\vspace{0.1in}
	\noindent \Huge \nohyphens{\titulotesis} \par
	\noindent \hrulefill \par
	\noindent
	\LARGE \nombreestudiante

	\begin{figure}[b!]
		%\makebox[\textwidth]{\includegraphics[height=13.25cm]{plantilla/portadacit.jpg}}
		\makebox[\textwidth]{
			\begin{overpic}[height=13.25cm]{\imagenportada}
				\put(63,0){\includegraphics[height=1.15in]{plantilla/fondologo_grande.png}}
				\put(64.5,2){\includegraphics[height=0.55in]{plantilla/logoUVGblanco.eps}}
			\end{overpic}
		}
		%\includegraphics[height=13.25cm]{plantilla/portadacit.jpg}
	\end{figure}
	\restoregeometry
\fi

% ==============================================================================
% PRIMERAS PÁGINAS (Carátulas más hojas de guarda)
% ==============================================================================
\ifdefined\CAPcaratula
	\newpage
	\cleardoublepage\phantomsection
	% \pdfbookmark{Carátula}{toc}
	\pagecolor{white}
	\color{black}
	\setcounter{page}{1}
	\pagenumbering{roman}
	\thispagestyle{empty}
	\begin{center}
		\LARGE UNIVERSIDAD DEL VALLE DE GUATEMALA\\
		\LARGE Facultad de \uvgfacultad \\[0.75cm]
	\end{center}
	\begin{figure}[h]
		\begin{center}
			\includegraphics[height=5.5 cm]{plantilla/escudoUVGnegro.eps}
			\vspace{0.5in}
		\end{center}
	\end{figure}
	\begin{center}
		\Large \textbf{\nohyphens{\titulotesis}} \\
		%\LARGE \textbf{\titulotesis} \\
		\vfill
		\Large \nohyphens{Trabajo de graduación presentado por \nombreestudiante \ para optar al grado académico de Licenciado en \uvgcarrera} \\
		\vfill
		\large Guatemala, \\
		\vspace{1em}
		\anoentrega
	\end{center}

	\ifdefined\printver
		\blankpage
		\blankpage

		\newpage
		\cleardoublepage\phantomsection
		\pagecolor{white}
		\color{black}
		\setcounter{page}{1}
		\pagenumbering{roman}
		\thispagestyle{empty}
		\begin{center}
			\LARGE UNIVERSIDAD DEL VALLE DE GUATEMALA\\
			\LARGE Facultad de \uvgfacultad \\[0.75cm]
		\end{center}
		\begin{figure}[h]
			\begin{center}
				\includegraphics[height=5.5 cm]{plantilla/escudoUVGnegro.eps}
				\vspace{0.5in}
			\end{center}
		\end{figure}
		\begin{center}
			\Large \textbf{\nohyphens{\titulotesis}} \\
			%\LARGE \textbf{\titulotesis} \\
			\vfill
			\Large \nohyphens{Trabajo de graduación presentado por \nombreestudiante \ para optar al grado académico de Licenciado en \uvgcarrera} \\
			\vfill
			\large Guatemala, \\
			\vspace{1em}
			\anoentrega
		\end{center}
	\fi
\fi

% ==============================================================================
% HOJA DE FIRMAS
% ==============================================================================
\ifdefined\CAPfirmas
	\newpage
	\cleardoublepage\phantomsection
	\thispagestyle{empty}
	\vspace*{0.5in}
	\large Vo.Bo.:\\[1cm]
	\begin{center}
		(f) \rule[1pt]{4 in}{1pt}\\
		\nombreasesor
	\end{center}
	\vspace{1in}

	Tribunal Examinador:\\[1cm]
	\begin{center}
		(f) \rule[1pt]{4 in}{1pt}\\
		\nombreasesor \\[1in]
		(f) \rule[1pt]{4 in}{1pt}\\
		\nombreprimerex \\[1in]
		% (f) \rule[1pt]{4 in}{1pt}\\
		% \nombresegundoex
	\end{center}
	\vspace{1in}

	% Fecha de aprobación: Guatemala, \rule[1pt]{0.5 in}{1pt} de \rule[1pt]{1
	% 	in}{1pt} de \anoaprobacion. \normalsize \fi
	Fecha de aprobación: Guatemala, noviembre de \anoaprobacion. \normalsize \fi

% Comentar para formato estilo libro en la numeración de páginas (NO 
% compatible con la guía UVG 2019)
\pagestyle{plain}
% ==============================================================================
% CONTENIDO DEL TRABAJO
% ==============================================================================
% DEDICATORIA
% ------------------------------------------------------------------------------
\ifdefined\CAPdedicatoria
	\newpage
	\cleardoublepage\phantomsection
	\chapter*{Dedicatoria}
	\ifdefined\parpordefecto
		\defaultparformat{a-dedicatoria}
	\else
		A mi madre, ejemplo de superación y esfuerzo desde el inicio de mi vida; la razón de mi persistencia y el motor que ha impulsado cada uno de mis logros.

A mis abuelos, cuyo cariño y apoyo incondicional han dado forma a la persona que soy.

A mi hermano, con la esperanza de que alcance todas las metas que se proponga y encuentre en su hermano mayor el ejemplo y apoyo que necesite en cada etapa de su vida.

A la Fundación Juan Bautista Gutiérrez, por brindarme la oportunidad de desarrollar mi carrera académica más allá de lo que alguna vez imaginé posible.

A mis compañeros de universidad, que mantuvieron el optimismo año tras año, avanzando juntos sin dejar a nadie atrás.

	\fi
	\addcontentsline{toc}{chapter}{Dedicatoria}
\fi
% PREFACIO
% ------------------------------------------------------------------------------
\ifdefined\CAPprefacio
	\newpage
	\cleardoublepage\phantomsection
	\chapter*{Prefacio}
	\ifdefined\parpordefecto
		\defaultparformat{b-prefacio}
	\else
		\input{b-prefacio}
	\fi
	\addcontentsline{toc}{chapter}{Prefacio}
\fi

% RESUMEN
% ------------------------------------------------------------------------------
\ifdefined\CAPresumen
	\newpage
	\cleardoublepage\phantomsection
	\chapter*{Resumen}
	\ifdefined\parpordefecto
		\defaultparformat{c-resumen}
	\else
		La inteligencia artificial (IA) ofrece un potencial transformador para la
educación, especialmente en contextos marcados por desigualdades sociales,
económicas y tecnológicas. En países como Guatemala, donde persiste una amplia
brecha educativa, la IA puede convertirse en una herramienta clave para
facilitar el acceso a aprendizajes significativos.

Uno de los ámbitos más desatendidos es la formación ciudadana y el desarrollo
de valores morales que, aunque incluidos en los programas educativos, suelen
abordarse de forma teórica y desvinculada de la realidad social. Es aquí donde
se plantea la importancia de una herramienta que brinde acompañamiento a los
estudiantes en la reflexión sobre su papel como ciudadanos y en la práctica de
valores como el respeto, la empatía y la responsabilidad. Este proyecto combina
tecnología accesible con un enfoque centrado en el usuario para fortalecer no
solo el aprendizaje, sino también la conciencia social y ética de los jóvenes.

La implementación de esta herramienta incluye el procesamiento de contenidos
educativos, el desarrollo de un asistente de IA adaptado a los contenidos
seleccionados, a través de una aplicación móvil interactiva y, finalmente, la
validación directa de la interacción con la herramienta para verificar que las
respuestas dadas coincidan con el material proporcionado. Como resultado, se
obtuvo una solución funcional, innovadora y escalable que demuestra cómo la IA
puede contribuir significativamente al fortalecimiento de la educación en
valores en entornos con recursos limitados.
	\fi
	\addcontentsline{toc}{chapter}{Resumen}
\fi

% ABSTRACT
% ------------------------------------------------------------------------------
\ifdefined\CAPabstract
	\newpage
	\cleardoublepage\phantomsection
	\chapter*{Abstract}
	\ifdefined\parpordefecto
		\defaultparformat{d-abstract}
	\else
		Artificial intelligence (AI) offers transformative potential for education,
especially in contexts marked by social, economic, and technological
inequalities. In countries like Guatemala, where a significant educational gap
persists, AI can become a key tool to facilitate access to meaningful learning.

One of the most neglected areas is civic education and the development of moral
values, which, although included in educational programs, are often addressed
theoretically and detached from social reality. This is where the importance of
a tool that supports students in reflecting on their role as citizens and in
practicing values such as respect, empathy, and responsibility arises. This
project combines both accessible technology and a user-centered approach to
strengthen not only learning but also the social and ethical awareness of young
people.

The implementation of this tool includes the digitization of educational
content, the development of an AI assistant adapted to the selected contents, and
finally, the direct validation of the interaction with the tool to ensure that
the answers provided align with the material given. As a result, a functional,
innovative, and scalable solution is sought that demonstrates how AI can
significantly contribute to strengthening values education in environments with
limited resources.
	\fi
	\addcontentsline{toc}{chapter}{Abstract}
\fi

% ÍNDICE GENERAL
% ------------------------------------------------------------------------------
\ifdefined\CAPindice
	\newpage
	\cleardoublepage\phantomsection
	\renewcommand{\contentsname}{Índice}
	%\phantomsection
	\pdfbookmark{\contentsname}{toc}
	%\pdfbookmark{Índice}{toc}
	\tableofcontents
\fi

% LISTADO DE CUADROS
% ------------------------------------------------------------------------------
\ifdefined\CAPcuadros
	\newpage
	\cleardoublepage\phantomsection
	\renewcommand{\listtablename}{Lista de cuadros}
	\listoftables
	\addcontentsline{toc}{chapter}{Lista de cuadros}
\fi

% LISTADO DE ALGORITMOS
% ------------------------------------------------------------------------------
\ifdefined\CAPalgoritmos
	\newpage
	\cleardoublepage\phantomsection
	\renewcommand{\listalgorithmname}{Lista de algoritmos} % nombre que aparece en el índice
	\listofalgorithms
	\addcontentsline{toc}{chapter}{Lista de algoritmos}
\fi

% LISTADO DE FIGURAS
% ------------------------------------------------------------------------------
\ifdefined\CAPfiguras
	\newpage
	\cleardoublepage\phantomsection
	\renewcommand{\listfigurename}{Lista de figuras}
	\listoffigures
	\addcontentsline{toc}{chapter}{Lista de figuras}
\fi

% GLOSARIO
% ------------------------------------------------------------------------------
\ifdefined\CAPglosario
	\glsaddall
	\newpage
	\printglossary[title=Abreviaturas,toctitle=Abreviaturas]
\fi

% ALCANCE
% ------------------------------------------------------------------------------
\ifdefined\CAPalcance
	\newpage
	\chapter{Alcances}
	\ifdefined\parpordefecto
		\defaultparformat{j-alcance}
	\else
		El proyecto \textit{Compañero Digital: Ocho a Dieciocho} es una iniciativa que
busca crear una herramienta de educación cívica basada en inteligencia
artificial (IA) dirigida a la juventud guatemalteca. El propósito principal
consiste en la construcción de una herramienta conversacional
(\textit{chatbot}) especializada, capaz de brindar orientación, información y
acompañamiento sobre temas de educación cívica informal, accesible desde una
aplicación móvil en cualquier momento y lugar. Como megaproyecto, el objetivo
final busca \textbf{revitalizar la educación informal cívica en Guatemala}
mediante tecnología moderna.

Bajo las espectativas funcionales del megaproyecto principal, el proyecto
\textit{Ciudadano Digital} constituye una base técnica inicial,centrada en
demostrar la viabilidad de las herramientas tecnológicas contempladas para el
desarrollo del producto final. Esta iteración busca definir la arquitectura
base que permita el procesamiento del contenido educativo, la interacción con
el usuario y un sistema independiente que pueda ser consumido por una
aplicación móvil, pero a su vez esté disponible para cualquier otro tipo de
implementación (por ejemplo, una página web). A su vez, se realiza el
perfilamiento base del usuario final objetivo, con el fin de mantener el
enfoque principal del megaproyecto original.

Al tratarse meramente de una validación técnica, este proyecto no contempla
pruebas de campo con usuarios reales. Las respuestas brindadas por el modelo se
validan únicamente en comparación con los contenidos utilizados para la
alimentación del conocimiento del mismo. Sin embargo, sí se valida la
integración adecuada entre los distintos componentes tecnológicos que
constituyen el producto final: modelo de lenguaje de gran escala (LLM, por sus
siglas en inglés), base de datos vectorial, base de datos relacional y
aplicación móvil final.

\textit{Ciudadano Digital} se centra únicamente en la construcción del Producto Mínimo Viable (MVP, por sus siglas en inglés) necesario para comprobar que la arquitectura propuesta puede operar de manera funcional y coherente. Se evalúa el procesamiento del corpus, el flujo de generación aumentada por recuperación (RAG, por sus siglas en inglés) y la conexión entre la aplicación móvil y el servidor desarrollado, sin abordar aún la implementación completa de los componentes pedagógicos, escalabilidad institucional ni las funciones avanzadas previstas para el proyecto final.


	\fi
\fi

% INTRODUCCIÓN
% ------------------------------------------------------------------------------
\ifdefined\CAPintroduccion
	\newpage
	\cleardoublepage
	\pagenumbering{arabic}
	\setcounter{page}{1}
	\chapter{Introducción}
	\ifdefined\parpordefecto
		\defaultparformat{f-introduccion}
	\else
		La formación en valores y ciudadanía constituye un componente esencial en la
construcción de sociedades democráticas, inclusivas y participativas. A través
de ella, los estudiantes desarrollan competencias cívicas como la empatía, la
responsabilidad social, el respeto a la diversidad y el compromiso con el bien
común. Organismos internacionales como la UNESCO (Organización de las Naciones
Unidas para la Educación, la Ciencia y la Cultura) y la OCDE (Organización para
la Cooperación y el Desarrollo Económicos) han destacado la necesidad de
fortalecer estos aprendizajes en un contexto global marcado por tensiones
sociales, crisis ambientales y transformaciones digitales profundas
\cite{unesco2021ethics,oecd2021skills}. Sin embargo, en muchos países de
América Latina esta dimensión formativa continúa siendo relegada frente a
enfoques centrados exclusivamente en resultados académicos cuantificables
\cite{worldbank2022revolution,rivas2023future}.

En Guatemala, el Currículo Nacional Base (CNB) reconoce la educación ciudadana
como un eje transversal, pero su aplicación efectiva enfrenta múltiples
desafíos: la escasa formación docente en metodologías críticas, la ausencia de
recursos digitales adaptados al contexto local y la persistencia de enfoques
tradicionales centrados en la memorización. Estas limitaciones dificultan que
los estudiantes logren vincular los contenidos cívicos con su vida cotidiana o
desarrollar una comprensión profunda de su papel como agentes de cambio
\cite{mineduc2020cnb,cien2019diagnostico}. A ello se suma una brecha
tecnológica significativa entre zonas urbanas y rurales, que restringe el
acceso equitativo a experiencias de aprendizaje innovadoras y limita las
oportunidades para fomentar la reflexión ética y la participación ciudadana
\cite{unesco2023monitoring,levy2025teachers}.

En este panorama, la inteligencia artificial (IA) emerge como una herramienta
con gran potencial transformador para la educación, especialmente cuando se
orienta hacia el fortalecimiento de habilidades humanas y el acompañamiento
moral, más que hacia la mera automatización de contenidos. La UNESCO subraya
que, para que la IA contribuya al desarrollo de sistemas educativos más justos
y democráticos, debe diseñarse bajo principios de equidad, inclusión y
supervisión humana, evitando reproducir sesgos o exclusiones
\cite{unesco2021ethics,unesco2021guidance}. Aplicada con criterios éticos y
pedagógicos, la IA puede servir como un medio para ampliar el acceso a
materiales formativos, personalizar experiencias de aprendizaje y acompañar
procesos de reflexión moral mediante un diálogo guiado y contextualizado
\cite{frontiers2024chatgpt,tulsiani2024chatgpt}.

En los últimos años, el avance de los modelos de lenguaje de gran escala (LLMs)
ha impulsado el desarrollo de asistentes conversacionales capaces de generar
tutorías personalizadas, ofrecer retroalimentación inmediata y adaptarse al
ritmo de cada estudiante \cite{elstad2024ai,frontiers2025education}. Estas
tecnologías abren la posibilidad de diseñar espacios de aprendizaje informal
donde los jóvenes puedan explorar dilemas morales, reflexionar sobre valores y
fortalecer su pensamiento crítico mediante la interacción con un sistema
empático y culturalmente pertinente.

Desde esta perspectiva, el proyecto \textit{Ciudadano Digital} busca integrar la
inteligencia artificial en la formación ciudadana y moral a través de una
aplicación accesible, diseñada para acompañar el aprendizaje ético de los
jóvenes en entornos digitales. El sistema, basado en un modelo de lenguaje
conectado a una base de datos vectorial con materiales educativos, éticos y
contextuales, genera respuestas personalizadas y fundamentadas que orientan la
reflexión. Su propósito no es sustituir al docente, sino complementar su labor
mediante un acompañamiento continuo que promueva la autonomía moral, la empatía
y la responsabilidad social. Además, su diseño prioriza la accesibilidad y la
escalabilidad, de modo que pueda funcionar eficazmente en dispositivos de bajo
costo y en entornos con recursos limitados. En conjunto, el proyecto pretende
fortalecer la educación cívica guatemalteca mediante el uso ético y
contextualizado de la IA, contribuyendo a una educación más inclusiva,
reflexiva y humanista en la era digital
\cite{unesco2021ethics,worldbank2022revolution,rivas2023future}.
	\fi
\fi

% JUSTIFICACIÓN
% ------------------------------------------------------------------------------
\ifdefined\CAPjustificacion
	\newpage
	\chapter{Justificación}
	\ifdefined\parpordefecto
		\defaultparformat{g-justificacion}
	\else
		La llegada de modelos conversacionales, como ChatGPT, ha revolucionado el aprendizaje, permitiendo tutorías personalizadas que adaptan el contenido al ritmo del estudiante y ofrecen retroalimentación inmediata, factores clave para mejorar la motivación y el rendimiento académico en contextos diversos \cite{frontiers2024chatgpt}. Sin embargo, investigaciones advierten sobre riesgos como la desinformación o el plagio si estos sistemas se emplean sin supervisión pedagógica, lo que subraya la necesidad de un diseño ético y guiado \cite{tulsiani2024chatgpt}. Además, la inteligencia artificial (IA) está transformando la gestión educativa al automatizar tareas administrativas como la calificación de exámenes y el seguimiento de asistencia, liberando tiempo para que los docentes se centren en procesos pedagógicos de mayor valor \cite{unesco2023monitoring}.

En América Latina, las plataformas de IA están extendiendo recursos digitales a zonas rurales, reduciendo brechas de cobertura; sin embargo, su escalabilidad sigue limitada por deficiencias en infraestructura y falta de formación técnica docente \cite{worldbank2022revolution,rivas2023future}. En el caso de Guatemala, el Acuerdo Ministerial 2810-2023 estableció el Programa Nacional de Educación en Valores para priorizar la formación ciudadana \cite{mineduc2023acuerdo}, aunque el Diagnóstico del CIEN, realizado en 2019 evidenció problemas persistentes en cobertura, eficiencia, calidad y ausencia de estrategias de tecnología educativa \cite{cien2019diagnostico}.

En este contexto, los modelos de lenguaje (LLMs) ofrecen una propuesta tecnológica altamente pertinente, en este caso, como tutores en valores y ciudadanía. Su capacidad para imitar patrones de diálogo humano, que incluyen desde el ajuste del tono al hablar hasta el nivel de complejidad, facilita interacciones conversacionales continuas que se acoplan al contexto que se esté tratando, similar a como lo haría un tutor humano para con el estudiante \cite{qin2024transforming}. Diversos estudios han demostrado que los LLMs pueden fomentar la auto-reflexión y el pensamiento crítico mediante estrategias como el método Socrático que, en algunos casos, alcanzan e incluso llegan a superar niveles de efectividad comparables a cuestionarios estructurados \cite{cordova2025aiagents}. En particular, la técnica de Retrieval-Augmented Generation (RAG), permite que los LLMs fundamenten sus respuestas en documentos específicos (por ejemplo, guías curriculares, casos de estudio resueltos), de manera que se reducen enormemente las respuestas imaginadas por la IA, a la vez que se mejora la confianza en la herramienta \cite{cordova2025aiagents,levonian2025safechats}.

Los LLMs, por tanto, se diferencian de otras herramientas de IA al ofrecer una tutoría que se adapta al estilo de aprendizaje y comprensión del estudiante, ofrece retroalimentación inmediata, y mantiene el diálogo necesario para estimular el juicio ético en casos de la vida cotidiana del estudiante; todo ello sin pretender sustituir la educación tradicional, sino complementando el aprendizaje teórico al permitir que el estudiante fortalezca su rol como ciudadano en el día a día. Para asegurar que este sistema contribuya a reducir en lugar de profundizar brechas, es necesario entrenarlo con materiales éticos y contextuales, y mantener supervisión continua por educadores y profesionales en el área \cite{unesco2021ethics,unesco2021guidance}. Estas condiciones permiten generar interacciones fundamentadas y culturalmente pertinentes, garantizando una tutoría ética y efectiva en valores y ciudadanía.
	\fi
\fi

% ANTECEDENTES
% ------------------------------------------------------------------------------
\ifdefined\CAPantecedentes
	\newpage
	\chapter{Antecedentes}
	\ifdefined\parpordefecto
		\defaultparformat{h-antecedentes}
	\else
		\input{h-antecedentes}
	\fi
\fi

% OBJETIVOS
% ------------------------------------------------------------------------------
\ifdefined\CAPobjetivos
	\newpage
	\chapter{Objetivos}
	\ifdefined\parpordefecto
		\defaultparformat{i-objetivos}
	\else
		\section{Objetivo general}
Desarrollar una herramienta tecnológica de educación informal orientada al
acompañamiento en la adquisición de aprendizajes sobre formación ciudadana y
valores morales.

\section{Objetivos específicos}
\begin{itemize}
      % \item Implementar un modelo LLM pre-entrenado y optimizado para obtener respuestas coherentes y acordes a la solicitud del usuario.
      \item Determinar si la aplicación de LLM pre-entrenados se puede adaptar a un ambiente de aprendizaje informal.
      \item Integrar una base de datos vectorial para almacenar y recuperar información
            actualizada que proporcione respuestas fundamentadas en el contenido
            preseleccionado.
      \item Desarrollar una interfaz gráfica atractiva para dispositivos móviles que permita
            la interacción entre el usuario y el modelo de inteligencia artifcial.
\end{itemize}
	\fi
\fi

% MARCO TEÓRICO
% ------------------------------------------------------------------------------
\ifdefined\CAPmarcoteorico
	\newpage
	\chapter{Marco teórico}
	\ifdefined\parpordefecto
		\defaultparformat{k-marco_teorico}
	\else
		La relación entre inteligencia artificial (IA) y educación se ha convertido en
un campo de estudio emergente que combina aportes de la pedagogía, la
psicología del aprendizaje y las ciencias de la computación. Diversas
investigaciones han demostrado que los sistemas basados en IA pueden desempeñar
funciones de apoyo al proceso educativo, desde la automatización de tareas
administrativas hasta la personalización de la enseñanza mediante algoritmos de
aprendizaje adaptativo \cite{elstad2024ai,frontiers2025education}. Sin embargo,
más allá de sus aplicaciones instrumentales, la IA plantea un nuevo paradigma
pedagógico que redefine la forma en que se conciben la enseñanza, la
interacción docente-estudiante y la construcción del conocimiento en entornos
digitales.

En este contexto, la formación en valores y ciudadanía adquiere especial
relevancia. Aunque tradicionalmente se ha abordado desde marcos filosóficos y
éticos, su integración con tecnologías emergentes permite explorar nuevas
formas de aprendizaje moral mediadas por el diálogo y la reflexión guiada. La
incorporación de sistemas inteligentes en este ámbito representa tanto una
oportunidad como un desafío: por un lado, posibilita acompañamientos
personalizados que estimulan el pensamiento crítico y la autorregulación ética;
por otro, exige garantizar la responsabilidad, transparencia y confiabilidad de
los modelos utilizados \cite{betterinternet2024,carter2024ethics}.

Desde esta convergencia entre tecnología y formación ética, la literatura
reciente destaca el potencial de los modelos de lenguaje de gran escala (LLMs,
por sus siglas en inglés) para generar entornos conversacionales que promuevan
la reflexión moral y la toma de decisiones fundamentadas
\cite{seibt2024llm,frontiers2025psychology}. Estos sistemas, diseñados bajo
principios éticos y pedagógicos, pueden convertirse en agentes de
acompañamiento educativo informal, capaces de sostener diálogos significativos
y culturalmente pertinentes. De esta manera, la IA no solo actúa como
herramienta tecnológica, sino como mediadora cognitiva y moral, lo que amplia
las posibilidades de aprendizaje y fortalece el desarrollo ciudadano en la era
digital.

\section{Educación ciudadana y valores}
La educación ciudadana constituye un proceso educativo integral orientado a
formar individuos capaces de ejercer sus derechos y deberes de manera
responsable, ética y crítica. Esta formación no se limita al conocimiento de
normas y leyes, sino que promueve valores como la solidaridad, la justicia y el
respeto por la diversidad, esenciales para la convivencia democrática
\cite{unesco2021global, schulz2010iccs}. Además, la educación ciudadana
incorpora competencias sociales y habilidades de pensamiento crítico,
fomentando la participación activa en la comunidad y la toma de decisiones
informadas \cite{bentley2018education}.

\subsection{Educación en valores}
La educación en valores constituye un enfoque pedagógico reconocido a nivel
internacional bajo diversas denominaciones, como educación moral, educación del
carácter o educación ética. Si bien cada una presenta matices particulares y
distintos énfasis, todas comparten la convicción fundamental de que la
formación en valores personales y cívicos representa una responsabilidad
legítima de las instituciones educativas a nivel mundial. En la actualidad,
este ámbito ya no se considera exclusivo del entorno familiar o religioso, pues
diversas investigaciones han evidenciado que una educación desvinculada de los
valores puede limitar de forma significativa el desarrollo integral del
estudiante, tanto en el plano ético como en el académico
\cite{lovat2009values}.

Asimismo, la educación en valores se concibe como un proceso formativo integral
que no solo promueve principios fundamentales de ética y ciudadanía, sino que
se posiciona como un componente esencial y transversal de la calidad educativa.
Lejos de tratarse de un aspecto aislado, establece una relación de mutua
interdependencia con la enseñanza de calidad, al punto de integrarse en una
dinámica de doble hélice que potencia el desarrollo personal, social y
académico del estudiante \cite{lovat2009values}.

\subsection{Formación ciudadana}
La formación ciudadana, bajo el concepto anglosajón de \textit{<<civic
    education>>}, es el conjunto de procesos, formales e informales, mediante los
cuales las personas desarrollan conocimientos, valores, actitudes, habilidades
y compromisos que les permiten participar activamente y de manera crítica en la
vida democrática y comunitaria; éste no está limitado al ámbito escolar ni a
una etapa específica de la vida del individuo, sino que se extiende a lo largo
de su ciclo vital e involucra diversos aspectos externos como la familia, los
medios de comunicación, su comunidad, instituciones educativas, etc.
\cite{crittenden2007civic}

Por lo tanto, la formación ciudadana no se limita a la transmisión de
contenidos normativos sobre el sistema político, sino que incorpora prácticas
educativas activas, como la discusión de temas controversiales, la
participación en acciones colectivas y la reflexión crítica, las cuales han
demostrado tener efectos significativos en el desarrollo de una ciudadanía
activa, consciente y empoderada \cite{crittenden2007civic}.

\subsection{Competencias cívicas fundamentales}
Las competencias cívicas fundamentales son un conjunto integrado de
disposiciones personales y capacidades que permiten a los individuos participar
activamente en sociedades democráticas diversas. De acuerdo con el Consejo de
Europa, estas competencias se organizan en torno a cuatro dimensiones
esenciales: \textbf{los valores} que guían el comportamiento ético; \textbf{las
    actitudes} que predisponen a la apertura y al respeto; \textbf{las habilidades}
necesarias para la interacción democrática; y \textbf{los conocimientos} y
\textbf{la comprensión crítica del mundo} social, político y cultural. Su
desarrollo es clave para convivir como iguales en contextos diversos y
democráticos \cite{barrett2016competences}.

\subsubsection{Valores}
Los valores son creencias fundamentales que orientan a las personas hacia metas
que consideran deseables en la vida. Funcionan como motores de acción y como
criterios que guían la toma de decisiones, al proporcionar marcos de referencia
sobre lo que se considera apropiado pensar o hacer en diversas situaciones.
Estos principios no se limitan a contextos específicos, sino que ofrecen
estándares para evaluar conductas, justificar posturas, elegir entre opciones,
planificar acciones e influir en otros \cite{barrett2016competences}.

\subsubsection{Actitudes}
Las actitudes representan la disposición mental general que una persona adopta
frente a individuos, grupos, instituciones, temas u objetos simbólicos. Esta
orientación suele estar compuesta por cuatro elementos interrelacionados: una
creencia o juicio cognitivo sobre el objeto, una respuesta emocional, una
valoración positiva o negativa, y una inclinación conductual específica hacia
dicho objeto \cite{barrett2016competences}.

\subsubsection{Habilidades}
Las habilidades son capacidades que permiten organizar y ejecutar de forma
eficiente patrones complejos de pensamiento o acción, adaptándolos al contexto
con el propósito de alcanzar un objetivo específico.
\cite{barrett2016competences}

\subsubsection{Conocimientos y Comprensión Crítica}
Los conocimientos representan el conjunto de información que una persona ha
adquirido, mientras que la comprensión crítica implica no solo entender esa
información, sino también valorar de forma reflexiva los sentimientos,
perspectivas y significados asociados a ella. Este tipo de comprensión es
esencial en contextos democráticos e interculturales, ya que permite analizar e
interpretar activamente las situaciones, superando respuestas automáticas o no
conscientes. En ese sentido, favorece la evaluación crítica de lo que se sabe y
de cómo se interpreta el mundo social y político \cite{barrett2016competences}.

\subsection{Educación moral}
La educación moral es el proceso educativo centrado en la moralidad, entendida
principalmente como la adhesión a normas morales y la creencia en su
justificación. Este enfoque puede implicar dos dimensiones fundamentales: por
un lado, la formación moral, que busca desarrollar en los individuos
disposiciones afectivas, conductuales y motivacionales alineadas con esas
normas; y por otro, la indagación moral, que promueve la reflexión crítica y la
construcción de creencias fundamentadas sobre la validez de dichas normas.
Ambas dimensiones pueden ser abordadas de manera complementaria, aunque
conceptualmente son distintas. Además, el autor reconoce que la moralidad
podría abarcar elementos adicionales, como ciertas virtudes o disposiciones
emocionales, cuya formación también puede formar parte significativa de la
educación moral \cite{hand2017moral}.

\subsubsection{Formación Moral}
La formación moral es una dimensión de la educación moral centrada en el
desarrollo de disposiciones afectivas y conductuales que llevan a una persona a
adherirse a normas morales y a responder emocionalmente a ellas. No se trata
únicamente de enseñar qué está bien o mal, sino de fomentar inclinaciones
internas que impulsen a actuar conforme a ciertos estándares, de forma estable
y espontánea. Estas disposiciones pueden incluir sentimientos de satisfacción
cuando se actúa moralmente, incomodidad al violar principios morales, y
expectativas de que otros también se comporten moralmente \cite{hand2017moral}.

Asimismo, este concepto puede abarcar el cultivo de virtudes, entendidas no
solo como inclinaciones a seguir normas, sino como capacidades para moderar
emociones humanas fundamentales. Bajo esta perspectiva, la formación moral no
se reduce a enseñar reglas, sino que apunta a moldear el carácter y las
emociones de forma que apoyen una vida moral \cite{hand2017moral}.

\subsubsection{Indagación Moral}
La indagación moral es la parte de la educación moral que se enfoca en
investigar y evaluar la justificación de las normas morales. Consiste en un
proceso cognitivo mediante el cual se analiza por qué una norma debería ser
aceptada, se examinan los argumentos que la sustentan y se reflexiona
críticamente sobre ellos. Creer en la justificación de una norma no es un
requisito para adherirse a ella, por lo que esta indagación es distinta de la
formación moral, que busca cultivar la adhesión emocional y conductual a esas
normas \cite{hand2017moral}.

En la enseñanza de la indagación moral, es posible adoptar un enfoque
directivo, orientando al individuo hacia una conclusión particular sobre la
validez de una norma, o un enfoque no directivo, en el que se facilita el
análisis y la discusión sin influir en la opinión final. Ambos métodos
promueven la capacidad del individuo para pensar críticamente sobre las normas
morales y su justificación, complementando así la formación moral.
\cite{hand2017moral}

\section{Aprendizaje informal y brecha educativa}
El aprendizaje informal constituye una estrategia educativa que ocurre fuera de
los entornos formales, como escuelas o universidades, y se produce de manera
espontánea en la vida cotidiana. Este tipo de educación fomenta la autonomía
del aprendiz, la creatividad y la resolución de problemas; contribuyendo a
reducir la brecha educativa, especialmente cuando el acceso a la educación
formal es limitado \cite{coombs1968world, mills2014informal}.

\subsection{Educación informal}
La educación informal se refiere a las formas de aprendizaje que ocurren de
manera natural en la vida cotidiana, en una amplia variedad de contextos
geográficos e históricos. Este tipo de educación no se limita a entornos
específicos, sino que suele surgir en espacios donde las personas se sienten
cómodas y con la libertad de socializar entre sí. Aunque este concepto es
asociado tradicionalmente con actividades fuera de la escuela, hoy en día la
educación informal también puede darse dentro de escuelas convencionales o en
organizaciones como el voluntariado juvenil o el movimiento scout.
\cite{mills2014informal}

Este tipo de educación se basa en el diálogo y la conversación, fomentando la
confianza, el respeto y la empatía. No busca imponer resultados específicos,
sino que promueve el aprendizaje a partir de las preocupaciones reales y
cotidianas de las personas; generando cambios positivos y significativos en sus
vidas. Además, la educación informal puede tener un carácter político,
inspirándose en enfoques críticos que buscan que las personas tomen conciencia
de las injusticias sociales y encuentren formas de superarlas, conectando lo
personal con temas sociales y políticos más amplios \cite{mills2014informal}.

\subsection{Autoformación guiada}
La autoformación guiada es un proceso intencional en el que el sujeto
desarrolla su aprendizaje autónomo con el apoyo de una institución, un educador
o un colectivo social. Aunque el aprendiz asume responsabilidad sobre sus
objetivos, recursos, métodos y ritmos, recibe orientación y acompañamiento que
facilitan el desarrollo de su capacidad de aprendizaje y autorregulación
\cite{mills2014informal}.

En este enfoque, la autoformación deja de ser un esfuerzo completamente
solitario o espontáneo para convertirse en una práctica educativa estructurada,
donde el apoyo externo configura las condiciones que permiten que el individuo
desarrolle su propio proyecto formativo y consolide su agencia como aprendiz
activo en contextos educativos no formales \cite{mills2014informal}.

\subsection{Brecha educativa y tecnológica}
La brecha educativa y tecnológica se refiere a las diferencias en el acceso y
aprovechamiento de recursos educativos y tecnológicos entre distintos grupos
sociales. Estas desigualdades afectan la calidad del aprendizaje, limitan la
participación en entornos digitales y pueden amplificar la exclusión social.
Factores como el acceso desigual a internet, dispositivos digitales y
capacitación docente contribuyen a esta brecha, la cual requiere estrategias
integrales de inclusión digital y políticas educativas que promuevan la equidad
\cite{van2005digital, unesco2023monitoring}.

\subsection{Tecnología como herramienta de inclusión educativa}
La tecnología educativa se ha consolidado como una herramienta estratégica para
promover la inclusión educativa, al facilitar el acceso a contenidos y recursos
didácticos a estudiantes con diversidad de contextos, habilidades y
necesidades. Plataformas digitales, dispositivos móviles y herramientas de
aprendizaje asistidas por inteligencia artificial permiten superar barreras
geográficas, socioeconómicas y culturales, contribuyendo a mejorar la equidad
en la educación \cite{teras2022education, unesco2023monitoring}.

\section{Fundamentos de Inteligencia Artificial en Educación}
La inteligencia artificial (IA), aplicada a la educación, ofrece oportunidades
para diseñar entornos de aprendizaje interactivos y personalizados. Una de las
estrategias más prometedoras es la implementación de métodos socráticos
digitales, donde los sistemas de IA guían a los estudiantes mediante preguntas
y diálogos reflexivos, estimulando el pensamiento crítico y la autonomía en la
construcción del conocimiento \cite{holmes2019ai, woolf2010building}.

\subsection{Inteligencia Artificial aplicada a la educación}
La IA en la educación permite automatizar tareas administrativas, ofrecer
tutorías personalizadas, monitorear el progreso de los estudiantes y adaptar
los contenidos a sus necesidades individuales. Estas aplicaciones han
demostrado mejorar la motivación, la eficiencia del aprendizaje y la calidad de
la enseñanza, siempre que se acompañen de supervisión pedagógica y criterios
éticos claros \cite{elstad2024ai, frontiers2025education, carter2024ethics}.

\subsection{Modelos de Lenguaje de Gran Escala}
Los modelos de lenguaje de gran escala (LLMs, por sus siglas en inglés) son
sistemas de inteligencia artificial entrenados con enormes volúmenes de texto
para comprender y generar lenguaje natural. Estos modelos permiten ofrecer
respuestas contextualizadas, realizar tutorías personalizadas y asistir en la
construcción de conocimiento mediante diálogo interactivo. Su potencial
educativo radica en la capacidad de proporcionar retroalimentación inmediata,
adaptada al nivel del estudiante, fomentando la reflexión crítica y la
autoformación \cite{brown2020language, raffel2020exploring}.

\subsection{Arquitectura y funcionamiento de sistemas RAG}
La Recuperación Aumentada por Búsqueda (RAG, por sus siglas en inglés) combina
modelos de lenguaje (LLM, por sus siglas en inglés) con motores de recuperación
de información (IR, por sus siglas en inglés), con el fin de proporcionar
respuestas fundamentadas en el contenido específico deseado. Esta técnica,
introducida por primera vez en 2020 en el artículo \textit{"Retrieval-Augmented
    Generation for Knowledge-Intensive NLP Tasks"}, permite la obtención de
respuestas fundamentadas en fuentes confiables, que bajo el contexto del
sistema promueve la reflexión ética y la resolución de dilemas morales basados
en evidencia. RAG amplía las capacidades de tutoría digital al integrar
conocimiento externo con generación de lenguaje natural
\cite{lewis2020retrieval, khandelwal2020generalization}.

\subsubsection{Arquitectura de RAG}

Un sistema RAG opera en dos fases principales:

\begin{enumerate}
    \item \textbf{Fase de indexación:} Los documentos fuente se procesan mediante:
          \begin{itemize}
              \item Segmentación (chunking) en fragmentos semánticamente coherentes.
              \item Generación de embeddings vectoriales para cada fragmento.
              \item Almacenamiento en bases de datos vectoriales con metadatos.
          \end{itemize}

          Esta fase corresponde con la obtención de conocimiento externo a la
          implementación del sistema. En el artículo original, la figura
          \ref{fig:flujo-rag-teoria} muestra cómo este proceso se implementa mediante la
          combinación de \textit{"memoria paramétrica"} (el LLM utilizado para recibir
          preguntas y generar respuestas) y la \textit{"memoria no paramétrica"} (índice
          vectorial del cual se obtiene el contexto para fundamentar la respuesta), con
          lo cual la generación final obtiene la información requerida para brindar al
          usuario un resultado fundamentado.

          \begin{figure}[H]
              \centering
              \includegraphics[width=0.8\textwidth]{assets/rag\_original.png}
              \caption{Vista general del enfoque aplicado en el artículo de \texttt{Lewis et al}. \cite{lewis2020retrieval}}
              \label{fig:flujo-rag-teoria}
          \end{figure}

    \item \textbf{Fase de inferencia:} Al recibir una consulta:
          \begin{itemize}
              \item Se genera un embedding de la pregunta del usuario.
              \item Se recuperan los K fragmentos más relevantes (típicamente un valor K de entre 3
                    y 10 elementos), conformando el contexto de la consulta.
              \item Se construye un prompt contextualizado que combina la pregunta original con los
                    fragmentos recuperados.
              \item El LLM genera una respuesta basándose exclusivamente en el contexto dado o
                    tomándolo como guía (el enfoque depende de la estructura utilizada para
                    construir el \textit{prompt}) de manera que todo resultado se ve anclado a las
                    fuentes verificadas previamente seleccionadas para alimentar el sistema.

                    Según la implementación, se pueden utilizar distintas estrategias para la
                    combinación de los fragmentos con la generación final. En el documento original
                    se menciona la diferencia de utilizar \textit{RAG sequence} (el modelo
                    selecciona un único documento sobre el cual basará su respuesta) frente a
                    \textit{RAG token} (el modelo genera la respuesta por pasos, seleccionando la
                    fuente a utilizar para cada token independiente, lo que permite combinar más de
                    una fuente).
          \end{itemize}
\end{enumerate}

\subsubsection{Estrategias de segmentación}
El método de segmentación (\textit{chunking}) seleccionado puede llegar a
afectar directamente la calidad de las respuestas obtenidas por el sistema. Las
estrategias más comunes incluyen:
\begin{itemize}
    \item \textbf{Segmentación por tamaño fijo:} se divide el texto en fragmentos de longitud uniforme, independiente de su semántica. Presenta la ventaja de que es el tipo de segmentación más fácil de implementar, ya que basta solamente con establecer un límite de palabras y separar todo el texto en dicho límite. Sin embargo, al no evaluar el sentido semántico en cada fragmento, podría dar lugar a rupturas de contexto o pérdida de sentido. \cite{wang-etal-2025-document}
    \item \textbf{Segmentación semántica:} esta segmentación divide el texto respetando unidades de sentido, identificadas mediante signos de puntuación, saltos de línea, identificación de encabezados, secciones, listas, etc. El propósito de esta estrategia es preservar la continuidad de contexto entre cada fragmento. \cite{wang-etal-2025-document}
    \item \textbf{Segmentación recursiva con solapamiento:} esta técnica divide el texto utilizando alguna técnica anterior, con la peculiaridad de incluir al inicio o al final una parte del fragmento contiguo. Es decir, se define un porcentaje de solapamiento que se refiere a qué tanto del fragmento siguiente (o anterior) se incluirá como parte del nuevo fragmento, de manera que se controle de forma explícita la continuidad. \cite{wang-etal-2025-document}
\end{itemize}

La estrategia de segmentación dependerá de la implementación del sistema RAG
que se desea realizar. Se debe tomar en cuenta que la estrategia utilizada
afectará directamente la calidad de las respuestas, ya que esto define cómo el
LLM obtendrá el contexto del cual se basará para brindar la respuesta a la
consulta dada. \cite{wang-etal-2025-document}

\subsubsection{Ventajas de RAG en educación}
La implementación de sistemas RAG en enfoques educativos, brinda varias
ventajas orientadas al uso de modelos de inteligencia artificial:
\begin{itemize}
    \item \textbf{Reduce alucinaciones} al obligar al sistema a utilizar respuestas fundamentadas en el \textit{corpus} definido para el proyecto. \cite{gupta2024comprehensivesurveyretrievalaugmentedgeneration}
    \item \textbf{Permite actualización del conocimiento} sin reentrenar el modelo; basta solamente con modificar, añadir o eliminar el corpus del proyecto para que el sistema utilice esta nueva información. \cite{gupta2024comprehensivesurveyretrievalaugmentedgeneration}
    \item \textbf{Facilita trazabilidad} y citación de fuentes, lo cual es particularmente importante en contextos educativos en los que es necesario fundamentar de dónde se obtiene toda la información proporcionada. \cite{gupta2024comprehensivesurveyretrievalaugmentedgeneration}
    \item Es apropiado para \textbf{dominios especializados} o corpus limitados, como
          asignaturas o materiales didácticos que no están bien cubiertos en los datos de
          entrenamiento general del LLM.
          \cite{gupta2024comprehensivesurveyretrievalaugmentedgeneration}
\end{itemize}

\subsubsection{Desafíos conocidos}
A pesar de sus beneficios, la implementación de RAG también presenta ciertos
retos a cubrir en proyectos educativos de alto impacto:
\begin{itemize}
    \item \textbf{Dependencia crítica} de la calidad del corpus. La calidad de las respuestas depende estrictamente de la calidad del contenido educativo utilizado para alimentar el modelo, por lo que si los documentos están mal organizados, contienen errores o están desactualizados; la recuperación de contexto será débil. \cite{zheng2025knowshiftqarobustragsystems}
    \item \textbf{Riesgo de fragmentación} que rompa la coherencia contextual. Utilizar una estrategia o combinación de estrategias de segmentación inapropiada, puede llevar a que el contexto obtenido por el modelo pierda de sentido, o bien, que una consulta que sí está relacionada con el contenido del corpus no pueda ser respondida. \cite{zheng2025knowshiftqarobustragsystems}
    \item \textbf{Limitaciones de la ventana de contexto del LLM} a pesar de que el sistema de recuperación diseñado obtenga una base contextual amplia, según el modelo LLM utilizado, generalmente se tiene un límite de tokens permitido para cada consulta, por lo que el \text{prompt} construido también cuenta con limitaciones de longitud y, por lo tanto, del nivel de especificación y claridad exigido al modelo. \cite{zheng2025knowshiftqarobustragsystems}
    \item Es apropiado para \textbf{Dificultad para sintetizar información de múltiples
              fragmentos dispersos} Al combinar varios fragmentos en un solo \textit{prompt},
          si no se ha seleccionado el \textit{corpus} cuidadosamente, se puede incurrir
          en contradicciones o redundancia en las consultas.
          \cite{zheng2025knowshiftqarobustragsystems}
\end{itemize}

\subsection{\textit{Embeddings} y representación semántica del texto} Los \textit{embeddings} son
representaciones vectoriales de palabras, frases o documentos que capturan sus
significados semánticos. Esta técnica permite que los sistemas de IA comparen y
recuperen información de manera eficiente, lo que permite medir la similitud
entre conceptos y facilita búsquedas semánticas. En educación, los
\textit{embeddings} permiten vincular preguntas de los estudiantes con
contenidos relevantes, apoyando la personalización del aprendizaje
\cite{mikolov2013efficient, le2014distributed}.

\subsection{Bases de datos vectoriales y búsqueda semántica}
Las bases de datos vectoriales permiten almacenar y consultar
\textit{embeddings} de manera eficiente, habilitando la búsqueda semántica en
grandes volúmenes de información. Este enfoque supera las limitaciones de las
búsquedas basadas en palabras clave, lo que permite que los estudiantes y sistemas
educativos accedan a contenidos relevantes de manera más precisa y
contextualizada, facilitando la recuperación de conocimiento en entornos
digitales \cite{johnson2019billion, han2023comprehensive}.

\subsection{\textit{Prompt Engineering} y diseño de instrucciones} La disciplina de \textit{Prompt Engineering}
consiste en el diseño de instrucciones efectivas para guiar el comportamiento
de modelos de lenguaje hacia objetivos específicos.
\cite{white2023promptpatterncatalogenhance} El enfoque principal es brindar al
modelo directivas claras con el fin de obtener el resultado final esperado,
enfocados en ser tan específicos y directos como el modelo permita.

\subsubsection{Componentes de un \textit{prompt} efectivo}
\begin{itemize}
    \item \textbf{System prompt:} Define el rol que el asistente debe adoptar ante cada consulta que se le solicite, así como el tono de las respuestas generadas y los límites que debe cumplir (por ejemplo, indicar que debe responder basado solamente en el \textit{corpus} del proyecto e ignorar todo lo externo). \cite{zhou2023largelanguagemodelshumanlevel}
    \item \textbf{Contexto:} Aquí se especifica la base que debe fundamentar todas las respuestas del modelo. En el caso de un sistema RAG, es aquí donde se incluyen todos los fragmentos recuperados de la fuente de datos creada previamente. \cite{zhou2023largelanguagemodelshumanlevel}
    \item \textbf{Instrucción:} Define la tarea específica que se espera que deba cumplir el asistente. Aquí puede ir la pregunta, solicitud de información o generación de contenido multimedia (si aplica). \cite{zhou2023largelanguagemodelshumanlevel}
    \item \textbf{Formato de salida:} Se debe definir también cómo se espera que el modelo responda, ya sea porque se busca obtener una respuesta segmentada en separadores identificables, o bien, para especificar un formato específico. \cite{zhou2023largelanguagemodelshumanlevel}
    \item \textbf{Ejemplos (opcional):} Si se quiere ser aún más explícito sobre cómo se espera que el modelo se comporte, se pueden indicar ejemplos claros del comportamiento que debe tener el modelo a partir de las consultas recibidas. \cite{zhou2023largelanguagemodelshumanlevel}
\end{itemize}

\subsubsection{Estrategias en contextos educativos}
Con el objetivo de enfocar el diseño de \textit{prompts} al campo de la
educación, se enfatizan prácticas específicas que permiten obtener el flujo de
pensamiento del asistente, establecer un tipo de interacción específica con el
estudiante (por ejemplo, el uso del método socrático) o solicitar las fuentes
utilizadas.
\begin{itemize}
    \item \textbf{\textit{Chain-of-thoutht prompting}:} Solicita al modelo el razonamiento que utilizó para responder la pregunta, se exige el paso a paso de cómo llegó hasta la respuesta brindada. \cite{NEURIPS2022_9d560961}
    \item \textbf{\textit{Socratic prompting}:} Indica al modelo que se debe guiar por el método socrático, el cual consiste en incentivar al usuario a obtener una respuesta final por sí mismo, en lugar de brindar una respuesta directa a la consulta dada.\cite{NEURIPS2022_9d560961}
    \item \textbf{\textit{Constitutional AI}:} Incorpora principios éticos en las instrucciones, por ejemplo, la omisión de palabras o temas sensibles.\cite{NEURIPS2022_9d560961}
    \item \textbf{\textit{Retrieval-aware prompting}:} Exige al modelo citar fuentes en todas sus respuestas. Puede ser útil, aunque también vale la pena analizar si conviene más esta estrategia o simplemente almacenar en los metadatos de los fragmentos los documentos originales.\cite{NEURIPS2022_9d560961}
\end{itemize}

\subsection{Tutoría personalizada con IA}
La tutoría personalizada con IA permite adaptar los contenidos y las
estrategias de enseñanza al nivel, intereses y ritmo de cada estudiante. Los
sistemas inteligentes analizan patrones de aprendizaje y ofrecen
retroalimentación inmediata, identificando áreas de dificultad y recomendando
recursos específicos. Esta personalización mejora la motivación, la retención
de conocimiento y promueve la autonomía del aprendiz \cite{woolf2010building,
    zawacki-richter2019systematic}.

\subsection{Método socrático aplicado a entornos digitales}
Los entornos digitales permiten implementar el método socrático mediante
sistemas de IA que guían a los estudiantes a través de preguntas reflexivas y
secuencias de razonamiento. Esta estrategia fomenta el pensamiento crítico y la
autonomía, ya que los alumnos deben analizar, argumentar y evaluar sus propias
respuestas antes de recibir retroalimentación. El uso de chatbots y asistentes
inteligentes basados en este método facilita un aprendizaje personalizado y
continuo, replicando la interacción dialógica propia del enfoque socrático
tradicional \cite{favero2024socratic, woolf2010building}.

\subsection{Métricas de evaluación de chatbots educativos}
La evaluación de asistentes conversacionales educativos requiere métricas más
allá de la precisión técnica, incorporando dimensiones pedagógicas incluso si
no son herramientas planificadas para su utilización en entornos formales
tradicionales de educación. \cite{10.3389/frai.2021.654924}

\subsubsection{Métricas de calidad de respuesta}
A nivel semántico, es importante identificar aspectos que pueden definir una
respuesta dada como exitosa o no, enfocados en la pregunta inicial dada, el
\textit{corpus} utilizado para la alimentación del modelo y las instrucciones
de tono y comprensión indicadas al modelo. \cite{ADAMOPOULOU2020100006}
\begin{itemize}
    \item \textbf{Relevancia:} ¿La respuesta aborda la pregunta planteada?
    \item \textbf{Precisión:} ¿La información es correcta?
    \item \textbf{Completitud:} ¿Cubre todos los aspectos necesarios?
    \item \textbf{Claridad:} ¿Es comprensible para el público objetivo?
\end{itemize}

\subsubsection{Métricas de desempeño técnico}
Por otro lado, a nivel técnico también es útil identificar el desempeño del
sistema en cuanto a tiempos de respuesta, integridad de los componentes, coste
de operaciones y recursos utilizados.
\begin{itemize}
    \item \textbf{Tasa de éxito} (\% de preguntas respondidas apropiadamente)
    \item \textbf{Precisión y Recall} en detección de preguntas fuera de alcance
    \item \textbf{Latencia} (tiempos de respuesta)
    \item \textbf{Consumo de tokens}
\end{itemize}

\subsection{Evaluación de calidad de respuestas en sistemas RAG}
Por su parte, los sistemas que implementan RAG presentan sus propios desafíos
específicos de evaluación relacionados con la recuperación y síntesis de la
información almacenada y recuperada. Es necesario identificar si la
recuperación semántica fue exitosa, si el almacenamiento de los fragmentos y
metadatos se está haciendo correctamente y si el modelo está tomando de
referencia los documentos apropiados dentro del \textit{corpus}.

\subsubsection{Métricas de recuperación (\textit{Retrieval})}
Para poder obtener estas métricas, las pruebas deben consistir de un conjunto de preguntas previamente analizadas, así como la identificación de qué fragmentos son realmente relevantes para abordar cada pregunta.
\begin{itemize}
    \item \textbf{Precisión@K:} Se refiere al porcentaje de fragmentos obtenidos que corresponden a los fragmentos relevantes identificados para la consulta.
    \item \textbf{Recall@K:} Se refiere al porcentaje de fragmentos relevantes recuperados por el sistema. Es decir, de todos los fragmentos relevantes identificados para la consulta, qué porcentaje fue seleccionado por el modelo.
    \item \textbf{\textit{Mean Reciprocal Rank} (MRR):} Del top K obtenido por el modelo, en qué posición se sitúa el primer fragmento relevante identificado. Se guía bajo la siguiente ecuación:
          \begin{equation}
              \text{MRR} = \frac{1}{|N|} \sum_{i=1}^{|N|} \frac{1}{\text{K}_i}
          \end{equation}

          En donde \texttt{K} el número de fragmentos recuperados, \texttt{i} representa
          cada fragmento relevante y \texttt{N} la posición del fragmento dentro del
          top-K.
\end{itemize}

\subsubsection{Métricas de generación (\textit{Generation})}
A diferencia de los sistemas puramente extractivos, los sistemas RAG deben ser
evaluados tanto por la calidad de la información generada como por su fidelidad
al contexto recuperado. Para ello, se emplean métricas que permiten analizar
si la respuesta es coherente, pertinente y derivada correctamente del material
de referencia.

\begin{itemize}
    \item \textbf{Faithfulness (Fidelidad):} Evalúa si la respuesta se limita a
          la información contenida en los documentos recuperados, evitando
          alucinaciones o afirmaciones no sustentadas.
          \cite{shuster2021retrievalaugmentationreduceshallucination}

    \item \textbf{Answer Relevance (Relevancia de la respuesta):} Determina el
          grado en que la respuesta generada responde efectivamente a la pregunta
          formulada, sin desviarse hacia información tangencial o irrelevante.
          \cite{bang2023multitaskmultilingualmultimodalevaluation}

    \item \textbf{Context Utilization (Utilización del contexto):} Mide la
          dependencia efectiva del modelo respecto al contexto recuperado. Una
          respuesta es de alta calidad cuando demuestra haber utilizado
          fragmentos relevantes del corpus y no únicamente el conocimiento previo
          del modelo.
          \cite{lewis2020retrieval}

    \item \textbf{Coherencia y fluidez:} Se relaciona con la estructura interna
          de la respuesta, su legibilidad y consistencia discursiva. En el contexto
          educativo, la claridad también constituye un indicador pedagógico de
          calidad.
          \cite{ADAMOPOULOU2020100006}
\end{itemize}

\subsection{Validación sin usuarios finales: enfoques y limitaciones}
La validación técnica de sistemas educativos sin participación de usuarios
finales presenta limitaciones inherentes, pero permite establecer la viabilidad
funcional del sistema antes de su despliegue \cite{luckin2016intelligence}.
Estos procedimientos son fundamentales en etapas tempranas del desarrollo.

\subsubsection{Enfoques de validación técnica:}
\begin{enumerate}
    \item \textbf{Validación basada en corpus:} Verifica que las respuestas
          provengan de los materiales fuente y que no contengan información ajena
          al contenido educativo.
    \item \textbf{Validación por expertos:} Permite asegurar la pertinencia
          pedagógica y detectar errores conceptuales o éticos en las respuestas.
    \item \textbf{Pruebas de estrés:} Incluyen preguntas fuera de alcance,
          ambiguas o sensibles para evaluar la resiliencia del modelo.
    \item \textbf{Simulación de usuarios:} Reproduce escenarios conversacionales
          con perfiles variados, como estudiantes de diferentes edades o niveles
          educativos.
          \cite{luckin2016intelligence,holmes2019ai,zawacki-richter2019systematic}
\end{enumerate}

\subsubsection{Limitaciones reconocidas:}
\begin{itemize}
    \item No mide el impacto pedagógico real ni el aprendizaje logrado.
    \item La validación manual puede estar sujeta a sesgos del evaluador.
    \item Los escenarios simulados no reflejan completamente la complejidad del uso real.
    \item No existe retroalimentación iterativa basada en la experiencia de usuarios
          finales.
          \cite{luckin2016intelligence,holmes2019ai,zawacki-richter2019systematic}
\end{itemize}

\subsubsection{Implicaciones:}
Los resultados obtenidos de esta validación deben interpretarse como una
demostración de factibilidad técnica y coherencia funcional, pero no como
evidencia de efectividad educativa. Estudios posteriores con usuarios reales
serán necesarios para medir impacto pedagógico y aceptación.
\cite{holmes2019ai}

\subsection{Congruencia y fundamentación en respuestas educativas}
En contextos educativos, además de la precisión técnica, se espera que el
sistema responda de forma congruente con las expectativas del rol asignado
(alumno, tutor o asistente). La \textbf{congruencia conversacional} se define
como el grado en que el sistema actúa según lo esperado ante cada tipo de
pregunta: responder apropiadamente cuando debe hacerlo y abstenerse en
situaciones fuera de alcance \cite{Li2021ConsistencyChatbot}.

Este tipo de métrica puede calcularse empíricamente como el porcentaje de
interacciones en las que el modelo actúa conforme al comportamiento esperado.
En el presente proyecto, dicha métrica complementa las tradicionales de
precisión y fidelidad, proporcionando una visión integral de la calidad del
comportamiento conversacional \cite{Li2021ConsistencyChatbot}.

\section{Ética y Responsabilidad en IA Educativa}
La ética en IA educativa aborda la responsabilidad en el diseño, implementación
y uso de sistemas inteligentes en contextos de aprendizaje. Incluye
consideraciones sobre privacidad de los datos, equidad, transparencia,
inclusión e impacto social. Garantizar que los estudiantes sean tratados de
manera justa y que los sistemas no reproduzcan sesgos existentes es crucial
para la confianza y efectividad de la educación asistida por IA
\cite{selwyn2019should, williamson2023social}.

\subsection{Principios éticos fundamentales en IA}
Los principios éticos fundamentales en IA incluyen transparencia, justicia, no
discriminación, responsabilidad, privacidad y seguridad. En el ámbito
educativo, estos principios guían el desarrollo de sistemas que respeten la
dignidad del estudiante, promuevan equidad en el aprendizaje y faciliten la
rendición de cuentas por parte de desarrolladores y educadores. La aplicación
de estos principios permite aprovechar el potencial de la IA sin comprometer la
integridad pedagógica \cite{jobin2019global, floridi2018ai}.

\subsection{Sesgos algorítmicos y culturales en contextos latinoamericanos}
La prevención de sesgos algorítmicos se centra en garantizar que los sistemas
de IA no reproduzcan ni amplifiquen desigualdades existentes en la educación.
Esto implica analizar los datos de entrenamiento, identificar posibles sesgos y
aplicar técnicas de mitigación, como ajuste de ponderaciones, diversificación
de datasets y pruebas de equidad en los resultados. La prevención de sesgos
asegura que todos los estudiantes reciban oportunidades de aprendizaje justas y
equitativas \cite{mehrabi2019survey, binns2018fairness}.

\subsection{Transparencia y explicabilidad en sistemas inteligentes}
La transparencia y explicabilidad son fundamentales para que docentes,
estudiantes y desarrolladores comprendan cómo un sistema de IA toma decisiones.
Esto incluye técnicas de interpretabilidad que permitan visualizar la lógica de
los modelos y justificar las recomendaciones que generan. En educación, la
explicabilidad ayuda a confiar en las decisiones automatizadas, facilita la
supervisión pedagógica y permite detectar errores o sesgos
\cite{doshi2017towards, lipton2018mythos}.

\subsection{Responsabilidad ante respuestas incorrectas o inadecuadas}
La responsabilidad en sistemas de IA educativa implica definir con claridad los
mecanismos para abordar errores, recomendaciones inadecuadas o información
potencialmente nociva generada por los algoritmos. Cuando un sistema
automatizado produce contenidos incorrectos, los efectos pueden ser
especialmente sensibles en contextos educativos, ya que influyen directamente
en el aprendizaje, la motivación y las decisiones académicas de los estudiantes
\cite{jobin2019global, floridi2018ai}.

Esta responsabilidad recae tanto en los desarrolladores, quienes deben
implementar modelos seguros, mecanismos de verificación y pruebas continuas,
como en los docentes y las instituciones que integran la tecnología. Esto
incluye ofrecer rutas de corrección, permitir retroalimentación humana y
asegurar canales claros para reportar fallos. De esta manera, la IA se integra
como una herramienta asistiva bajo supervisión profesional, en lugar de delegar
completamente la evaluación y orientación pedagógica \cite{jobin2019global,
    floridi2018ai}.

\subsection{Privacidad y seguridad en aplicaciones educativas para menores}
El uso de aplicaciones educativas basadas en IA en contextos escolares requiere
un enfoque riguroso de protección de datos, especialmente cuando se trata de
menores de edad. La información académica, conductual y biométrica recopilada
por estos sistemas constituye un activo sensible que debe ser gestionado bajo
principios de minimización de datos, consentimiento informado y almacenamiento
seguro \cite{selwyn2019should, williamson2023social}.

Organismos internacionales han enfatizado la importancia de resguardar la
identidad digital de los estudiantes, evitar usos secundarios no autorizados y
garantizar que los datos no se utilicen para prácticas discriminatorias o
comerciales. Las instituciones tienen la responsabilidad de establecer
políticas claras de acceso, supervisar proveedores tecnológicos y aplicar
estándares robustos de ciberseguridad. La protección de los datos de menores no
solo es una obligación legal y ética, sino también una condición necesaria para
preservar la confianza en entornos educativos mediados por IA
\cite{selwyn2019should, williamson2023social}.

\subsection{Supervisión pedagógica en sistemas automatizados}
A pesar de la autonomía de los sistemas de IA, la supervisión pedagógica es
esencial para garantizar la calidad del aprendizaje. Docentes y tutores deben
monitorear el funcionamiento de los sistemas automatizados, evaluar la
relevancia y exactitud de las respuestas generadas, y ajustar los parámetros de
personalización según las necesidades de los estudiantes. Este enfoque mixto
asegura que la tecnología complemente, y no reemplace, la guía educativa
\cite{holmes2019ai, luckin2016intelligence}.

\section{Aprendizaje Móvil en Contextos de Recursos Limitados}
El aprendizaje móvil (\textit{mobile learning} o m-learning) se ha convertido
en un medio clave para ampliar el acceso a experiencias educativas,
especialmente en regiones donde las limitaciones tecnológicas, de
infraestructura o económicas dificultan el aprendizaje tradicional. En
contextos con recursos limitados, los dispositivos móviles permiten superar
barreras geográficas y temporales, democratizando oportunidades de acceso a
información, formación técnica y herramientas de apoyo educativo
\cite{traxler2007defining, unesco2013policy}.

Sin embargo, la implementación efectiva del aprendizaje móvil requiere
considerar retos como la disponibilidad de dispositivos, la alfabetización
digital de los usuarios, los costos de conectividad y las brechas de
infraestructura. El diseño de soluciones educativas móviles sostenibles debe
responder a estos factores para garantizar accesibilidad, pertinencia cultural
y equidad tecnológica. \cite{unesco2013policy}

\subsection{Panorama del aprendizaje móvil en Guatemala y Centroamérica}
El crecimiento del aprendizaje móvil en Guatemala y Centroamérica ha sido
gradual pero progresivo, impulsado por iniciativas de digitalización,
comunidades tecnológicas emergentes y el interés institucional por modernizar
procesos educativos y productivos. Según el BID, la región ha avanzado
significativamente en adopción tecnológica, pero aún enfrenta brechas en cuanto
a infraestructura digital, capacidad de investigación e inversión en innovación
\cite{worldbank2022revolution}.

\subsection{Diseño de experiencias móviles para usuarios con baja alfabetización digital}
El diseño de experiencias educativas móviles para usuarios con baja
alfabetización digital requiere estrategias centradas en usabilidad,
simplicidad y acompañamiento formativo. UNESCO destaca que las interfaces
visuales claras, los flujos guiados y los recursos multimedia accesibles pueden
favorecer la participación de usuarios principiantes
\cite{unesco2021reimagining}.

\subsection{Consideraciones de conectividad intermitente y consumo de datos}
En muchos contextos latinoamericanos, incluidos sectores rurales de Guatemala,
el acceso a Internet es costoso e intermitente. Por ello, las aplicaciones
educativas móviles deben optimizar el consumo de datos, ofrecer funcionalidad
fuera de línea y emplear técnicas de sincronización diferida para resguardar el
progreso del usuario cuando no haya conexión \cite{shrestha2010offline,
    unesco2013policy, android_offline_first}.

Prácticas recomendadas incluyen compresión de recursos multimedia,
almacenamiento local temporal, caching inteligente y utilización de formatos
eficientes. La capacidad de operar con conectividad limitada no solo reduce
barreras de acceso, sino que también mejora la adopción sostenida de
herramientas educativas en zonas marginadas \cite{col_offline_design_model}.

\subsection{Aplicaciones móviles en la educación}
Las aplicaciones móviles educativas permiten acceder a recursos y experiencias
de aprendizaje en cualquier momento y lugar. Integradas con IA, estas apps
pueden ofrecer tutorías personalizadas, seguimiento del progreso,
retroalimentación inmediata y gamificación del aprendizaje. Su portabilidad y
accesibilidad contribuyen a reducir la brecha educativa y facilitan la
inclusión digital \cite{traxler2009learning, crompton2018use}.

\section{Tecnologías de Implementación}
El uso responsable de IA en educación implica enseñar a los estudiantes a
utilizar herramientas inteligentes sin vulnerar normas éticas ni académicas.
Esto incluye fomentar la autoría propia, la citación adecuada de fuentes y el
desarrollo de habilidades de pensamiento crítico para interpretar la
información generada por la IA. La integridad académica asegura que la
tecnología complemente el aprendizaje sin reemplazar la reflexión y el esfuerzo
personal \cite{bretag2019academic, eaton2023postplagiarism}.

\subsection{Arquitectura cliente-servidor (\textit{frontend}/\textit{backend})}
La arquitectura cliente-servidor es un modelo de diseño de software en el que
el cliente (por ejemplo, una app móvil o navegador web) solicita servicios al
servidor, el cual procesa la información, ejecuta lógica de negocio y responde
con datos. En educación digital, esta arquitectura permite centralizar recursos
educativos, gestionar bases de datos y ofrecer aplicaciones interactivas
seguras y escalables. El \textit{frontend} se encarga de la interfaz y la experiencia de
usuario, mientras que el \textit{backend} gestiona la lógica, la seguridad y la
integración con IA y bases de datos \cite{tanenbaum2007distributed,
    hwang2011cloud}.

\subsubsection{Patrón de diseño MVC}
El patrón de diseño \textit{Model-View-Controller} (MVC) es una arquitectura
ampliamente utilizada para la construcción de interfaces de usuario. Su
objetivo principal es separar la representación de la información de la lógica
que la gestiona, promoviendo modularidad, reutilización y facilidad de
mantenimiento. \cite{reenskaug1979mvc}

En este patrón, el \textbf{Model} contiene los datos y la lógica de negocio; la
\textbf{View} es responsable de mostrar la información al usuario; y el
\textbf{Controller} actúa como intermediario, recibiendo entradas del usuario y
coordinando las actualizaciones entre el modelo y la vista. Esta separación
permite que cambios en la interfaz no afecten directamente a la lógica del
sistema y viceversa. \cite{reenskaug1979mvc}

El patrón MVC tuvo sus orígenes en el entorno \textit{Smalltalk-80} y ha sido
ampliamente adoptado en múltiples \textit{frameworks} modernos de desarrollo
tanto web como de escritorio debido a su capacidad de estructurar aplicaciones
complejas de forma eficiente \cite{reenskaug1979mvc}.

\subsection{Frameworks de desarrollo móvil: Kotlin y ecosistema Android}
Kotlin es un lenguaje de programación moderno y seguro que se utiliza para el
desarrollo de aplicaciones Android. Presenta características como tipado
estático, interoperabilidad con Java, sintaxis concisa y soporte nativo en
Android Studio. Su uso permite crear aplicaciones robustas, escalables y
fáciles de mantener, integrando librerías modernas y \textit{frameworks} de IA
para educación digital \cite{leiva2018kotlin, fedirchuk2018kotlin}.

\subsubsection{Patrón de diseño MVVM}
El patrón \textit{Model-View-ViewModel} (MVVM) es una arquitectura de software
que separa de forma clara la lógica de negocio de la interfaz de usuario,
promoviendo el desacoplamiento y facilitando la mantenibilidad del código.
\cite{fowler2015mvvm}

En este patrón, el \textbf{Model} encapsula los datos y reglas de negocio; la
\textbf{View} representa la interfaz de usuario; y el \textbf{ViewModel} actúa
como un intermediario que gestiona el estado de la vista, expone datos al
usuario y maneja la lógica de presentación. La comunicación suele realizarse
mediante mecanismos de enlace de datos (\textit{data binding}), lo que permite
que la interfaz se actualice automáticamente ante cambios en los datos.
\cite{fowler2015mvvm}

\subsection{Bases de datos vectoriales y su contraste con bases relacionales}
Las bases de datos vectoriales almacenan representaciones numéricas
(\textit{embeddings}) de información, permitiendo búsquedas semánticas rápidas
y precisas. En cambio, las bases de datos relacionales organizan información en
tablas con relaciones explícitas y consultas estructuradas. Para educación
digital basada en IA, las bases vectoriales permiten recuperar contenido
relevante según el significado, mientras que las relacionales son útiles para
gestión de usuarios, cursos y registros administrativos. Integrar ambos tipos
optimiza tanto la eficiencia semántica como la consistencia estructural de los
datos \cite{johnson2019billion, chaudhuri1995self}.

\subsection{APIs de IA: integración de modelos conversacionales}
Las APIs de IA, como \textit{OpenAI} y \textit{Gemini}, permiten integrar
modelos de lenguaje conversacionales en aplicaciones educativas. Estos
servicios ofrecen capacidades de generación de texto, comprensión de lenguaje
natural y tutoría personalizada, facilitando la interacción del estudiante con
sistemas de IA. La integración se realiza mediante solicitudes a la API, manejo
de \textit{tokens} y adaptación de respuestas al contexto educativo,
permitiendo desarrollar tutores digitales eficientes y éticos
\cite{openai2023api, google2024gemini}.
	\fi
\fi

% METODOLOGÍA
% ------------------------------------------------------------------------------
\ifdefined\CAPmetodologia
	\newpage
	\chapter{Metodología}
	\ifdefined\parpordefecto
		\defaultparformat{l-metodologia}
	\else
		El desarrollo del proyecto \textit{Ciudadano Digital} se llevó a cabo bajo el
marco de trabajo SCRUM, un enfoque ágil ampliamente utilizado en ingeniería de
software que permite la entrega incremental de productos funcionales mediante
ciclos cortos de desarrollo denominados \textit{sprints}. Esta metodología fue
seleccionada debido a su flexibilidad, capacidad de adaptación a cambios en los
requerimientos y enfoque en la mejora continua, elementos clave en un proyecto
de innovación educativa..

A lo largo del proceso, se definieron seis \textit{sprints} principales, cada
uno con objetivos concretos y entregables verificables, orientados a la
obtención progresiva de un prototipo funcional y validado de la aplicación.
Cada \textit{sprint} tuvo una duración de entre tres y cuatro semanas,
ajustándose según la complejidad técnica y la carga académica del periodo
correspondiente.

Cada ciclo SCRUM siguió las fases de planificación, desarrollo, revisión y
retrospectiva, bajo los siguientes principios:

\begin{itemize}
      \item Planificación (Planeación del Sprint): se definieron los objetivos y alcance
            del \textit{sprint}, así como las tareas específicas necesarias para cumplir la
            meta establecida.
      \item Desarrollo (Ejecución del Sprint): se ejecutaron las tareas asignadas con
            enfoque en la funcionalidad incremental, priorizando siempre la obtención de
            resultados medibles.
      \item Revisión (Revisión del Sprint): al cierre de cada \textit{sprint}, se evaluó el
            cumplimiento de los objetivos, la calidad del producto obtenido y la
            satisfacción de los criterios de aceptación definidos.
      \item Retrospectiva (\textit{Sprint Retrospective}): se analizaron los aprendizajes
            obtenidos, los obstáculos encontrados y las oportunidades de mejora para el
            siguiente \textit{sprint}.
\end{itemize}

El enfoque SCRUM permitió mantener un flujo de trabajo iterativo, controlado y
adaptable, asegurando que cada componente técnico se ejecutara tomando como
enfoque la experiencia esperada del usuario objetivo. En este caso, el usuario
fue representado a través de una \textbf{Persona} descrita con base en un
proceso de investigación y perfilamiento descrito en el primer \textit{sprint}.

A partir del segundo \textit{sprint}, los entregables se enfocaron en la
construcción progresiva del sistema técnico, desde la recopilación y
procesamiento de contenido educativo, hasta la implementación del servidor, el
desarrollo de la interfaz móvil y las fases finales de validación y
documentación.

El producto mínimo viable (MVP, por sus siglas en inglés) obtenido al finalizar
el último \textit{sprint} constituye una versión funcional del asistente
inteligente de educación ciudadana, capaz de interactuar con el usuario,
contextualizar sus preguntas y generar respuestas basadas en la información
previamente curada y vectorizada.

\section{Enfoque metodológico aplicado al contexto del proyecto}
La metodología utilizada en esta primera iteración de \textit{Ciudadano
      Digital} se basó en SCRUM, adaptada a las particularidades de un proyecto
sociotecnológico cuyo objetivo principal fue validar la viabilidad técnica de
integrar un sistema de generación aumentada por recuperación (RAG, por sus
siglas en inglés) dentro de un entorno educativo. Para ello, se incorporaron
actividades orientadas tanto al desarrollo incremental del software como al
análisis contextual del futuro usuario final, sin que esto implicara evaluar
impactos pedagógicos en esta etapa. La estructura de seis \textit{sprints}
permitió organizar el trabajo de manera secuencial y enfocada, atendiendo los
componentes esenciales del sistema.

En cada \textit{sprint} se integraron tareas específicas relacionadas con el
levantamiento de requisitos tecnológicos, la definición del perfil preliminar
del usuario objetivo y la verificación de compatibilidad entre las decisiones
técnicas y las necesidades operativas del macroproyecto. La definición del
usuario objetivo sirvió como referencia para orientar los objetivos de los
primeros \textit{sprints}, mientras que los siguientes se concentraron en el
procesamiento del corpus, la construcción del servidor, el desarrollo de la
interfaz móvil y la validación funcional básica. Esta estructura permitió que
las decisiones técnicas avanzaran de forma ordenada y coherente con el
propósito de esta iteración.

\subsection{Estructura de los \textit{Sprints}}
La fase metodológica del desarrollo del proyecto estuvo compuesta por seis \textit{sprints}, cada uno con un enfoque y duración específicos (Cuadro \ref{tab:estructura-sprints}). El propósito de esta organización fue facilitar la gestión del proyecto, permitiendo una entrega progresiva de resultados y la incorporación de retroalimentación continua, la cual se llevó a cabo al finalizar cada \textit{sprint} evaluando si se alcanzó el objetivo establecido para el mismo a través de las tareas completadas.
\begin{table}[H]
      \centering
      \renewcommand{\arraystretch}{1.2}
      \begin{tabular}{|l|p{8cm}|c|}
            \hline
            \textbf{\textit{Sprint}} & \textbf{Meta Principal}                                  & \textbf{Duración Estimada} \\ \hline
            \textit{Sprint} 1        & Identificación del perfil de usuario objetivo (Persona). & 3 semanas                  \\ \hline
            \textit{Sprint} 2        & Recolección y procesamiento del contenido educativo.     & 4 semanas                  \\ \hline
            \textit{Sprint} 3        & Construcción e implementación del servidor.              & 4 semanas                  \\ \hline
            \textit{Sprint} 4        & Desarrollo de la interfaz móvil en Kotlin.               & 4 semanas                  \\ \hline
            \textit{Sprint} 5        & Pruebas y validación funcional.                          & 3 semanas                  \\ \hline
            \textit{Sprint} 6        & Documentación, presentación y cierre del proyecto.       & 3 semanas                  \\ \hline
      \end{tabular}
      \caption[Estructura de los \textit{Sprints}]{Estructura de los \textit{sprints} del proyecto, incluyendo la meta principal y duración estimada de cada uno.}
      \label{tab:estructura-sprints}
\end{table}

Al final, cada \textit{sprint} culminó con un entregable verificable que sirvió
como criterio de avance para el siguiente ciclo, asegurando así la trazabilidad
y coherencia entre la visión inicial del proyecto y el producto final obtenido.

\section{\textit{Sprint} 1: Identificación del perfil de usuario objetivo}
\textbf{Duración estimada:} 3 semanas

Este \textit{sprint} tuvo como objetivo desarrollar un perfil de usuario
(Persona) que sirviera como insumo accionable para orientar las decisiones de
diseño interactivo y priorización técnica del proyecto. Dado que no fue posible
realizar entrevistas ni trabajo de campo, el perfil se elaboró exclusivamente a
partir del análisis de fuentes documentales que reflejan la situación actual de
los estudiantes en el país, considerando aspectos demográficos, académicos y
sociales. Con base en esta información, se construyó una ficha de
\textbf{Persona} completa, acompañada de criterios de diseño alineados con las
necesidades y características identificadas.

\subsection{Ejecución}
La culminación del \textit{sprint} se evaluó tomando en cuenta la culminación
exitosa de las siguientes tareas:

\begin{enumerate}
      \item \textbf{Investigación documental}
            \begin{itemize}
                  \item Revisión de informes académicos y/o gubernamentales sobre educación ciudadana,
                        competencias cívicas y valores en jóvenes guatemaltecos.
                  \item Consulta de programas educativos oficiales, como el Currículo Nacional Base
                        (CNB) y materiales de formación en valores del Ministerio de Educación de
                        Guatemala, así como contenido internacional enfocado en brindar una educación
                        más completa \cite{mineduc2020cnb,mineduc2019guia,Toro2010-iq}.
                  \item Análisis de estudios internacionales de organismos como UNESCO (Organización de
                        las Naciones Unidas para la Educación, la Ciencia y la Cultura) y CIEN (Centro
                        de Investigaciones Económicas Nacionales) sobre hábitos digitales, desigualdad
                        educativa y desarrollo de competencias ciudadanas en adolescentes y jóvenes
                        \cite{unesco2023monitoring,unesco2021reimagining,cien2019diagnostico}.
            \end{itemize}

      \item \textbf{Análisis e interpretación de la información}
            \begin{itemize}
                  \item Sistematización de datos demográficos, educativos y tecnológicos relevantes
                        para el contexto juvenil guatemalteco.
                  \item Identificación de patrones generales de comportamiento, motivaciones,
                        frustraciones y objetivos (enfocados en aspiraciones cívicas), a partir de
                        tendencias reportadas en las fuentes analizadas.
                  \item Construcción de categorías de análisis que permitieran traducir los hallazgos
                        documentales en insumos para el diseño centrado en el usuario.
            \end{itemize}

      \item \textbf{Definición del perfil Persona}
            \begin{itemize}
                  \item Elaboración de una ficha de usuario basada en la interpretación crítica de los
                        datos documentales, con los siguientes componentes:
                        \begin{itemize}
                              \item \textbf{Perfil base:} edad estimada, nivel educativo, ubicación, etnia, acceso tecnológico y contexto social.
                              \item \textbf{Motivaciones:} interés por la participación comunitaria y el aprendizaje de ciudadanía.
                              \item \textbf{Frustraciones:} barreras de acceso a recursos educativos y desconfianza en la calidad o adecuación de los materiales disponibles.
                              \item \textbf{Objetivos:} las metas que el usuario quisiera conseguir a través de sus motivaciones y frustraciones, bajo el contexto de educación en valores y formación ciudadana.
                              \item \textbf{Consideraciones especiales:} limitaciones de conectividad, recursos económicos y brechas culturales.
                        \end{itemize}
                  \item Producción de una ficha visual que sirviera como base para las decisiones de
                        diseño en \textit{sprints} posteriores.
            \end{itemize}

      \item \textbf{Documentación de criterios de diseño}
            \begin{itemize}
                  \item Derivación de recomendaciones de diseño UX basadas en el perfil construido:
                        tono comunicativo, estructura de funciones, rol a asumir por el asistente, y
                        adaptabilidad tecnológica.
                  \item Identificación de necesidades prioritarias que el asistente debe ser capaz de
                        abordar a través de la interacción pregunta-respuesta.
            \end{itemize}

\end{enumerate}

\subsection{Resultado final}
Como resultado de este primer \textit{sprint}, se construyó un perfil de
\textbf{Persona} detallado, basado en fuentes documentales, que permitió
comprender las necesidades, barreras y expectativas del usuario objetivo frente
a una herramienta de apoyo educativo.

\begin{itemize}
      \item Edad promedio: 13-17 años.
      \item Contexto educativo: estudiantes de nivel medio y universitario inicial.
      \item Motivaciones: aprender de forma práctica y reflexiva, mejorar su comprensión de
            ciudadanía y valores.
      \item Frustraciones: enseñanza teórica, falta de espacios de diálogo y escasez de
            herramientas interactivas.
      \item Competencias digitales: nivel bajo a medio en uso de aplicaciones y
            herramientas digitales.
      \item Contexto de uso de la aplicación: dispositivos móviles, principalmente Android,
            con sesiones cortas de interacción y preferencia por contenidos dinámicos y
            cercanos a su realidad.
\end{itemize}

Este perfil se utilizó como base para orientar el diseño conversacional, las
estrategias de análisis documental y los lineamientos pedagógicos que guiarán
las siguientes etapas del desarrollo del proyecto.

\section{\textit{Sprint} 2: Recolección y procesamiento del contenido educativo inicial}
\textbf{Duración estimada:} 4 semanas

Este \textit{sprint} se centró en recopilar, procesar y estructurar el
contenido educativo inicial que alimentará al asistente virtual de inteligencia
artificial, con la finalidad de garantizar que el sistema pueda generar
respuestas precisas y contextualizadas sobre formación ciudadana y valores
morales, basándose en información confiable y organizada de manera semántica.
Se combinó la selección documental, curación de contenido, segmentación
temática y almacenamiento vectorial de manera sistemática, asegurando la
trazabilidad y calidad de los datos utilizados.

\subsection{Ejecución}

Para cumplir el objetivo se desarrolló un proceso sistemático dividido en
cuatro etapas principales: selección documental, curación, segmentación
temática, y vectorización (a través de OpenAI) con almacenamiento en Pinecone.
Este flujo se diseñó de forma reproducible para permitir futuras ampliaciones o
actualizaciones del corpus de información.

\begin{enumerate}

      \item \textbf{Selección documental}
            \begin{itemize}
                  \item \textbf{Identificación de fuentes oficiales y confiables:} se recopilaron documentos emitidos por el Ministerio de Educación de Guatemala, tales como contenidos contemplados en el \textbf{Currículo Nacional Base (CNB)} para los grados educativos abarcados por el rango de edad establecido, así como guías orientacionales dirigidas a los educadores, con el fin de que el asistente también tenga conocimiento de cómo interactuar con los usuarios objetivo de forma correcta.
                  \item \textbf{Revisión de fuentes internacionales:} se incorporaron publicaciones y estudios de entidades internacionales como la OEA (Organización de Estados Americanos), la UNESCO (Organización de las Naciones Unidas para la Educación, la Ciencia y la Cultura) o universidades extranjeras. Mediante este contenido, se buscó alimentar aún más el conocimiento teórico del asistente, así como diversificar las fuentes de información a contextos internacionales.
                  \item \textbf{Estudios complementarios:} Además de las fuentes mencionadas, también se incluye la recopilación de libros educativos de entidades independientes (tales como IGER, Instituto Guatemalteco de Educación Radifónica) así como de autores externos. \cite{iger2024cienciasciudadana}
                  \item \textbf{Registro de metadatos:} cada documento fue almacenado en un contenedor tipo \textbf{Amazon S3} (\textit{Simple Storage Service}), mientras que sus metadatos asociados (título, autor, año de publicación) así como la ruta de almacenamiento relativa dentro del contenedor, fueron almacenados en la base de datos vectorial, siguiendo el siguiente esquema:
                        \begin{itemize}
                              \item Identificador único (\texttt{DocumentID})
                              \item Identificador del usuario que sube el documento (\texttt{UserID})
                              \item Título del documento (\texttt{Title})
                              \item Fuente o Autor (\texttt{Author})
                              \item Año de publicación (\texttt{Year})
                              \item Categoría temática (\texttt{Category})
                              \item Ruta dentro del contenedor S3 (\texttt{Document URL})
                        \end{itemize}
                        Este registro garantiza la trazabilidad desde la fuente original hasta el fragmento vectorizado. Cabe aclarar que el uso de variables en inglés corresponde a buenas prácticas de programación, con el fin de que el código pueda ser comprendido a nivel global en caso de ser necesario \cite{batubara2025impact}.
            \end{itemize}

      \item \textbf{Curación y digitalización}
            \begin{itemize}
                  \item \textbf{Conversión de documentos:} los archivos se transformaron a texto plano (\texttt{.txt}) con codificación UTF-8 mediante herramientas como \texttt{NFKD} o \texttt{Tesseract OCR} (esta última extrae de forma automática el texto reconocible de imágenes o documentos PDF escaneados).
                  \item \textbf{Limpieza y normalización:} se eliminaron saltos de línea innecesarios, espacios vacíos múltiples y caracteres especiales, a través del Algoritmo \ref{alg:limpiar-texto}.

                        \begin{algorithm}[H]
                              \caption{Proceso de limpieza y normalización profunda de texto}
                              \label{alg:limpiar-texto}
                              \begin{algorithmic}[1]
                                    \Procedure{LimpiarTexto}{texto}
                                    \State texto $\gets$ NormalizarUnicode(texto, \guillemetleft{}NFKD\guillemetright{})
                                    \State texto $\gets$ EliminarCaracteresNoASCII(texto)
                                    \State texto $\gets$ Reemplazar(texto, \{\guillemetleft{}\textbackslash r\guillemetright{}, \guillemetleft{}\textbackslash n\guillemetright{}, \guillemetleft{}\textbackslash t\guillemetright{}\}, \guillemetleft{} \guillemetright{})
                                    \State texto $\gets$ EliminarCaracteresEspeciales(texto, \guillemetleft{}manteniendo letras, números y puntuación básica\guillemetright{})
                                    \State texto $\gets$ ReemplazarMúltiplesEspaciosPorUno(texto)
                                    \State texto $\gets$ EliminarEspaciosExtremos(texto)
                                    \State \Return texto
                                    \EndProcedure
                              \end{algorithmic}
                        \end{algorithm}

                  \item \textbf{Estandarización de formato:} se uniformaron títulos y subtítulos con reglas jerárquicas para facilitar la segmentación automática, como se muestra en el Algoritmo \ref{alg:estandarizar-texto}.
                        \begin{algorithm}[H]
                              \caption{Estandarización de títulos y numeración en texto}
                              \label{alg:estandarizar-texto}
                              \begin{algorithmic}[1]
                                    \Procedure{EstandarizarFormato}{texto}
                                    \State texto $\gets$ ReemplazarMarkdownConTitulo(texto)
                                    \State texto $\gets$ ReemplazarNumeracionConTitulo(texto)
                                    \State texto $\gets$ ConvertirTitulosMayusculas(texto)
                                    \State texto $\gets$ UniformarNumeracion(texto)
                                    \State texto $\gets$ EliminarEspaciosExtremos(texto)
                                    \State \Return texto
                                    \EndProcedure
                              \end{algorithmic}
                        \end{algorithm}

                  \item \textbf{Validación de integridad:} se verificó que los textos conservaran coherencia y completitud, eliminando duplicados o secciones ilegibles, siguiendo el flujo del Algoritmo \ref{alg:validar-integridad}.

                        \begin{algorithm}[H]
                              \caption{Validación de integridad de texto}
                              \label{alg:validar-integridad}
                              \begin{algorithmic}[1]
                                    \Procedure{ValidarIntegridad}{texto}
                                    \State líneas $\gets$ DividirEnLineas(texto)
                                    \State líneas\_limpias $\gets$ ListaVacía()
                                    \For{cada línea en líneas}
                                    \If{Longitud(Trim(linea)) < 3}
                                    \State Continuar
                                    \EndIf
                                    \State caracteres\_válidos $\gets$ ContarCaracteresAlfanumericosYEspacios(linea)
                                    \If{caracteres\_válidos / Max(Longitud(linea), 1) > 0.6}
                                    \State Añadir(linea, líneas\_limpias)
                                    \EndIf
                                    \EndFor
                                    \State líneas\_sin\_duplicados $\gets$ EliminarDuplicados(líneas\_limpias)
                                    \State texto\_limpio $\gets$ UnirLineas(líneas\_sin\_duplicados)
                                    \State \Return texto\_limpio
                                    \EndProcedure
                              \end{algorithmic}
                        \end{algorithm}
            \end{itemize}

      \item \textbf{Segmentación temática}
            \begin{itemize}
                  \item \textbf{Diseño del esquema de categorías:} se definieron seis temas guía iniciales: \textit{ética y moral}, \textit{participación ciudadana}, \textit{derechos humanos}, \textit{convivencia y respeto}, \textit{responsabilidad social} y \textit{cultura digital}. Esta lista puede incrementarse con el tiempo, a medida que el modelo procese una mayor cantidad de archivos y no sea capaz de incluirlos en una de las categorías predefinidas.
                  \item \textbf{División en fragmentos:} los textos fueron segmentados automáticamente en bloques de 20 a 150 palabras, conservando coherencia semántica.
                  \item \textbf{Etiquetado y registro:} cada fragmento se asoció a una categoría temática y se describió con los siguientes metadatos:
                        \texttt{document\_id} (identificador único del documento en la base de datos relacional), \texttt{text} (contenido original del documento), \texttt{source} (título original del documento), \texttt{author} (autor o institución que realizó el documento), \texttt{year} (año de publicación del documento original), \texttt{category} (categoría temática asociada al fragmento), \texttt{sha1} (\textit{hash} único del fragmento, para evitar duplicados), \texttt{uploaded\_at} (fecha y hora de publicación del fragmento, en formato ISO).
            \end{itemize}

      \item \textbf{Vectorización y almacenamiento en Pinecone}
            \begin{itemize}
                  \item \textbf{Generación de representaciones numéricas:} cada fragmento fue procesado con el modelo \textit{text-embedding-3-small} de OpenAI, generando vectores de 1536 dimensiones.
                  \item \textbf{Normalización final:} la asignación del metadato \textit{sha1} en cada vector, permitió validar la ausencia de duplicados, lo que asegura la unicidad de cada vector en la base de datos vectorial.
                  \item \textbf{Creación del índice vectorial:} se configuró un índice en Pinecone con los parámetros:
                        \begin{itemize}
                              \item \texttt{name = \guillemetleft{}ciudadano-digital\guillemetright{}}
                              \item \texttt{namespace = \guillemetleft{}ciudadania\guillemetright{}}
                              \item \texttt{metric = \guillemetleft{}cosine\guillemetright{}}
                              \item \texttt{dimension = 1536}
                        \end{itemize}
                  \item \textbf{Inserción de vectores:} la representación numérica de cada fragmento se insertó junto con sus metadatos para permitir consultas semánticas eficientes. El proceso completo se muestra en el Algoritmo \ref{alg:vectorizar-fragmentos}. Destaca la presencia de la variable \textbf{BATCH\_SIZE}, la cual corresponde a una constante utilizada para definir el tamaño del paquete de vectores enviado a la base de datos. Esto busca evitar enviar vectores individuales que podrían ralentizar el proceso de almacenamiento y elevar los costos relacionados a la base de datos vectorial.

                        \begin{algorithm}[H]
                              \caption{Vectorización de fragmentos}
                              \label{alg:vectorizar-fragmentos}
                              \begin{algorithmic}[1]
                                    \Procedure{VectorizarFragmentos}{fragmentos, identificador, fuente, autor, año, BATCH\_SIZE}
                                    \State lote $\gets$ ListaVacía()
                                    \For{cada frag en fragmentos}
                                    \State sha1 $\gets$ CalcularSHA1(frag)
                                    \If{FragmentoYaIndexado(sha1) \textbf{or} EsVacio(frag)}
                                    \State Continuar
                                    \EndIf
                                    \State categoria $\gets$ ClasificarCategoria(frag)
                                    \State embedding $\gets$ GenerarEmbedding(modelo=\guillemetleft{}text-embedding-3-small\guillemetright{}, texto=frag)
                                    \State metadatos $\gets$ CrearDiccionario(\{
                                    \guillemetleft{}document\_id\guillemetright{}: identificador,
                                    \guillemetleft{}text\guillemetright{}: frag,
                                    \guillemetleft{}source\guillemetright{}: fuente,
                                    \guillemetleft{}author\guillemetright{}: autor,
                                    \guillemetleft{}year\guillemetright{}: año,
                                    \guillemetleft{}category\guillemetright{}: categoria,
                                    \guillemetleft{}sha1\guillemetright{}: sha1,
                                    \guillemetleft{}uploaded\_at\guillemetright{}: FechaHoraActual()
                                    \})
                                    \State AñadirAlLote(lote, CrearVector(id=GenerarUUID(), valores=embedding, metadatos=metadatos))
                                    \If{Longitud(lote) $\ge$ BATCH\_SIZE}
                                    \State RegistrarLote(lote, namespace=\guillemetleft{}ciudadanía\guillemetright{})
                                    \State lote $\gets$ ListaVacía()
                                    \EndIf
                                    \EndFor
                                    \If{lote \textbf{no vacío}}
                                    \State RegistrarLote(lote)
                                    \EndIf
                                    \EndProcedure
                              \end{algorithmic}
                        \end{algorithm}

                  \item \textbf{Implementación de flujo para procesamiento de archivos}
                        El sistema obtenido a través del proceso anterior, permite procesar
                        diversos formatos de archivo:
                        \begin{itemize}
                              \item Documentos PDF (.pdf)
                              \item Documentos de Word (.doc, .docx)
                              \item Presentaciones (.ppt, .pptx)
                              \item Texto plano (.txt)
                              \item Markdown (.md)
                              \item Imágenes y documentos escaneados (mediante Tesseract OCR)
                        \end{itemize}
                        La Figura \ref{fig:procesamiento-docs} ilustra el flujo completo de
                        procesamiento:

                        \begin{figure}[H]
                              \centering
                              \includegraphics[width=0.9\textwidth]{assets/procesamiento-docs.png}
                              \caption{Flujo de procesamiento de documentos para generación del corpus.}
                              \label{fig:procesamiento-docs}
                        \end{figure}

                        El proceso se divide en dos etapas principales:

                        \begin{enumerate}
                              \item \textbf{Extracción y normalización del contenido:} En esta etapa se realiza la lectura y extracción de texto desde los formatos soportados. El flujo incluye la limpieza del texto, la estandarización del formato (identificación de títulos, encabezados y listas para preservar la semántica), la validación de la integridad (omisión de duplicados, normalización de caracteres) y la fragmentación semántica. El producto final resultante está compuesto por fragmentos de texto plano autocontenidos.
                              \item \textbf{Generación de representaciones numéricas e indexación vectorial:} Cada fragmento de texto se transforma en un vector mediante el modelo de OpenAI \texttt{text-embedding-3-small}. A cada representación numérica de texto, se le asocian metadatos relevantes (identificador del documento original, título, categoría temática, texto original, autor y año de publicación). Posteriormente, se realiza una operación de \textbf{\textit{upsert}} (es decir, insertar o actualizar el registro según corresponda) en el índice vectorial de Pinecone. Simultáneamente, se registra en la base de datos relacional la trazabilidad entre el vector y el documento original almacenado en un contenedor Amazon S3.
                        \end{enumerate}
            \end{itemize}
\end{enumerate}

\subsection{Resultado final}
Al finalizar el \textit{Sprint} 2, se obtuvo:

\begin{itemize}
      \item Una base documental curada y segmentada en categorías temáticas.
      \item Representaciones numéricas (\textit{embeddings}) generadas para cada fragmento
            de texto, con metadatos completos para garantizar trazabilidad.
      \item Un índice en Pinecone listo para consultas semánticas, capaz de proporcionar
            contexto preciso al asistente virtual para cualquier pregunta del usuario.
      \item Establecimiento de un flujo reproducible de selección, curación, segmentación y
            vectorización de contenido para una continua actualización del corpus del
            proyecto.
\end{itemize}

Este \textit{sprint} permitió sentar las bases para un sistema de respuesta
contextualizada, alineado con los objetivos de formación ciudadana y valores
morales definidos en el proyecto, asegurando que el asistente virtual cuente
con información confiable, organizada y accesible para generar respuestas
pertinentes y fundamentadas.

\section{\textit{Sprint} 3: Construcción e implementación del servidor}
\textbf{Duración estimada:} 4 semanas

Este \textit{sprint} se enfocó en el diseño, construcción e implementación de
la arquitectura del servidor del asistente virtual, garantizando la integración
de bases de datos relacionales y vectoriales, y estableciendo la comunicación
segura y eficiente con el modelo de lenguaje de gran escala (LLM, por sus
siglas en inglés) mediante un flujo RAG (\textit{Retrieval-Augmented
      Generation}). Se definieron módulos claros bajo el patrón de diseño MVC
(Modelo-Vista-Controlador), así como servicios complementarios internos en
Python tanto para la curación y procesamiento de documentos, como para procesar
consultas y generar respuestas contextualizadas con base en los mismos.

\subsection{Ejecución}
\begin{enumerate}
      \item \textbf{Diseño de arquitectura}
            \begin{itemize}
                  \item Se adoptó el patrón de diseño \textbf{Modelo–Vista–Controlador (MVC)},
                        siguiendo la estructura de la Figura \ref{fig:mvc}. Esta arquitectura permite
                        separar las responsabilidades del sistema, facilitando el mantenimiento y
                        escalabilidad.

                        \begin{figure}[H]
                              \centering
                              \includegraphics[width=0.7\textwidth]{assets/MVC.png}
                              \caption{Arquitectura MVC (Modelo, Vista, Controlador).}
                              \label{fig:mvc}
                        \end{figure}

                  \item \textbf{Modelos:}
                        \begin{itemize}
                              \item Representan entidades del sistema: usuarios, chats, mensajes, sesiones,
                                    documentos, categorías.
                              \item Cada modelo incluye operaciones de creación, lectura, actualización y
                                    eliminación de registros (CRUD, por sus siglas en inglés) que se ejecutan
                                    directamente sobre la base de datos según sea requerido.
                              \item A través de la comunicación con los controladores, los modelos gestionan la
                                    persistencia y recuperación de datos de manera eficiente, siguiendo la lógica
                                    de negocio definida.
                        \end{itemize}
                  \item \textbf{Vistas:} En la implementación del modelo MVC (Modelo-Vista-Controlador), las vistas corresponden a los puntos de conexión expuestos por el API, también llamados rutas o \textbf{\textit{endpoints}}. A través de la consulta a estos puntos de conexión, se tiene acceso a las funciones definidas por el servidor, como registro de usuraios, inicio de sesión, listado de chats, envío de preguntas, entre otros.
                  \item \textbf{Controladores:} Gestionan la lógica de negocio: validación de datos, comunicación con modelos,
                        manejo de errores y generación de respuestas. El Algoritmo \ref{alg:controlador-solicitudes} ilustra el flujo básico de un controlador típico.
                        \begin{algorithm}[H]
                              \caption{Controlador de solicitudes}
                              \label{alg:controlador-solicitudes}
                              \begin{algorithmic}[1]
                                    \Procedure{Controlador}{request}
                                    \State Validar(request.datos)
                                    \State resultado $\gets$ modelo.operacion(request)
                                    \State \textbf{devolver}(resultado)
                                    \EndProcedure
                              \end{algorithmic}
                        \end{algorithm}
                  \item \textbf{Módulos auxiliares:}
                        \begin{itemize}
                              \item \textbf{\textit{Middlewares} (Software intermedio):} corresponden a funciones que son ejecutadas antes de realizar la acción principal de cada punto de conexión. Se encargan de validar la autenticación y seguridad antes de pasar al controlador. El Algoritmo \ref{alg:validar-token} muestra un ejemplo de middleware para validar tokens JWT (JSON \textit{Web Token}, el término \textbf{JSON} corresponde a las siglas en inglés de \textbf{notación de objetos de JavaScript}, constituye un formato de texto sencillo para el intercambio de datos).
                                    \begin{algorithm}[H]
                                          \caption{Validación de token}
                                          \label{alg:validar-token}
                                          \begin{algorithmic}[1]
                                                \Procedure{ValidarTokenRequest}{request}
                                                \If{not ValidarToken(request.token)}
                                                \State DevolverError(401, \guillemetleft{}Token inválido\guillemetright{})
                                                \Else
                                                \State Continuar(request)
                                                \EndIf
                                                \EndProcedure
                                          \end{algorithmic}
                                    \end{algorithm}
                              \item \textbf{Helpers (ayudantes):} funciones auxiliares reutilizadas a lo largo del código, como:
                                    \begin{itemize}
                                          \item encriptarContraseña(contraseña)
                                          \item generarToken(usuarioID)
                                          \item formatearFecha(fecha)
                                    \end{itemize}
                        \end{itemize}
            \end{itemize}

      \item \textbf{Diseño y construcción de bases de datos}
            \subsubsection{Base de datos relacional}
            Esta será la encargada de almacenar la información estructurada del sistema,
            como usuarios, chats, mensajes y sesiones, de manera que se mantenga la
            persistencia de datos y se facilite la gestión de las interacciones del usuario
            con el asistente virtual. La Figura \ref{fig:diagrama-bd-relacional} muestra el
            diagrama entidad-relación (ER) de la base de datos relacional diseñada.

            \begin{figure}[H]
                  \centering
                  \includegraphics[width=0.9\textwidth]{assets/database\_uml.png}
                  \caption{Diagrama entidad-relación (ER) de la base de datos relacional.}
                  \label{fig:diagrama-bd-relacional}
            \end{figure}

            \begin{itemize}
                  \item Motor: PostgreSQL en AWS RDS (\textit{Relational Database Service}).
                  \item Entidades: usuarios, chats, mensajes, sesiones, documentos, categorías, códigos
                        de recuperación.
                  \item Relaciones:
                        \begin{itemize}
                              \item Un usuario puede tener varios chats.
                              \item Cada chat contiene múltiples mensajes.
                              \item Un usuario puede tener varias sesiones.
                              \item Un usuario puede subir varios documentos.
                              \item Una categoría puede incluir múltiples documentos.
                              \item Un usuario solo puede tener un único código de recuperación a la vez.
                        \end{itemize}
                  \item Mantenimiento: restricciones de claves foráneas, eliminación en cascada y
                        validaciones para asegurar la integridad referencial.
            \end{itemize}

            \subsubsection{Base de datos vectorial}
            Para llevar a cabo búsquedas semánticas eficientes y recuperar fragmentos de
            documentos relevantes en función de las preguntas del usuario (sistema
            \textit{Retrieval-Augmented Generation}), se implementó una base de datos
            vectorial utilizando Pinecone. Esta base almacena las representaciones
            numéricas generadas previamente a partir de los fragmentos de texto,
            permitiendo consultas rápidas y precisas basadas en similitud semántica. La
            métrica de similitud de coseno fue seleccionada para evaluar la cercanía entre
            vectores, optimizando la relevancia de los resultados obtenidos.

            \begin{itemize}
                  \item Motor: Pinecone, con métrica de similitud coseno.
                  \item Contenido: representaciones numéricas de fragmentos de documentos con metadatos
                        (fuente, categoría, documento, bloque, relevancia).
                  \item Obtención de los fragmentos más relevantes en relación al vector generado a
                        partir de la pregunta realizada. Esto lo realiza Pinecone de forma automática,
                        dependiendo del método de similitud configurado al momento de crear el índice.
                        En este caso, se indicó que se utilice la métrica de similitud coseno, puesto
                        que dicha métrica mide el ángulo de inclinación entre vectores, lo que
                        determina qué tan similar es la dirección a la que se dirigen (orientación
                        semántica). La fórmula que guía este cálculo es la siguiente:
                        \[\text{Similitud Coseno} (A, B) = \frac{A \cdot B}{\|A\| \|B\|} = \frac{\sum_{i=1}^{n} A_i B_i}{\sqrt{\sum_{i=1}^{n} A_i^2} \sqrt{\sum_{i=1}^{n} B_i^2}} \]
            \end{itemize}

            En esta etapa del desarrollo se definió un límite de \textbf{0.4}, lo que
            indica que, para que una representación numérica devuelta por Pinecone sea
            tomada como válida, esta debe tener una coincidencia de, por lo menos, el 40\%
            con la representación numérica de la pregunta. Esta barrera puede variar a
            medida que se enriquece el corpus, ya que a mayor variedad hay más
            oportunidades de similitud entre vectores.

      \item \textbf{Implementación del modelo LLM y flujo RAG}

            Se implementó un flujo RAG (\textit{Retrieval-Augmented Generation}) que
            permite al asistente virtual generar respuestas fundamentadas en los contenidos
            educativos previamente indexados. Este flujo integra los siguientes
            componentes:

            \begin{enumerate}
                  \item \textbf{API en NodeJS:} Actúa como servidor principal, orquestando la
                        comunicación entre la aplicación móvil y los servicios de procesamiento
                        de lenguaje natural.
                  \item \textbf{Microservicio en Python:} Gestiona la interacción con el modelo
                        LLM de OpenAI para generar respuestas y procesar documentos.
                  \item \textbf{API de OpenAI:} Utilizada para la generación de representaciones
                        numéricas (\textit{embeddings}) y respuestas contextualizadas.
                  \item \textbf{Base de datos vectorial (Pinecone):} Almacena y permite la
                        recuperación eficiente de las representaciones numéricas generadas a
                        partir de los contenidos educativos.
                  \item \textbf{Base de datos relacional (PostgreSQL):} Gestiona usuarios,
                        mensajes, sesiones e información de los documentos procesados, asegurando
                        la trazabilidad desde los vectores hasta el documento original.
            \end{enumerate}

            El flujo completo del proceso RAG, desde la recepción de una pregunta en la
            aplicación móvil hasta la devolución de la respuesta, se ilustra en la Figura
            \ref{fig:flujo-rag}.

            \begin{figure}[H]
                  \centering
                  \includegraphics[width=0.9\textwidth]{assets/RAG.png}
                  \caption{Flujo de procesamiento de preguntas mediante RAG.}
                  \label{fig:flujo-rag}
            \end{figure}

            \subsubsection{Arquitectura de almacenamiento distribuido}

            El almacenamiento de información a lo largo del sistema se distribuye en tres
            elementos principales que trabajan de forma coordinada:

            \begin{itemize}
                  \item \textbf{Base de datos relacional (PostgreSQL):} gestor principal de los
                        datos estructurados del sistema; almacena información de usuarios, chats,
                        mensajes, sesiones, códigos de recuperación de contraseñas y metadatos
                        básicos de los documentos procesados.
                  \item \textbf{Base de datos vectorial (Pinecone):} encargada de almacenar las
                        representaciones numéricas generadas a partir de los documentos
                        seleccionados. Constituye la base del funcionamiento RAG, permitiendo
                        recuperar el contexto necesario según la pregunta realizada para que el
                        modelo LLM genere respuestas fundamentadas.
                  \item \textbf{Almacenamiento de documentos originales (AWS S3):} contiene todos
                        los documentos fuente cargados al sistema, permitiendo su posterior
                        consulta o descarga por usuarios con rol de \textbf{Administrador}.
            \end{itemize}

            Esta arquitectura distribuida garantiza trazabilidad completa desde cada vector
            indexado hasta su documento fuente, mientras mantiene la eficiencia en las
            búsquedas semánticas y la integridad de los datos originales.

            \subsubsection{Flujo detallado de procesamiento de consultas}

            El microservicio de Python gestiona el flujo RAG completo, coordinando la
            interacción entre la base vectorial y el modelo de lenguaje de gran escala para
            generar respuestas contextualizadas y fundamentadas. El proceso se divide en
            tres etapas principales:

            \begin{enumerate}
                  \item \textbf{Recepción de la pregunta en NodeJS:}
                        \begin{enumerate}
                              \item El usuario envía una pregunta en texto plano a través de la interfaz de la
                                    aplicación.
                              \item NodeJS recibe la pregunta y prepara la solicitud para el microservicio Python.
                        \end{enumerate}

                  \item \textbf{Procesamiento en Python:}
                        \begin{enumerate}
                              \item El microservicio Python recibe la pregunta enviada desde NodeJS.
                              \item Genera una representación numérica (\textit{embedding}) del texto de la
                                    pregunta utilizando el modelo \textit{text-embedding-3-small} de OpenAI,
                                    transformando la información textual en un vector semántico de 1536
                                    dimensiones.
                              \item Se realiza una consulta al índice de Pinecone con la representación numérica
                                    generada, recuperando los cinco fragmentos más relevantes del corpus
                                    vectorizado mediante similitud coseno, que servirán como contexto para la
                                    respuesta.
                              \item Si no se encuentran fragmentos relevantes, se procede a generar una respuesta
                                    estándar indicando la falta de información suficiente.
                              \item Si la búsqueda semántica obtuvo resultados, se extrae el texto plano almacenado
                                    en los metadatos de los vectores y se combina con la pregunta original,
                                    construyendo una petición que resume la información relevante y plantea la
                                    consulta al modelo de lenguaje de gran escala.
                              \item Envía la petición al modelo LLM de OpenAI para generar la respuesta
                                    contextualizada.
                              \item Genera un objeto JSON que incluye la pregunta original, la respuesta obtenida y
                                    el tiempo de procesamiento, garantizando trazabilidad de la interacción.
                        \end{enumerate}

                  \item \textbf{Devolución de la respuesta a NodeJS:}
                        \begin{enumerate}
                              \item NodeJS recibe el JSON con la respuesta generada por el LLM.
                              \item Formatea y entrega la respuesta al usuario final a través de la interfaz de la
                                    aplicación.
                              \item Simultáneamente, guarda la pregunta y la respuesta en la base de datos
                                    relacional para mantener un historial completo de interacciones.
                        \end{enumerate}
            \end{enumerate}

            El flujo completo de procesamiento de preguntas se formaliza en el Algoritmo
            \ref{alg:flujo-completo-preguntas}.

            \begin{algorithm}[H]
                  \caption{Flujo completo de procesamiento de preguntas}
                  \label{alg:flujo-completo-preguntas}
                  \begin{algorithmic}[1]
                        \Procedure{ProcesarPregunta}{usuario}
                        \State pregunta $\gets$ RecibirPregunta(usuario)
                        \State EnviarPreguntaAPython(pregunta)
                        \State embedding $\gets$ GenerarEmbedding(pregunta)
                        \State contexto $\gets$ ConsultarPinecone(embedding, topK=5)
                        \State prompt $\gets$ ConstruirPrompt(pregunta, contexto)
                        \State respuesta $\gets$ ConsultarLLM(prompt)
                        \State jsonRespuesta $\gets$ ArmarJSON(pregunta, respuesta, tiempoProcesamiento)
                        \State EnviarAlCliente(jsonRespuesta)
                        \State GuardarMensaje(usuario, respuesta)
                        \EndProcedure
                  \end{algorithmic}
            \end{algorithm}
\end{enumerate}

\subsection{Resultado final}
Al finalizar este \textit{sprint}, se obtuvo:

\begin{itemize}
      \item Servidor modular bajo MVC (Modelo-Vista-Controlador), con rutas, controladores
            y modelos independientes.
      \item Base de datos relacional (PostgreSQL) con integridad referencial y seguridad.
      \item Base vectorial en Pinecone, indexada y lista para búsquedas semánticas
            eficientes.
      \item Servicio Python que integra recuperación contextual y generación ética de
            respuestas mediante LLM.
      \item Flujo completo validado: desde envío de pregunta hasta devolución de respuesta
            fundamentada.
\end{itemize}

Este \textit{sprint} consolidó la infraestructura técnica del sistema,
asegurando operación confiable, trazabilidad de datos y escalabilidad futura
para el asistente educativo.

\section{\textit{Sprint} 4: Desarrollo de la interfaz móvil en Kotlin}
\textbf{Duración estimada:} 4 semanas

Este \textit{sprint} tuvo como objetivo diseñar e implementar la interfaz móvil
de la aplicación del asistente educativo, utilizando Kotlin para garantizar
integración nativa con Android y un flujo de interacción intuitivo para el
usuario. Esto permite una interacción eficiente con el asistente al integrar
las funcionalidades proporcionadas por el servidor a través de servicios puntos
de conexión (\textit{endpoints}). El diseño de la arquitectura se basó en el
patrón de diseño MVVM (Modelo-Vista-Modelo de Vista), lo que permitió separar
responsabilidades, facilitar la escalabilidad del código y mantener una clara
independencia entre la lógica de negocio, la gestión de datos y la capa de
presentación.

\subsection{Ejecución}
\begin{enumerate}
      \item \textbf{Diseño de arquitectura}
            \begin{itemize}
                  \item Se adoptó el patrón de diseño \textbf{Modelo-Vista-Modelo de Vista (MVVM, por
                              sus siglas en inglés)}, siguiendo la estructura de la Figura \ref{fig:mvvm}.
                        Esta arquitectura permite separar las responsabilidades del sistema,
                        facilitando el mantenimiento y escalabilidad. Mediante este patrón de diseño se
                        logra una comunicación reactiva entre la interfaz de usuario y las fuentes de
                        datos, a través del uso de componentes de arquitectura de Android como Datos
                        Observables (\textit{LiveData}), Vista de modelo (\textit{ViewModel}) y
                        Repositorio (\textit{Repository}).

                        \begin{figure}[H]
                              \centering
                              \includegraphics[width=0.7\textwidth]{assets/MVVM.png}
                              \caption{Arquitectura MVVM (Modelo, Vista, Modelo de Vista).}
                              \label{fig:mvvm}
                        \end{figure}

                  \item \textbf{Vista:}
                        \begin{itemize}
                              \item Implementada a través de Actividades y Fragmentos organizados modularmente.
                              \item Cada fragmento representa una sección funcional del sistema (por ejemplo:
                                    inicio de sesión, chat, documentos, perfil).
                              \item Se aplicó el patrón de navegación basado en las librerías de Android
                                    \textit{NavHostFragment} y \textit{SafeArgs}, garantizando transiciones seguras
                                    y controladas entre vistas.
                              \item Se usaron componentes de diseño basados en \textit{Material Design 3}, el
                                    estandar de diseño de Google para mantener consistencia visual, accesibilidad y
                                    adaptabilidad en distintos tamaños de pantalla
                                    \cite{material_design_m3_website}.
                              \item Cada Fragmento o Actividad cuenta con su propia plantilla, definida en formato
                                    XML (Lenguaje de Marcado Extensible, por sus siglas en inglés), que especifica
                                    la disposición de los elementos visuales.
                        \end{itemize}
                  \item \textbf{Modelo de Vista:}
                        \begin{itemize}
                              \item Actúa como intermediario entre la vista y las fuentes de datos, manejando la
                                    lógica de presentación.
                              \item Emplea Datos Observables (\textit{LiveData}) y Flujo de Estado
                                    (\textit{StateFlow}) para notificar automáticamente a la vista sobre cambios en
                                    los datos.
                              \item Encapsula la interacción con los repositorios, garantizando que la vista
                                    permanezca libre de lógica de negocio.
                              \item El Algoritmo \ref{alg:viewmodel-flujo} ilustra el flujo básico de un Modelo de
                                    VIsta típico.

                                    \begin{algorithm}[H]
                                          \caption{Flujo básico de un Modelo de Vista}
                                          \label{alg:viewmodel-flujo}
                                          \begin{algorithmic}[1]
                                                \Procedure{ObtenerMensajes}{chatId}
                                                \State \textbf{emitir}(\texttt{Estado.Cargando})
                                                \State datos $\gets$ repositorio.obtenerMensajes(chatId)
                                                \State \textbf{emitir}(\texttt{Estado.Exitoso(datos)})
                                                \EndProcedure
                                          \end{algorithmic}
                                    \end{algorithm}

                        \end{itemize}
                  \item \textbf{Modelo (Repositorio y Fuentes de Datos):}
                        \begin{itemize}
                              \item Los repositorios centralizan el acceso a las fuentes de datos, tanto locales
                                    como remotas.
                              \item Se definieron dos capas de origen de datos:
                                    \begin{enumerate}
                                          \item \textbf{Datos Locales:} implementada con \textit{Room Database}, que gestiona entidades como usuarios, mensajes, chats y documentos. Esta capa permite el acceso a información sin conexión a internet (historiales de chat, información del usuario, listado de documentos disponibles), así como el almacenamiento persistente de la información obtenida mediante otras fuentes de datos que sí requieran comunicación con internet.
                                          \item \textbf{Datos Remotos:} implementada mediante \textit{Retrofit} y \textit{OkHttp}, se utiliza para consumir los puntos de conexión (\textit{endpoints}) del servidor desarrollado en el \textit{sprint} anterior.
                                    \end{enumerate}
                              \item Los repositorios determinan la fuente más apropiada según la disponibilidad de
                                    conexión y estado de sincronización.
                        \end{itemize}

                  \item \textbf{Gestión de dependencias:}
                        \begin{itemize}
                              \item Se empleó \textit{Hilt (Dagger)} para la inyección de dependencias,
                                    simplificando la creación de instancias de Modelos de Vistas, repositorios y
                                    servicios. Este enfoque garantiza bajo acoplamiento entre componentes y
                                    favorece la escalabilidad del sistema.
                              \item La configuración de los módulos de \textit{Hilt} se realizó en la carpeta
                                    \texttt{di/}, donde se definieron las dependencias necesarias para la
                                    aplicación, como clientes \textit{Retrofit}, bases de datos \textit{Room} y
                                    repositorios.
                              \item Se utilizaron otras dependencias básicas como \textit{AppCompat} y
                                    \textit{ConstraintLayout} para asegurar compatibilidad y flexibilidad en el
                                    diseño de la interfaz, así como \textit{NavigationFragment} para gestionar la
                                    navegación entre pantallas.
                              \item Por su parte, el uso de \textit{ThreeTen} permitió mantener la compatibilidad
                                    de manejo de fechas para versiones antiguas de Android (con soporte desde
                                    Android 7 en adelante), mientras que \textit{Glide} facilitó la carga y gestión
                                    eficiente de archivos dentro de la aplicación.
                        \end{itemize}
                  \item \textbf{Manejo de estado y persistencia:}
                        \begin{itemize}
                              \item Se usó el alcance del modelo de vista (\textit{ViewModelScope}) y el alcance de
                                    corrutina (\textit{CoroutineScope}) para ejecutar tareas asíncronas sin
                                    bloquear la interfaz.
                              \item Se implementó almacenamiento persistente mediante Preferencias Compartidas
                                    (\textit{SharedPreferences}) y Room, de manera que incluso sin internet la
                                    aplicación aún fuera utilizable, a pesar de no poder realizar peticiones al
                                    asistente ya que esta función sí requiere comunicación con internet para
                                    realizar solicitudes a \textit{OpenAI}.
                              \item La base de datos local permite al usuario acceder a su historial de chats,
                                    datos personales e historial de documentos (en el caso de usuarios con rol
                                    \textbf{Administrador}) aún sin conexión a internet.
                        \end{itemize}
            \end{itemize}

      \item \textbf{Diseño de flujo de interacción}
            \begin{itemize}
                  \item \textbf{Actividades}
                        \begin{itemize}
                              \item \textit{SplashActivity}: esta es la vista principal de la aplicación. Si bien visualmente tan solo muestra el logo del proyecto, su función principal es verificar si el usuario ya ha iniciado sesión previamente (mediante un token JWT almacenado en las Preferencias Compartidas o \textit{SharedPreferences}) o bien si debe ser redirigido a la pantalla de login. A partir de esta lógica se permite que el usuario no deba iniciar sesión cada vez que abre la aplicación, mejorando la experiencia de uso.
                              \item \textit{UnloggedActivity}: esta actividad contiene los fragmentos relacionados con la autenticación del usuario, incluyendo \textit{LoginFragment}, \textit{RegisterFragment} y \textit{RecoverPasswordFragment}. Cada fragmento maneja su propia lógica de validación y comunicación con el servidor para gestionar el acceso seguro a la aplicación.
                              \item \textit{MainActivity}: esta actividad permite el acceso a las funciones principales de la aplicación una vez el usuario ya haya sido autenticado. Como componente principal, alberga el \textit{NavHostFragment}, el cual gestiona la navegación entre los distintos fragmentos funcionales, que incluyen el chat con el asistente, perfil de usuario y visualización de documentos. Por otro lado, esta actividad también incluye un menú lateral que facilita el acceso rápido al historial de chats del usuario.
                        \end{itemize}
                  \item \textbf{Fragmentos}
                        \begin{itemize}
                              \item \textit{LoginFragment}: permite al usuario ingresar sus credenciales (correo electrónico y contraseña) para autenticarse en el sistema. Incluye validaciones de formato y manejo de errores en caso de credenciales incorrectas.
                              \item \textit{RegisterFragment}: permite a nuevos usuarios crear una cuenta proporcionando información básica como correo electrónico, nombre, apellido, número de teléfono (con código de teléfono, en caso sea un número extranjero), contraseña y confirmación de contraseña. Incluye validaciones para asegurar la integridad y unicidad de los datos ingresados. La solicitud de un número telefónico cumple únicamente una función de verificación de identidad, ya que el sistema de mensajería SMS no fue implementado en esta fase del proyecto. Por el contrario, el correo electrónico sí es indispensable, ya que se utiliza para la recuperación de contraseña y notificaciones importantes como la creación o eliminación de documentos.
                              \item \textit{SendRecoveryFragment}: permite a los usuarios solicitar un código de recuperación de contraseña, el cual es enviado al correo electrónico registrado. El código será enviado \textbf{únicamente} si se encuentra una cuenta asociada al correo proporcionado.
                              \item \textit{VerifyCodeFragment}: permite a los usuarios ingresar el código de recuperación recibido por correo electrónico, así como validar la validez del mismo.
                              \item \textit{ResetPasswordFragment}: permite a los usuarios establecer una nueva contraseña tras haber validado el código de recuperación.
                              \item \textit{ChatFragment}: permite la interacción directa con el asistente virtual. Incluye un campo de texto para ingresar preguntas, un botón para enviar las consultas y una vista de lista que muestra el historial de mensajes intercambiados con el asistente.
                              \item \textit{ProfileFragment}: permite al usuario visualizar y actualizar su información personal, como nombre, apellido, correo electrónico y número de teléfono. La contraseña solo puede ser modificada mediante el flujo de recuperación de contraseña.
                              \item\textit{ DocumentsFragment}: permite a los usuarios con rol \textbf{Administrador} visualizar el listado de documentos educativos cargados en el sistema, así como acceder al documento original de ser requerido, o bien, agregar o eliminar documentos según sea necesario.
                        \end{itemize}
                  \item Implementación de navegación mediante \textit{Navigation Component},
                        garantizando consistencia y control del back stack.
                  \item Integración de indicadores de carga y estado de conexión, ofreciendo
                        retroalimentación inmediata al usuario sobre la consulta al LLM.
            \end{itemize}

      \item \textbf{Integración con el servidor y servicios de Python}
            \begin{itemize}
                  \item Consumo de puntos de conexión del servidor para autenticación, gestión de
                        sesiones, envío de preguntas y recuperación de respuestas.
                  \item Procesamiento de respuestas JSON, parseo y renderizado en la interfaz de
                        usuario de manera clara y comprensible.
                  \item Manejo de errores y reconexión ante fallos de red, asegurando robustez en la
                        experiencia de usuario.
            \end{itemize}
\end{enumerate}

\subsection{Resultado final}
Al finalizar este \textit{sprint}, se obtuvo:
\begin{itemize}
      \item Aplicación móvil funcional en Android, con integración nativa mediante Kotlin y
            comunicación estable con el servidor.
      \item Estructura modular clara (\textbf{Datos}, \textbf{Inyección de Dependencias},
            \textbf{Ayudantes}, \textbf{UI}, \textbf{Recursos}) que facilita mantenimiento
            y escalabilidad.
      \item Flujo de interacción optimizado para el usuario, incluyendo envío de preguntas,
            recepción de respuestas contextuales y visualización de documentos y
            administración de documentos para usuarios autorizados.
\end{itemize}

Este \textit{sprint} permitió contar con una interfaz móvil operativa, lista
para el despliegue y pruebas de usabilidad, estableciendo las bases para las
fases finales de evaluación y documentación del proyecto.

\section{\textit{Sprint} 5: Pruebas y validación}
\textbf{Duración estimada:} 3 semanas

Este \textit{sprint} se centró en validar el funcionamiento integral del
sistema, asegurando la correcta interacción entre la aplicación móvil, el
servidor, la base de datos relacional y vectorial y el modelo de lenguaje (LLM,
por sus siglas en inglés). Además, se buscó evaluar la calidad, precisión y
confiabilidad de las respuestas generadas por el asistente virtual mediante la
metodología y métricas propuestas en las secciones 4.3.9 y 4.3.10 de este
documento.

\subsection{Ejecución}
\begin{enumerate}
      \item \textbf{Pruebas funcionales del sistema}
            \begin{itemize}
                  \item Verificación de la comunicación entre la aplicación móvil (Kotlin), el servidor
                        (NodeJS), la base de datos relacional (PostgreSQL) y la base de datos vectorial
                        (Pinecone). Se considera exitoso si cada módulo responde correctamente a las
                        solicitudes y envía los datos esperados al módulo contiguo.
                  \item Pruebas de puntos de conexión mediante Postman para asegurar correcta
                        autenticación de usuarios, envío de preguntas, recuperación de respuestas y
                        gestión de historial de chats. Se considera exitoso si cada uno de los puntos
                        de conexión funciona como se espera, incluyendo la validación de parámetros,
                        restricciones de acceso, formato de respuestas y manejo de errores.
                  \item Comprobación de la integridad de los datos entre los distintos módulos,
                        incluyendo creación, lectura, actualización y eliminación de información
                        (CRUD). Se considera exitoso si todos los datos manipulados a través del
                        cliente se reflejan correctamente en la base de datos relacional y vectorial,
                        manteniendo consistencia y precisión. Se evalúa también la integridad
                        referencial a los documentos originales almacenados en el contenedor S3; es
                        decir, si un elemento se crea o se elimina en cualquiera de las tres fuentes de
                        datos, este cambio debe reflejarse en las otras dos.
            \end{itemize}

      \item \textbf{Pruebas de calidad y confiabilidad de las respuestas}
            \begin{itemize}
                  \item Se definió un conjunto de 45 preguntas de prueba, basadas en el corpus y
                        distribuidas en categorías como ética, ciudadanía, formación ciudadana o
                        democracia. También se incluyeron consultas que evalúan la capacidad del
                        asistente para manejar situaciones hipotéticas o dilemas morales en los cuales
                        podría incurrir el usuario. Por otro lado, se incluyeron 10 preguntas de
                        control que no están relacionadas con el contenido base, para evaluar la
                        capacidad del modelo de rechazar consultas fuera de contexto y verificar la no
                        alucinación del mismo.
                  \item Se validaron las respuestas generadas por el asistente al compararlas con el
                        contenido base utilizado para conformar el corpus, ya que no basta solamente
                        con que el modelo sea capaz de responder coherentemente, sino que toda
                        información proporcionada debe estar alineada con el contenido educativo
                        recopilado.
                  \item Se identificaron los casos en que el modelo no proporciona información
                        suficiente o presenta inconsistencias, al haber tomado en cuenta aquellas
                        preguntas que el modelo sí se espera que sea capaz de responder, pero que no lo
                        hace correctamente. Estos casos se documentan para su posterior análisis y
                        ajuste en futuras iteraciones.
                  \item Se evaluó el tiempo de respuesta transcurrido desde que el usuario envía la
                        consulta hasta que recibe la respuesta generada por el asistente.
            \end{itemize}
\end{enumerate}

\subsection{Resultado final}
Al concluir este \textit{sprint}, se logró:

\begin{itemize}
      \item Una validación completa de la integración entre cliente (\textit{frontend}),
            servidor (\textit{backend}) y bases de datos, garantizando estabilidad y
            funcionalidad del sistema, así como el correcto funcionamiento de todos los
            componentes involucrados antes de incurrir en un análisis específico para
            identificar puntos de mejora. Se verifica que todos los componentes técnicos
            funcionen según lo previsto, encaminados al cumplimiento de los objetivos del
            proyecto.
      \item Obtener resultados cuantitativos a partir de las pruebas de calidad y
            confiabilidad que permiten determinar el estado actual del modelo RAG
            (\textit{Retrieval-Augmented Generation}) en términos de
            \begin{itemize}
                  \item \textbf{Eficiencia:} Tiempo de respuesta del modelo.
                  \item \textbf{Exito} (\%): Porcentaje de los casos de prueba se comportó como se esperaba.
                  \item \textbf{Nivel de congruencia} (\%): Porcentaje de preguntas que el modelo supo responder correctamente, basado de forma estricta en el contenido proporcionado.
            \end{itemize}
      \item La implementación de mejoras en la interfaz y en la lógica de generación de
            peticiones para optimizar la experiencia de usuario y la pertinencia
            pedagógica.
      \item La documentación de resultados de prueba y recomendaciones para mantenimiento
            futuro y escalabilidad del proyecto.
\end{itemize}

Este \textit{sprint} permitió asegurar que la aplicación estuviera lista para
su uso efectivo público, a través de proporcionar respuestas precisas y
contextualizadas, así como el establecimiento de las bases para fases futuras
de despliegue y monitoreo continuo del asistente virtual.

\section{\textit{Sprint} 6: Documentación y presentación}
\textbf{Duración estimada:} 3 semanas

Este \textit{sprint} se centró en consolidar toda la documentación generada
durante el desarrollo del proyecto y preparar la presentación final del
asistente virtual de formación ciudadana y valores morales. El objetivo fue
garantizar que tanto los resultados como los procesos utilizados quedaran
claramente registrados, así como asegurar que el producto final estuviera
disponible para revisión, prueba y entrega formal al cliente.

\subsection{Ejecución}
\begin{enumerate}
      \item \textbf{Elaboración del informe final por medio de}
            \begin{itemize}
                  \item La integración de la información de todos los \textit{sprints} previos en un
                        documento único, estructurado y coherente.
                  \item La inclusión de resultados de cada \textit{sprint}, análisis de hallazgos,
                        decisiones de diseño y mejoras implementadas.
                  \item La redacción de conclusiones generales y recomendaciones para futuras
                        iteraciones, escalabilidad o mejoras del asistente virtual.
                  \item El formateo del documento en LaTeX, asegurando uniformidad, claridad y
                        cumplimiento de estándares académicos y de presentación profesional.
            \end{itemize}

      \item \textbf{Preparación de la presentación final, que incluye}
            \begin{itemize}
                  \item El desarrollo del material visual que resuma el proyecto, incluyendo diagramas
                        de arquitectura, capturas de pantalla del prototipo móvil, flujo de interacción
                        y ejemplos de uso del asistente.
                  \item La elaboración de una presentación estructurada para explicar el proceso de
                        desarrollo, resultados obtenidos y funcionalidades del sistema.
                  \item El ensayo de la presentación y ajuste de contenido para garantizar claridad,
                        concisión y relevancia para el público objetivo.
            \end{itemize}

      \item \textbf{Traslado y entrega de documentación al cliente, a través de}
            \begin{itemize}
                  \item La consolidación de repositorios de código, documentos de investigación, fichas
                        de usuario, diagramas de arquitectura y demás materiales generados.
                  \item La entrega formal de toda la documentación y repositorios a la Fundación de
                        Scouts de Guatemala, asegurando que puedan acceder a todos los recursos para
                        pruebas, mantenimiento y futuras actualizaciones.
                  \item El registro de la entrega, incluyendo inventario de archivos, versión final de
                        documentación y evidencia de disponibilidad del producto para pruebas finales.
            \end{itemize}
\end{enumerate}

\subsection{Resultado final}
Como resultado de este \textit{sprint}, se logró generar:

\begin{itemize}
      \item Un informe final consolidado, claro y completo que documenta todo el proceso de
            desarrollo, análisis y resultados del proyecto.
      \item El material de presentación profesional listo para exponer ante el cliente y
            otros interesados.
      \item La entrega formal de toda la documentación y repositorios al cliente,
            asegurando disponibilidad total de recursos para pruebas, evaluación y futuras
            mejoras.
      \item El registro de la entrega y validación de que el producto final está operativo
            y listo para su uso y pruebas definitivas.
\end{itemize}

Este \textit{sprint} concluyó con la transferencia completa del conocimiento y
del producto, cerrando oficialmente el ciclo de desarrollo inicial del
asistente virtual y dejando una base sólida para el mantenimiento y
escalabilidad futura del proyecto.
	\fi
\fi

% CAPÍTULOS
% ------------------------------------------------------------------------------
% \newpage
% \ifdefined\parpordefecto
% 	\defaultparformat{j-capitulos}
% \else
% 	\input{j-capitulos}
% \fi

% RESULTADOS
% ------------------------------------------------------------------------------
\ifdefined\CAPresultados
	\newpage
	\chapter{Resultados}
	\ifdefined\parpordefecto
		\defaultparformat{m-resultados}
	\else
		Esta fase del proyecto \textit{Ciudadano Digital} culminó con la implementación
de un prototipo funcional de un asistente virtual para la educación informal en
ciudadanía y valores morales. El sistema integra de manera coherente
componentes de procesamiento de lenguaje natural (NLP, por sus siglas en
inglés), recuperación semántica de contexto y una interfaz nativa móvil para
Android. Si bien los resultados no constituyen una validación empírica del
impacto educativo, al no contar con pruebas de campo reales con usuarios
finales, esta primera versión demostró la viabilidad técnica y conceptual del
proyecto. Asimismo, se verificó su alineación con el marco teórico y los
objetivos específicos, destacando los logros técnicos y de diseño alcanzados
durante los sprints definidos.

\section{Definición de usuario objetivo (Persona)}
Durante el primer \textit{sprint} se elaboró una ficha técnica detallada del
\textbf{perfil Persona}, ilustrada en la Figura \ref{fig:persona}.

\begin{figure}[H]
      \centering
      \includegraphics[width=0.8\textwidth]{assets/persona.png}
      \caption{Ficha de \textbf{Perfil Persona} con datos básicos del usuario objetivo.}
      \label{fig:persona}
\end{figure}

A partir de este perfil, mediante el cual se identificaron las motivaciones,
objetivos, frustraciones, necesidades y comportamientos digitales de la
población objetivo, se definieron los siguientes aspectos centrales de la
aplicación:

\begin{itemize}
      \item \textbf{Plataforma:} En esta fase, el desarrollo se concentró en el sistema operativo Android. Se obtuvo compatibilidad para dispositivos de este sistema operativo a partir de la versión \textbf{Android 7}.
      \item \textbf{Enfoque:} La interacción pregunta-respuesta se diseñó bajo la filosofía del método socrático, lo que permitió que en cada interacción con el asistente se se obtenga también una lista de preguntas sugeridas.
      \item \textbf{Diseño:} La interfaz sigue los principios de \textit{Material Design}, esto al priorizar un diseño minimalista y enfocado en la funcionalidad del sistema
\end{itemize}

% \section{Implementación de flujo para procesamiento de archivos}
% El sistema obtenido a través del segundo \textit{sprint}, permite procesar
% diversos formatos de archivo:

% \begin{itemize}
%       \item Documentos PDF (.pdf)
%       \item Documentos de Word (.doc, .docx)
%       \item Presentaciones (.ppt, .pptx)
%       \item Texto plano (.txt)
%       \item Markdown (.md)
%       \item Imágenes y documentos escaneados (mediante Tesseract OCR)
% \end{itemize}

% La Figura \ref{fig:procesamiento-docs} ilustra el flujo completo de
% procesamiento:

% \begin{figure}[H]
%       \centering
%       \includegraphics[width=0.9\textwidth]{assets/procesamiento-docs.png}
%       \caption{Flujo de procesamiento de documentos para generación del corpus.}
%       \label{fig:procesamiento-docs}
% \end{figure}

% El proceso se divide en dos etapas principales:

% \begin{enumerate}
%       \item \textbf{Extracción y normalización del contenido:} En esta etapa se realiza la lectura y extracción de texto desde los formatos soportados. El flujo incluye la limpieza del texto, la estandarización del formato (identificación de títulos, encabezados y listas para preservar la semántica), la validación de la integridad (omisión de duplicados, normalización de caracteres) y la fragmentación semántica. El producto final resultante está compuesto por fragmentos de texto plano autocontenidos.
%       \item \textbf{Generación de representaciones numéricas e indexación vectorial:} Cada fragmento de texto se transforma en un vector mediante el modelo de OpenAI \texttt{text-embedding-3-small}. A cada representación numérica de texto, se le asocian metadatos relevantes (identificador del documento original, título, categoría temática, texto original, autor y año de publicación). Posteriormente, se realiza una operación de upsert en el índice vectorial de Pinecone. Simultáneamente, se registra en la base de datos relacional la trazabilidad entre el vector y el documento original almacenado en un contenedor Amazon S3.
% \end{enumerate}

% \section{Implementación del modelo LLM y flujo RAG}
% En el tercer \textit{sprint}, se implementó un flujo RAG
% (\textit{Retrieval-Augmented Generation}) que permite al asistente virtual
% generar respuestas fundamentadas en los contenidos educativos previamente
% indexados. Este flujo integra los siguientes componentes:
% \begin{enumerate}
%       \item \textbf{API en NodeJS:} Actúa como servidor principal, orquestando la comunicación entre la aplicación móvil y los servicios de procesamiento de lenguaje natural.
%       \item \textbf{Microservicio en Python:} Gestiona la interacción con el modelo LLM (\textit{Large Language Model}) de OpenAI para generar respuestas y procesar documentos.
%       \item \textbf{API de OpenAI:} Utilizada para la generación de representaciones numéricas (\textit{embeddings}) y respuestas.
%       \item \textbf{Base de datos vectorial (Pinecone):} Almacena y permite la recuperación de las representaciones numéricas generadas a partir de los contenidos educativos.
%       \item \textbf{Base de datos relacional (PostgreSQL):} Gestiona usuarios, mensajes, sesiones e información de los documentos procesados, asegurando la trazabilidad desde los vectores hasta el documento original.

% \end{enumerate}
% El flujo completo del proceso RAG, desde la recepción de una pregunta en la aplicación móvil hasta la devolución de la respuesta, se ilustra en la Figura \ref{fig:flujo-rag}.

% \begin{figure}[H]
%       \centering
%       \includegraphics[width=0.8\textwidth]{assets/RAG.png}
%       \caption{Flujo de procesamiento de preguntas mediante RAG.}
%       \label{fig:flujo-rag}
% \end{figure}

% \subsection{Integración de la Base de Datos Vectorial y Relacional}
% El almacenamiento de información a lo largo del sistema se describe en 3
% elementos principales:
% \begin{itemize}
%       \item \textbf{Base de datos relacional (PostgreSQL):} gestor principal de los datos del sistema; aquí se almacenan los datos de los usuarios, chats, mensajes, sesiones, códigos de recuperación de contraseñas y metadatos básicos de los documentos procesados.
%       \item \textbf{Base de datos vectorial (Pinecone):} encargada de almacenar las representaciones numéricas de texto generadas a partir de los documentos seleccionados. Esta es la base del funcionamiento RAG, ya que permite recuperar el contexto necesario, según la pregunta realizada, para que el modelo LLM pueda generar respuestas fundamentadas.
%       \item \textbf{Documentos originales (Contenedor AWS S3):} aquí se almacenan todos los documentos originales cargados al sistema, lo que permite su posterior consulta o descarga por parte de los usuarios con rol de \textbf{Administrador}.
% \end{itemize}

\section{Corpus vectorizado}
El sistema definido en los \textit{sprints} 2 y 3 permitió el procesamiento y
gestión de los documentos con los que se alimenta el modelo. Con ello, se
tomaron los documentos seleccionados al inicio del \textit{sprint} 2, se
sometieron a procesamiento, limpieza e indexación, todo a través de solicitudes
realizadas desde el cliente, validando la correcta integración de sistemas
internos y cliente-servidor.

Las pruebas realizadas verificaron la sincronización entre todas las fuentes de
datos a través del servidor, asegurando que las consultas se asociaran con el
contenido adecuado. La Figura \ref{fig:preview-pinecone} muestra el índice
vectorial en Pinecone después de cargar las representaciones numéricas
(\textit{embeddings}) de los fragmentos de texto.

\begin{figure}[H]
      \centering
      \includegraphics[width=0.8\textwidth]{assets/pinecone-preview.png}
      \caption{Vista del índice vectorial en Pinecone con las representaciones numéricas (\textit{embeddings}) cargadas.}
      \label{fig:preview-pinecone}
\end{figure}

\section{Interfaz móvil en Kotlin (Cliente)}
Durante el cuarto \textit{sprint}, se desarrolló la aplicación móvil nativa
para dispositivos Android utilizando el lenguaje de programación Kotlin,
implementando el patrón de diseño MVVM (Modelo-Vista-Modelo de Vista) para
asegurar una arquitectura modular y mantenible. Como resultado, se obtuvo una
interfaz funcional, amigable con el usuario y cuyo uso no requiere estar
familizarizado previamente con herramientas parecidas.

\subsection{Funcionalidades Comunes}
La aplicación cuenta con 2 tipos de roles diferentes: \textbf{Usuario} y
\textbf{Administrador}, cuya gestión debe ser realizada desde la base de datos
al definir en el campo \textbf{\guillemetleft{}role\guillemetright{}} para cada
usuario. Las funcionalidades básicas a las que tiene acceso tanto el usuario
como el administrador, son:
\begin{itemize}
      \item \textbf{Conversaciomes:} El usuario puede crear chats, enviar mensajes y recibir respuestas del modelo. No se requieren accesos especiales para esta función, tan solo contar con una cuenta de usuario sin importar el rol.
      \item \textbf{Perfil de Usuario:} Los usuarios pueden ver y modificar su propia informació personal. A excepción de la contraseña, tienen acceso a cambiar sus nombres, apellidos, fecha de nacimiento, correo electrónico o número de teléfono.
      \item \textbf{Cerrar Sesión:} Todos los usuarios pueden elegir terminar su sesión actual de forma automática. Si no lo hacen, la sesión se cerrará igualmente de forma automática tras transcurrir 1 hora de inactividad en la aplicación.
\end{itemize}

\subsection{Funcionalidades del Administrador}
Por otro lado, también hay funciones que solamente el administrador puede
realizar, con el fin de mantener el control de aspectos delicados de la
aplicación, que en este caso corresponden a mantener la seguridad y
consistencia del corpus del modelo. Las funcionalidades específicas que solo
puede realizar el administrador, son:
\begin{itemize}
      \item \textbf{Ver documentos:} Los usuarios con rol administrador pueden acceder al panel de gestión de documentos, en donde se muestran todos los archivos cargados en la base de datos. Desde aquí, tienen la opción de ver el documento original en su navegador.
      \item \textbf{Añadir documentos:} De igual forma, los usuarios administradores pueden agregar documentos al corpus del modelo. Para ello, en la vista de gestión de documentos, deben seleccionar la opción \textbf{\guillemetleft{}Agregar\guillemetright{}} e ingresar los datos solicitados: Nombre, Autor, Año, Rango de edades. Finalmente, se debe cargar el documento y presionar \textbf{\guillemetleft{}Guardar\guillemetright{}}. El procesamiento del archivo se realiza en segundo plano, por lo que al usuario le llegará una notificación por correo electrónico cuando esté listo y cargado en el sistema.
      \item \textbf{Eliminar documentos:} Finalmente, el usuario administrador también puede remover documentos del corpus si le parece apropiado o si ya no se desea que forme parte del corpus del sistema. Igualmente, esta eliminación se procesa en segundo plano, así que cuando esté listo, le llegará una notificación por correo electrónico.
\end{itemize}

La Figura \ref{fig:app-interfaces} muestra algunas de las pantallas principales
de la aplicación móvil desarrollada.

\begin{figure}[H]
      \centering
      \begin{minipage}{0.22\textwidth}
            \centering
            \includegraphics[width=\textwidth]{assets/app-login.jpeg}
            \subcaption{Pantalla de inicio de sesión.}
      \end{minipage}
      \hfill
      \begin{minipage}{0.22\textwidth}
            \centering
            \includegraphics[width=\textwidth]{assets/app-chat.jpeg}
            \subcaption{Pantalla de chat con asistente virtual.}
      \end{minipage}
      \hfill
      \begin{minipage}{0.22\textwidth}
            \centering
            \includegraphics[width=\textwidth]{assets/app-documents.jpeg}
            \subcaption{Pantalla de gestión de documentos.}
      \end{minipage}
      \hfill
      \begin{minipage}{0.22\textwidth}
            \centering
            \includegraphics[width=\textwidth]{assets/app-messages.jpeg}
            \subcaption{Pantalla de chat con mensajes.}
      \end{minipage}
      \caption{Interfaz de la aplicación móvil desarrollada.}
      \label{fig:app-interfaces}
\end{figure}

El diseño de la interfaz emplea los principios y componentes de
\textit{Material Design}, junto con \textit{Navigation Component} para la
navegación y \textit{Hilt} para la inyección de dependencias.

\section{Pruebas y Validación}
La implementación del prototipo concluyó con una evaluación exhaustiva del
sistema integrado. Se realizaron pruebas a los puntos de conexión mediante
Postman, los cuales cubren la autenticación, recuperación de contraseña, envío
de mensajes, historial de chats y administración de documentos.

En el Anexo \ref{tab:consultas-resultados}, se presenta una serie de 45
preguntas relacionadas al tema central del proyecto, abarcando conceptos de
ciudadanía, formación ciudadana y valores morales. Adicionalmente, se
realizaron 10 pregutnas de control (no relacionadas con el tema de civismo ni
con los documentos que conforman el corpus), diseñadas para evaluar la
\textbf{no alucinación} del modelo; es decir, que no responda cuando no deba
hacerlo.

Para cada pregunta, independientemente si era de control o no, se tomó el
tiempo de respuesta, así como las referencias devueltas por el sistema (Anexo
\ref{tab:consultas-referencias}). La columna \texttt{Exitosa} se refiere a si la pregunta fue o no
fue respondida por el modelo, mientras que la columna \texttt{Congruente}
establece la congruencia según el tipo de consulta realizada:

\begin{itemize}
      \item \textbf{Consulta:} Estas se refieren a preguntas que el modelo \textbf{sí debería responder} ya que estan basadas estrictamente en el contenido del corpus proporcionado.
      \item \textbf{Control:} Se trata de preguntas deliberadamente fuera del contexto del sistema, las cuales el modelo \textbf{no debería responder}. En este caso, la congruencia es positiva cuando el modelo efectivamente se niega a responder por estar fuera de su alcance y propósito.
\end{itemize}

Como se mencionó, el Anexo \ref{tab:consultas-referencias} muestra la relación
entre los documentos que se espera que el modelo seleccione para responder cada
pregunta, frente a las referencias que finalmente utilizó para extraer el
contexto necesario. Se evidencia que el comportamiento en este caso presenta 3
variantes detectables:
\begin{itemize}
      \item \textbf{Coincidencia exacta:} Se refiere a los casos en los que las referencias esperadas y las extraídas por el modelo coinciden por completo, como se evidencia en las filas 13, 16, 17, 22, 23, 34, 39, 42, 46, 48 y 53. Destaca el caso mostrado en la fila 13, puesto que si bien las referencias utilizadas son las mismas que las esperadas, difiere el orden de prioridad interpretado por el modelo.
      \item\textbf{Coincidencia parcial:} Estos son casos que incluyen algunas de las referencias esperadas pero también el uso de referencias adicionales; se evidencia este comportamiento en las filas 1, 7, 11, 14, 18, 26, 28, 33, 36, 38, 41, 43, 44, 47, 49, 51, 52 y 55.
      \item \textbf{Coincidencia nula:} Estos casos son aquellos en los que las referencias utilizadas difieren completamente de las referencias esperadas, como se evidencia en las filas 3, 6, 8, 12, 19 y 31.
\end{itemize}

\subsection{Métricas Encontradas}

A partir de los resultados, se calcularon las siguientes métricas relacionadas
con el desempeño del sistema:

En primer lugar, el Anexo \ref{tab:consultas-resultados} muestra que, de 45
consultas reales (excluyendo las de control), 35 fueron respondidas
exitosamente, lo que representa una tasa de éxito global del 77.78\%. Para las
consultas de control, ninguna fue respondida, lo que resulta en una tasa de
éxito del 100\% para este tipo.

Respecto a la congruencia, de las 55 consultas totales, 45 mostraron un
comportamiento congruente con lo establecido, lo que equivale a una congruencia
fáctica global del 81.81\%.

Para la latencia, el tiempo de respuesta mínimo fue de 3.303 segundos y el
máximo de 19.121 segundos, con una desviación estándar aproximada de 2.96. El
tiempo de respuesta promedio fue de 7.8184 segundos. La comparación de estos
datos con modelos comerciales será abordada en la sección de discusión.

Analizando el Anexo \ref{tab:consultas-referencias}, de las 35 consultas
respondidas correctamente, la distribución por tipo de coincidencia fue:

\begin{itemize}
      \item Coincidencia Exacta: 11 (31.43\%)
      \item Coincidencia Parcial: 18 (51.43\%)
      \item Coincidencia Nula: 6 (17.14\%)
\end{itemize}

Las métricas calculadas se resumen en el Cuadro \ref{tab:metricas}.

\begin{table}[H]
      \centering
      \caption{Métricas de desempeño del sistema}
      \label{tab:metricas}
      \begin{tabular}{|l|c|}
            \hline
            \textbf{Métrica}     & \textbf{Valor} \\
            \hline
            Tasa de éxito global & 77.78\%        \\
            Congruencia fáctica  & 81.81\%        \\
            Latencia promedio    & 7.8184         \\
            \hline
      \end{tabular}
\end{table}


	\fi
\fi

% DISCUSIÓN
% ------------------------------------------------------------------------------
\ifdefined\CAPdiscusion
	\newpage
	\chapter{Discusión}
	\ifdefined\parpordefecto
		\defaultparformat{n-discusion}
	\else
		El desarrollo del proyecto \textit{Ciudadano Digital} permitió obtener un
prototipo funcional de un asistente virtual orientado a la educación ciudadana
y en valores morales, lo que evidencia la viabilidad técnica y conceptual de la
propuesta inicial, así como el desarrollo de una arquitectura tecnológica y un
flujo de trabajo adecuados para lograr el alcance de los objetivos planteados.
La integración de un mecanismo de recuperación aumentada de generación (RAG,
por sus siglas en inglés) con modelos de lenguaje de gran escala (LLM, por sus
siglas en inglés) demostró ser una estrategia efectiva para ofrecer respuestas
contextualizadas y fundamentadas en un corpus educativo que, por su parte,
también cuenta con una metodología específica para la selección y procesado del
mismo.

Como se mencionó al inicio de este documento, el desarrollo de
\textit{Ciudadano Digital} contempla la base técnica que busca demostrar la
viabilidad tecnológica de las herramientas para conformar un sistema de
generación aumentada por recuperación (RAG) efectivo. Para este punto, como se
mencionó en la justificación, se optó por centrar el funcionamiento de la
aplicación para su disponibilidad en dispositivos bajo el sistema operativo
Android. Esta decisiónresponde a criterios de accesibilidady equidad
tecnológica, puesto que los dispositivos bajo este sistema operativo
representan la mayoría del mercado móvil en regiones marginadas, por lo que se
alinea con el público objetivo identificado. El diseño implementado para la
aplicación enfatiza la legibilidad, los contrastes adecuados, los controles
táctiles amplios y los flujos de navegación sencillos, todo ello a través de
los principios dictados por el sistema de diseño de Google, \textit{Material
    Design}.

Por otro lado, la interacción pregunta-respuesta diseñada bajo la filosofía del
método socrático promueve no solamente la resolución de consultas, sino también
el uso de preguntas sugeridas que enriquezcan el conocimiento del usuario y lo
motiven a construir sus propias conclusiones a partir de la información
proporcionada.

Desde el punto de vista técnico, la implementación exitosa de la arquitectura
propuesta valida el propósito central del proyecto: la posibilidad de integrar
componentes de procesamiento de lenguaje natural, bases de datos vectoriales e
interfaces móviles para crear un asistente educativo funcional. Bajo la
elección de la metodología RAG, se permitió superar una limitación fundamental
de los LLMs convencionales: la propensión a la alucinación mediante la
contextualización estricta de las respuestas en el corpus educativo. El sistema
construido logró integrar exitosamente las tecnologías planteadas desde el
inicio del proyecto; se logró la generación de un corpus dedicado a la
fundamentación de las respuestas obtenidas del modelo, esto a través de la base
de datos vectorial Pinecone que, a su vez, se comunica correctamente con la
base de datos relacional y con el contenedor de archivos. Todo este sistema
permite la obtención apropiada de respuestas basadas únicamente en los
documentos seleccionados y procesados, garantizando que no exista alucinación
por parte del modelo y que este se limite a responder únicamente las cuestiones
que estén relacionadas con el contenido de la base de datos vectorial.

Regresando a la Figura \ref{fig:flujo-rag}, el flujo de generación aumentada
por reucperación (RAG) se ejecuta de manera sistemática al dividir
correctamente las tareas entre los distintos componentes del sistema. Sin
embargo, si bien se cumplieron las expectativas técnicas de implementación,
curado y procesamiento de datos, se identifican claramente nuevas oportunidades
de mejora a abordar en iteraciones futuras del proyecto, tales como la
optimización del tiempo de respuesta, la mejora en la gestión de errores y la
incorporación de mecanismos de supervisión pedagógica que garanticen
contínuamente la validación de los documentos utilizados para alimentar el
modelo.

En el plano semántico, el sistema logró una tasa de éxito del 77.78\% en
consultas válidas y un 100\% de efectividad en el rechazo de preguntas de
control (no relacionadas al tema de civismo). Esto evidencia que el mecanismo
de recuperación semántica funciona como barrera efectiva contra respuestas no
fundamentadas. Este resultado es crucial en contextos educativos, donde la
precisión fáctica y la confiabilidad de la información son requisitos
fundamentales. A su vez, la latencia promedio de \textbf{7.8184 segundos} señala un área
de optimización prioritaria; futuras iteraciones podrían explorar estrategias
como la precomputación de representaciones numéricas de texto
(\textit{embeddings}) para consultas frecuentes o la implementación de cachés
semánticos que detecten similitudes entre preguntas, de manera que se logre
reducir los tiempos de respuesta.

En comparación con mediciones independientes realizadas a modelos de referencia
como \textit{GPT-4-0125} y \textit{GPT-4-1106}, cuyos tiempos de respuesta promedio se sitúan
aproximadamente en \textbf{7.16s} y \textbf{6.36s} respectivamente, la latencia del sistema
desarrollado (7.8184 s) se mantiene dentro de un rango competitivo para
arquitecturas basadas en RAG \cite{openai_performance_analysis_2024}. Sin embargo, estas pruebas también revelan
diferencias relevantes en variabilidad: mientras que los modelos de GPT
reportan desviaciones estándar cercanas a \textbf{0.59s} y \textbf{1.45s} \cite{openai_performance_analysis_2024}, el comportamiento
del prototipo presenta fluctuaciones asociadas a su naturaleza distribuida, que
depende de la recuperación semántica, el acceso a la base vectorial y la
posterior generación de la respuesta. Esta comparación refuerza la importancia
de explorar mecanismos de optimización orientados a reducir la latencia global del sistema.

Adicionalmente, se obtuvo una congruencia fáctica del 81.81\%; es decir, una
mayor parte de las pruebas realizadas presentaron el comportamiento esperado.
Esta métrica fue obtenida a través de las preguntas de \textbf{consulta} y
preguntas de \textbf{control} (temas no pertinentes). En el caso de las
consultas, la congruencia resulta negativa cuando el modelo no es capaz de
responder, ya que se rompe el comportamiento esperado de encontrar el contexto
específico para contestar la pregunta. Por otro lado, en el caso de las
preguntas de control, la congruencia es positiva cuando el modelo efectivamente
se niega a responder, ya que se espera que identifique que esta petición está
fuera de su alcance y propósito.

Si bien se detectan áreas de mejora a abarcar en próximas fases del proyecto,
las métricas cuantitativas obtenidas por el sistema, permiten demostrar que
este alcanzó un desempeño adecuado en la generación de respuestas fundamentadas
y coherentes con el material curado.

Por otro lado, al analizar las respuestas generadas por el sistema y su
comparación con las fuentes extraídas y seleccionadas, se revelan patrones
significativos sobre el comportamiento del flujo RAG. La fidelidad efectiva del
82.86\% (considerando coincidencias exactas y parciales) indica que el sistema
logra, en la mayoría de los casos planteados, recuperar y utilizar información
relevante del corpus. Sin embargo, la distribución de los tipos de coincidencia
(31.43\% exacta, 51.43\% parcial y 17.14\% nula) merece una consideración
detallada y una iteración dedicada exhaustivamente a la optimización de la
calidad de respuestas obtenidas.

Cabe aclarar que la coincidencia exacta se define como aquella en la que la
consulta realizada al asistente recuperó información proveniente de las mismas
referencias esperadas. Los casos en los que se obtienen las mismas referencias
pero en distinto orden de prioridad pueden deberse a que la selección manual
inicial (realizada sin el acompañamiento de un experto en el dominio de
educación) no ponderó adecuadamente el peso conceptual de cada referencia. Por
ello, al obtener el contexto mediante Pinecone, la prioridad de las referencias
puede variar.

Por otro lado, los casos de coincidencia parcial evidencian que el modelo puede
utilizar referencias que no fueron consideradas durante la selección manual o
que algunas referencias incluidas en dicha selección (nuevamente, limitada por
la falta de un especialista en educación) no obtuvieron un puntaje suficiente
para ser seleccionadas como contexto por Pinecone. Esto explica por qué el
sistema eligió ciertos fragmentos distintos a los previstos.

Las coincidencias parciales, que representan más de la mitad de los casos,
sugieren que el sistema posee flexibilidad semántica para identificar
información relevante más allá de lo estrictamente anticipado. Este
comportamiento puede interpretarse positivamente como capacidad de inferencia
contextual, aunque también subraya la importancia de revisar la estrategia de
segmentación empleada, especialmente considerando que sin la guía de un experto
en educación podrían haberse definido fragmentos o criterios de segmentación
subóptimos. Esto debe tomarse en cuenta para mejorar futuras iteraciones del
sistema.

Por su parte, los casos de coincidencia nula (17.14\%), donde el sistema
responde correctamente pero con referencias diferentes a las esperadas, podrían
indicar redundancia temática en el corpus, o bien, limitaciones en el proceso
de recuperación. Iteracioes posteriores podrían dedicarse a la incorporación de
análisis de similitud semántica entre documentos, incluido en el flujo de
procesamiento, de manera que se pueda determinar si estas discrepancias
representan variaciones conceptuales legítimas o si dependen directamente de
estrategias de mejora en el indexado vectorial.

Desde una perspectiva pedagógica, si bien el proyecto fue ideado no como una
herramienta de enseñanza directa, sino como un asistente secundario que
acompañe a los estudiantes en las inquietudes que los aquejen al salir del
salón de clases, el proyecto demuestra el potencial de los asistentes basados
en IA para complementar los procesos de formación ciudadana. El enfoque
socrático implementado, evidenciado en respuestas que invitan a la reflexión
más que a la memorización, alinea con los principios de educación moral y con
las prácticas de tutoría inteligente.

No obstante, la tasa de éxito del 77.78\% también revela limitaciones
significativas: en las pruebas funcionales se identificaron 7 solicitudes
fallidas relacionadas con la conexión a internet, las cuales no se incluyeron
en el análisis porque el enfoque estaba orientado a la fiabilidad semántica de
las respuestas; sin embargo, el Anexo \ref{tab:consultas-resultados} muestra un
total de 10 consultas que no pudieron ser respondidas, equivalentes al 18.18\%
del total. Este hallazgo sugiere brechas en la cobertura temática del corpus
disponible al momento de las pruebas y, además, refuerza la importancia de
contar con un experto en el dominio educativo, cuya ausencia limitó la
definición de contenidos esenciales y la capacidad del sistema para manejar
consultas complejas o ambiguas.

Desde una perspectiva visual, el diseño móvil basado en \textit{Material
    Design} resultó adecuado para un contexto de recursos limitados, facilitando la
accesibilidad y la usabilidad. No obstante, se identifican oportunidades de
mejora en la interfaz gráfica y en la experiencia de usuario que podrían
optimizarse mediante pruebas de usabilidad con usuarios reales, a fin de
garantizar que la herramienta sea intuitiva, accesible y atractiva para los
jóvenes.

Es importante señalar que, debido a la ausencia de interacción directa con el
usuario objetivo durante esta fase, el diseño se fundamentó únicamente en
buenas prácticas y documentación disponible, lo cual podría no reflejar de
forma precisa las necesidades y preferencias reales de los estudiantes
guatemaltecos. Por ello, sería recomendable realizar estudios específicos sobre
el impacto de colores, tipografías, distribución de elementos y flujos de
navegación en la experiencia del usuario final. A su vez, resalta la
importancia de contar con un especialista en diseño de aplicaciones que asegure
un enfoque adecuado para priorizar tanto la usabilidad y apariencia visual
agradable, de manera que la aplicación cumpla con su propósito a la vez que
resulta amigable para el usuario objetivo

Resulta pertinente afirmar, entonces, que el proyecto \textit{Ciudadano
    Digital} busca sentar las bases para la mejora continua del proyecto, enfocado
en implementar la estructura tecnológica básica necesaria para la interacción
con el modelo, la generación continua del corpus y los criterios de diseño
iniciales para continuar el desarrollo de la aplicación móvil. Se logró la
implementación técnica esperada, tomando en cuenta las limitaciones
relacionadas con la falta de retroalimentación directa y validación empírica
con usuarios finales. Sin embargo, esta primera fase del proyecto permite
demostrar cómo la integración de tecnologías de inteligencia artificial puede
contribuir al fortalecimiento de la educación en valores y formación ciudadana
en contextos con recursos limitados, lo que abre la puerta a futuras
investigaciones y desarrollos en este campo.
	\fi
\fi

% CONCLUSIONES
% ------------------------------------------------------------------------------
\ifdefined\CAPconclusiones
	\newpage
	\chapter{Conclusiones}
	\ifdefined\parpordefecto
		\defaultparformat{o-conclusiones}
	\else
		El desarrollo del proyecto \textit{Ciudadano Digital} permitió cumplir
satisfactoriamente la creación de una herramienta tecnológica de educación
informal orientada al acompañamiento en la adquisición de aprendizajes sobre
formación ciudadana y valores morales; puesto que el prototipo construido
demuestra la factibilidad técnica y conceptual de emplear modelos grandes de
lenguaje (LLM, por sus siglas en inglés) como apoyo en procesos educativos no
formales, con especial atención en la coherencia discursiva, pertinencia
temática y alineación con los principios establecidos en el marco teórico.

Asimismo, se logró alcanzar la implementación de un modelo LLM preentrenado y
optimizado mediante la integración de un flujo RAG (siglas en inglés para generación mejorada por recuperación), el cual permite generar respuestas coherentes y contextualizadas
con base en los documentos educativos procesados. Los resultados obtenidos en
la validación técnica y semántica evidencian un comportamiento consistente y un
adecuado control de respuestas no relacionadas con el corpus del proyecto,
evitando así la alucinación. Esto respalda el cumplimiento del primer objetivo
específico planteado y respalda la viabilidad del enfoque adoptado, asíc omo su
escalabilidad para futuras iteraciones del proyecto.

Se alcanzó la integración de una base de datos vectorial mediante la
implementación de Pinecone como índice semántico. Esto permitió el
almacenamiento y recuperación eficiente de fragmentos relevantes a partir de
los contenidos educativos almacenados. Se garantiza la trazabilidad entre los
vectores, metadatos y documentos originales mediante la conexión con la base de
datos relacional y el uso de contenedores de tipo S3. Esta integración facilitó
la optimización del proceso de recuperación de información y permitió el
desarrollo de un flujo RAG completo y funcional.

Se obtuvo como producto final una aplicación móvil nativa para dispositivos
Android a través de Kotlin y el patrón de diseño MVVM (Modelo-Vsta-Modelo de
Vista), logrando alcanzar una interfaz clara, funcional y visualmente coherente
con los principios de \textit{Material Design}. Esta herramienta facilita la
interacción del usuario con todos los elementos que componen el sistema
principal, lo que da paso a una portabilidad y accesibilidad adecuada para el
público objetivo planteado, ya que permite el acceso al flujo RAG desde
cualquier lugar y en cualquier momento.

Al haber desarrollado el flujo RAG mediante un sistema modular y desacoplado,
el producto final presenta una arquitectura flexible y escalable, encapsulado
en una API capaz de ser utilizada por cualquier otro cliente. Este proyecto no
está limitado únicamente a la aplicación móvil desarrollada, sino que queda
abierto a su uso tanto en plataformas web como en otros entornos tecnológicos.

El prototipo constituye una base sólida para futuras fases del proyecto,
demostrando la viabilidad de integrar inteligencia artificial generativa en
entornos educativos informales. Si bien se reconoce que los resultados se
limitan al ámbito técnico y conceptual, los logros alcanzados hasta esta fase
actúan como incentivo para avanzar hacia etapas de validación con usuarios
reales para evaluar el impacto pedagógico, la pertinencia cultural de las
respuestas y la efectividad del acompañamiento en la formación ciudadana.

	\fi
\fi

% RECOMENDACIONES
% ------------------------------------------------------------------------------
\ifdefined\CAPrecomendaciones
	\newpage
	\chapter{Recomendaciones}
	\ifdefined\parpordefecto
		\defaultparformat{p-recomendaciones}
	\else
		Se plantean las siguientes recomendaciones para continuar el desarrollo y
validación de \textit{Ciudadano Digital}:

\begin{itemize}
      \item Optimizar el flujo de obtención de respuestas por parte del modelo. En esta
            fase, la interacción con los servicios secundarios de Python se llevó a cabo
            mediante ejecución directa de los scripts. Esto puede ser optimizado a partir
            de la creación de una API interna que ofrezca mayor eficiencia y escalabilidad
            en la comunicación entre los componentes del sistema. Esto puede lograrse a
            través de herramientas como \textbf{Flask}.
      \item Realizar pruebas de campo con usuarios finales (estudiantes), así como la
            validación pertinente de los objetivos, respuestas e impacto a largo plazo con
            expertos en el área educativa. De esta forma se podrá evaluar la efectividad
            del prototipo en entornos educativos reales al medir el impacto real en la
            comprensión de valores ciudadanos y formación moral.
      \item Ampliación, diversificación y validación continua del conjunto de documentos
            utilizado para armar el corpus de la aplicación, con el fin de mejorar la
            pertinencia y profundidad de las respuestas generadas por el modelo y mantener
            un enfoque actualizado a partir del conocimiento que surja a lo largo del
            tiempo.
      \item Incorporar mecanismos de retroalimentación dentro de la aplicación que permitan
            a los usuarios evaluar la utilidad de las respuestas. Esto permitirá que el
            equipo técnico pueda ajustar continuamente el modelo según las necesidades de
            los usuarios.
      \item Integrar funcionalidades adicionales en el prototipo, como la capacidad de
            generar actividades interactivas o evaluaciones formativas basadas en los
            contenidos presentados, con el fin de fomentar un aprendizaje activo y
            participativo entre los estudiantes.
      \item Incluir la posibilidad de compartir la aplicación con otros usuarios, por medio
            de foros o espacios de discusión apoyados por el propio asistente. De esta
            manera se podría promover el diálogo y la reflexión colectiva sobre los temas
            de ciudadanía digital y valores cívicos, permitiendo alcanzar avances como
            sociedad.
\end{itemize}
	\fi
\fi

% BIBLIOGRAFÍA
% ------------------------------------------------------------------------------
\ifdefined\CAPbibliografia
	\newpage
	\cleardoublepage\phantomsection
	\chapter{\bibname}
	\printbibliography[heading=none]
\fi

% ANEXOS
% ------------------------------------------------------------------------------
\ifdefined\CAPanexos
	\newpage
	\chapter{Anexos}
	\ifdefined\parpordefecto
		\defaultparformat{r-anexos}
	\else
		{
\renewcommand{\tablename}{Anexo}
\setcounter{table}{0}

\begin{longtable}{|p{4cm}|c|c|c|c|}
    \caption{Resultados de las consultas y controles con sus tiempos de respuesta, éxito y congruencia.}
    \label{tab:consultas-resultados}                                                                                                                                                                                                                                                                            \\
    \hline
    \textbf{Consulta}                                                                                                                                                                                               & \textbf{Tiempo de respuesta (s)} & \textbf{Exitosa} & \textbf{Congruente} & \textbf{Tipo} \\
    \hline
    \endfirsthead
    \hline
    \textbf{Consulta}                                                                                                                                                                                               & \textbf{Tiempo de respuesta (s)} & \textbf{Exitosa} & \textbf{Congruente} & \textbf{Tipo} \\
    \hline
    \endhead
    Si soy joven, ¿realmente ya se me considera un \guillemetleft{}ciudadano\guillemetright{}, o solo el \guillemetleft{}ciudadano del futuro\guillemetright{}?                                                     & 8.44                             & Sí               & Sí                  & Consulta      \\
    \hline
    ¿Cómo puedo desarrollar mi autonomía personal para tomar decisiones sin que otros me dominen?                                                                                                                   & 9.229                            & No               & No                  & Consulta      \\
    \hline
    ¿Qué hago si no confío en los políticos o en las instituciones? ¿Sigo siendo un buen ciudadano si soy escéptico?                                                                                                & 19.121                           & Sí               & Sí                  & Consulta      \\
    \hline
    ¿Cuáles son las habilidades básicas que necesito aprender para poder influir en la vida pública?                                                                                                                & 5.764                            & No               & No                  & Consulta      \\
    \hline
    Dame el listado de todos los departamentos de Guatemala                                                                                                                                                         & 9.242                            & No               & Sí                  & Control       \\
    \hline
    ¿Cómo puedo saber si mi identidad personal se está construyendo de forma sana, siendo que la sociedad cambia tan rápido?                                                                                        & 11.634                           & Sí               & Sí                  & Consulta      \\
    \hline
    ¿Qué es el sentimiento de pertenencia a una comunidad y por qué es esencial para ser un buen ciudadano?                                                                                                         & 14.656                           & Sí               & Sí                  & Consulta      \\
    \hline
    ¿Qué diferencia hay entre la ciudadanía civil, política y social, y en cuál tengo más influencia ahora?                                                                                                         & 7.099                            & Sí               & Sí                  & Consulta      \\
    \hline
    ¿Es normal que me quede sin ideas o sin interés al discutir problemas complejos? ¿Cómo supero la apatía?                                                                                                        & 5.644                            & No               & No                  & Consulta      \\
    \hline
    ¡Es una emergencia! ¿Cómo aplico un torniquete?                                                                                                                                                                 & 3.711                            & No               & Sí                  & Control       \\
    \hline
    ¿Qué valores universales deberíamos promover hoy en día?                                                                                                                                                        & 11.849                           & Sí               & Sí                  & Consulta      \\
    \hline
    ¿Cómo puedo practicar la tolerancia y el respeto en un mundo donde hay tantas opiniones y culturas diferentes?                                                                                                  & 5.871                            & Sí               & Sí                  & Consulta      \\
    \hline
    ¿Cómo puedo ser una persona responsable si la sociedad en general parece muy individualista?                                                                                                                    & 7.452                            & Sí               & Sí                  & Consulta      \\
    \hline
    ¿Por qué se dice que el respeto por los derechos humanos debe ser la base de la educación ciudadana?                                                                                                            & 9.316                            & Sí               & Sí                  & Consulta      \\
    \hline
    ¿Cuál es el procedimiento de emergencia si se incendia un televisor?                                                                                                                                            & 7.205                            & No               & Sí                  & Control       \\
    \hline
    ¿Es suficiente con tener buenos sentimientos para ser ético, o tengo que entrenar mi juicio moral?                                                                                                              & 8.444                            & Sí               & Sí                  & Consulta      \\
    \hline
    ¿Cómo se pueden distinguir los valores que benefician a mi grupo de los valores que son buenos para toda la humanidad?                                                                                          & 9.604                            & Sí               & Sí                  & Consulta      \\
    \hline
    ¿Por qué la educación en valores no puede ser solo una asignatura, sino que debe impregnar toda la escuela?                                                                                                     & 9.409                            & Sí               & Sí                  & Consulta      \\
    \hline
    ¿Cómo se relacionan los valores con la prevención de la violencia en mi vida diaria?                                                                                                                            & 8.664                            & Sí               & Sí                  & Consulta      \\
    \hline
    Le di chocolate a mi perro, ¿cómo puedo salvarlo?                                                                                                                                                               & 3.376                            & No               & Sí                  & Control       \\
    \hline
    ¿Qué pasa si mi comunidad tiene problemas de corrupción o impunidad? ¿Qué puedo hacer?                                                                                                                          & 6.959                            & No               & No                  & Consulta      \\
    \hline
    ¿Qué son los servicios públicos y cómo podemos asegurarnos de que el Estado cumpla su responsabilidad de proporcionarlos?                                                                                       & 8.207                            & Sí               & Sí                  & Consulta      \\
    \hline
    ¿Qué diferencia hay entre la Demografía, la Sociología y la Antropología, y cómo me sirven para entender mi comunidad?                                                                                          & 6.967                            & Sí               & Sí                  & Consulta      \\
    \hline
    ¿Por qué es importante que yo sepa sobre los impuestos si todavía no trabajo?                                                                                                                                   & 3.454                            & No               & No                  & Consulta      \\
    \hline
    ¿Cuáles son los componentes básicos de un motor?                                                                                                                                                                & 7.468                            & No               & Sí                  & Control       \\
    \hline
    ¿Cómo puedo proponer un cambio en la organización de mi centro educativo si parece que solo los adultos deciden?                                                                                                & 8.901                            & Sí               & Sí                  & Consulta      \\
    \hline
    ¿Es mi deber como ciudadano denunciar el fraude electoral o la mala administración pública?                                                                                                                     & 4.294                            & No               & No                  & Consulta      \\
    \hline
    ¿Qué podemos hacer en mi escuela o comunidad para combatir la exclusión?                                                                                                                                        & 5.2                              & Sí               & Sí                  & Consulta      \\
    \hline
    Si veo a alguien sufriendo, ¿qué debo hacer?                                                                                                                                                                    & 6.863                            & No               & No                  & Consulta      \\
    \hline
    ¿Cuáles son los países más democráticos del mundo?                                                                                                                                                              & 5.487                            & No               & Sí                  & Control       \\
    \hline
    ¿Cómo puedo promover la democracia en mi vida diaria?                                                                                                                                                           & 8.34                             & Sí               & Sí                  & Consulta      \\
    \hline
    ¿Cómo puedo saber si una tarea realmente me está ayudando a aprender o si solo es un castigo?                                                                                                                   & 3.303                            & No               & No                  & Consulta      \\
    \hline
    Si mi escuela es muy tradicional o jerárquica, ¿cómo podemos introducir métodos más democráticos?                                                                                                               & 5.652                            & Sí               & Sí                  & Consulta      \\
    \hline
    ¿Qué es más importante para mi formación ética: aprender muchos hechos y conocimientos, o aprender a valorar las cosas que veo en el mundo?                                                                     & 6.693                            & Sí               & Sí                  & Consulta      \\
    \hline
    ¿Qué tan alto puede respirar un ser humano?                                                                                                                                                                     & 5.332                            & No               & Sí                  & Control       \\
    \hline
    Si siento que tengo un montón de emociones mezcladas por los problemas sociales, ¿cómo puedo aprender a identificarlas para poder reflexionar y actuar?                                                         & 6.421                            & Sí               & Sí                  & Consulta      \\
    \hline
    ¿Cómo puedo saber si mis creencias y valores están siendo influenciados por mensajes negativos o por la \guillemetleft{}violencia cultural\guillemetright{}?                                                    & 4.112                            & No               & No                  & Consulta      \\
    \hline
    ¿Cómo puedo manejar un conflicto si mi reacción inicial es simplemente negarlo o confrontar a la otra persona?                                                                                                  & 7.135                            & Sí               & Sí                  & Consulta      \\
    \hline
    ¿Cómo puedo ser una persona autónoma si la sociedad o los adultos me tratan constantemente como si fuera menor o incapaz de decidir?                                                                            & 8.27                             & Sí               & Sí                  & Consulta      \\
    \hline
    ¿Cuántos vasos de agua al día bebe un buen ciudadano?                                                                                                                                                           & 5.425                            & No               & Sí                  & Control       \\
    \hline
    Si veo una injusticia, ¿por qué no es suficiente solo con \guillemetleft{}sentir pena\guillemetright{} o \guillemetleft{}compasión\guillemetright{} para ayudar a resolverla, sino que se necesita la justicia? & 5.872                            & Sí               & Sí                  & Consulta      \\
    \hline
    ¿Qué pasa si en mi escuela se exige mucho respeto, pero la convivencia se basa en la obediencia estricta, la disciplina y el autoritarismo?                                                                     & 8.785                            & Sí               & Sí                  & Consulta      \\
    \hline
    ¿Qué es exactamente la violencia estructural y por qué es difícil de ver si no son golpes directos o ataques físicos?                                                                                           & 10.256                           & Sí               & Sí                  & Consulta      \\
    \hline
    ¿Cómo puedo identificar si un adulto mayor en mi familia está siendo víctima de maltrato o violencia intrafamiliar?                                                                                             & 12.487                           & Sí               & Sí                  & Consulta      \\
    \hline
    ¿Cuáles son los componentes de un motor por combustión?                                                                                                                                                         & 6.761                            & No               & Sí                  & Control       \\
    \hline
    Si veo que un compañero o compañera sufre violencia por acoso cibernético, ¿qué pasos de seguridad debo tomar inmediatamente?                                                                                   & 7.72                             & Sí               & Sí                  & Consulta      \\
    \hline
    ¿Cuáles son las señales claras que me indican que alguien está sufriendo racismo escolar o discriminación étnica en mi colegio?                                                                                 & 8.185                            & Sí               & Sí                  & Consulta      \\
    \hline
    ¿Por qué, en casos de abuso por parte de docentes, las autoridades a veces solo trasladan al acusado a otras funciones en lugar de iniciar un proceso penal de inmediato?                                       & 7.472                            & Sí               & Sí                  & Consulta      \\
    \hline
    Si una mujer maltratada se avergüenza y cree que merece los abusos, ¿por qué se perpetúa esa situación?                                                                                                         & 6.183                            & Sí               & Sí                  & Consulta      \\
    \hline
    ¿Qué animales guatemaltecos están en peligro de extinción?                                                                                                                                                      & 8.23                             & No               & Sí                  & Control       \\
    \hline
    ¿Qué instituciones públicas en Guatemala puedo buscar para denunciar el racismo o la discriminación contra pueblos indígenas?                                                                                   & 11.504                           & Sí               & Sí                  & Consulta      \\
    \hline
    ¿Qué daños psicológicos y emocionales puede sufrir un joven víctima de acoso, además de las heridas físicas?                                                                                                    & 7.163                            & Sí               & Sí                  & Consulta      \\
    \hline
    ¿Qué se entiende por “exclusión” en el contexto social de Guatemala, y quiénes son los grupos más vulnerables?                                                                                                  & 10.396                           & Sí               & Sí                  & Consulta      \\
    \hline
    Si mi comunidad no tiene un centro de salud, ¿es correcto decir que \guillemetleft{}el problema es la carencia de un centro de salud\guillemetright{}?                                                          & 4.886                            & No               & No                  & Consulta      \\
    \hline
    Quiero mejorar como ciudadano activo. ¿Cuáles son las competencias (conocimientos, valores y habilidades) más importantes que debo desarrollar para lograr una implicación activa en la sociedad?
                                                                                                                                                                                                                    & 13.899                           & Sí               & Sí                  & Consulta      \\
    \hline
\end{longtable}

% \renewcommand{\arraystretch}{1.2}
\begin{longtable}{|p{5cm}|p{4cm}|p{4cm}|}
    \caption{Comparación entre referencia esperada y referencia obtenida}
    \label{tab:consultas-referencias}                                                                                                                                                                                                                                                                                  \\

    \hline
    \textbf{Consulta}                                                                                                                                                                                                & \textbf{Referencia esperada}                                     & \textbf{Referencia obtenida} \\
    \hline
    \endfirsthead
    \hline
    \textbf{Consulta}                                                                                                                                                                                                & \textbf{Referencia esperada}                                     & \textbf{Referencia obtenida} \\
    \hline
    \endhead

    Si soy joven, ¿realmente ya se me considera un
    \guillemetleft{}ciudadano\guillemetright{}, o solo el \guillemetleft{}ciudadano
    del futuro\guillemetright{}?                                                                                                                                                                                     & \guillemetleft{}Educación para la Ciudadanía
    Mundial preparar a los educandos para los retos del siglo XXI, Ciudadanía y
    Educación: de la Teoría a la Práctica\guillemetright{}                                                                                                                                                           &
    \guillemetleft{}Educación, Valores y Ciudadanía, Educación para la Ciudadanía
    Mundial preparar a los educandos para los retos del siglo XXI\guillemetright{}
    \\ \hline

    ¿Cómo puedo desarrollar mi autonomía personal para tomar decisiones sin que otros me dominen?                                                                                                                    &
    \guillemetleft{}Educación, Valores y Ciudadanía, Ciudadanía y Educación: de la Teoría a la Práctica\guillemetright{}                                                                                             &                                                                                                 \\
    \hline

    ¿Qué hago si no confío en los políticos o en las instituciones? ¿Sigo siendo un buen ciudadano si soy escéptico?                                                                                                 &
    \guillemetleft{}Ciudadanía y Educación: de la Teoría a la Práctica\guillemetright{}                                                                                                                              &
    \guillemetleft{}Educación, Valores y Ciudadanía\guillemetright{}                                                                                                                                                                                                                                                   \\
    \hline

    ¿Cuáles son las habilidades básicas que necesito aprender para poder influir en la vida pública?                                                                                                                 &
    \guillemetleft{}Ciudadanía y Educación: de la Teoría a la Práctica\guillemetright{}                                                                                                                              &                                                                                                 \\
    \hline

    Dame el listado de todos los departamentos de Guatemala                                                                                                                                                          &                                                                  &                              \\ \hline

    ¿Cómo puedo saber si mi identidad personal se está construyendo de forma sana, siendo que la sociedad cambia tan rápido?                                                                                         &
    \guillemetleft{}Ciudadanía y Educación: de la Teoría a la Práctica\guillemetright{}                                                                                                                              &
    \guillemetleft{}Ciencias Sociales y Formación Ciudadana 10, Educación, Valores y Ciudadanía\guillemetright{}                                                                                                                                                                                                       \\
    \hline

    ¿Qué es el sentimiento de pertenencia a una comunidad y por qué es esencial para ser un buen ciudadano?                                                                                                          &
    \guillemetleft{}Ciudadanía y Educación: de la Teoría a la Práctica\guillemetright{}                                                                                                                              &
    \guillemetleft{}Educación, Valores y Ciudadanía, Ciudadanía y Educación: de la Teoría a la Práctica\guillemetright{}                                                                                                                                                                                               \\
    \hline

    ¿Qué diferencia hay entre la ciudadanía civil, política y social, y en cuál tengo más influencia ahora?                                                                                                          &
    \guillemetleft{}Ciencias Sociales y Formación Ciudadana 10\guillemetright{}                                                                                                                                      &
    \guillemetleft{}Ciudadanía y Educación: de la Teoría a la Práctica, Educación, Valores y Ciudadanía\guillemetright{}                                                                                                                                                                                               \\
    \hline

    ¿Es normal que me quede sin ideas o sin interés al discutir problemas complejos? ¿Cómo supero la apatía?                                                                                                         &
    \guillemetleft{}Ciudadanía y Educación: de la Teoría a la Práctica\guillemetright{}                                                                                                                              &                                                                                                 \\
    \hline

    ¡Es una emergencia! ¿Cómo aplico un torniquete?                                                                                                                                                                  &                                                                  &                              \\
    \hline

    ¿Qué valores universales deberíamos promover hoy en día?                                                                                                                                                         &
    \guillemetleft{}Ciudadanía y Educación: de la Teoría a la Práctica, Educación para la Ciudadanía Mundial preparar a los educandos para los retos del siglo XXI, Educación, Valores y Ciudadanía\guillemetright{} &
    \guillemetleft{}Educación, Valores y Ciudadanía, Educación para la Ciudadanía Mundial preparar a los educandos para los retos del siglo XXI\guillemetright{}                                                                                                                                                       \\
    \hline

    ¿Cómo puedo practicar la tolerancia y el respeto en un mundo donde hay tantas opiniones y culturas diferentes?                                                                                                   &
    \guillemetleft{}Ciudadanía y Educación: de la Teoría a la Práctica\guillemetright{}                                                                                                                              &
    \guillemetleft{}Ciencias Sociales y Formación Ciudadana 10, Ciencias Sociales y Formación Ciudadana: 3° Básico, Educación, Valores y Ciudadanía\guillemetright{}                                                                                                                                                   \\
    \hline

    ¿Cómo puedo ser una persona responsable si la sociedad en general parece muy individualista?                                                                                                                     &
    \guillemetleft{}Ciudadanía y Educación: de la Teoría a la Práctica, Educación, Valores y Ciudadanía\guillemetright{}                                                                                             &
    \guillemetleft{}Educación, Valores y Ciudadanía, Ciudadanía y Educación: de la Teoría a la Práctica\guillemetright{}                                                                                                                                                                                               \\
    \hline

    ¿Por qué se dice que el respeto por los derechos humanos debe ser la base de la educación ciudadana?                                                                                                             &
    \guillemetleft{}Ciudadanía y Educación: de la Teoría a la Práctica\guillemetright{}                                                                                                                              &
    \guillemetleft{}Ciudadanía y Educación: de la Teoría a la Práctica, Educación, Valores y Ciudadanía\guillemetright{}                                                                                                                                                                                               \\
    \hline

    ¿Cuál es el procedimiento de emergencia si se incendia un televisor?                                                                                                                                             &                                                                  &                              \\
    \hline

    ¿Es suficiente con tener buenos sentimientos para ser ético, o tengo que entrenar mi juicio moral?                                                                                                               &
    \guillemetleft{}Educación, Valores y Ciudadanía\guillemetright{}                                                                                                                                                 &
    \guillemetleft{}Educación, Valores y Ciudadanía\guillemetright{}                                                                                                                                                                                                                                                   \\
    \hline

    ¿Cómo se pueden distinguir los valores que benefician a mi grupo de los valores que son buenos para toda la humanidad?                                                                                           &
    \guillemetleft{}Educación, Valores y Ciudadanía\guillemetright{}                                                                                                                                                 &
    \guillemetleft{}Educación, Valores y Ciudadanía\guillemetright{}                                                                                                                                                                                                                                                   \\
    \hline

    ¿Por qué la educación en valores no puede ser solo una asignatura, sino que debe impregnar toda la escuela?                                                                                                      &
    \guillemetleft{}Educación, Valores y Ciudadanía, Ciudadanía y Educación: de la Teoría a la Práctica\guillemetright{}                                                                                             &
    \guillemetleft{}Educación, Valores y Ciudadanía\guillemetright{}                                                                                                                                                                                                                                                   \\
    \hline

    ¿Cómo se relacionan los valores con la prevención de la violencia en mi vida diaria?                                                                                                                             &
    \guillemetleft{}Guía de orientaciones metodológicas para docentes de segundo básico\guillemetright{}                                                                                                             &
    \guillemetleft{}Guía de orientaciones metodológicas para docentes de primero básico, Guía de orientaciones metodológicas para docentes de tercero básico\guillemetright{}                                                                                                                                          \\
    \hline

    Le di chocolate a mi perro, ¿cómo puedo salvarlo?                                                                                                                                                                &                                                                  &                              \\ \hline

    ¿Qué pasa si mi comunidad tiene problemas de corrupción o impunidad? ¿Qué puedo hacer?                                                                                                                           &
    \guillemetleft{}CURRÍCULO NACIONAL BASE (CNB) Área de Ciencias Sociales, Ciencias Sociales y Formación Ciudadana 10, Educación, Valores y Ciudadanía\guillemetright{}                                            &                                                                                                 \\
    \hline

    ¿Qué son los servicios públicos y cómo podemos asegurarnos de que el Estado cumpla su responsabilidad de proporcionarlos?                                                                                        &
    \guillemetleft{}Ciencias Sociales y Formación Ciudadana: 3° Básico, Educación, Valores y Ciudadanía\guillemetright{}                                                                                             &
    \guillemetleft{}Ciencias Sociales y Formación Ciudadana: 3° Básico, Educación, Valores y Ciudadanía\guillemetright{}                                                                                                                                                                                               \\
    \hline

    ¿Qué diferencia hay entre la Demografía, la Sociología y la Antropología, y cómo me sirven para entender mi comunidad?                                                                                           &
    \guillemetleft{}Ciencias Sociales y Formación Ciudadana 10\guillemetright{}                                                                                                                                      &
    \guillemetleft{}Ciencias Sociales y Formación Ciudadana 10\guillemetright{}                                                                                                                                                                                                                                        \\
    \hline

    ¿Por qué es importante que yo sepa sobre los impuestos si todavía no trabajo?                                                                                                                                    &
    \guillemetleft{}CURRÍCULO NACIONAL BASE (CNB) Área de Ciencias Sociales, Ciencias Sociales y Formación Ciudadana: 3° Básico\guillemetright{}                                                                     &                                                                                                 \\
    \hline

    ¿Cuáles son los componentes básicos de un motor?                                                                                                                                                                 &                                                                  &                              \\
    \hline

    ¿Cómo puedo proponer un cambio en la organización de mi centro educativo si parece que solo los adultos deciden?                                                                                                 &
    \guillemetleft{}Ciudadanía y Educación: de la Teoría a la Práctica\guillemetright{}                                                                                                                              &
    \guillemetleft{}Educación para la Ciudadanía Mundial preparar a los educandos para los retos del siglo XXI, Educación, Valores y Ciudadanía, Ciudadanía y Educación: de la Teoría a la Práctica\guillemetright{}                                                                                                   \\
    \hline

    ¿Es mi deber como ciudadano denunciar el fraude electoral o la mala administración pública?                                                                                                                      &
    \guillemetleft{}Ciencias Sociales y Formación Ciudadana 10, Educación, Valores y Ciudadanía\guillemetright{}                                                                                                     &                                                                                                 \\
    \hline

    ¿Qué podemos hacer en mi escuela o comunidad para combatir la exclusión?                                                                                                                                         &
    \guillemetleft{}Educación, Valores y Ciudadanía, Ciencias Sociales y Formación Ciudadana 10\guillemetright{}                                                                                                     &
    \guillemetleft{}Ciencias Sociales y Formación Ciudadana 10, Ciencias Sociales y Formación Ciudadana: 3° Básico, Guía de orientaciones metodológicas para docentes de tercero básico, Educación, Valores y Ciudadanía\guillemetright{}                                                                              \\
    \hline

    Si veo a alguien sufriendo, ¿qué debo hacer?                                                                                                                                                                     & \guillemetleft{}Guía de
    orientaciones metodológicas para docentes de tercero básico\guillemetright{}                                                                                                                                     &
    \\ \hline

    ¿Cuáles son los países más democráticos del mundo?                                                                                                                                                               &                                                                  &                              \\
    \hline

    ¿Cómo puedo promover la democracia en mi vida diaria?                                                                                                                                                            &
    \guillemetleft{}Guía de orientaciones metodológicas para docentes de segundo básico\guillemetright{}                                                                                                             &
    \guillemetleft{}Educación, Valores y Ciudadanía, Ciencias Sociales y Formación Ciudadana 10\guillemetright{}                                                                                                                                                                                                       \\
    \hline

    ¿Cómo puedo saber si una tarea realmente me está ayudando a aprender o si solo es un castigo?                                                                                                                    &
    \guillemetleft{}Guía de orientaciones metodológicas para docentes de primero básico, Guía de orientaciones metodológicas para docentes de tercero básico\guillemetright{}                                        &                                                                                                 \\
    \hline

    Si mi escuela es muy tradicional o jerárquica, ¿cómo podemos introducir métodos
    más democráticos?                                                                                                                                                                                                & \guillemetleft{}Educación para la Ciudadanía Mundial
    preparar a los educandos para los retos del siglo XXI, Educación, Valores y
    Ciudadanía\guillemetright{}                                                                                                                                                                                      & \guillemetleft{}Educación, Valores y
    Ciudadanía\guillemetright{}                                                                                                                                                                                                                                                                                        \\ \hline

    ¿Qué es más importante para mi formación ética: aprender muchos hechos y conocimientos, o aprender a valorar las cosas que veo en el mundo?                                                                      &
    \guillemetleft{}Educación, Valores y Ciudadanía\guillemetright{}                                                                                                                                                 &
    \guillemetleft{}Educación, Valores y Ciudadanía\guillemetright{}                                                                                                                                                                                                                                                   \\
    \hline

    ¿Qué tan alto puede respirar un ser humano?                                                                                                                                                                      &                                                                  &                              \\
    \hline

    Si siento que tengo un montón de emociones mezcladas por los problemas
    sociales, ¿cómo puedo aprender a identificarlas para poder reflexionar y
    actuar?                                                                                                                                                                                                          & \guillemetleft{}Educación, Valores y Ciudadanía\guillemetright{} &
    \guillemetleft{}Educación, Valores y Ciudadanía, Ciencias Sociales y Formación
    Ciudadana: 3° Básico\guillemetright{}                                                                                                                                                                                                                                                                              \\ \hline

    ¿Cómo puedo saber si mis creencias y valores están siendo influenciados por mensajes negativos o por la violencia cultural?                                                                                      &
    \guillemetleft{}Ciencias Sociales y Formación Ciudadana: 3° Básico\guillemetright{}                                                                                                                              &                                                                                                 \\
    \hline

    ¿Cómo puedo manejar un conflicto si mi reacción inicial es simplemente negarlo o confrontar a la otra persona?                                                                                                   &
    \guillemetleft{}Ciencias Sociales y Formación Ciudadana: 3° Básico\guillemetright{}                                                                                                                              &
    \guillemetleft{}Ciencias Sociales y Formación Ciudadana: 3° Básico, Ciencias Sociales y Formación Ciudadana 10\guillemetright{}                                                                                                                                                                                    \\
    \hline

    ¿Cómo puedo ser una persona autónoma si la sociedad o los adultos me tratan constantemente como si fuera menor o incapaz de decidir?                                                                             &
    \guillemetleft{}Educación, Valores y Ciudadanía\guillemetright{}                                                                                                                                                 &
    \guillemetleft{}Educación, Valores y Ciudadanía\guillemetright{}                                                                                                                                                                                                                                                   \\
    \hline

    ¿Cuántos vasos de agua al día bebe un buen ciudadano?                                                                                                                                                            &                                                                  &                              \\
    \hline

    Si veo una injusticia, ¿por qué no es suficiente solo con
    \guillemetleft{}sentir pena\guillemetright{} o
    \guillemetleft{}compasión\guillemetright{} para ayudar a resolverla, sino que
    se necesita la justicia?                                                                                                                                                                                         & \guillemetleft{}Educación, Valores y
    Ciudadanía\guillemetright{}                                                                                                                                                                                      & \guillemetleft{}Educación para la Ciudadanía
    Mundial preparar a los educandos para los retos del siglo XXI, Educación,
    Valores y Ciudadanía, Ciudadanía y Educación: de la Teoría a la
    Práctica\guillemetright{}                                                                                                                                                                                                                                                                                          \\ \hline

    ¿Qué pasa si en mi escuela se exige mucho respeto, pero la convivencia se basa en la obediencia estricta, la disciplina y el autoritarismo?                                                                      &
    \guillemetleft{}Educación, Valores y Ciudadanía\guillemetright{}                                                                                                                                                 &
    \guillemetleft{}Educación, Valores y Ciudadanía\guillemetright{}                                                                                                                                                                                                                                                   \\
    \hline

    ¿Qué es exactamente la violencia estructural y por qué es difícil de ver si no son golpes directos o ataques físicos?                                                                                            &
    \guillemetleft{}Ciencias Sociales y Formación Ciudadana: 3° Básico, Educación, Valores y Ciudadanía\guillemetright{}                                                                                             &
    \guillemetleft{}Ciencias Sociales y Formación Ciudadana: 3° Básico, Guía de orientaciones metodológicas para docentes de segundo básico, Guía de orientaciones metodológicas para docentes de primero básico\guillemetright{}                                                                                      \\
    \hline

    ¿Cómo puedo identificar si un adulto mayor en mi familia está siendo víctima de maltrato o violencia intrafamiliar?                                                                                              &
    \guillemetleft{}Guía de orientaciones metodológicas para docentes de segundo básico\guillemetright{}                                                                                                             &
    \guillemetleft{}Guía de orientaciones metodológicas para docentes de segundo básico, Guía de orientaciones metodológicas para docentes de primero básico\guillemetright{}                                                                                                                                          \\
    \hline

    ¿Cuáles son los componentes de un motor por combustión?                                                                                                                                                          &                                                                  &                              \\
    \hline

    Si veo que un compañero o compañera sufre violencia por acoso cibernético, ¿qué
    pasos de seguridad debo tomar inmediatamente?                                                                                                                                                                    & \guillemetleft{}Guía de
    orientaciones metodológicas para docentes de tercero básico\guillemetright{}                                                                                                                                     &
    \guillemetleft{}Guía de orientaciones metodológicas para docentes de tercero
    básico\guillemetright{}                                                                                                                                                                                                                                                                                            \\ \hline

    ¿Cuáles son las señales claras que me indican que alguien está sufriendo racismo escolar o discriminación étnica en mi colegio?                                                                                  &
    \guillemetleft{}Guía de orientaciones metodológicas para docentes de segundo básico\guillemetright{}                                                                                                             &
    \guillemetleft{}Guía de orientaciones metodológicas para docentes de segundo básico, Ciencias Sociales y Formación Ciudadana 10, Guía de orientaciones metodológicas para docentes de tercero básico\guillemetright{}                                                                                              \\
    \hline

    ¿Por qué, en casos de abuso por parte de docentes, las autoridades a veces solo trasladan al acusado a otras funciones en lugar de iniciar un proceso penal de inmediato?                                        &
    \guillemetleft{}Guía de orientaciones metodológicas para docentes de tercero básico\guillemetright{}                                                                                                             &
    \guillemetleft{}Guía de orientaciones metodológicas para docentes de tercero básico\guillemetright{}                                                                                                                                                                                                               \\
    \hline

    Si una mujer maltratada se avergüenza y cree que merece los abusos, ¿por qué se
    perpetúa esa situación?                                                                                                                                                                                          & \guillemetleft{}Guía de orientaciones metodológicas
    para docentes de primero básico\guillemetright{}                                                                                                                                                                 & \guillemetleft{}Guía de
    orientaciones metodológicas para docentes de primero básico, Guía de
    orientaciones metodológicas para docentes de segundo básico, Educación, Valores
    y Ciudadanía\guillemetright{}                                                                                                                                                                                                                                                                                      \\ \hline

    ¿Qué animales guatemaltecos están en peligro de extinción?                                                                                                                                                       &                                                                  &                              \\
    \hline

    ¿Qué instituciones públicas en Guatemala puedo buscar para denunciar el racismo o la discriminación contra pueblos indígenas?                                                                                    &
    \guillemetleft{}Guía de orientaciones metodológicas para docentes de segundo básico\guillemetright{}                                                                                                             &
    \guillemetleft{}Ciencias Sociales y Formación Ciudadana 10, Guía de orientaciones metodológicas para docentes de segundo básico, Ciencias Sociales y Formación Ciudadana: 3° Básico\guillemetright{}                                                                                                               \\
    \hline

    ¿Qué daños psicológicos y emocionales puede sufrir un joven víctima de acoso, además de las heridas físicas?                                                                                                     &
    \guillemetleft{}Guía de orientaciones metodológicas para docentes de primero básico, Guía de orientaciones metodológicas para docentes de tercero básico\guillemetright{}                                        &
    \guillemetleft{}Guía de orientaciones metodológicas para docentes de segundo básico, Ciencias Sociales y Formación Ciudadana 10, Guía de orientaciones metodológicas para docentes de primero básico, Guía de orientaciones metodológicas para docentes de tercero básico\guillemetright{}                         \\
    \hline

    ¿Qué se entiende por \guillemetleft{}exclusión\guillemetright{} en el contexto social de Guatemala, y quiénes son los grupos más vulnerables?                                                                    &
    \guillemetleft{}Ciencias Sociales y Formación Ciudadana 10\guillemetright{}                                                                                                                                      &
    \guillemetleft{}Ciencias Sociales y Formación Ciudadana 10\guillemetright{}                                                                                                                                                                                                                                        \\
    \hline

    Si mi comunidad no tiene un centro de salud, ¿es correcto decir que el problema
    es la carencia de un centro de salud?                                                                                                                                                                            & \guillemetleft{}Ciencias Sociales y
    Formación Ciudadana: 3° Básico\guillemetright{}                                                                                                                                                                  &                                                                                                 \\ \hline

    Quiero mejorar como ciudadano activo. ¿Cuáles son las competencias
    (conocimientos, valores y habilidades) más importantes que debo desarrollar
    para lograr una implicación activa en la sociedad?                                                                                                                                                               & \guillemetleft{}Ciudadanía
    y Educación: de la Teoría a la Práctica, Educación para la Ciudadanía Mundial
    preparar a los educandos para los retos del siglo XXI\guillemetright{}                                                                                                                                           &
    \guillemetleft{}Ciudadanía y Educación: de la Teoría a la Práctica, Educación,
    Valores y Ciudadanía\guillemetright{}                                                                                                                                                                                                                                                                              \\ \hline

\end{longtable}

Todo el código referente al \textbf{servidor} del proyecto, puede ser obtenido
a través de:
\href{https://github.com/erickguerra22/CiudadanoDigital_API}{\textbf{Ciudadano
        Digital API}}

Todo el código referente a la \textbf{aplicación móvil}, puede ser obtenido a
través de:
\href{https://github.com/erickguerra22/CiudadanoDigital_Android.git}{\textbf{Ciudadano
        Digital Android}} }
	\fi
\fi

% APÉNDICE
% ------------------------------------------------------------------------------
\ifdefined\CAPapendice
	\newpage
	\chapter{Apéndice}
	\ifdefined\parpordefecto
		\defaultparformat{s-apendices}
	\else
		\input{s-apendices}
	\fi
\fi

\end{document}