La formación en valores y ciudadanía constituye un componente fundamental en la construcción de sociedades democráticas, inclusivas y participativas. A través de ella, los estudiantes desarrollan competencias cívicas como la empatía, la responsabilidad social, el respeto a la diversidad y el compromiso con el bien común. Diversos organismos internacionales, como la UNESCO y la OCDE, han destacado la importancia de reforzar estos aprendizajes en contextos educativos cada vez más desafiantes, marcados por tensiones sociales, crisis ambientales y transformaciones digitales profundas \cite{unesco2021ethics,oecd2021skills}. No obstante, en muchos países de América Latina, esta dimensión formativa continúa siendo relegada frente a enfoques centrados únicamente en resultados académicos medibles \cite{worldbank2022revolution,rivas2023future}.

En el caso de Guatemala, si bien el Currículo Nacional Base reconoce la educación ciudadana como un eje transversal, su aplicación efectiva enfrenta múltiples obstáculos; como la falta de metodologías activas, el uso limitado de recursos digitales y la escasa formación docente en enfoques críticos y reflexivos. Estas condiciones dificultan que los estudiantes puedan vincular los contenidos cívicos con sus experiencias cotidianas o desarrollar una comprensión profunda de su papel como agentes de cambio en sus comunidades \cite{mineduc2020cnb,cien2019diagnostico}. A ello se suma una brecha tecnológica significativa entre zonas urbanas y rurales, lo cual limita las oportunidades para introducir enfoques innovadores que promuevan aprendizajes significativos en valores y ciudadanía \cite{unesco2023monitoring,levy2025teachers}.

Ante este escenario, la inteligencia artificial (IA) se presenta como una herramienta con potencial transformador en el ámbito educativo, particularmente cuando se orienta hacia el fortalecimiento de habilidades humanas y no únicamente hacia la automatización de contenidos. La UNESCO ha enfatizado que, para que estas tecnologías contribuyan a sistemas educativos más justos y democráticos, deben diseñarse bajo principios de equidad, inclusión y supervisión humana, evitando reproducir sesgos o exclusiones preexistentes \cite{unesco2021ethics,unesco2021guidance}. Aplicada con criterio ético y pedagógico, la IA puede ser utilizada como un recurso para ampliar el acceso a materiales formativos, personalizar la experiencia de aprendizaje y acompañar procesos de reflexión moral desde una lógica de diálogo \cite{frontiers2024chatgpt,tulsiani2024chatgpt}.

En base a todo lo expuesto, la solución propuesta consiste en un sistema basado en un modelo |de lenguaje (LLM) apoyado en una base de datos vectorial que almacena documentos éticos, pedagógicos y contextuales mediante técnicas de embedding. Este tutor virtual, accesible desde dispositivos de bajo costo y alta accesibilidad, genera respuestas personalizadas y fundamentadas que acompañan el aprendizaje informal en valores y ciudadanía, estimulando la reflexión crítica y la toma de decisiones éticas. Más que sustituir al docente, este sistema busca potenciar la iniciativa del estudiante, ofreciendo un recurso inclusivo que aporte un acompañamiento directo en su día a día, diseñado en base a material validado por profesionales expertos para asegurar un enfoque humanizado y contextualizado a las realidades actuales del país \cite{unesco2021ethics,worldbank2022revolution,rivas2023future}.