La formación en valores y ciudadanía constituye un componente esencial en la
construcción de sociedades democráticas, inclusivas y participativas. A través
de ella, los estudiantes desarrollan competencias cívicas como la empatía, la
responsabilidad social, el respeto a la diversidad y el compromiso con el bien
común. Organismos internacionales como la UNESCO (Organización de las Naciones
Unidas para la Educación, la Ciencia y la Cultura) y la OCDE (Organización para
la Cooperación y el Desarrollo Económicos) han destacado la necesidad de
fortalecer estos aprendizajes en un contexto global marcado por tensiones
sociales, crisis ambientales y transformaciones digitales profundas
\cite{unesco2021ethics,oecd2021skills}. Sin embargo, en muchos países de
América Latina esta dimensión formativa continúa siendo relegada frente a
enfoques centrados exclusivamente en resultados académicos cuantificables
\cite{worldbank2022revolution,rivas2023future}.

En Guatemala, el Currículo Nacional Base (CNB) reconoce la educación ciudadana
como un eje transversal, pero su aplicación efectiva enfrenta múltiples
desafíos: la escasa formación docente en metodologías críticas, la ausencia de
recursos digitales adaptados al contexto local y la persistencia de enfoques
tradicionales centrados en la memorización. Estas limitaciones dificultan que
los estudiantes logren vincular los contenidos cívicos con su vida cotidiana o
desarrollar una comprensión profunda de su papel como agentes de cambio
\cite{mineduc2020cnb,cien2019diagnostico}. A ello se suma una brecha
tecnológica significativa entre zonas urbanas y rurales, que restringe el
acceso equitativo a experiencias de aprendizaje innovadoras y limita las
oportunidades para fomentar la reflexión ética y la participación ciudadana
\cite{unesco2023monitoring,levy2025teachers}.

En este panorama, la inteligencia artificial (IA) emerge como una herramienta
con gran potencial transformador para la educación, especialmente cuando se
orienta hacia el fortalecimiento de habilidades humanas y el acompañamiento
moral, más que hacia la mera automatización de contenidos. La UNESCO subraya
que, para que la IA contribuya al desarrollo de sistemas educativos más justos
y democráticos, debe diseñarse bajo principios de equidad, inclusión y
supervisión humana, evitando reproducir sesgos o exclusiones
\cite{unesco2021ethics,unesco2021guidance}. Aplicada con criterios éticos y
pedagógicos, la IA puede servir como un medio para ampliar el acceso a
materiales formativos, personalizar experiencias de aprendizaje y acompañar
procesos de reflexión moral mediante un diálogo guiado y contextualizado
\cite{frontiers2024chatgpt,tulsiani2024chatgpt}.

En los últimos años, el avance de los modelos de lenguaje de gran escala (LLMs)
ha impulsado el desarrollo de asistentes conversacionales capaces de generar
tutorías personalizadas, ofrecer retroalimentación inmediata y adaptarse al
ritmo de cada estudiante \cite{elstad2024ai,frontiers2025education}. Estas
tecnologías abren la posibilidad de diseñar espacios de aprendizaje informal
donde los jóvenes puedan explorar dilemas morales, reflexionar sobre valores y
fortalecer su pensamiento crítico mediante la interacción con un sistema
empático y culturalmente pertinente.

Desde esta perspectiva, el proyecto \textit{Ciudadano Digital} busca integrar la
inteligencia artificial en la formación ciudadana y moral a través de una
aplicación accesible, diseñada para acompañar el aprendizaje ético de los
jóvenes en entornos digitales. El sistema, basado en un modelo de lenguaje
conectado a una base de datos vectorial con materiales educativos, éticos y
contextuales, genera respuestas personalizadas y fundamentadas que orientan la
reflexión. Su propósito no es sustituir al docente, sino complementar su labor
mediante un acompañamiento continuo que promueva la autonomía moral, la empatía
y la responsabilidad social. Además, su diseño prioriza la accesibilidad y la
escalabilidad, de modo que pueda funcionar eficazmente en dispositivos de bajo
costo y en entornos con recursos limitados. En conjunto, el proyecto pretende
fortalecer la educación cívica guatemalteca mediante el uso ético y
contextualizado de la IA, contribuyendo a una educación más inclusiva,
reflexiva y humanista en la era digital
\cite{unesco2021ethics,worldbank2022revolution,rivas2023future}.