El desarrollo del proyecto \textit{Ciudadano Digital} permitió obtener un
prototipo funcional de asistente virtual orientado a la educación ciudadana y
en valores morales, lo que evidencia la viabilidad técnica y conceptual de la
propuesta inicial, así como el desarrollo de una arquitectura tecnológica y un
flujo de trabajo adecuados para lograr el alcance de los objetivos planteados.
La integración de un mecanismo de recuperación aumentada de generación (RAG)
con modelos de lenguaje grande (LLM) demostró ser una estrategia efectiva para
ofrecer respuestas contextualizadas y fundamentadas en un \textit{corpus}
educativo que, por su parte, también cuenta con una metodología específica para
la selección y procesado del mismo.

Si bien los resultados obtenidos se ven limitados a un entorno de validación
externa, puesto que no se realizaron pruebas con usuarios finales, permiten
analizar el desempeño del sistema a partir de 3 perspectivas reconocibles:
técnica, semántica y visual.

Desde el punto de vista técnico, el sistema construido logró integrar
exitosamente las tecnologías planteadas desde el inicio del proyecto; se logró
la generación de un corpus dedicado a la fundamentación de las respuestas
obtenidas del modelo, esto a través de la base de datos vectorial
\textit{Pinecone} que, a su vez, se comunica correctamente con la base de datos
relacional y con el contenedor de archivos. Todo este sistema permite la
obtención apropiada de respuestas basadas únicamente en los documentos
seleccionados y procesados, garantizando que no exista alucinación por parte
del modelo y que este se limite a responder únicamente las cuestiones que estén
relacionadas con el contenido de la base de datos vectorial.

Como se evidencia en la Figura \ref{fig:flujo-rag}, el flujo de recuperación
aumentada de generación (RAG) se ejecuta de manera sistemática al dividir
correctamente las tareas entre los distintos componentes del sistema.

Sin embargo, si bien se cumplieron las expectativas técnicas de implementación,
curado y procesamiento de datos, se identifican claramente nuevas oportunidades
de mejora a abordar en iteraciones futuras del proyecto, tales como la
optimización del tiempo de respuesta, la mejora en la gestión de errores y la
incorporación de mecanismos de supervisión pedagógica que garanticen
contínuamente la validación de los documentos utilizados para alimentar el
modelo.

En el plano semántico; con una tasa de éxito del 77.5\%, acompañado de un 82\%
de congruencia en las respuestas obtenidas, el sistema demostró un desempeño
adecuado en la generación de respuestas fundamentadas y coherentes con el
material curado. No obstante, estos resultados también indican áreas de
oportunidad para mejorar la precisión y profundidad de las respuestas,
especialmente en temas complejos o ambigüos, así como en temáticas más
específicas a la realidad sociocultural y contextual del usuario que realiza la
petición.

Desde una perspectiva pedagógica, si bien el proyecto fue ideado no como una
herramienta de enseñanza directa, sino como un asistente secundario que
acompañe a los estudiantes en las inquietudes que los aquejen al salir del
salón de clases, se reconoce la importancia de evaluar el impacto real que este
tipo de tecnologías pueden tener en el aprendizaje y desarrollo ético de los
jóvenes. En este sentido, futuras iteraciones deberían centrarse en la
implementación de pruebas piloto con estudiantes y educadores, para medir no
solo la eficacia técnica del asistente, sino también su capacidad para fomentar
la reflexión crítica y el compromiso ciudadano.

Finalmente, desde una perspectiva visual, el diseño de la aplicación móvil fue
construido a partir de componentes y directrices de diseño centradas en la
experiencia de usuario usual de un propietario de dispositivos
\textit{Android}, pensado así para asegurar un mayor alcance entre la población
objetivo. No obstante, se identifican oportunidades de mejora en la interfaz
gráfica y en la experiencia de usuario, las cuales podrían optimizarse a través
de pruebas de usabilidad con usuarios reales, para garantizar que la
herramienta sea intuitiva, accesible y atractiva para los jóvenes. Cabe
resaltar que, al no realizarse contacto directo con el usuario objetivo en
ningún momento del desarrollo de esta fase, si bien se tomaron en cuenta buenas
prácticas de diseño y la información documentada, esto podría no reflejar de
manera fiel las necesidades y preferencias reales de los estudiantes
guatemaltecos que se pretende que utilicen la herramienta; valdría la pena
realizar estudios centrados en cómo los colores, tipografías, distribución de
elementos y flujos de navegación impactan en la experiencia del usuario final.

Resulta pertinente afirmar, entonces, que el proyecto \textit{Sapien - Ciudadano Digital} busca sentar las bases para la mejora continua del proyecto, enfocado en implementar la estructura tecnológica básica necesaria para la interacción con el modelo, la generación continua del \textit{corpus} y los criterios de diseño iniciales para continuar el desarrollo de la aplicación móvil. Se logró la implementación técnica esperada, tomando en cuenta las limitaciones relacionadas con la falta de retroalimentación directa y validación empírica con usuarios finales. Sin embargo, esta primera fase del proyecto permite demostrar cómo la integración de tecnologías de inteligencia artificial puede contribuir al fortalecimiento de la educación en valores y formación ciudadana en contextos con recursos limitados, lo que abre la puerta a futuras investigaciones y desarrollos en este campo.