Los resultados obtenidos durante el desarrollo del proyecto \textit{Ciudadano
    Digital} permiten analizar el alcance y las implicaciones de esta primera fase
de implementación. El prototipo cumplió con los objetivos técnicos
establecidos, demostrando la viabilidad de integrar un sistema conversacional
educativo basado en inteligencia artificial, aunque los resultados también
evidencian limitaciones propias de un producto en etapa inicial. En esta
sección se discuten los hallazgos más relevantes en relación con los objetivos,
el enfoque metodológico y los aspectos técnicos y pedagógicos del sistema.

\section{Desempeño técnico y coherencia del sistema}
Los resultados de las pruebas internas mostraron que la aplicación alcanzó un
nivel de estabilidad moderado, con un tiempo promedio de respuesta entre seis y
quinde segundos y una tasa de error cercana al 20\%. Esto confirma la correcta
integración de los componentes tecnológicos: aplicación móvil,
\textit{backend}, microservicio y bases de datos; lo cual constituye un avance
importante dentro de los objetivos específicos planteados.

No obstante, el comportamiento del sistema ante preguntas ambiguas o carentes
de contexto evidenció que el modelo aún depende en gran medida de la calidad
del \textit{corpus} documental y del proceso de recuperación semántica; en
varios casos, las respuestas resultaron genéricas o conservadoras, reflejando
los límites del enfoque RAG (\textit{Retrieval-Augmented Generation}) cuando el
contexto disponible es escaso o poco representativo. Este hallazgo resalta la
necesidad de ampliar el conjunto de datos y mejorar los mecanismos de
indexación para futuras iteraciones.

A pesar de estas limitaciones, los indicadores de desempeño sugieren que el
sistema es técnicamente funcional y adaptable, con una arquitectura sólida que
permite su expansión progresiva. El flujo de comunicación entre los módulos y
la estabilidad general del entorno confirman que el diseño propuesto es
adecuado para la escala de uso prevista en esta etapa.

\section{Pertinencia y calidad de las respuestas}
En el ámbito semántico, las pruebas internas y la validación experta revelaron
un desempeño consistente pero refinable. Las respuestas del asistente se
caracterizaron por mantener una estructura coherente, un tono respetuoso y una
orientación reflexiva, elementos fundamentales en un entorno educativo. Sin
embargo, los evaluadores señalaron que, en ciertos casos, el sistema ofrecía
respuestas demasiado breves o con escasa profundidad argumentativa,
especialmente en dilemas morales que requerían contextualización cultural o
análisis más matizado.

Estos resultados sugieren que el modelo logró asimilar los principios del
enfoque socrático al plantear reflexiones y contra-preguntas, pero todavía no
alcanza una interacción plenamente adaptativa al perfil del usuario. La calidad
del diálogo depende del tipo de pregunta, la extensión del contexto y el grado
de ambigüedad del tema tratado.

La valoración promedio de los expertos, situada entre 3.5 y 4.0 sobre 5 puntos
en los distintos criterios, refleja un desempeño favorable pero aún limitado en
términos pedagógicos. En este punto, la discusión se centra en la tensión entre
la coherencia formal de la respuesta y su profundidad conceptual, un equilibrio
que será clave en las siguientes fases del proyecto.

\section{Aspectos pedagógicos y éticos}
Uno de los hallazgos más relevantes se relaciona con la dimensión pedagógica
del sistema. Aunque no se realizaron pruebas con usuarios reales, las
observaciones de los expertos permiten inferir que \textit{Ciudadano Digital}
mantiene una coherencia ética adecuada y un enfoque comunicativo respetuoso,
evitando juicios morales o afirmaciones dogmáticas. Este comportamiento se
alinea con el propósito educativo del proyecto, orientado a fomentar la
reflexión personal y el razonamiento moral autónomo.

Sin embargo, la falta de validación con estudiantes o docentes impide
determinar con precisión el nivel de comprensión, motivación o impacto
educativo real del sistema; la percepción de los evaluadores indica que el
asistente podría ser un recurso de apoyo valioso, pero que su efectividad
dependerá de la mediación pedagógica y del diseño de estrategias de uso en
contextos reales de aprendizaje.

De esta manera, los resultados invitan a reflexionar sobre el papel que pueden
desempeñar los sistemas de inteligencia artificial en la educación moral: no
como sustitutos del docente, sino como espacios complementarios de diálogo y
reflexión guiada.

\section{Limitaciones y observaciones derivadas de la validación}
El análisis de los resultados debe situarse dentro de los límites de esta fase.
La ausencia de pruebas empíricas con público objetivo es la principal
restricción, ya que impide evaluar indicadores de aprendizaje, aceptación o
eficacia pedagógica. Por otro lado, el \textit{corpus} documental utilizado,
aunque cuidadosamente seleccionado, abarca una muestra limitada de temas dentro
de la educación ciudadana, lo cual restringe el alcance temático de las
respuestas.

Las pruebas internas también evidenciaron áreas de mejora técnica, entre ellas
la optimización del manejo de errores, la mejora en la gestión de solicitudes
simultáneas y la ampliación del contexto recuperado por el índice vectorial. A
nivel funcional, la interfaz móvil resultó adecuada para las pruebas internas,
pero requiere de ajustes en accesibilidad y retroalimentación interactiva para
un uso prolongado o masivo.

Estas limitaciones no demeritan los avances alcanzados, sino que orientan el
rumbo de los siguientes ciclos de desarrollo, reafirmando el carácter
exploratorio y progresivo del proyecto.

\section{Consideraciones sobre la metodología SCRUM}
El uso de la metodología SCRUM favoreció una organización estructurada del
proceso de desarrollo. Dividir el trabajo en \textit{sprints} permitió
identificar prioridades, corregir errores tempranos y mantener una visión
incremental del producto. Los resultados obtenidos muestran que este enfoque
ágil contribuyó a equilibrar los aspectos técnicos y conceptuales, logrando
avances tangibles en tiempos relativamente cortos.

La naturaleza iterativa de SCRUM también facilitó la integración de la
retroalimentación experta en cada etapa, lo que enriqueció el diseño del
sistema y permitió ajustes progresivos en la estructura y en la lógica de
interacción. Sin embargo, la dependencia de tiempos cortos y entregas continuas
exigió una planificación rigurosa para evitar la dispersión de tareas o la
sobrecarga técnica hacia el final del proceso.

En conjunto, la metodología permitió mantener una visión realista y controlada
del desarrollo, priorizando la funcionalidad y la validación antes que la
expansión prematura del sistema.


\noindent\rule{\linewidth}{0.5pt}

En síntesis, los resultados obtenidos permiten sostener que \textit{Ciudadano
    Digital} ha superado con éxito su fase de validación técnica, alcanzando un
nivel funcional estable y coherente con sus propósitos educativos. A pesar de
ello, los hallazgos también evidencian desafíos pendientes en la ampliación del
\textit{corpus}, la profundización semántica de las respuestas y la evaluación
del impacto pedagógico real. El sistema muestra un equilibrio prometedor entre
viabilidad tecnológica y orientación ética, aunque aún requiere ajustes para
consolidarse como herramienta formativa de alcance práctico.