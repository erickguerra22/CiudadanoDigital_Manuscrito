El desarrollo del proyecto \textit{Ciudadano Digital} permitió obtener un
prototipo funcional de un asistente virtual orientado a la educación ciudadana
y en valores morales, lo que evidencia la viabilidad técnica y conceptual de la
propuesta inicial, así como el desarrollo de una arquitectura tecnológica y un
flujo de trabajo adecuados para lograr el alcance de los objetivos planteados.
La integración de un mecanismo de recuperación aumentada de generación (RAG,
por sus siglas en inglés) con modelos de lenguaje de gran escala (LLM, por sus
siglas en inglés) demostró ser una estrategia efectiva para ofrecer respuestas
contextualizadas y fundamentadas en un corpus educativo que, por su parte,
también cuenta con una metodología específica para la selección y procesado del
mismo.

Como se mencionó al inicio de este documento, el desarrollo de
\textit{Ciudadano Digital} contempla la base técnica que busca demostrar la
viabilidad tecnológica de las herramientas para conformar un sistema de
generación aumentada por recuperación (RAG) efectivo. Para este punto, como se
mencionó en la justificación, se optó por centrar el funcionamiento de la
aplicación para su disponibilidad en dispositivos bajo el sistema operativo
Android. Esta decisiónresponde a criterios de accesibilidady equidad
tecnológica, puesto que los dispositivos bajo este sistema operativo
representan la mayoría del mercado móvil en regiones marginadas, por lo que se
alinea con el público objetivo identificado. El diseño implementado para la
aplicación enfatiza la legibilidad, los contrastes adecuados, los controles
táctiles amplios y los flujos de navegación sencillos, todo ello a través de
los principios dictados por el sistema de diseño de Google, \textit{Material
    Design}.

Por otro lado, la interacción pregunta-respuesta diseñada bajo la filosofía del
método socrático promueve no solamente la resolución de consultas, sino también
el uso de preguntas sugeridas que enriquezcan el conocimiento del usuario y lo
motiven a construir sus propias conclusiones a partir de la información
proporcionada.

Desde el punto de vista técnico, la implementación exitosa de la arquitectura
propuesta valida el propósito central del proyecto: la posibilidad de integrar
componentes de procesamiento de lenguaje natural, bases de datos vectoriales e
interfaces móviles para crear un asistente educativo funcional. Bajo la
elección de la metodología RAG, se permitió superar una limitación fundamental
de los LLMs convencionales: la propensión a la alucinación mediante la
contextualización estricta de las respuestas en el corpus educativo. El sistema
construido logró integrar exitosamente las tecnologías planteadas desde el
inicio del proyecto; se logró la generación de un corpus dedicado a la
fundamentación de las respuestas obtenidas del modelo, esto a través de la base
de datos vectorial Pinecone que, a su vez, se comunica correctamente con la
base de datos relacional y con el contenedor de archivos. Todo este sistema
permite la obtención apropiada de respuestas basadas únicamente en los
documentos seleccionados y procesados, garantizando que no exista alucinación
por parte del modelo y que este se limite a responder únicamente las cuestiones
que estén relacionadas con el contenido de la base de datos vectorial.

Regresando a la Figura \ref{fig:flujo-rag}, el flujo de generación aumentada
por reucperación (RAG) se ejecuta de manera sistemática al dividir
correctamente las tareas entre los distintos componentes del sistema. Sin
embargo, si bien se cumplieron las expectativas técnicas de implementación,
curado y procesamiento de datos, se identifican claramente nuevas oportunidades
de mejora a abordar en iteraciones futuras del proyecto, tales como la
optimización del tiempo de respuesta, la mejora en la gestión de errores y la
incorporación de mecanismos de supervisión pedagógica que garanticen
contínuamente la validación de los documentos utilizados para alimentar el
modelo.

En el plano semántico, el sistema logró una tasa de éxito del 77.78\% en
consultas válidas y un 100\% de efectividad en el rechazo de preguntas de
control (no relacionadas al tema de civismo). Esto evidencia que el mecanismo
de recuperación semántica funciona como barrera efectiva contra respuestas no
fundamentadas. Este resultado es crucial en contextos educativos, donde la
precisión fáctica y la confiabilidad de la información son requisitos
fundamentales. A su vez, la latencia promedio de \textbf{7.8184 segundos} señala un área
de optimización prioritaria; futuras iteraciones podrían explorar estrategias
como la precomputación de representaciones numéricas de texto
(\textit{embeddings}) para consultas frecuentes o la implementación de cachés
semánticos que detecten similitudes entre preguntas, de manera que se logre
reducir los tiempos de respuesta.

En comparación con mediciones independientes realizadas a modelos de referencia
como \textit{GPT-4-0125} y \textit{GPT-4-1106}, cuyos tiempos de respuesta promedio se sitúan
aproximadamente en \textbf{7.16s} y \textbf{6.36s} respectivamente, la latencia del sistema
desarrollado (7.8184 s) se mantiene dentro de un rango competitivo para
arquitecturas basadas en RAG \cite{openai_performance_analysis_2024}. Sin embargo, estas pruebas también revelan
diferencias relevantes en variabilidad: mientras que los modelos de GPT
reportan desviaciones estándar cercanas a \textbf{0.59s} y \textbf{1.45s} \cite{openai_performance_analysis_2024}, el comportamiento
del prototipo presenta fluctuaciones asociadas a su naturaleza distribuida, que
depende de la recuperación semántica, el acceso a la base vectorial y la
posterior generación de la respuesta. Esta comparación refuerza la importancia
de explorar mecanismos de optimización orientados a reducir la latencia global del sistema.

Adicionalmente, se obtuvo una congruencia fáctica del 81.81\%; es decir, una
mayor parte de las pruebas realizadas presentaron el comportamiento esperado.
Esta métrica fue obtenida a través de las preguntas de \textbf{consulta} y
preguntas de \textbf{control} (temas no pertinentes). En el caso de las
consultas, la congruencia resulta negativa cuando el modelo no es capaz de
responder, ya que se rompe el comportamiento esperado de encontrar el contexto
específico para contestar la pregunta. Por otro lado, en el caso de las
preguntas de control, la congruencia es positiva cuando el modelo efectivamente
se niega a responder, ya que se espera que identifique que esta petición está
fuera de su alcance y propósito.

Si bien se detectan áreas de mejora a abarcar en próximas fases del proyecto,
las métricas cuantitativas obtenidas por el sistema, permiten demostrar que
este alcanzó un desempeño adecuado en la generación de respuestas fundamentadas
y coherentes con el material curado.

Por otro lado, al analizar las respuestas generadas por el sistema y su
comparación con las fuentes extraídas y seleccionadas, se revelan patrones
significativos sobre el comportamiento del flujo RAG. La fidelidad efectiva del
82.86\% (considerando coincidencias exactas y parciales) indica que el sistema
logra, en la mayoría de los casos planteados, recuperar y utilizar información
relevante del corpus. Sin embargo, la distribución de los tipos de coincidencia
(31.43\% exacta, 51.43\% parcial y 17.14\% nula) merece una consideración
detallada y una iteración dedicada exhaustivamente a la optimización de la
calidad de respuestas obtenidas.

Cabe aclarar que la coincidencia exacta se define como aquella en la que la
consulta realizada al asistente recuperó información proveniente de las mismas
referencias esperadas. Los casos en los que se obtienen las mismas referencias
pero en distinto orden de prioridad pueden deberse a que la selección manual
inicial (realizada sin el acompañamiento de un experto en el dominio de
educación) no ponderó adecuadamente el peso conceptual de cada referencia. Por
ello, al obtener el contexto mediante Pinecone, la prioridad de las referencias
puede variar.

Por otro lado, los casos de coincidencia parcial evidencian que el modelo puede
utilizar referencias que no fueron consideradas durante la selección manual o
que algunas referencias incluidas en dicha selección (nuevamente, limitada por
la falta de un especialista en educación) no obtuvieron un puntaje suficiente
para ser seleccionadas como contexto por Pinecone. Esto explica por qué el
sistema eligió ciertos fragmentos distintos a los previstos.

Las coincidencias parciales, que representan más de la mitad de los casos,
sugieren que el sistema posee flexibilidad semántica para identificar
información relevante más allá de lo estrictamente anticipado. Este
comportamiento puede interpretarse positivamente como capacidad de inferencia
contextual, aunque también subraya la importancia de revisar la estrategia de
segmentación empleada, especialmente considerando que sin la guía de un experto
en educación podrían haberse definido fragmentos o criterios de segmentación
subóptimos. Esto debe tomarse en cuenta para mejorar futuras iteraciones del
sistema.

Por su parte, los casos de coincidencia nula (17.14\%), donde el sistema
responde correctamente pero con referencias diferentes a las esperadas, podrían
indicar redundancia temática en el corpus, o bien, limitaciones en el proceso
de recuperación. Iteracioes posteriores podrían dedicarse a la incorporación de
análisis de similitud semántica entre documentos, incluido en el flujo de
procesamiento, de manera que se pueda determinar si estas discrepancias
representan variaciones conceptuales legítimas o si dependen directamente de
estrategias de mejora en el indexado vectorial.

Desde una perspectiva pedagógica, si bien el proyecto fue ideado no como una
herramienta de enseñanza directa, sino como un asistente secundario que
acompañe a los estudiantes en las inquietudes que los aquejen al salir del
salón de clases, el proyecto demuestra el potencial de los asistentes basados
en IA para complementar los procesos de formación ciudadana. El enfoque
socrático implementado, evidenciado en respuestas que invitan a la reflexión
más que a la memorización, alinea con los principios de educación moral y con
las prácticas de tutoría inteligente.

No obstante, la tasa de éxito del 77.78\% también revela limitaciones
significativas: en las pruebas funcionales se identificaron 7 solicitudes
fallidas relacionadas con la conexión a internet, las cuales no se incluyeron
en el análisis porque el enfoque estaba orientado a la fiabilidad semántica de
las respuestas; sin embargo, el Anexo \ref{tab:consultas-resultados} muestra un
total de 10 consultas que no pudieron ser respondidas, equivalentes al 18.18\%
del total. Este hallazgo sugiere brechas en la cobertura temática del corpus
disponible al momento de las pruebas y, además, refuerza la importancia de
contar con un experto en el dominio educativo, cuya ausencia limitó la
definición de contenidos esenciales y la capacidad del sistema para manejar
consultas complejas o ambiguas.

Desde una perspectiva visual, el diseño móvil basado en \textit{Material
    Design} resultó adecuado para un contexto de recursos limitados, facilitando la
accesibilidad y la usabilidad. No obstante, se identifican oportunidades de
mejora en la interfaz gráfica y en la experiencia de usuario que podrían
optimizarse mediante pruebas de usabilidad con usuarios reales, a fin de
garantizar que la herramienta sea intuitiva, accesible y atractiva para los
jóvenes.

Es importante señalar que, debido a la ausencia de interacción directa con el
usuario objetivo durante esta fase, el diseño se fundamentó únicamente en
buenas prácticas y documentación disponible, lo cual podría no reflejar de
forma precisa las necesidades y preferencias reales de los estudiantes
guatemaltecos. Por ello, sería recomendable realizar estudios específicos sobre
el impacto de colores, tipografías, distribución de elementos y flujos de
navegación en la experiencia del usuario final. A su vez, resalta la
importancia de contar con un especialista en diseño de aplicaciones que asegure
un enfoque adecuado para priorizar tanto la usabilidad y apariencia visual
agradable, de manera que la aplicación cumpla con su propósito a la vez que
resulta amigable para el usuario objetivo

Resulta pertinente afirmar, entonces, que el proyecto \textit{Ciudadano
    Digital} busca sentar las bases para la mejora continua del proyecto, enfocado
en implementar la estructura tecnológica básica necesaria para la interacción
con el modelo, la generación continua del corpus y los criterios de diseño
iniciales para continuar el desarrollo de la aplicación móvil. Se logró la
implementación técnica esperada, tomando en cuenta las limitaciones
relacionadas con la falta de retroalimentación directa y validación empírica
con usuarios finales. Sin embargo, esta primera fase del proyecto permite
demostrar cómo la integración de tecnologías de inteligencia artificial puede
contribuir al fortalecimiento de la educación en valores y formación ciudadana
en contextos con recursos limitados, lo que abre la puerta a futuras
investigaciones y desarrollos en este campo.