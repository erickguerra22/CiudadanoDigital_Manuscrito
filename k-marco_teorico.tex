La relación entre inteligencia artificial (IA) y educación se ha convertido en
un campo de estudio emergente que combina aportes de la pedagogía, la
psicología del aprendizaje y las ciencias de la computación. Diversas
investigaciones han demostrado que los sistemas basados en IA pueden desempeñar
funciones de apoyo al proceso educativo, desde la automatización de tareas
administrativas hasta la personalización de la enseñanza mediante algoritmos de
aprendizaje adaptativo \cite{elstad2024ai,frontiers2025education}. Sin embargo,
más allá de sus aplicaciones instrumentales, la IA plantea un nuevo paradigma
pedagógico que redefine la forma en que se conciben la enseñanza, la
interacción docente-estudiante y la construcción del conocimiento en entornos
digitales.

En este contexto, la formación en valores y ciudadanía adquiere especial
relevancia. Aunque tradicionalmente se ha abordado desde marcos filosóficos y
éticos, su integración con tecnologías emergentes permite explorar nuevas
formas de aprendizaje moral mediadas por el diálogo y la reflexión guiada. La
incorporación de sistemas inteligentes en este ámbito representa tanto una
oportunidad como un desafío: por un lado, posibilita acompañamientos
personalizados que estimulan el pensamiento crítico y la autorregulación ética;
por otro, exige garantizar la responsabilidad, transparencia y confiabilidad de
los modelos utilizados \cite{betterinternet2024,carter2024ethics}.

Desde esta convergencia entre tecnología y formación ética, la literatura
reciente destaca el potencial de los modelos grandes de lenguaje (LLMs, por sus
siglas en inglés) para generar entornos conversacionales que promuevan la
reflexión moral y la toma de decisiones fundamentadas
\cite{seibt2024llm,frontiers2025psychology}. Estos sistemas, diseñados bajo
principios éticos y pedagógicos, pueden convertirse en agentes de
acompañamiento educativo informal, capaces de sostener diálogos significativos
y culturalmente pertinentes. De esta manera, la IA no solo actúa como
herramienta tecnológica, sino como mediadora cognitiva y moral, lo que amplia
las posibilidades de aprendizaje y fortalece el desarrollo ciudadano en la era
digital.

\section{Educación ciudadana y valores}
La educación ciudadana constituye un proceso educativo integral orientado a
formar individuos capaces de ejercer sus derechos y deberes de manera
responsable, ética y crítica. Esta formación no se limita al conocimiento de
normas y leyes, sino que promueve valores como la solidaridad, la justicia y el
respeto por la diversidad, esenciales para la convivencia democrática
\cite{unesco2021global, schulz2010iccs}. Además, la educación ciudadana
incorpora competencias sociales y habilidades de pensamiento crítico,
fomentando la participación activa en la comunidad y la toma de decisiones
informadas \cite{bentley2018education}.

\subsection{Educación en valores}
La educación en valores constituye un enfoque pedagógico reconocido a nivel
internacional bajo diversas denominaciones, como educación moral, educación del
carácter o educación ética. Si bien cada una presenta matices particulares y
distintos énfasis, todas comparten la convicción fundamental de que la
formación en valores personales y cívicos representa una responsabilidad
legítima de las instituciones educativas a nivel mundial. En la actualidad,
este ámbito ya no se considera exclusivo del entorno familiar o religioso, pues
diversas investigaciones han evidenciado que una educación desvinculada de los
valores puede limitar de forma significativa el desarrollo integral del
estudiante, tanto en el plano ético como en el académico
\cite{lovat2009values}.

Asimismo, la educación en valores se concibe como un proceso formativo integral
que no solo promueve principios fundamentales de ética y ciudadanía, sino que
se posiciona como un componente esencial y transversal de la calidad educativa.
Lejos de tratarse de un aspecto aislado, establece una relación de mutua
interdependencia con la enseñanza de calidad, al punto de integrarse en una
dinámica de doble hélice que potencia el desarrollo personal, social y
académico del estudiante \cite{lovat2009values}.

\subsection{Formación ciudadana}
La formación ciudadana, bajo el concepto anglosajón de
\textit{\guillemetleft{}civic education\guillemetright{}}, es el conjunto de
procesos, formales e informales, mediante los cuales las personas desarrollan
conocimientos, valores, actitudes, habilidades y compromisos que les permiten
participar activamente y de manera crítica en la vida democrática y
comunitaria; éste no está limitado al ámbito escolar ni a una etapa específica
de la vida del individuo, sino que se extiende a lo largo de su ciclo vital e
involucra diversos aspectos externos como la familia, los medios de
comunicación, su comunidad, instituciones educativas, etc.
\cite{crittenden2007civic}

Por lo tanto, la formación ciudadana no se limita a la transmisión de
contenidos normativos sobre el sistema político, sino que incorpora prácticas
educativas activas, como la discusión de temas controversiales, la
participación en acciones colectivas y la reflexión crítica, las cuales han
demostrado tener efectos significativos en el desarrollo de una ciudadanía
activa, consciente y empoderada \cite{crittenden2007civic}.

\subsection{Competencias cívicas fundamentales}
Las competencias cívicas fundamentales son un conjunto integrado de
disposiciones personales y capacidades que permiten a los individuos participar
activamente en sociedades democráticas diversas. De acuerdo con el Consejo de
Europa, estas competencias se organizan en torno a cuatro dimensiones
esenciales: \textbf{los valores} que guían el comportamiento ético; \textbf{las
    actitudes} que predisponen a la apertura y al respeto; \textbf{las habilidades}
necesarias para la interacción democrática; y \textbf{los conocimientos} y
\textbf{la comprensión crítica del mundo} social, político y cultural. Su
desarrollo es clave para convivir como iguales en contextos diversos y
democráticos \cite{barrett2016competences}.

\subsubsection{Valores}
Los valores son creencias fundamentales que orientan a las personas hacia metas
que consideran deseables en la vida. Funcionan como motores de acción y como
criterios que guían la toma de decisiones, al proporcionar marcos de referencia
sobre lo que se considera apropiado pensar o hacer en diversas situaciones.
Estos principios no se limitan a contextos específicos, sino que ofrecen
estándares para evaluar conductas, justificar posturas, elegir entre opciones,
planificar acciones e influir en otros \cite{barrett2016competences}.

\subsubsection{Actitudes}
Las actitudes representan la disposición mental general que una persona adopta
frente a individuos, grupos, instituciones, temas u objetos simbólicos. Esta
orientación suele estar compuesta por cuatro elementos interrelacionados: una
creencia o juicio cognitivo sobre el objeto, una respuesta emocional, una
valoración positiva o negativa, y una inclinación conductual específica hacia
dicho objeto \cite{barrett2016competences}.

\subsubsection{Habilidades}
Las habilidades son capacidades que permiten organizar y ejecutar de forma
eficiente patrones complejos de pensamiento o acción, adaptándolos al contexto
con el propósito de alcanzar un objetivo específico.
\cite{barrett2016competences}

\subsubsection{Conocimientos y Comprensión Crítica}
Los conocimientos representan el conjunto de información que una persona ha
adquirido, mientras que la comprensión crítica implica no solo entender esa
información, sino también valorar de forma reflexiva los sentimientos,
perspectivas y significados asociados a ella. Este tipo de comprensión es
esencial en contextos democráticos e interculturales, ya que permite analizar e
interpretar activamente las situaciones, superando respuestas automáticas o no
conscientes. En ese sentido, favorece la evaluación crítica de lo que se sabe y
de cómo se interpreta el mundo social y político \cite{barrett2016competences}.

\subsection{Educación moral}
La educación moral es el proceso educativo centrado en la moralidad, entendida
principalmente como la adhesión a normas morales y la creencia en su
justificación. Este enfoque puede implicar dos dimensiones fundamentales: por
un lado, la formación moral, que busca desarrollar en los individuos
disposiciones afectivas, conductuales y motivacionales alineadas con esas
normas; y por otro, la indagación moral, que promueve la reflexión crítica y la
construcción de creencias fundamentadas sobre la validez de dichas normas.
Ambas dimensiones pueden ser abordadas de manera complementaria, aunque
conceptualmente son distintas. Además, el autor reconoce que la moralidad
podría abarcar elementos adicionales, como ciertas virtudes o disposiciones
emocionales, cuya formación también puede formar parte significativa de la
educación moral \cite{hand2017moral}.

\subsubsection{Formación Moral}
La formación moral es una dimensión de la educación moral centrada en el
desarrollo de disposiciones afectivas y conductuales que llevan a una persona a
adherirse a normas morales y a responder emocionalmente a ellas. No se trata
únicamente de enseñar qué está bien o mal, sino de fomentar inclinaciones
internas que impulsen a actuar conforme a ciertos estándares, de forma estable
y espontánea. Estas disposiciones pueden incluir sentimientos de satisfacción
cuando se actúa moralmente, incomodidad al violar principios morales, y
expectativas de que otros también se comporten moralmente \cite{hand2017moral}.

Asimismo, este concepto puede abarcar el cultivo de virtudes, entendidas no
solo como inclinaciones a seguir normas, sino como capacidades para moderar
emociones humanas fundamentales. Bajo esta perspectiva, la formación moral no
se reduce a enseñar reglas, sino que apunta a moldear el carácter y las
emociones de forma que apoyen una vida moral \cite{hand2017moral}.

\subsubsection{Indagación Moral}
La indagación moral es la parte de la educación moral que se enfoca en
investigar y evaluar la justificación de las normas morales. Consiste en un
proceso cognitivo mediante el cual se analiza por qué una norma debería ser
aceptada, se examinan los argumentos que la sustentan y se reflexiona
críticamente sobre ellos. Creer en la justificación de una norma no es un
requisito para adherirse a ella, por lo que esta indagación es distinta de la
formación moral, que busca cultivar la adhesión emocional y conductual a esas
normas \cite{hand2017moral}.

En la enseñanza de la indagación moral, es posible adoptar un enfoque
directivo, orientando al individuo hacia una conclusión particular sobre la
validez de una norma, o un enfoque no directivo, en el que se facilita el
análisis y la discusión sin influir en la opinión final. Ambos métodos
promueven la capacidad del individuo para pensar críticamente sobre las normas
morales y su justificación, complementando así la formación moral.
\cite{hand2017moral}

\section{Aprendizaje informal y brecha educativa}
El aprendizaje informal constituye una estrategia educativa que ocurre fuera de
los entornos formales, como escuelas o universidades, y se produce de manera
espontánea en la vida cotidiana. Este tipo de educación fomenta la autonomía
del aprendiz, la creatividad y la resolución de problemas; contribuyendo a
reducir la brecha educativa, especialmente cuando el acceso a la educación
formal es limitado \cite{coombs1968world, mills2014informal}.

\subsection{Educación informal}
La educación informal se refiere a las formas de aprendizaje que ocurren de
manera natural en la vida cotidiana, en una amplia variedad de contextos
geográficos e históricos. Este tipo de educación no se limita a entornos
específicos, sino que suele surgir en espacios donde las personas se sienten
cómodas y con la libertad de socializar entre sí. Aunque este concepto es
asociado tradicionalmente con actividades fuera de la escuela, hoy en día la
educación informal también puede darse dentro de escuelas convencionales o en
organizaciones como el voluntariado juvenil o el movimiento scout.
\cite{mills2014informal}

Este tipo de educación se basa en el diálogo y la conversación, fomentando la
confianza, el respeto y la empatía. No busca imponer resultados específicos,
sino que promueve el aprendizaje a partir de las preocupaciones reales y
cotidianas de las personas; generando cambios positivos y significativos en sus
vidas. Además, la educación informal puede tener un carácter político,
inspirándose en enfoques críticos que buscan que las personas tomen conciencia
de las injusticias sociales y encuentren formas de superarlas, conectando lo
personal con temas sociales y políticos más amplios \cite{mills2014informal}.

\subsection{Autoformación guiada}
La autoformación guiada es un proceso intencional en el que el sujeto
desarrolla su aprendizaje autónomo con el apoyo de una institución, un educador
o un colectivo social. Aunque el aprendiz asume responsabilidad sobre sus
objetivos, recursos, métodos y ritmos, recibe orientación y acompañamiento que
facilitan el desarrollo de su capacidad de aprendizaje y autorregulación
\cite{mills2014informal}.

En este enfoque, la autoformación deja de ser un esfuerzo completamente
solitario o espontáneo para convertirse en una práctica educativa estructurada,
donde el apoyo externo configura las condiciones que permiten que el individuo
desarrolle su propio proyecto formativo y consolide su agencia como aprendiz
activo en contextos educativos no formales \cite{mills2014informal}.

\subsection{Brecha educativa y tecnológica}
La brecha educativa y tecnológica se refiere a las diferencias en el acceso y
aprovechamiento de recursos educativos y tecnológicos entre distintos grupos
sociales. Estas desigualdades afectan la calidad del aprendizaje, limitan la
participación en entornos digitales y pueden amplificar la exclusión social.
Factores como el acceso desigual a internet, dispositivos digitales y
capacitación docente contribuyen a esta brecha, la cual requiere estrategias
integrales de inclusión digital y políticas educativas que promuevan la equidad
\cite{van2005digital, unesco2023monitoring}.

\subsection{Tecnología como herramienta de inclusión educativa}
La tecnología educativa se ha consolidado como una herramienta estratégica para
promover la inclusión educativa, al facilitar el acceso a contenidos y recursos
didácticos a estudiantes con diversidad de contextos, habilidades y
necesidades. Plataformas digitales, dispositivos móviles y herramientas de
aprendizaje asistidas por inteligencia artificial permiten superar barreras
geográficas, socioeconómicas y culturales, contribuyendo a mejorar la equidad
en la educación \cite{teras2022education, unesco2023monitoring}.

\section{Fundamentos de Inteligencia Artificial en Educación}
La inteligencia artificial (IA), aplicada a la educación, ofrece oportunidades
para diseñar entornos de aprendizaje interactivos y personalizados. Una de las
estrategias más prometedoras es la implementación de métodos socráticos
digitales, donde los sistemas de IA guían a los estudiantes mediante preguntas
y diálogos reflexivos, estimulando el pensamiento crítico y la autonomía en la
construcción del conocimiento \cite{holmes2019ai, woolf2010building}.

\subsection{Inteligencia Artificial aplicada a la educación}
La IA en la educación permite automatizar tareas administrativas, ofrecer
tutorías personalizadas, monitorear el progreso de los estudiantes y adaptar
los contenidos a sus necesidades individuales. Estas aplicaciones han
demostrado mejorar la motivación, la eficiencia del aprendizaje y la calidad de
la enseñanza, siempre que se acompañen de supervisión pedagógica y criterios
éticos claros \cite{elstad2024ai, frontiers2025education, carter2024ethics}.

\subsection{Alucinación en Inteligencia Artificial}
Este concepto se refiere a la generación de información incorrecta, falsa o
inconsistente por parte de modelos de lenguaje, incluso cuando estos parecen
confiables y coherentes. Este fenómeno ocurre principalmente debido a la forma
en que los modelos aprenden a partir de grandes volúmenes de datos: no
comprenden la realidad de manera literal, sino que predicen la siguiente
palabra más probable según patrones estadísticos aprendidos. Como resultado,
pueden combinar información de manera inapropiada, inferir hechos inexistentes
o extrapolar contenido de forma errónea. La alucinación es especialmente
crítica en aplicaciones donde la precisión es esencial, como la educación, la
medicina o la generación de informes, ya que puede inducir a errores o
desinformación si no se detecta y corrige adecuadamente \cite{Ji_2023}.

\subsection{Modelos Grandes de Lenguaje}
Los modelos grandes de lenguaje (\textit{Large Language Models} o LLMs) son
sistemas de inteligencia artificial entrenados con enormes volúmenes de texto
para comprender y generar lenguaje natural. Estos modelos permiten ofrecer
respuestas contextualizadas, realizar tutorías personalizadas y asistir en la
construcción de conocimiento mediante diálogo interactivo. Su potencial
educativo radica en la capacidad de proporcionar retroalimentación inmediata,
adaptada al nivel del estudiante, fomentando la reflexión crítica y la
autoformación \cite{brown2020language, raffel2020exploring}.

\subsection{Arquitectura y funcionamiento de sistemas RAG}
La Generación Mejorada por Recupperación (RAG, por sus siglas en inglés)
combina modelos grandes de lenguaje (LLMs) con motores de recuperación de
información, con el fin de proporcionar respuestas fundamentadas en el
contenido específico deseado. Esta técnica, introducida por primera vez en 2020
en el artículo \textit{\guillemetleft{}Retrieval-Augmented Generation for
    Knowledge-Intensive NLP Tasks\guillemetright{}} por Lewis et al, permite la
obtención de respuestas fundamentadas en fuentes confiables, que bajo el
contexto del sistema promueve la reflexión ética y la resolución de dilemas
morales basados en evidencia. La generación mejorada por recuperación amplía
las capacidades de tutoría digital al integrar conocimiento externo con
generación de lenguaje natural \cite{lewis2020retrieval,
    khandelwal2020generalization}.

\subsubsection{Representaciones numéricas (\textit{embeddings}) y representación semántica del texto} Los \textit{embeddings} son
representaciones vectoriales de palabras, frases o documentos que capturan sus
significados semánticos. Esta técnica permite que los sistemas de IA comparen y
recuperen información de manera eficiente, lo que permite medir la similitud
entre conceptos y facilita búsquedas semánticas. En educación, estas
representaciones numéricas permiten vincular preguntas de los estudiantes con
contenidos relevantes, apoyando la personalización del aprendizaje
\cite{mikolov2013efficient, le2014distributed}.

\subsubsection{Arquitectura de Generación Mejorada por Recuperación}

Un sistema RAG opera en dos fases principales:

\begin{enumerate}
    \item \textbf{Fase de indexación:} Los documentos fuente se procesan mediante:
          \begin{itemize}
              \item Segmentación (\textit{chunking}) en fragmentos semánticamente coherentes.
              \item Generación de representaciones numéricas (\textit{embeddings}) para cada
                    fragmento.
              \item Almacenamiento en bases de datos vectoriales con metadatos.
          \end{itemize}

          Esta fase corresponde con la obtención de conocimiento externo a la
          implementación del sistema. En el artículo de Lewis et al, se presenta una
          figura (reproducida en la Figura \ref{fig:flujo-rag-teoria}) que muestra cómo
          este proceso se implementa mediante la combinación de
          \textbf{\guillemetleft{}memoria paramétrica\guillemetright{}} (el LLM utilizado
          para recibir preguntas y generar respuestas) y la
          \textbf{\guillemetleft{}memoria no paramétrica\guillemetright{}} (índice
          vectorial del cual se obtiene el contexto para fundamentar la respuesta), con
          lo cual la generación final obtiene la información requerida para brindar al
          usuario un resultado fundamentado.

          \begin{figure}[H]
              \centering
              \includegraphics[width=0.8\textwidth]{assets/rag\_original.png}
              \caption{Vista general del enfoque aplicado en el artículo de Lewis et al. \cite{lewis2020retrieval}}
              \label{fig:flujo-rag-teoria}
          \end{figure}

          La figura \ref{fig:flujo-rag-teoria} ilustra la arquitectura general de un
          sistema de recuperación y generación entrenado de extremo a extremo. El proceso
          inicia con una \textbf{consulta de entrada} \(x\), por ejemplo, solicitar la
          respuesta a una pregunta. Esta consulta es procesada por un \textbf{codificador
              de consultas} (\emph{Query Encoder}), el cual transforma \(x\) en una
          representación densa \(q(x)\).

          A continuación, esta representación es enviada al \textbf{módulo recuperador no
              paramétrico} \(p_{\eta}\), el cual opera sobre un índice de documentos. El
          recuperador utiliza \emph{Maximum Inner Product Search} (MIPS) para identificar
          los documentos cuyas representaciones \(d(z)\) presentan mayor similitud con el
          vector de consulta. El índice contiene múltiples documentos representados en
          forma vectorial y el recuperador selecciona los más relevantes, denotados como
          \(z_1, z_2, z_3, \dots\).

          Los documentos seleccionados se redireccionan hacia el \textbf{generador
              paramétrico} \(p_{\theta}\), el cual recibe tanto la representación de la
          consulta como la de los documentos recuperados. Este generador produce una
          distribución sobre posibles salidas, es decir, posibles respuestas a la
          pregunta solicitada.

          Finalmente, el sistema realiza la \textbf{marginalización} sobre todos los
          documentos recuperados, combinando las contribuciones de cada uno de ellos para
          producir la salida final \(y\). Esta salida es la que toma la forma de la
          respuesta completa a la pregunta original.

    \item \textbf{Fase de inferencia:} Al recibir una consulta:
          \begin{itemize}
              \item Se genera una representación numérica (\textit{embedding}) de la pregunta del
                    usuario.
              \item Se recuperan los K fragmentos más relevantes (típicamente un valor K de entre 3
                    y 10 elementos), conformando el contexto de la consulta.
              \item Se construye una petición contextualizada que combina la pregunta original con
                    los fragmentos recuperados.
              \item El LLM genera una respuesta basándose exclusivamente en el contexto dado o
                    tomándolo como guía (el enfoque depende de la estructura utilizada para
                    construir la petición) de manera que todo resultado se ve anclado a las fuentes
                    verificadas previamente seleccionadas para alimentar el sistema.

                    Según la implementación, se pueden utilizar distintas estrategias para la
                    combinación de los fragmentos con la generación final. En el artículo de Lewis
                    et al, se menciona la diferencia de utilizar una \textbf{RAG por secuencia }
                    (el modelo selecciona un único documento sobre el cual basará su respuesta)
                    frente al enfoque de \textbf{RAG por token} (el modelo genera la respuesta por
                    pasos, seleccionando la fuente a utilizar para cada token independiente, lo que
                    permite combinar más de una fuente).
          \end{itemize}
\end{enumerate}

\subsubsection{Estrategias de segmentación}
El método de segmentación (\textit{chunking}) seleccionado puede llegar a
afectar directamente la calidad de las respuestas obtenidas por el sistema. Las
estrategias más comunes incluyen \cite{wang-etal-2025-document}:
\begin{itemize}
    \item \textbf{Segmentación por tamaño fijo:} se divide el texto en fragmentos de longitud uniforme, independiente de su semántica. Presenta la ventaja de que es el tipo de segmentación más fácil de implementar, ya que basta solamente con establecer un límite de palabras y separar todo el texto en dicho límite. Sin embargo, al no evaluar el sentido semántico en cada fragmento, podría dar lugar a rupturas de contexto o pérdida de sentido.
    \item \textbf{Segmentación semántica:} esta segmentación divide el texto respetando unidades de sentido, identificadas mediante signos de puntuación, saltos de línea, identificación de encabezados, secciones, listas, etc. El propósito de esta estrategia es preservar la continuidad de contexto entre cada fragmento.
    \item \textbf{Segmentación recursiva con solapamiento:} esta técnica divide el texto utilizando alguna técnica anterior, con la peculiaridad de incluir al inicio o al final una parte del fragmento contiguo. Es decir, se define un porcentaje de solapamiento que se refiere a qué tanto del fragmento siguiente (o anterior) se incluirá como parte del nuevo fragmento, de manera que se controle de forma explícita la continuidad.
\end{itemize}

La estrategia de segmentación dependerá de la implementación del sistema RAG
que se desea realizar. Se debe tomar en cuenta que la estrategia utilizada
afectará directamente la calidad de las respuestas, ya que esto define cómo el
LLM obtendrá el contexto del cual se basará para brindar la respuesta a la
consulta dada.

\subsubsection{Ventajas de RAG en educación}
La implementación de sistemas de generación mejorada por recuperación en
enfoques educativos, brinda varias ventajas orientadas al uso de modelos de
inteligencia artificial
\cite{gupta2024comprehensivesurveyretrievalaugmentedgeneration}:
\begin{itemize}
    \item \textbf{Reduce alucinaciones} al obligar al sistema a utilizar respuestas fundamentadas en el corpus definido para el proyecto.
    \item \textbf{Permite actualización del conocimiento} sin reentrenar el modelo; basta solamente con modificar, añadir o eliminar el corpus del proyecto para que el sistema utilice esta nueva información.
    \item \textbf{Facilita trazabilidad} y citación de fuentes, lo cual es particularmente importante en contextos educativos en los que es necesario fundamentar de dónde se obtiene toda la información proporcionada.
    \item Es apropiado para \textbf{dominios especializados} o corpus limitados, como
          asignaturas o materiales didácticos que no están bien cubiertos en los datos de
          entrenamiento general del LLM.
\end{itemize}

\subsubsection{Desafíos conocidos}
A pesar de sus beneficios, la implementación de generación mejorada por
recuperación también presenta ciertos retos a cubrir en proyectos educativos de
alto impacto \cite{zheng2025knowshiftqarobustragsystems}:
\begin{itemize}
    \item \textbf{Dependencia crítica} de la calidad del corpus. La calidad de las respuestas depende estrictamente de la calidad del contenido educativo utilizado para alimentar el modelo, por lo que si los documentos están mal organizados, contienen errores o están desactualizados; la recuperación de contexto será débil.
    \item \textbf{Riesgo de fragmentación} que rompa la coherencia contextual. Utilizar una estrategia o combinación de estrategias de segmentación inapropiada, puede llevar a que el contexto obtenido por el modelo deje de tener sentido, o bien, que una consulta que sí está relacionada con el contenido del corpus no pueda ser respondida.
    \item \textbf{Limitaciones de la ventana de contexto del LLM} a pesar de que el sistema de recuperación diseñado obtenga una base contextual amplia, según el modelo LLM utilizado, generalmente se tiene un límite de tokens permitido para cada consulta, por lo que el la petición construida también cuenta con limitaciones de longitud y, por lo tanto, del nivel de especificación y claridad exigido al modelo.
    \item \textbf{Dificultad para sintetizar información de múltiples
              fragmentos dispersos} Al combinar varios fragmentos en una sola petición,
          si no se ha seleccionado el corpus cuidadosamente, se puede incurrir
          en contradicciones o redundancia en las consultas.

\end{itemize}

\subsection{Bases de datos vectoriales y búsqueda semántica}
Las bases de datos vectoriales permiten almacenar y consultar representaciones
numéricas (\textit{embeddings}) de manera eficiente, habilitando la búsqueda
semántica en grandes volúmenes de información. Este enfoque supera las
limitaciones de las búsquedas basadas en palabras clave, lo que permite que los
estudiantes y sistemas educativos accedan a contenidos relevantes de manera más
precisa y contextualizada, facilitando la recuperación de conocimiento en
entornos digitales \cite{johnson2019billion, han2023comprehensive}.

\subsection{Ingeniería de peticiones y diseño de instrucciones} La ingeniería de peticiones consiste en el diseño de instrucciones efectivas
para guiar el comportamiento de modelos de lenguaje hacia objetivos
específicos. \cite{white2023promptpatterncatalogenhance} El enfoque principal
es brindar al modelo directivas claras con el fin de obtener el resultado final
esperado, enfocados en ser tan específicos y directos como el modelo permita.

\subsubsection{Componentes de una petición efectiva} Al construir una petición que será recibida por un modelo grande de lenguaje
(LLM), se deben tomar en cuenta varios componentes que, en conjunto, serán los
que definan la interacción entre el modelo y el usuario. El proceso de
ingeniería de peticiones consiste en optimizar el lenguaje utilizado en la
instrucción, enfocado en guiar el comportamiento de los modelos de lenguaje y
obtener el mejor desempeño posible. Esto implica entender que los modelos
procesan las instrucciones como programas en lenguaje natural, por lo que la
precisión y coherencia de las respuestas se ven afectadas directamente por la
estructura y redacción de cada parte de la petición.
\cite{schulhoff2025promptreportsystematicsurvey}

Los componentes más comunes son:

\begin{itemize}
    \item \textbf{Rol (Persona):} Si bien es un componente ampliamente discutido, corresponde a la sección de la petición que permite asignar un rol específico a la inteligencia artificial (IA). A través de la definición de un rol específico, se determina el tono de las respuestas generadas y los límites que debe cumplir en su interacción con el usuario.
    \item \textbf{Información Adicional:} Comunmente identificado también como el \textbf{contexto} de la petición. Se refiere a todo el contenido adicional que la IA debe tomar en cuenta para generar la respuesta, ya sea interacciones previas, o bien, el contenido que delimite la respuesta final (por ejemplo, a través de un sistema de generación mejorada por recuperación).
    \item \textbf{Directiva:} Define la intención central de la petición. Aquí puede ir la pregunta, solicitud de información o generación de contenido multimedia (si aplica).
    \item \textbf{Formato de salida:} Comunmente, es deseable que la IA generativa sea capaz de devolver la información solicitada en un formato o patrón específico. Este aspecto también incluye las \textbf{instrucciones de estilo}, las cuales son un tipo de formato de salida utilizado para modificar la salida estilísticamente en lugar de estructuralmente (por ejemplo, solicitar una respuesta apropiada para una persona de edad específica).
    \item \textbf{Ejemplos (opcional):} Los ejemplos, también conocidos como \textbf{tomas}, actúan como demostraciones explícitas que guían a la IA para lograr un resultado específico para la tarea indicada. Este aspecto es especialmente útil en técnicas como el aprendizeje en contexto.
\end{itemize}

\subsubsection{Estrategias en contextos educativos}
Con el objetivo de enfocar el diseño de peticiones al campo de la educación, se
enfatizan prácticas específicas que permiten obtener el flujo de pensamiento
del asistente, establecer un tipo de interacción específica con el estudiante
(por ejemplo, el uso del método socrático) o solicitar las fuentes utilizadas
\cite{NEURIPS2022_9d560961}.

Algunas de las estrategias más comunes son:

\begin{itemize}
    \item \textbf{\textit{Chain-of-thought prompting} (Petición de Cadena de Pensamiento):} Solicita al modelo que describa el razonamiento que utilizó para responder la pregunta, se exige el paso a paso de cómo llegó hasta la respuesta brindada.
    \item \textbf{\textit{Socratic prompting} (Petición Socrática):} Indica al modelo que se debe guiar por el método socrático, el cual consiste en incentivar al usuario a obtener una respuesta final por sí mismo, en lugar de brindar una respuesta directa a la consulta dada.
    \item \textbf{\textit{Constitutional AI} (IA Constitucional):} Incorpora principios éticos en las instrucciones, por ejemplo, la omisión de palabras o temas sensibles.
    \item \textbf{\textit{Retrieval-aware prompting} (Petición Consciente de Recuperación):} Exige al modelo citar fuentes en todas sus respuestas. Puede ser útil, aunque también vale la pena analizar si conviene más esta estrategia o simplemente almacenar en los metadatos de los fragmentos los documentos originales.
\end{itemize}

\subsection{Tutoría personalizada con IA}
La tutoría personalizada con IA permite adaptar los contenidos y las
estrategias de enseñanza al nivel, intereses y ritmo de cada estudiante. Los
sistemas inteligentes analizan patrones de aprendizaje y ofrecen
retroalimentación inmediata, identificando áreas de dificultad y recomendando
recursos específicos. Esta personalización mejora la motivación, la retención
de conocimiento y promueve la autonomía del aprendiz \cite{woolf2010building,
    zawacki-richter2019systematic}.

\subsection{Método socrático aplicado a entornos digitales}
Los entornos digitales permiten implementar el método socrático mediante
sistemas de IA que guían a los estudiantes a través de preguntas reflexivas y
secuencias de razonamiento. Esta estrategia fomenta el pensamiento crítico y la
autonomía, ya que los alumnos deben analizar, argumentar y evaluar sus propias
respuestas antes de recibir retroalimentación. El uso de chatbots y asistentes
inteligentes basados en este método facilita un aprendizaje personalizado y
continuo, replicando la interacción dialógica propia del enfoque socrático
tradicional \cite{favero2024socratic, woolf2010building}.

\subsection{Métricas de evaluación de chatbots educativos}
\label{subsec:metricas-chatbots}
La evaluación de asistentes conversacionales educativos requiere métricas más
allá de la precisión técnica, incorporando dimensiones pedagógicas incluso si
no son herramientas planificadas para su utilización en entornos formales
tradicionales de educación. \cite{10.3389/frai.2021.654924}

\subsubsection{Métricas de calidad de respuesta}
En el diseño y evaluación de sistemas conversacionales, la calidad de la
respuesta se relaciona con la capacidad de la herramienta para comprender la
intención del usuario, mantener la coherencia del diálogo y producir respuestas
naturales y adecuadas. La efectividad de un chatbot depende de componentes como
el análisis de mensajes, la gestión del diálogo y la generación del lenguaje
natural, los cuales influyen directamente en la calidad percibida por el
usuario. \cite{ADAMOPOULOU2020100006}

Varias de las métricas más utilizadas son:

\begin{itemize}
    \item \textbf{Relevancia del contexto:} Evalúa qué tan pertinentes son los documentos recuperados para la consulta inicial. \cite{saad-falcon-etal-2024-ares}.
    \item \textbf{Fidelidad de la respuesta:} Mide si la salida generada está fundamentada en la evidencia recuperada, evitando alucinaciones.\cite{saad-falcon-etal-2024-ares}.
    \item \textbf{Relevancia de la respuesta:} Valora si la respuesta aborda adecuadamente la consulta del usuario \cite{saad-falcon-etal-2024-ares}.
\end{itemize}

\subsubsection{Métricas de evaluación técnica}
Para evaluar sistemas de RAG desde una perspectiva técnica, se consideran
métricas que capturan el rendimiento operacional del sistema en escenarios
prácticos.\cite{Yu_2025}

Las más comunes son:

\begin{itemize}
    \item \textbf{Tasa de éxito:} Porcentaje de consultas en las que el sistema genera respuestas correctas y apropiadas. Esta métrica evalúa la efectividad global del sistema en tareas específicas, siendo fundamental para aplicaciones donde la precisión es crítica \cite{chen2023benchmarkinglargelanguagemodels, xiong2024benchmarkingretrievalaugmentedgenerationmedicine}.
    \item \textbf{Latencia:} Tiempo total transcurrido desde el envío de la consulta hasta la recepción de la respuesta completa, incluyendo las fases de recuperación y generación. Esta métrica es crucial para la experiencia de usuario en aplicaciones interactivas y sistemas en tiempo real \cite{10.1145/3539618.3591687}.
\end{itemize}

\subsection{Evaluación de calidad de respuestas en sistemas RAG}
\label{subsec:calidad-rag}
Los sistemas que implementan RAG presentan desafíos específicos de evaluación
relacionados con la recuperación y síntesis de información. A diferencia de los
sistemas puramente extractivos, los sistemas RAG deben ser evaluados tanto por
la calidad de la información generada como por su fidelidad al contexto
recuperado \cite{lewis2020retrieval}.

\subsubsection{Fidelidad al contenido base (Faithfulness)}
En el contexto de sistemas RAG educativos, la fidelidad evalúa si la respuesta
se limita estrictamente a la información contenida en los documentos
recuperados, evitando alucinaciones o afirmaciones no sustentadas por el corpus
proporcionado. \cite{es-etal-2024-ragas}

Esta métrica es particularmente crítica en contextos educativos, donde la
precisión fáctica y la trazabilidad de la información son requisitos
fundamentales. Se mide comparando el contenido de las respuestas generadas con
el material documentado en el corpus base, verificando que toda afirmación esté
respaldada por los documentos fuente. \cite{es-etal-2024-ragas}

\subsubsection{Relevancia de la respuesta (Answer Relevance)}
La relevancia de la respuesta determina el grado en que la respuesta generada
responde efectivamente a la pregunta formulada, sin desviarse hacia información
tangencial o irrelevante. Esta métrica evalúa la pertinencia contextual de la
respuesta respecto a la consulta original.
\cite{su2021improvequeryfocusedabstractive}

En sistemas educativos, una respuesta puede ser técnicamente correcta y basada
en el contenido, pero si no aborda adecuadamente la pregunta del usuario,
pierde su valor pedagógico. \cite{su2021improvequeryfocusedabstractive}

\subsection{Congruencia y fundamentación en respuestas educativas}
\label{subsec:congruencia-respuestas}
En contextos educativos, además de la precisión técnica, se espera que los
sistemas basados en modelos de lenguaje mantengan consistencia con el
conocimiento verificado y respondan de manera fundamentada. En este sentido, la
\textbf{congruencia fáctica} puede entenderse como el grado en que las
respuestas del sistema son coherentes con los hechos documentados en una base
de conocimiento estructurada, evitando errores o afirmaciones sin evidencia
empírica \cite{wang2024earthflatunveilingfactual}.

El \textbf{nivel de congruencia fáctica} puede calcularse como el porcentaje de
consultas en las que el sistema responde correctamente de acuerdo con el
conocimiento base. Esta medida complementa las métricas tradicionales de
exactitud o fidelidad al evaluar específicamente la capacidad del sistema para
mantenerse dentro de los límites del conocimiento validado, evitando
desviaciones o \textbf{alucinaciones} que puedan inducir a error en entornos
educativos \cite{wang2024earthflatunveilingfactual}.

\subsection{Validación técnica sin usuarios finales}
La validación técnica de sistemas educativos sin participación de usuarios
finales presenta limitaciones inherentes, pero permite establecer la viabilidad
funcional del sistema antes de su despliegue. \cite{luckin2016intelligence}

Este enfoque de validación se basa en la definición de conjuntos de prueba
estructurados que incluyen tanto preguntas dentro del dominio esperado como
preguntas de control fuera del alcance del sistema. La validación basada en el
corpus verifica que las respuestas provengan efectivamente de los materiales
fuente y que no contengan información ajena al contenido educativo.
\cite{holmes2019ai}

Si bien este tipo de validación no mide el impacto pedagógico real ni el
aprendizaje logrado, permite demostrar la factibilidad técnica y coherencia
funcional del sistema antes de proceder a estudios con usuarios reales.
\cite{luckin2016intelligence,holmes2019ai,zawacki-richter2019systematic}

\section{Ética y Responsabilidad en la IA Educativa} La ética en la IA educativa aborda la responsabilidad en el diseño,
implementación y uso de sistemas inteligentes en contextos de aprendizaje, lo
cual incluye consideraciones sobre privacidad de los datos, equidad,
transparencia, inclusión e impacto social. Garantizar que los estudiantes sean
tratados de manera justa y que los sistemas no reproduzcan sesgos existentes es
crucial para la confianza y efectividad de la educación asistida por IA
\cite{selwyn2019should, williamson2023social}.

\subsection{Principios éticos fundamentales en la IA}
Los principios éticos fundamentales en la IA incluyen transparencia, justicia,
no discriminación, responsabilidad, privacidad y seguridad. En el ámbito
educativo, estos principios guían el desarrollo de sistemas que respeten la
dignidad del estudiante, promuevan equidad en el aprendizaje y faciliten la
rendición de cuentas por parte de desarrolladores y educadores. La aplicación
de estos principios permite aprovechar el potencial de la IA sin comprometer la
integridad pedagógica \cite{jobin2019global, floridi2018ai}.

\subsection{Sesgos algorítmicos y culturales en contextos latinoamericanos}
La prevención de sesgos algorítmicos se centra en garantizar que los sistemas
de IA no reproduzcan ni amplifiquen desigualdades existentes en la educación.
Esto implica analizar los datos de entrenamiento, identificar posibles sesgos y
aplicar técnicas de mitigación, como ajuste de ponderaciones, diversificación
de conjuntos de datos y pruebas de equidad en los resultados. La prevención de
sesgos asegura que todos los estudiantes reciban oportunidades de aprendizaje
justas y equitativas \cite{mehrabi2019survey, binns2018fairness}.

\subsection{Transparencia y explicabilidad en sistemas inteligentes}
La transparencia y explicabilidad son fundamentales para que docentes,
estudiantes y desarrolladores comprendan cómo un sistema de IA toma decisiones.
Esto incluye técnicas de interpretabilidad que permitan visualizar la lógica de
los modelos y justificar las recomendaciones que generan. En educación, la
explicabilidad ayuda a confiar en las decisiones automatizadas, facilita la
supervisión pedagógica y permite detectar errores o sesgos
\cite{doshi2017towards, lipton2018mythos}.

\subsection{Responsabilidad ante respuestas incorrectas o inadecuadas}
La responsabilidad en sistemas de IA educativa implica definir con claridad los
mecanismos para abordar errores, recomendaciones inadecuadas o información
potencialmente nociva generada por los algoritmos. Cuando un sistema
automatizado produce contenidos incorrectos, los efectos pueden ser
especialmente sensibles en contextos educativos, ya que influyen directamente
en el aprendizaje, la motivación y las decisiones académicas de los estudiantes
\cite{jobin2019global, floridi2018ai}.

La responsabilidad recae tanto en los desarrolladores, quienes deben
implementar modelos seguros, mecanismos de verificación y pruebas continuas,
como en los docentes y las instituciones que integran la tecnología. Esto
incluye ofrecer rutas de corrección, permitir retroalimentación humana y
asegurar canales claros para reportar fallos. De esta manera, la IA se integra
como una herramienta asistiva bajo supervisión profesional, en lugar de delegar
completamente la evaluación y orientación pedagógica \cite{jobin2019global,
    floridi2018ai}.

\subsection{Privacidad y seguridad en aplicaciones educativas para menores}
El uso de aplicaciones educativas basadas en IA en contextos escolares requiere
un enfoque riguroso de protección de datos, especialmente cuando se trata de
menores de edad. La información académica, conductual y biométrica recopilada
por estos sistemas constituye un activo sensible que debe ser gestionado bajo
principios de minimización de datos, consentimiento informado y almacenamiento
seguro \cite{selwyn2019should, williamson2023social}.

Organismos internacionales han enfatizado la importancia de resguardar la
identidad digital de los estudiantes, evitar usos secundarios no autorizados y
garantizar que los datos no se utilicen para prácticas discriminatorias o
comerciales. Las instituciones tienen la responsabilidad de establecer
políticas claras de acceso, supervisar proveedores tecnológicos y aplicar
estándares robustos de ciberseguridad. La protección de los datos de menores no
solo es una obligación legal y ética, sino también una condición necesaria para
preservar la confianza en entornos educativos mediados por IA
\cite{selwyn2019should, williamson2023social}.

\subsection{Supervisión pedagógica en sistemas automatizados}
A pesar de la autonomía de los sistemas de IA, la supervisión pedagógica es
esencial para garantizar la calidad del aprendizaje. Docentes y tutores deben
monitorear el funcionamiento de los sistemas automatizados, evaluar la
relevancia y exactitud de las respuestas generadas, y ajustar los parámetros de
personalización según las necesidades de los estudiantes. Este enfoque mixto
asegura que la tecnología complemente, y no reemplace, la guía educativa
\cite{holmes2019ai, luckin2016intelligence}.

\section{Aprendizaje Móvil en Contextos de Recursos Limitados}
El aprendizaje móvil (\textit{mobile learning} o m-learning) se ha convertido
en un medio clave para ampliar el acceso a experiencias educativas,
especialmente en regiones donde las limitaciones tecnológicas, de
infraestructura o económicas dificultan el aprendizaje tradicional. En
contextos con recursos limitados, los dispositivos móviles permiten superar
barreras geográficas y temporales, democratizando oportunidades de acceso a
información, formación técnica y herramientas de apoyo educativo
\cite{traxler2007defining, unesco2013policy}.

Sin embargo, la implementación efectiva del aprendizaje móvil requiere
considerar retos como la disponibilidad de dispositivos, la alfabetización
digital de los usuarios, los costos de conectividad y las brechas de
infraestructura. El diseño de soluciones educativas móviles sostenibles debe
responder a estos factores para garantizar accesibilidad, pertinencia cultural
y equidad tecnológica. \cite{unesco2013policy}

\subsection{Panorama del aprendizaje móvil en Guatemala y Centroamérica}
El crecimiento del aprendizaje móvil en Guatemala y Centroamérica ha sido
gradual pero progresivo, impulsado por iniciativas de digitalización,
comunidades tecnológicas emergentes y el interés institucional por modernizar
procesos educativos y productivos. Según el BID, la región ha avanzado
significativamente en adopción tecnológica, pero aún enfrenta brechas en cuanto
a infraestructura digital, capacidad de investigación e inversión en innovación
\cite{worldbank2022revolution}.

\subsection{Diseño de experiencias móviles para usuarios con baja alfabetización digital}
El diseño de experiencias educativas móviles para usuarios con baja
alfabetización digital requiere estrategias centradas en usabilidad,
simplicidad y acompañamiento formativo. UNESCO destaca que las interfaces
visuales claras, los flujos guiados y los recursos multimedia accesibles pueden
favorecer la participación de usuarios principiantes
\cite{unesco2021reimagining}.

\subsection{Consideraciones de conectividad intermitente y consumo de datos}
En muchos contextos latinoamericanos, incluidos sectores rurales de Guatemala,
el acceso a Internet es costoso e intermitente. Por ello, las aplicaciones
educativas móviles deben optimizar el consumo de datos, ofrecer funcionalidad
fuera de línea y emplear técnicas de sincronización diferida para resguardar el
progreso del usuario cuando no haya conexión \cite{shrestha2010offline,
    unesco2013policy, android_offline_first}.

Prácticas recomendadas incluyen compresión de recursos multimedia,
almacenamiento local temporal, caching inteligente y utilización de formatos
eficientes. La capacidad de operar con conectividad limitada no solo reduce
barreras de acceso, sino que también mejora la adopción sostenida de
herramientas educativas en zonas marginadas \cite{col_offline_design_model}.

\subsection{Aplicaciones móviles en la educación}
Las aplicaciones móviles educativas permiten acceder a recursos y experiencias
de aprendizaje en cualquier momento y lugar. Integradas con IA, estas apps
pueden ofrecer tutorías personalizadas, seguimiento del progreso,
retroalimentación inmediata y gamificación del aprendizaje. Su portabilidad y
accesibilidad contribuyen a reducir la brecha educativa y facilitan la
inclusión digital \cite{traxler2009learning, crompton2018use}.

\section{Tecnologías de Implementación}
El uso responsable de IA en educación implica enseñar a los estudiantes a
utilizar herramientas inteligentes sin vulnerar normas éticas ni académicas.
Esto incluye fomentar la autoría propia, la citación adecuada de fuentes y el
desarrollo de habilidades de pensamiento crítico para interpretar la
información generada por la IA. La integridad académica asegura que la
tecnología complemente el aprendizaje sin reemplazar la reflexión y el esfuerzo
personal \cite{bretag2019academic, eaton2023postplagiarism}.

\subsection{Interfaz de Programación de Aplicaciones (\textit{Application Programming Interface} - API)}
Las \textbf{APIs} permiten que dos sistemas de software
se comuniquen entre sí mediante un conjunto definido de reglas, conocidos como
\textit{endpoints} o \textbf{puntos de conexión}. Según la definición clásica
de Fielding y Taylor, una API proporciona un mecanismo estandarizado para que
aplicaciones distintas intercambien datos o funcionalidades sin necesidad de
conocer la implementación interna de cada sistema \cite{10.1145/514183.514185}.
En el ámbito de la inteligencia artificial, las APIs ofrecen acceso directo a
modelos avanzados sin requerir que el desarrollador entrene o despliegue
modelos por su cuenta.

\subsection{Pinecone}
Pinecone es una base de datos vectorial administrada, en la nube y diseñada
para búsquedas semánticas a gran escala. Permite almacenar, indexar y consultar
representaciones numéricas (\textit{embeddings}) de alta dimensionalidad con
latencias muy bajas. Su arquitectura está optimizada para tareas como
generación mejorada por recuperación (RAG), sistemas de recomendación,
clasificación semántica y búsqueda inteligente. Proporciona escalabilidad
automática, indexación eficiente y una API sencilla para integración en
aplicaciones de aprendizaje de máquinas (\textit{machine learning})
\cite{pinecone_docs}.

\subsection{PostgreSQL}
PostgreSQL es un sistema de gestión de bases de datos relacional de código
abierto; reconocido por su robustez, flexibilidad y estricto cumplimiento de
estándares SQL. Está diseñado para manejar cargas de trabajo complejas,
consultas avanzadas y grandes volúmenes de datos, a la vez que ofrece
características como transacciones ACID (siglas en inglés para Atomicidad,
Consistencia, Aislamiento y Durabilidad), extensibilidad mediante funciones
personalizadas, índices especializados y soporte para tipos de datos avanzados.
Su arquitectura orientada a la confiabilidad y la integridad de los datos lo
convierte en una opción preferida en aplicaciones empresariales, científicas y
de software moderno que requieren estabilidad y alto rendimiento
\cite{postgresqlDoc}.

\subsection{NotebookLM}
NotebookLM es una herramienta de inteligencia artificial que permite a los
usuarios \guillemetleft{}conversar\guillemetright{} con sus propios documentos
mediante técnicas de modelos grandes de lenguaje. A diferencia de un motor de
búsqueda, NotebookLM analiza los textos cargados por el usuario y construye una
base de conocimiento personalizada. A partir de esto, puede generar resúmenes,
responder preguntas complejas sobre el contenido, extraer ideas clave,
organizar la información en guías de estudio, e incluso convertir textos
extensos en resúmenes de audio o notas estructuradas. \cite{notebooklm2025}

Esta herramienta resulta especialmente útil para estudiantes, investigadores o
profesionales que manejan gran cantidad de información, ya que facilita la
comprensión y navegación de textos largos o complejos, reduciendo el tiempo
necesario para extraer lo esencial. Además, al apoyarse únicamente en los
documentos proporcionados por el usuario, NotebookLM ayuda a mantener la
precisión y trazabilidad de la información, lo que puede mitigar el riesgo de
errores comunes en modelos de lenguaje no especializados.
:contentReference[oaicite:2]{index=2}

\subsection{Postman}
Postman es una herramienta de desarrollo utilizada para diseñar, probar y
documentar APIs. Ofrece una interfaz gráfica intuitiva para enviar solicitudes
HTTP (siglas en inglés para Protocolo de Transferencia de HiperTexto), revisar
respuestas, automatizar pruebas, organizar colecciones de puntos de conexión y
colaborar en equipos de desarrollo. Permite trabajar con distintos métodos
HTTP, autenticación, variables de entorno y generación automática de
documentación. \cite{postman_tool}

\subsection{Servicio de Almacenamiento Simple (\textit{Simple Storage Service} o S3)} S3 es un servicio de almacenamiento de objetos altamente escalable,
ofrecido por los servicios web de Amazon (\textit{Amazon Web Services} o AWS).
Permite guardar y recuperar cualquier cantidad de datos desde cualquier lugar a
través de Internet, proporcionando durabilidad de casi el 100\%, alta
disponibilidad y control granular de acceso. S3 organiza los datos en
contenedores llamados \textit{\guillemetleft{}buckets\guillemetright} y ofrece
diversas clases de almacenamiento optimizadas para distintos niveles de costo y
frecuencia de acceso. \cite{aws_s3_docs}

\subsection{Arquitectura cliente-servidor (\textit{frontend}/\textit{backend})}
La arquitectura cliente-servidor es un modelo de diseño de software en el que
el cliente (por ejemplo, una app móvil o navegador web) solicita servicios al
servidor, el cual procesa la información, ejecuta la lógica de negocio y responde
con datos. En educación digital, esta arquitectura permite centralizar recursos
educativos, gestionar bases de datos y ofrecer aplicaciones interactivas
seguras y escalables. El cliente se encarga de la interfaz y la experiencia de
usuario, mientras que el servidor gestiona la lógica, la seguridad y la
integración con IA y bases de datos \cite{tanenbaum2007distributed,
    hwang2011cloud}.

\subsubsection{Patrón de diseño MVC}
El patrón de diseño \textbf{Modelo-Vista-Controlador}
(\textit{Model-View-Controller} o MVC) es una arquitectura ampliamente
utilizada para la construcción de interfaces de usuario. Su objetivo principal
es separar la representación de la información de la lógica que la gestiona,
promoviendo modularidad, reutilización y facilidad de mantenimiento.
\cite{reenskaug1979mvc}

En este patrón, el \textbf{Modelo} contiene los datos y la lógica de negocio;
la \textbf{Vista} es responsable de mostrar la información al usuario; y el
\textbf{Controlador} actúa como intermediario, recibiendo entradas del usuario
y coordinando las actualizaciones entre el modelo y la vista. Esta separación
permite que cambios en la interfaz no afecten directamente a la lógica del
sistema y viceversa. \cite{reenskaug1979mvc}

El patrón MVC tuvo sus orígenes en el entorno \textit{Smalltalk-80} y ha sido
ampliamente adoptado en múltiples marcos (\textit{frameworks}) modernos de
desarrollo, tanto web como de escritorio, esto debido a su capacidad de
estructurar aplicaciones complejas de forma eficiente \cite{reenskaug1979mvc}.

\subsubsection{NodeJS como plataforma para el desarrollo del lado del servidor}
NodeJS es una plataforma de ejecución de JavaScript basada en un modelo
orientado a eventos y operaciones no bloqueantes, lo que permite manejar
múltiples conexiones concurrentes con alta eficiencia y baja latencia. Su
extenso ecosistema de paquetes mediante npm (Gestor de Paquetes de Node, por
sus siglas en inglés) facilita integrar funcionalidades para redes, bases de
datos, autenticación y servicios web, convirtiéndolo en una opción ampliamente
adoptada para construir APIs (Interfaces de Programación de Aplicaciones),
aplicaciones en tiempo real y arquitecturas modernas de microservicios.
\cite{nodejs_documentation_2023}

\subsection{Códigos de estado HTTP}
Son respuestas numéricas estandarizadas que un servidor envía al cliente para
indicar el resultado de una solicitud, realizada mediante el protocolo HTTP.
Estos códigos permiten identificar si la petición fue exitosa, si requiere
acciones adicionales, si hubo errores del cliente o del servidor, o si existen
problemas de redirección. Se organizan en cinco categorías principales:
respuestas informativas (de 100 a 199), exitosas (de 200 a 299), redirecciones
(de 300 a 399), errores del cliente (de 400 a 499) y errores del servidor (de
500 a 599). Su uso adecuado facilita la comunicación clara entre aplicaciones y
servicios web, lo que permite un manejo correcto de errores y una interacción
más confiable en arquitecturas distribuidas \cite{rfc9110}.

\subsection{Ping}
Esta es una herramienta de diagnóstico de red que permite verificar la
conectividad entre dos dispositivos mediante el envío de mensajes ICMP (siglas
en inglés para Protocolo de Mensajes de Control de Internet, protocolo
utilizado para diagnosticar problemas de comunicación). Su propósito principal
es determinar si un dispositivo es accesible, así como medir parámetros como el
tiempo de ida y vuelta (RTT o \textit{Round-Trip Time}) y la pérdida de
paquetes. Ping es ampliamente utilizado para identificar problemas de red
básicos, ya que ayuda a detectar fallos de comunicación, latencias inusuales o
interrupciones en la ruta entre origen y destino. Su funcionamiento se basa en
el protocolo ICMP, definido en los estándares fundamentales de Internet
\cite{rfc792}.

\subsection{Marcos de desarrollo móvil: Kotlin y ecosistema Android}
Kotlin es un lenguaje de programación moderno y seguro que se utiliza para el
desarrollo de aplicaciones Android. Presenta características como tipado
estático, interoperabilidad con Java, sintaxis concisa y soporte nativo en
Android Studio. Su uso permite crear aplicaciones robustas, escalables y
fáciles de mantener, integrando librerías modernas y marcos de IA para
educación digital \cite{leiva2018kotlin, fedirchuk2018kotlin}.

\subsubsection{Patrón de diseño MVVM}
El patrón de diseño \textbf{Modelo-Vista-Modelo de Vista}
(\textit{Model-View-ViewModel} o MVVM) es una arquitectura de software que
separa de forma clara la lógica de negocio de la interfaz de usuario,
promoviendo el desacoplamiento y facilitando la mantenibilidad del código.
\cite{fowler2015mvvm}

En este patrón, el \textbf{Modelo} encapsula los datos y reglas de negocio; la
\textbf{Vista} representa la interfaz de usuario; y el \textbf{Modelo de Vista}
actúa como un intermediario que gestiona el estado de la vista, expone datos al
usuario y maneja la lógica de presentación. La comunicación suele realizarse
mediante mecanismos de enlace de datos (\textit{data binding}), lo que permite
que la interfaz se actualice automáticamente ante cambios en los datos.
\cite{fowler2015mvvm}

\subsubsection{Actividades (\textit{Activities})}
En Android, una Actividad o \textit{Activity} es uno de los componentes fundamentales de una aplicación y representa una única pantalla con la que el usuario puede interactuar. Cada Actividad administra su propio ciclo de vida, el cual incluye estados como creación, inicio, pausa, reanudación y destrucción. Son el componente encargado de renderizar la interfaz de usuario, manejar eventos y coordinar la navegación hacia otras partes de la aplicación. Las Activities actúan como puntos de entrada principales y permiten organizar la aplicación de forma modular, definiendo flujos independientes dentro del sistema. \cite{android_activities}

\subsubsection{Fragmentos (\textit{Fragments})}
En Android, un Fragmento o \textit{Fragment} es un componente modular de la interfaz de usuario que representa una parte reutilizable de una Actividad. Cada Fragmento posee su propio ciclo de vida, su propia lógica y su propio diseño, pero siempre existe dentro del contexto de una Actividad que lo hospeda. Los Fragmentos permiten construir interfaces flexibles y adaptables, especialmente en pantallas grandes, dividiendo una Actividad en múltiples secciones que pueden combinarse dinámicamente, reemplazarse o reutilizarse en diversos flujos de navegación. Su uso facilita el diseño responsivo, la reutilización de componentes y la separación de responsabilidades dentro de una aplicación Android. \cite{android_fragments}

\subsubsection{Android Jetpack}
Android Jetpack es un conjunto de bibliotecas, herramientas y componentes de
arquitectura diseñados para facilitar el desarrollo de aplicaciones Android
modernas, robustas y mantenibles. Organizado en cuatro pilares: Arquitectura
(\textit{Architecture}), Interfaz de Usuario (\textit{User Interface} o UI),
Comportamiento (\textit{Behavior}) y base (\textit{Foundation}). Jetpack
proporciona soluciones listas para usar que simplifican tareas comunes como
manejo del ciclo de vida, persistencia de datos, navegación, inyección de
dependencias y diseño de interfaces reactivas. Sus componentes son modulares,
retrocompatibles y siguen prácticas recomendadas, lo que permite a los
desarrolladores construir aplicaciones más escalables, seguras y con menos
código repetitivo. \cite{android_jetpack}

\subsubsection{AppCompat}
AppCompat es una librería de compatibilidad de Android Jetpack que permite
utilizar componentes modernos de la interfaz de usuario en versiones antiguas
del sistema operativo. Proporciona implementaciones retrocompatibles de
elementos clave y comportamientos visuales actualizados, lo que garantiza una
apariencia coherente y funcionalidades recientes incluso en dispositivos con
versiones anteriores del sistema operativo. AppCompat es esencial para mantener
compatibilidad retroactiva y asegurar que una misma aplicación funcione de
forma uniforme en una amplia gama de dispositivos Android.
\cite{android_appcompat}

\subsubsection{Componente de Navegación (\textit{Navigation Component})}
Este es una parte de Android Jetpack diseñada para gestionar la navegación dentro de una aplicación Android de forma estructurada, segura y declarativa. Permite definir en un único gráfico (llamado gráfico de navegación o \textit{Navigation Graph}) todos los destinos y acciones de navegación, facilitando el manejo de transiciones entre Fragmentos, Actividades y pantallas desplegables. Ofrece funcionalidades como animaciones de transición, paso de argumentos tipados (\textit{SafeArgs}), y control de la pila de navegación (\textit{back stack}) sin necesidad de manipular transiciones manuales entre fragmentos. Esto reduce errores comunes, simplifica el código y promueve arquitecturas más limpias y mantenibles. \cite{android_navigation_component}

\subsubsection{NavHostFragment}
Es un contenedor especializado del Componente de Navegación de Android Jetpack
que actúa como el anfitrión donde se muestran los destinos definidos en el
gráfico de navegación. Es el componente responsable de gestionar los cambios de
pantalla al navegar entre Fragmentos, controlar automáticamente el ciclo de
vida de cada destino, la pila de navegación, las transiciones y la integración
con el controlador de navegación (\textit{NavController}). En esencia, sirve
como el punto central que conecta la interfaz de usuario con la lógica de
navegación declarada, permitiendo manejar flujos complejos sin necesidad de
administrar manualmente transacciones de fragmentos.
\cite{navhostfragment_androidx}

\subsubsection{NavigationFragment}
Al utilizar Android Jetpack, este suele ser el término para definir un
fragmento que funciona como destino dentro del Componente de Navegación, aunque
no existe una clase oficial con ese nombre en la librería. En la práctica, un
\textit{NavigationFragment} es cualquier fragmento diseñado para integrarse con
el gráfico de navegación y ser gestionado por un \textit{NavHostFragment}.
Estos fragmentos participan en flujos de navegación declarativos, reciben
argumentos tipados (por medio de \textit{SafeArgs}), controlan su propio ciclo
de vida y pueden iniciar acciones de navegación mediante un
\textit{NavController}. \cite{android_navigation_component}

\subsubsection{NavController}
Este es el componente central del Componente de Navegación de Android Jetpack,
encargado de orquestar las acciones de navegación dentro de una aplicación.
Actúa como intermediario entre el \textit{NavHostFragment} y el gráfico de
navegación, ya que interpreta las acciones declaradas en el gráfico y ejecuta
las transiciones correspondientes entre destinos. Además, administra la pila de
navegación, permite navegar hacia destinos específicos, manejar argumentos
tipados (\textit{SafeArgs}) y coordinar animaciones o comportamientos
especiales definidos en la navegación. \cite{android_navcontroller}

\subsubsection{SafeArgs}
Este es un complemento del Componente de Navegación de Android Jetpack, el cual
permite pasar argumentos entre destinos de manera tipada y segura en tiempo de
compilación. Genera clases y métodos automáticamente a partir del gráfico de
navegación, lo que evita errores comunes asociados al uso de paquetes
(\textit{Bundle}) manuales y proporciona una API clara. Con \textit{SafeArgs},
se permite el envío de datos entre Fragmentos y Actividades usando objetos
generados, lo que garantiza que los tipos coincidan, que los argumentos
requeridos estén presentes y que se reduzca significativamente el riesgo de
fallos en la navegación. \cite{safe_args_android_navigation}

\subsubsection{Serializador (\textit{Serializable})}
Es una interfaz de Java que permite convertir un objeto en una secuencia de
\textit{bytes} con el fin de almacenarlo o transmitirlo. En Android, aunque es
compatible, no es el método más eficiente para pasar datos entre componentes,
ya que su rendimiento es inferior al de \textit{Parcelable}. Su principal
ventaja es la simplicidad: basta con implementar la interfaz para habilitar la
serialización automática del objeto. \cite{java_serializable}

\subsubsection{Parcelador (\textit{Parcelable})}
Es una interfaz específica de Android diseñada para serializar objetos de
manera más rápida y eficiente que el Serializador (\textit{Serializable}). Permite descomponer un
objeto en un paquete llamado \textit{Parcel} para transferirlo entre
Actividades, Fragmentos o servicios. Requiere implementar métodos explícitos
para escribir y reconstruir el objeto, lo que reduce la sobrecarga y mejora el
rendimiento, especialmente en dispositivos móviles donde la optimización es
clave. \cite{android_parcelable}

\subsubsection{Paquete (\textit{Bundle})}
En Android, un \textit{Bundle} es una estructura de datos de tipo llave-valor, utilizada para almacenar y transferir información entre componentes del sistema. Está optimizado para manejar tipos de datos primitivos y objetos que implementan \textit{Parcelable} o \textit{Serializable}. Los \textit{Bundles} se emplean comúnmente para pasar argumentos en la navegación, preservar el estado durante cambios de configuración y enviar datos en eventos del ciclo de vida. Constituyen un mecanismo ligero y eficiente que permite empaquetar información de forma temporal dentro de la arquitectura de componentes de Android. \cite{android_bundle}

\subsubsection{Material Design}
Corresponde a un sistema de diseño creado por \textbf{Google}, el cual
proporciona una guía integral para crear interfaces de usuario consistentes,
accesibles y visualmente atractivas. Se basa en principios como la metáfora del
material, animaciones significativas, jerarquías claras y uso intencional de
colores y tipografías. En Android, sus componentes están disponibles a través
de la librería \textit{Material Components for Android}, la cual ofrece
\textit{widgets} o complementos modernos, patrones de navegación y estilos
personalizables para construir experiencias coherentes en todo el ecosistema
Android. \cite{material_design_m3_website}

\subsubsection{Lenguaje de Marcado Extensible (\textit{Extensible Markup Language} - XML)}
Este es un lenguaje diseñado para almacenar y transportar datos de forma estructurada y legible tanto para humanos como para máquinas. A diferencia de HTML (\textit{HyperText Markup Language} o Lenguaje de Marcado de Hipertexto), su propósito no es describir la presentación, sino definir la estructura y el significado de la información mediante etiquetas personalizables. XML es extensible, jerárquico y basado en texto, lo que lo convierte en un estándar ampliamente utilizado para intercambio de datos, configuraciones y definición de documentos. En Android, XML se utiliza para describir interfaces de usuario, recursos de diseño, valores, animaciones y configuraciones del sistema. \cite{w3c_xml}

\subsubsection{Plantillas con XML}
En Android, las plantillas con XML se refieren a la definición de interfaces de
usuario mediante archivos XML que describen la estructura, diseño y atributos
visuales de cada pantalla o componente. Estos archivos especifican elementos
como elementos de texto, botones, listas, así como sus propiedades (márgenes,
tamaños, colores, disposiciones, etc.). El uso de XML permite separar la lógica
del diseño, mantener una arquitectura más limpia y reutilizar componentes de
interfaz. \textbf{Android Studio} proporciona herramientas visuales y
plantillas predefinidas que facilitan su creación y edición.
\cite{android_xml_layouts}

\subsubsection{ConstraintLayout}
\textit{ConstraintLayout} es un administrador de diseño (\textit{layout manager}) de Android altamente flexible que permite posicionar y organizar vistas mediante restricciones. Estas restricciones definen relaciones entre vistas o entre una vista y su contenedor, lo que permite crear interfaces complejas y responsivas sin necesidad de anidar múltiples plantillas. Ofrece un excelente rendimiento y herramientas visuales como el \textit{Layout Editor} (editor de plantilla), que facilitan la construcción de diseños adaptados a distintas resoluciones y orientaciones de pantalla. \cite{android_constraintlayout}

\subsubsection{Glide}
Se trata de una librería de carga y manejo de imágenes para Android que
facilita la descarga, almacenamiento en caché y visualización eficiente de
imágenes, ya sea desde la web, recursos locales o almacenamiento interno. Está
optimizada para rendimiento y uso de memoria, para ofrecer integración con
listas, animaciones suaves y soporte para imágenes GIF (siglas en inglés para
Formato de Intercambio de Gráficos). Su API simple permite transformar
imágenes, redimensionarlas y cargarlas de forma asíncrona, por lo que es una
herramienta ampliamente utilizada en aplicaciones modernas.
\cite{glide_library}

\subsubsection{Base de Datos \textit{Room}}
Room es una biblioteca de persistencia de datos de Android Jetpack que proporciona una capa de abstracción sobre \textit{SQLite}, lo que facilita el uso de bases de datos locales de forma segura y eficiente. Ofrece validación en tiempo de compilación para consultas SQL (siglas en inglés para Lenguaje de Consulta Estructurada), integración fluida con datos observables (\textit{LiveData}), mapeo automático de entidades, y un ORM (siglas en inglés para Mapeo Objeto-Relacional) ligero que reduce el código repetitivo. Room promueve buenas prácticas al asegurar operaciones seguras en hilos separados y facilitar migraciones y manejo estructurado de datos persistentes. \cite{android_room}

\subsubsection{Mapeo Objeto-Relacional (\textit{Object-Relational Mapping})}
Esta es una técnica que permite relacionar objetos de un lenguaje de programación orientado a objetos con tablas de una base de datos relacional, lo que automatiza la conversión entre ambos modelos. Esto facilita el acceso y la manipulación de datos sin necesidad de escribir consultas SQL manuales, a la vez que reduce errores, mejora la mantenibilidad y promueve un diseño más limpio al desacoplar la lógica de negocio de la capa de persistencia.  \cite{fowler2003patterns}

\subsubsection{Preferencias Compartidas (\textit{SharedPreferences})}
Es un mecanismo ligero de almacenamiento llave-valor en Android, utilizado para
guardar datos persistentes simples como configuraciones, preferencias del
usuario o estados pequeños. La información se almacena en archivos privados XML
de la aplicación y puede leerse o escribirse de manera rápida. Es ideal para
guardar datos que no requieren una base de datos completa, como cadenas de
texto o valores numéricos simples. \cite{android_sharedpreferences}

\subsubsection{ThreeTen}
Es una adaptación para Android de la biblioteca \textit{java.time} introducida
en Java 8. Provee un sistema moderno y robusto para manejar fechas, tiempos,
zonas horarias y duraciones con mayor precisión y seguridad que las clases
tradicionales. La versión más comúnmente utilizada en Android es
\textit{ThreeTenABP} (\textit{Android Backport}), la cual permite usar la API
de tiempo nativa en dispositivos con versiones antiguas de Android.
\cite{threeten_abp}

\subsubsection{Retrofit}
Retrofit es una librería de cliente HTTP para Android y Java que permite
consumir \textit{APIs REST} (siglas en inglés para Transferencia de Estado
Representacional) de forma sencilla. Soporta conversión transparente de JSON
(siglas en inglés para Notación de Objetos de JavaScript) a objetos
Kotlin/Java, manejo asíncrono con corrutinas, interceptores, autenticación y
manejo de errores. Es una herramienta fundamental para aplicaciones móviles que
interactúan con servicios web. \cite{retrofit_library}

\subsubsection{OkHTTP}
Es una librería de cliente HTTP que permite realizar solicitudes de red
eficientes y confiables en Android y Java. Ofrece características avanzadas
como conexión persistente, enrutamiento transparente, manejo automático de
compresión y un potente sistema de interceptores para personalizar solicitudes
y respuestas. Se integra comúnmente con \textit{Retrofit} para proporcionar la
capa de transporte sobre la cual se ejecutan las llamadas HTTP en Android.
\cite{okhttp_library}

\subsubsection{Transferencia de Estado Representacional (\textit{Representational State Transfer} o REST)} Es un estilo arquitectónico para el diseño de servicios web basado en
principios como el uso uniforme de recursos, operaciones estándar HTTP (GET,
POST, PUT, DELETE), ausencia de estado (\textit{statelessness}), y
representaciones intercambiables de información (JSON, XML, etc.). Los
servicios \textit{RESTful} buscan ser simples, escalables y fácilmente
consumibles por clientes diversos. Es uno de los enfoques más utilizados para
APIs modernas debido a su flexibilidad y compatibilidad multiplataforma.
\cite{10.5555/932295}

\subsubsection{Datos Observables (\textit{LiveData})}
Corresponde a una clase observable de Android Jetpack diseñada para contener y
notificar datos de manera reactiva y consciente del ciclo de vida. Solo
notifica cambios a componentes activos, lo que evita fugas de memoria y errores
derivados de actualizaciones fuera del ciclo adecuado. Se utiliza comúnmente
junto con el modelo de vista (\textit{ViewModel}) para mantener datos
persistentes ante rotaciones y cambios de configuración.
\cite{android_livedata}

\subsubsection{Flujo de Estado \textit{StateFlow}}
Es un flujo observable y orientado al estado, provisto por las corrutinas de Kotlin. Representa un valor de estado que siempre está disponible y que emite actualizaciones a todos los suscriptores. A diferencia de los datos observables (\textit{LiveData}), el Flujo de Estado forma parte de la librería de corrutinas y funciona en cualquier capa de la arquitectura, no solo en la capa de UI (siglas en inglés para Interfaz de Usuario). Es ampliamente utilizado en patrones como MVVM (Modelo-Vista-Modelo de Vista) para exponer un estado reactivo de forma concurrente y segura. \cite{stateflow_kotlin}

\subsubsection{Alcance de Modelo de Vista (\textit{ViewModelScope})}
Es un alcance (\textit{scope}) de corrutinas provisto por Android Jetpack que
permite ejecutar tareas asincrónicas dentro de un Modelo de Vista. Todas las
corrutinas lanzadas en este alcance se cancelan automáticamente cuando el
Modelo de Vista es destruido, lo que evita fugas de memoria y simplifica la gestión
del ciclo de vida. Es ideal para operaciones como llamadas a API, acceso a
bases de datos o procesamiento de datos que deben sobrevivir a cambios de
configuración. \cite{androidviewmodelscope}

\subsubsection{Alcance de Corrutina (\textit{CoroutineScope})}
Este es un componente fundamental de las corrutinas de Kotlin que define el
contexto en el cual se ejecutan las mismas, lo que incluye su ciclo de vida y
reglas de cancelación. Permite estructurar tareas asincrónicas de manera
organizada, para que se garantice que todas las corrutinas lanzadas dentro del
mismo alcance puedan cancelarse de forma conjunta. Su uso es esencial para
evitar fugas de memoria y para mantener controlado el comportamiento
asincrónico de una aplicación, especialmente en arquitecturas basadas en
Modelos de Vista, Actividades o servicios. \cite{kotlincoroutinescope}

\subsubsection{Inyección de Dependencias (\textit{Dependency Injection} o DI)} Consiste en un patrón de diseño que consiste en proporcionar a un objeto
las dependencias que necesita en lugar de que él mismo las construya. Este
enfoque fomenta la modularidad, la reutilización de código, las pruebas
unitarias y la separación de responsabilidades. En Android, la DI permite
administrar servicios como repositorios, fuentes de datos, controladores de red
o Modelos de Vista de forma centralizada y escalable, lo que reduce el
acoplamiento y facilita el mantenimiento a largo plazo de la aplicación.
\cite{dependency_injection_principle}

\subsubsection{Hilt}
Este es un marco oficial de Android basado en \textit{Dagger} que proporciona
una solución estandarizada para la inyección de dependencias. Simplifica la
configuración de DI mediante anotaciones, ciclos de vida integrados y módulos
predefinidos para componentes clave del sistema. Gracias a su integración con
Jetpack, \textit{Hilt} reduce significativamente el código repetitivo respecto
al uso de \textit{Dagger} puro, facilita las pruebas y asegura una
inicialización eficiente de dependencias en toda la aplicación.
\cite{hilt_dagger}

\subsubsection{Dagger}
Es un marco de inyección de dependencias para Java y Android que genera código
estático en tiempo de compilación. Esto lo hace extremadamente eficiente en
rendimiento y seguro en términos de tipos, lo que evita la reflexión y reduce
el costo en tiempo de ejecución. \textit{Dagger} organiza las dependencias
mediante módulos y componentes, esto permite crear grafos complejos de objetos
administrados automáticamente. Aunque su configuración puede ser detallada,
ofrece un control fino sobre la construcción y el ciclo de vida de las
dependencias, lo que representa una solución robusta para aplicaciones grandes
y escalables. \cite{dagger_framework}

\subsection{Bases de datos vectoriales y su contraste con bases relacionales}
Las bases de datos vectoriales almacenan representaciones numéricas
(\textit{embeddings}) de información, permitiendo búsquedas semánticas rápidas
y precisas. En cambio, las bases de datos relacionales organizan información en
tablas con relaciones explícitas y consultas estructuradas. Para educación
digital basada en IA, las bases vectoriales permiten recuperar contenido
relevante según el significado, mientras que las relacionales son útiles para
gestión de usuarios, cursos y registros administrativos. Integrar ambos tipos
optimiza tanto la eficiencia semántica como la consistencia estructural de los
datos \cite{johnson2019billion, chaudhuri1995self}. El cuadro
\ref{tab:comparacion-basesdedatos} resume el contraste entre ambos conceptos.

\begin{table}[h!]
    \centering
    \caption{Resumen de Contraste: Bases de Datos Relacionales vs. Vectoriales}
    \label{tab:comparacion-basesdedatos}
    \resizebox{\textwidth}{!}{%
        \begin{tabular}{|l|l|l|}
            \hline
            \textbf{Característica}      & \textbf{Base de Datos Relacional}                 & \textbf{Base de Datos Vectorial}                                     \\
            \hline
            \textbf{Elemento Almacenado} & Filas y Columnas (\textit{Tablas})                & \textbf{Representaciones numéricas} (vectores o \textit{Embeddings}) \\
            \hline
            \textbf{Organización}        & \textbf{Estructurada}, con esquema fijo           & \textbf{Densamente Numérica}, sin esquema fijo                       \\
            \hline
            \textbf{Tipo de Búsqueda}    & \textbf{Exacta}, basada en valores (SQL)          & \textbf{Por Similitud}, basada en significado                        \\
            \hline
            \textbf{Aplicación Clave}    & \textbf{Gestión} (Usuarios, Mensajes, Documentos) & \textbf{Recuperación Semántica} (Contenido)                          \\
            \hline
        \end{tabular}
    }
\end{table}

\subsection{APIs de IA: integración de modelos conversacionales}
Las APIs de IA, como OpenAI y Gemini, permiten integrar modelos de lenguaje
conversacionales en aplicaciones educativas. Estos servicios ofrecen
capacidades de generación de texto, comprensión de lenguaje natural y tutoría
personalizada, facilitando la interacción del estudiante con sistemas de IA. La
integración se realiza mediante solicitudes a la API, manejo de \textit{tokens}
y adaptación de respuestas al contexto educativo, permitiendo desarrollar
tutores digitales eficientes y éticos \cite{openai2023api, google2024gemini}.