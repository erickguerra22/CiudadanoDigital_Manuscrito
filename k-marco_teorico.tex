La relación entre inteligencia artificial (IA) y educación se ha convertido en
un campo de estudio emergente que combina aportes de la pedagogía, la
psicología del aprendizaje y las ciencias de la computación. Diversas
investigaciones han demostrado que los sistemas basados en IA pueden desempeñar
funciones de apoyo al proceso educativo, desde la automatización de tareas
administrativas hasta la personalización de la enseñanza mediante algoritmos de
aprendizaje adaptativo \cite{elstad2024ai,frontiers2025education}. Sin embargo,
más allá de sus aplicaciones instrumentales, la IA plantea un nuevo paradigma
pedagógico que redefine la forma en que se conciben la enseñanza, la
interacción docente-estudiante y la construcción del conocimiento en entornos
digitales.

En este contexto, la formación en valores y ciudadanía adquiere especial
relevancia. Aunque tradicionalmente se ha abordado desde marcos filosóficos y
éticos, su integración con tecnologías emergentes permite explorar nuevas
formas de aprendizaje moral mediadas por el diálogo y la reflexión guiada. La
incorporación de sistemas inteligentes en este ámbito representa tanto una
oportunidad como un desafío: por un lado, posibilita acompañamientos
personalizados que estimulan el pensamiento crítico y la autorregulación ética;
por otro, exige garantizar la responsabilidad, transparencia y confiabilidad de
los modelos utilizados \cite{betterinternet2024,carter2024ethics}.

Desde esta convergencia entre tecnología y formación ética, la literatura
reciente destaca el potencial de los modelos de lenguaje de gran escala (LLMs, por sus siglas en inglés)
para generar entornos conversacionales que promuevan la reflexión moral y la
toma de decisiones fundamentadas \cite{seibt2024llm,frontiers2025psychology}.
Estos sistemas, diseñados bajo principios éticos y pedagógicos, pueden
convertirse en agentes de acompañamiento educativo informal, capaces de
sostener diálogos significativos y culturalmente pertinentes. De esta manera,
la IA no solo actúa como herramienta tecnológica, sino como mediadora cognitiva
y moral, ampliando las posibilidades de aprendizaje y fortaleciendo el
desarrollo ciudadano en la era digital.

\section{Educación ciudadana y valores}
La educación ciudadana constituye un proceso educativo integral orientado a
formar individuos capaces de ejercer sus derechos y deberes de manera
responsable, ética y crítica. Esta formación no se limita al conocimiento de
normas y leyes, sino que promueve valores como la solidaridad, la justicia y el
respeto por la diversidad, esenciales para la convivencia democrática
\cite{unesco2021global, schulz2010iccs}. Además, la educación ciudadana
incorpora competencias sociales y habilidades de pensamiento crítico,
fomentando la participación activa en la comunidad y la toma de decisiones
informadas \cite{bentley2018education}.

\subsection{Educación en valores}
La educación en valores constituye un enfoque pedagógico reconocido a nivel
internacional bajo diversas denominaciones, como educación moral, educación del
carácter o educación ética. Si bien cada una presenta matices particulares y
distintos énfasis, todas comparten la convicción fundamental de que la
formación en valores personales y cívicos representa una responsabilidad
legítima de las instituciones educativas a nivel mundial. En la actualidad,
este ámbito ya no se considera exclusivo del entorno familiar o religioso, pues
diversas investigaciones han evidenciado que una educación desvinculada de los
valores puede limitar de forma significativa el desarrollo integral del
estudiante, tanto en el plano ético como en el académico
\cite{lovat2009values}.

Asimismo, la educación en valores se concibe como un proceso formativo integral
que no solo promueve principios fundamentales de ética y ciudadanía, sino que
se posiciona como un componente esencial y transversal de la calidad educativa.
Lejos de tratarse de un aspecto aislado, establece una relación de mutua
interdependencia con la enseñanza de calidad, al punto de integrarse en una
dinámica de doble hélice que potencia el desarrollo personal, social y
académico del estudiante \cite{lovat2009values}.

\subsection{Formación ciudadana}
La formación ciudadana, bajo el concepto anglosajón de \textit{<<civic
    education>>}, es el conjunto de procesos, formales e informales, mediante los
cuales las personas desarrollan conocimientos, valores, actitudes, habilidades
y compromisos que les permiten participar activamente y de manera crítica en la
vida democrática y comunitaria; éste no está limitado al ámbito escolar ni a
una etapa específica de la vida del individuo, sino que se extiende a lo largo
de su ciclo vital e involucra diversos aspectos externos como la familia, los
medios de comunicación, su comunidad, instituciones educativas, etc.
\cite{crittenden2007civic}

Por lo tanto, la formación ciudadana no se limita a la transmisión de
contenidos normativos sobre el sistema político, sino que incorpora prácticas
educativas activas, como la discusión de temas controversiales, la
participación en acciones colectivas y la reflexión crítica, las cuales han
demostrado tener efectos significativos en el desarrollo de una ciudadanía
activa, consciente y empoderada \cite{crittenden2007civic}.

\subsection{Competencias cívicas fundamentales}
Las competencias cívicas fundamentales son un conjunto integrado de
disposiciones personales y capacidades que permiten a los individuos participar
activamente en sociedades democráticas diversas. De acuerdo con el Consejo de
Europa, estas competencias se organizan en torno a cuatro dimensiones
esenciales: \textbf{los valores} que guían el comportamiento ético; \textbf{las
    actitudes} que predisponen a la apertura y al respeto; \textbf{las habilidades}
necesarias para la interacción democrática; y \textbf{los conocimientos} y
\textbf{la comprensión crítica del mundo} social, político y cultural. Su
desarrollo es clave para convivir como iguales en contextos diversos y
democráticos \cite{barrett2016competences}.

\subsubsection{Valores}
Los valores son creencias fundamentales que orientan a las personas hacia metas
que consideran deseables en la vida. Funcionan como motores de acción y como
criterios que guían la toma de decisiones, al proporcionar marcos de referencia
sobre lo que se considera apropiado pensar o hacer en diversas situaciones.
Estos principios no se limitan a contextos específicos, sino que ofrecen
estándares para evaluar conductas, justificar posturas, elegir entre opciones,
planificar acciones e influir en otros \cite{barrett2016competences}.

\subsubsection{Actitudes}
Las actitudes representan la disposición mental general que una persona adopta
frente a individuos, grupos, instituciones, temas u objetos simbólicos. Esta
orientación suele estar compuesta por cuatro elementos interrelacionados: una
creencia o juicio cognitivo sobre el objeto, una respuesta emocional, una
valoración positiva o negativa, y una inclinación conductual específica hacia
dicho objeto \cite{barrett2016competences}.

\subsubsection{Habilidades}
Las habilidades son capacidades que permiten organizar y ejecutar de forma
eficiente patrones complejos de pensamiento o acción, adaptándolos al contexto
con el propósito de alcanzar un objetivo específico.
\cite{barrett2016competences}

\subsubsection{Conocimientos y Comprensión Crítica}
Los conocimientos representan el conjunto de información que una persona ha
adquirido, mientras que la comprensión crítica implica no solo entender esa
información, sino también valorar de forma reflexiva los sentimientos,
perspectivas y significados asociados a ella. Este tipo de comprensión es
esencial en contextos democráticos e interculturales, ya que permite analizar e
interpretar activamente las situaciones, superando respuestas automáticas o no
conscientes. En ese sentido, favorece la evaluación crítica de lo que se sabe y
de cómo se interpreta el mundo social y político \cite{barrett2016competences}.

\subsection{Educación moral}
La educación moral es el proceso educativo centrado en la moralidad, entendida
principalmente como la adhesión a normas morales y la creencia en su
justificación. Este enfoque puede implicar dos dimensiones fundamentales: por
un lado, la formación moral, que busca desarrollar en los individuos
disposiciones afectivas, conductuales y motivacionales alineadas con esas
normas; y por otro, la indagación moral, que promueve la reflexión crítica y la
construcción de creencias fundamentadas sobre la validez de dichas normas.
Ambas dimensiones pueden ser abordadas de manera complementaria, aunque
conceptualmente son distintas. Además, el autor reconoce que la moralidad
podría abarcar elementos adicionales, como ciertas virtudes o disposiciones
emocionales, cuya formación también puede formar parte significativa de la
educación moral \cite{hand2017moral}.

\subsubsection{Formación Moral}
La formación moral es una dimensión de la educación moral centrada en el
desarrollo de disposiciones afectivas y conductuales que llevan a una persona a
adherirse a normas morales y a responder emocionalmente a ellas. No se trata
únicamente de enseñar qué está bien o mal, sino de fomentar inclinaciones
internas que impulsen a actuar conforme a ciertos estándares, de forma estable
y espontánea. Estas disposiciones pueden incluir sentimientos de satisfacción
cuando se actúa moralmente, incomodidad al violar principios morales, y
expectativas de que otros también se comporten moralmente \cite{hand2017moral}.

Asimismo, este concepto puede abarcar el cultivo de virtudes, entendidas no
solo como inclinaciones a seguir normas, sino como capacidades para moderar
emociones humanas fundamentales. Bajo esta perspectiva, la formación moral no
se reduce a enseñar reglas, sino que apunta a moldear el carácter y las
emociones de forma que apoyen una vida moral \cite{hand2017moral}.

\subsubsection{Indagación Moral}
La indagación moral es la parte de la educación moral que se enfoca en
investigar y evaluar la justificación de las normas morales. Consiste en un
proceso cognitivo mediante el cual se analiza por qué una norma debería ser
aceptada, se examinan los argumentos que la sustentan y se reflexiona
críticamente sobre ellos. Creer en la justificación de una norma no es un
requisito para adherirse a ella, por lo que esta indagación es distinta de la
formación moral, que busca cultivar la adhesión emocional y conductual a esas
normas \cite{hand2017moral}.

En la enseñanza de la indagación moral, es posible adoptar un enfoque
directivo, orientando al individuo hacia una conclusión particular sobre la
validez de una norma, o un enfoque no directivo, en el que se facilita el
análisis y la discusión sin influir en la opinión final. Ambos métodos
promueven la capacidad del individuo para pensar críticamente sobre las normas
morales y su justificación, complementando así la formación moral.
\cite{hand2017moral}

\section{Aprendizaje informal y brecha educativa}
El aprendizaje informal constituye una estrategia educativa que ocurre fuera de
los entornos formales, como escuelas o universidades, y se produce de manera
espontánea en la vida cotidiana. Este tipo de educación fomenta la autonomía
del aprendiz, la creatividad y la resolución de problemas; contribuyendo a
reducir la brecha educativa, especialmente cuando el acceso a la educación
formal es limitado \cite{coombs1968world, mills2014informal}.

\subsection{Educación informal}
La educación informal se refiere a las formas de aprendizaje que ocurren de
manera natural en la vida cotidiana, en una amplia variedad de contextos
geográficos e históricos. Este tipo de educación no se limita a entornos
específicos, sino que suele surgir en espacios donde las personas se sienten
cómodas y con la libertad de socializar entre sí. Aunque este concepto es
asociado tradicionalmente con actividades fuera de la escuela, hoy en día la
educación informal también puede darse dentro de escuelas convencionales o en
organizaciones como el voluntariado juvenil o el movimiento scout.
\cite{mills2014informal}

Este tipo de educación se basa en el diálogo y la conversación, fomentando la
confianza, el respeto y la empatía. No busca imponer resultados específicos,
sino que promueve el aprendizaje a partir de las preocupaciones reales y
cotidianas de las personas; generando cambios positivos y significativos en sus
vidas. Además, la educación informal puede tener un carácter político,
inspirándose en enfoques críticos que buscan que las personas tomen conciencia
de las injusticias sociales y encuentren formas de superarlas, conectando lo
personal con temas sociales y políticos más amplios \cite{mills2014informal}.

\subsection{Autoformación guiada}
La autoformación guiada es un proceso intencional en el que el sujeto
desarrolla su aprendizaje autónomo con el apoyo de una institución, un educador
o un colectivo social. Aunque el aprendiz asume responsabilidad sobre sus
objetivos, recursos, métodos y ritmos, recibe orientación y acompañamiento que
facilitan el desarrollo de su capacidad de aprendizaje y autorregulación
\cite{mills2014informal}.

En este enfoque, la autoformación deja de ser un esfuerzo completamente
solitario o espontáneo para convertirse en una práctica educativa estructurada,
donde el apoyo externo configura las condiciones que permiten que el individuo
desarrolle su propio proyecto formativo y consolide su agencia como aprendiz
activo en contextos educativos no formales \cite{mills2014informal}.

\subsection{Brecha educativa y tecnológica}
La brecha educativa y tecnológica se refiere a las diferencias en el acceso y
aprovechamiento de recursos educativos y tecnológicos entre distintos grupos
sociales. Estas desigualdades afectan la calidad del aprendizaje, limitan la
participación en entornos digitales y pueden amplificar la exclusión social.
Factores como el acceso desigual a internet, dispositivos digitales y
capacitación docente contribuyen a esta brecha, la cual requiere estrategias
integrales de inclusión digital y políticas educativas que promuevan la equidad
\cite{van2005digital, unesco2023monitoring}.

\subsection{Tecnología como herramienta de inclusión educativa}
La tecnología educativa se ha consolidado como una herramienta estratégica para
promover la inclusión educativa, al facilitar el acceso a contenidos y recursos
didácticos a estudiantes con diversidad de contextos, habilidades y
necesidades. Plataformas digitales, dispositivos móviles y herramientas de
aprendizaje asistidas por inteligencia artificial permiten superar barreras
geográficas, socioeconómicas y culturales, contribuyendo a mejorar la equidad
en la educación \cite{teras2022education, unesco2023monitoring}.

\section{Inteligencia Artificial y método socrático digital}
La inteligencia artificial (IA), aplicada a la educación, ofrece oportunidades
para diseñar entornos de aprendizaje interactivos y personalizados. Una de las
estrategias más prometedoras es la implementación de métodos socráticos
digitales, donde los sistemas de IA guían a los estudiantes mediante preguntas
y diálogos reflexivos, estimulando el pensamiento crítico y la autonomía en la
construcción del conocimiento \cite{holmes2019ai, woolf2010building}.

\subsection{Inteligencia Artificial en la educación}
La IA en la educación permite automatizar tareas administrativas, ofrecer
tutorías personalizadas, monitorear el progreso de los estudiantes y adaptar
los contenidos a sus necesidades individuales. Estas aplicaciones han
demostrado mejorar la motivación, la eficiencia del aprendizaje y la calidad de
la enseñanza, siempre que se acompañen de supervisión pedagógica y criterios
éticos claros \cite{elstad2024ai, frontiers2025education, carter2024ethics}.

\subsection{Modelos de Lenguaje de Gran Escala}
Los modelos de lenguaje de gran escala (LLMs, por sus siglas en inglés) son sistemas de inteligencia
artificial entrenados con enormes volúmenes de texto para comprender y generar
lenguaje natural. Estos modelos permiten ofrecer respuestas contextualizadas,
realizar tutorías personalizadas y asistir en la construcción de conocimiento
mediante diálogo interactivo. Su potencial educativo radica en la capacidad de
proporcionar retroalimentación inmediata, adaptada al nivel del estudiante,
fomentando la reflexión crítica y la autoformación \cite{brown2020language,
    raffel2020exploring}.

\subsection{Recuperación aumentada por búsqueda}
La Recuperación Aumentada por Búsqueda (RAG, por sus siglas en inglés) combina modelos de lenguaje con
motores de búsqueda para proporcionar información precisa y contextualizada. En
educación en valores, esta técnica permite que los estudiantes reciban
respuestas fundamentadas en fuentes confiables, promoviendo la reflexión ética
y la resolución de dilemas morales basados en evidencia. RAG amplía las
capacidades de tutoría digital al integrar conocimiento externo con generación
de lenguaje natural \cite{lewis2020retrieval, khandelwal2020generalization}.

\subsection{\textit{Embeddings} y representación vectorial del texto} Los \textit{embeddings} son
representaciones vectoriales de palabras, frases o documentos que capturan sus
significados semánticos. Esta técnica permite que los sistemas de IA comparen y
recuperen información de manera eficiente, midiendo la similitud entre
conceptos y facilitando búsquedas semánticas. En educación, los
\textit{embeddings} permiten vincular preguntas de los estudiantes con
contenidos relevantes, apoyando la personalización del aprendizaje
\cite{mikolov2013efficient, le2014distributed}.

\subsection{Bases de datos vectoriales y búsqueda semántica}
Las bases de datos vectoriales permiten almacenar y consultar
\textit{embeddings} de manera eficiente, habilitando la búsqueda semántica en
grandes volúmenes de información. Este enfoque supera las limitaciones de las
búsquedas basadas en palabras clave, permitiendo que los estudiantes y sistemas
educativos accedan a contenidos relevantes de manera más precisa y
contextualizada, facilitando la recuperación de conocimiento en entornos
digitales \cite{johnson2019billion, han2023comprehensive}.

\subsection{Método socrático aplicado a entornos digitales}
Los entornos digitales permiten implementar el método socrático mediante
sistemas de IA que guían a los estudiantes a través de preguntas reflexivas y
secuencias de razonamiento. Esta estrategia fomenta el pensamiento crítico y la
autonomía, ya que los alumnos deben analizar, argumentar y evaluar sus propias
respuestas antes de recibir retroalimentación. El uso de chatbots y asistentes
inteligentes basados en este método facilita un aprendizaje personalizado y
continuo, replicando la interacción dialógica propia del enfoque socrático
tradicional \cite{favero2024socratic, woolf2010building}.

\subsection{Tutoría personalizada con IA}
La tutoría personalizada con IA permite adaptar los contenidos y las
estrategias de enseñanza al nivel, intereses y ritmo de cada estudiante. Los
sistemas inteligentes analizan patrones de aprendizaje y ofrecen
retroalimentación inmediata, identificando áreas de dificultad y recomendando
recursos específicos. Esta personalización mejora la motivación, la retención
de conocimiento y promueve la autonomía del aprendiz \cite{woolf2010building,
    zawacki-richter2019systematic}.

\subsection{Supervisión pedagógica en sistemas automatizados}
A pesar de la autonomía de los sistemas de IA, la supervisión pedagógica es
esencial para garantizar la calidad del aprendizaje. Docentes y tutores deben
monitorear el funcionamiento de los sistemas automatizados, evaluar la
relevancia y exactitud de las respuestas generadas, y ajustar los parámetros de
personalización según las necesidades de los estudiantes. Este enfoque mixto
asegura que la tecnología complemente, y no reemplace, la guía educativa
\cite{holmes2019ai, luckin2016intelligence}.

\section{Ética en IA educativa}
La ética en IA educativa aborda la responsabilidad en el diseño, implementación
y uso de sistemas inteligentes en contextos de aprendizaje. Incluye
consideraciones sobre privacidad de los datos, equidad, transparencia,
inclusión e impacto social. Garantizar que los estudiantes sean tratados de
manera justa y que los sistemas no reproduzcan sesgos existentes es crucial
para la confianza y efectividad de la educación asistida por IA
\cite{selwyn2019should, williamson2023social}.

\subsection{Principios éticos fundamentales en IA}
Los principios éticos fundamentales en IA incluyen transparencia, justicia, no
discriminación, responsabilidad, privacidad y seguridad. En el ámbito
educativo, estos principios guían el desarrollo de sistemas que respeten la
dignidad del estudiante, promuevan equidad en el aprendizaje y faciliten la
rendición de cuentas por parte de desarrolladores y educadores. La aplicación
de estos principios permite aprovechar el potencial de la IA sin comprometer la
integridad pedagógica \cite{jobin2019global, floridi2018ai}.

\subsection{Prevención de sesgos algorítmicos}
La prevención de sesgos algorítmicos se centra en garantizar que los sistemas
de IA no reproduzcan ni amplifiquen desigualdades existentes en la educación.
Esto implica analizar los datos de entrenamiento, identificar posibles sesgos y
aplicar técnicas de mitigación, como ajuste de ponderaciones, diversificación
de datasets y pruebas de equidad en los resultados. La prevención de sesgos
asegura que todos los estudiantes reciban oportunidades de aprendizaje justas y
equitativas \cite{mehrabi2019survey, binns2018fairness}.

\subsection{Transparencia y explicabilidad en sistemas inteligentes}
La transparencia y explicabilidad son fundamentales para que docentes,
estudiantes y desarrolladores comprendan cómo un sistema de IA toma decisiones.
Esto incluye técnicas de interpretabilidad que permitan visualizar la lógica de
los modelos y justificar las recomendaciones que generan. En educación, la
explicabilidad ayuda a confiar en las decisiones automatizadas, facilita la
supervisión pedagógica y permite detectar errores o sesgos
\cite{doshi2017towards, lipton2018mythos}.

\section{Integridad académica y uso responsable de IA}
El uso responsable de IA en educación implica enseñar a los estudiantes a
utilizar herramientas inteligentes sin vulnerar normas éticas ni académicas.
Esto incluye fomentar la autoría propia, la citación adecuada de fuentes y el
desarrollo de habilidades de pensamiento crítico para interpretar la
información generada por la IA. La integridad académica asegura que la
tecnología complemente el aprendizaje sin reemplazar la reflexión y el esfuerzo
personal \cite{bretag2019academic, eaton2023postplagiarism}.

\subsection{Aplicaciones móviles en la educación}
Las aplicaciones móviles educativas permiten acceder a recursos y experiencias
de aprendizaje en cualquier momento y lugar. Integradas con IA, estas apps
pueden ofrecer tutorías personalizadas, seguimiento del progreso,
retroalimentación inmediata y gamificación del aprendizaje. Su portabilidad y
accesibilidad contribuyen a reducir la brecha educativa y facilitan la
inclusión digital \cite{traxler2009learning, crompton2018use}.

\subsection{Arquitectura cliente-servidor (\textit{frontend}/\textit{backend})}
La arquitectura cliente-servidor es un modelo de diseño de software en el que
el cliente (por ejemplo, una app móvil o navegador web) solicita servicios al
servidor, el cual procesa la información, ejecuta lógica de negocio y responde
con datos. En educación digital, esta arquitectura permite centralizar recursos
educativos, gestionar bases de datos y ofrecer aplicaciones interactivas
seguras y escalables. El \textit{frontend} se encarga de la interfaz y la experiencia de
usuario, mientras que el \textit{backend} gestiona la lógica, la seguridad y la
integración con IA y bases de datos \cite{tanenbaum2007distributed,
    hwang2011cloud}.

\subsection{Kotlin como lenguaje para desarrollo Android}
Kotlin es un lenguaje de programación moderno y seguro que se utiliza para el
desarrollo de aplicaciones Android. Presenta características como tipado
estático, interoperabilidad con Java, sintaxis concisa y soporte nativo en
Android Studio. Su uso permite crear aplicaciones robustas, escalables y
fáciles de mantener, integrando librerías modernas y \textit{frameworks} de IA
para educación digital \cite{leiva2018kotlin, fedirchuk2018kotlin}.

\subsection{\textit{Python} y \textit{Flask} como herramientas para \textit{backend} educativo}
\textit{Python} es un lenguaje de programación versátil y de alto nivel,
ampliamente usado en educación y ciencia de datos. \textit{Flask} es un
micro-\textit{framework} de \textit{Python} que permite construir aplicaciones
web y APIs de manera rápida y sencilla. Combinados, \textit{Python} y
\textit{Flask} son ideales para crear \textit{backends} educativos que integren
IA, bases de datos vectoriales y servicios de tutoría digital, asegurando
flexibilidad, escalabilidad y facilidad de mantenimiento
\cite{grinberg2018flask, lutz2013learning}.

\subsection{Bases de datos vectoriales y su contraste con bases relacionales}
Las bases de datos vectoriales almacenan representaciones numéricas
(\textit{embeddings}) de información, permitiendo búsquedas semánticas rápidas
y precisas. En cambio, las bases de datos relacionales organizan información en
tablas con relaciones explícitas y consultas estructuradas. Para educación
digital basada en IA, las bases vectoriales permiten recuperar contenido
relevante según el significado, mientras que las relacionales son útiles para
gestión de usuarios, cursos y registros administrativos. Integrar ambos tipos
optimiza tanto la eficiencia semántica como la consistencia estructural de los
datos \cite{johnson2019billion, chaudhuri1995self}.

\subsection{APIs de IA (\textit{OpenAI} y \textit{Gemini}): integración de modelos conversacionales}
Las APIs de IA, como \textit{OpenAI} y \textit{Gemini}, permiten integrar modelos de lenguaje
conversacionales en aplicaciones educativas. Estos servicios ofrecen
capacidades de generación de texto, comprensión de lenguaje natural y tutoría
personalizada, facilitando la interacción del estudiante con sistemas de IA. La
integración se realiza mediante solicitudes a la API, manejo de \textit{tokens} y
adaptación de respuestas al contexto educativo, permitiendo desarrollar tutores
digitales eficientes y éticos \cite{openai2023api, google2024gemini}.