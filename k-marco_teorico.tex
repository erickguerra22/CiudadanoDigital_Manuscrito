La relación entre inteligencia artificial (IA) y educación se ha convertido en
un campo de estudio emergente que combina aportes de la pedagogía, la
psicología del aprendizaje y las ciencias de la computación. Diversas
investigaciones han demostrado que los sistemas basados en IA pueden desempeñar
funciones de apoyo al proceso educativo, desde la automatización de tareas
administrativas hasta la personalización de la enseñanza mediante algoritmos de
aprendizaje adaptativo \cite{elstad2024ai,frontiers2025education}. Sin embargo,
más allá de sus aplicaciones instrumentales, la IA plantea un nuevo paradigma
pedagógico que redefine la forma en que se conciben la enseñanza, la
interacción docente-estudiante y la construcción del conocimiento en entornos
digitales.

En este contexto, la formación en valores y ciudadanía adquiere especial
relevancia. Aunque tradicionalmente se ha abordado desde marcos filosóficos y
éticos, su integración con tecnologías emergentes permite explorar nuevas
formas de aprendizaje moral mediadas por el diálogo y la reflexión guiada. La
incorporación de sistemas inteligentes en este ámbito representa tanto una
oportunidad como un desafío: por un lado, posibilita acompañamientos
personalizados que estimulan el pensamiento crítico y la autorregulación ética;
por otro, exige garantizar la responsabilidad, transparencia y confiabilidad de
los modelos utilizados \cite{betterinternet2024,carter2024ethics}.

Desde esta convergencia entre tecnología y formación ética, la literatura
reciente destaca el potencial de los modelos de lenguaje de gran escala (LLMs,
por sus siglas en inglés) para generar entornos conversacionales que promuevan
la reflexión moral y la toma de decisiones fundamentadas
\cite{seibt2024llm,frontiers2025psychology}. Estos sistemas, diseñados bajo
principios éticos y pedagógicos, pueden convertirse en agentes de
acompañamiento educativo informal, capaces de sostener diálogos significativos
y culturalmente pertinentes. De esta manera, la IA no solo actúa como
herramienta tecnológica, sino como mediadora cognitiva y moral, lo que amplia
las posibilidades de aprendizaje y fortalece el desarrollo ciudadano en la era
digital.

\section{Educación ciudadana y valores}
La educación ciudadana constituye un proceso educativo integral orientado a
formar individuos capaces de ejercer sus derechos y deberes de manera
responsable, ética y crítica. Esta formación no se limita al conocimiento de
normas y leyes, sino que promueve valores como la solidaridad, la justicia y el
respeto por la diversidad, esenciales para la convivencia democrática
\cite{unesco2021global, schulz2010iccs}. Además, la educación ciudadana
incorpora competencias sociales y habilidades de pensamiento crítico,
fomentando la participación activa en la comunidad y la toma de decisiones
informadas \cite{bentley2018education}.

\subsection{Educación en valores}
La educación en valores constituye un enfoque pedagógico reconocido a nivel
internacional bajo diversas denominaciones, como educación moral, educación del
carácter o educación ética. Si bien cada una presenta matices particulares y
distintos énfasis, todas comparten la convicción fundamental de que la
formación en valores personales y cívicos representa una responsabilidad
legítima de las instituciones educativas a nivel mundial. En la actualidad,
este ámbito ya no se considera exclusivo del entorno familiar o religioso, pues
diversas investigaciones han evidenciado que una educación desvinculada de los
valores puede limitar de forma significativa el desarrollo integral del
estudiante, tanto en el plano ético como en el académico
\cite{lovat2009values}.

Asimismo, la educación en valores se concibe como un proceso formativo integral
que no solo promueve principios fundamentales de ética y ciudadanía, sino que
se posiciona como un componente esencial y transversal de la calidad educativa.
Lejos de tratarse de un aspecto aislado, establece una relación de mutua
interdependencia con la enseñanza de calidad, al punto de integrarse en una
dinámica de doble hélice que potencia el desarrollo personal, social y
académico del estudiante \cite{lovat2009values}.

\subsection{Formación ciudadana}
La formación ciudadana, bajo el concepto anglosajón de \textit{<<civic
    education>>}, es el conjunto de procesos, formales e informales, mediante los
cuales las personas desarrollan conocimientos, valores, actitudes, habilidades
y compromisos que les permiten participar activamente y de manera crítica en la
vida democrática y comunitaria; éste no está limitado al ámbito escolar ni a
una etapa específica de la vida del individuo, sino que se extiende a lo largo
de su ciclo vital e involucra diversos aspectos externos como la familia, los
medios de comunicación, su comunidad, instituciones educativas, etc.
\cite{crittenden2007civic}

Por lo tanto, la formación ciudadana no se limita a la transmisión de
contenidos normativos sobre el sistema político, sino que incorpora prácticas
educativas activas, como la discusión de temas controversiales, la
participación en acciones colectivas y la reflexión crítica, las cuales han
demostrado tener efectos significativos en el desarrollo de una ciudadanía
activa, consciente y empoderada \cite{crittenden2007civic}.

\subsection{Competencias cívicas fundamentales}
Las competencias cívicas fundamentales son un conjunto integrado de
disposiciones personales y capacidades que permiten a los individuos participar
activamente en sociedades democráticas diversas. De acuerdo con el Consejo de
Europa, estas competencias se organizan en torno a cuatro dimensiones
esenciales: \textbf{los valores} que guían el comportamiento ético; \textbf{las
    actitudes} que predisponen a la apertura y al respeto; \textbf{las habilidades}
necesarias para la interacción democrática; y \textbf{los conocimientos} y
\textbf{la comprensión crítica del mundo} social, político y cultural. Su
desarrollo es clave para convivir como iguales en contextos diversos y
democráticos \cite{barrett2016competences}.

\subsubsection{Valores}
Los valores son creencias fundamentales que orientan a las personas hacia metas
que consideran deseables en la vida. Funcionan como motores de acción y como
criterios que guían la toma de decisiones, al proporcionar marcos de referencia
sobre lo que se considera apropiado pensar o hacer en diversas situaciones.
Estos principios no se limitan a contextos específicos, sino que ofrecen
estándares para evaluar conductas, justificar posturas, elegir entre opciones,
planificar acciones e influir en otros \cite{barrett2016competences}.

\subsubsection{Actitudes}
Las actitudes representan la disposición mental general que una persona adopta
frente a individuos, grupos, instituciones, temas u objetos simbólicos. Esta
orientación suele estar compuesta por cuatro elementos interrelacionados: una
creencia o juicio cognitivo sobre el objeto, una respuesta emocional, una
valoración positiva o negativa, y una inclinación conductual específica hacia
dicho objeto \cite{barrett2016competences}.

\subsubsection{Habilidades}
Las habilidades son capacidades que permiten organizar y ejecutar de forma
eficiente patrones complejos de pensamiento o acción, adaptándolos al contexto
con el propósito de alcanzar un objetivo específico.
\cite{barrett2016competences}

\subsubsection{Conocimientos y Comprensión Crítica}
Los conocimientos representan el conjunto de información que una persona ha
adquirido, mientras que la comprensión crítica implica no solo entender esa
información, sino también valorar de forma reflexiva los sentimientos,
perspectivas y significados asociados a ella. Este tipo de comprensión es
esencial en contextos democráticos e interculturales, ya que permite analizar e
interpretar activamente las situaciones, superando respuestas automáticas o no
conscientes. En ese sentido, favorece la evaluación crítica de lo que se sabe y
de cómo se interpreta el mundo social y político \cite{barrett2016competences}.

\subsection{Educación moral}
La educación moral es el proceso educativo centrado en la moralidad, entendida
principalmente como la adhesión a normas morales y la creencia en su
justificación. Este enfoque puede implicar dos dimensiones fundamentales: por
un lado, la formación moral, que busca desarrollar en los individuos
disposiciones afectivas, conductuales y motivacionales alineadas con esas
normas; y por otro, la indagación moral, que promueve la reflexión crítica y la
construcción de creencias fundamentadas sobre la validez de dichas normas.
Ambas dimensiones pueden ser abordadas de manera complementaria, aunque
conceptualmente son distintas. Además, el autor reconoce que la moralidad
podría abarcar elementos adicionales, como ciertas virtudes o disposiciones
emocionales, cuya formación también puede formar parte significativa de la
educación moral \cite{hand2017moral}.

\subsubsection{Formación Moral}
La formación moral es una dimensión de la educación moral centrada en el
desarrollo de disposiciones afectivas y conductuales que llevan a una persona a
adherirse a normas morales y a responder emocionalmente a ellas. No se trata
únicamente de enseñar qué está bien o mal, sino de fomentar inclinaciones
internas que impulsen a actuar conforme a ciertos estándares, de forma estable
y espontánea. Estas disposiciones pueden incluir sentimientos de satisfacción
cuando se actúa moralmente, incomodidad al violar principios morales, y
expectativas de que otros también se comporten moralmente \cite{hand2017moral}.

Asimismo, este concepto puede abarcar el cultivo de virtudes, entendidas no
solo como inclinaciones a seguir normas, sino como capacidades para moderar
emociones humanas fundamentales. Bajo esta perspectiva, la formación moral no
se reduce a enseñar reglas, sino que apunta a moldear el carácter y las
emociones de forma que apoyen una vida moral \cite{hand2017moral}.

\subsubsection{Indagación Moral}
La indagación moral es la parte de la educación moral que se enfoca en
investigar y evaluar la justificación de las normas morales. Consiste en un
proceso cognitivo mediante el cual se analiza por qué una norma debería ser
aceptada, se examinan los argumentos que la sustentan y se reflexiona
críticamente sobre ellos. Creer en la justificación de una norma no es un
requisito para adherirse a ella, por lo que esta indagación es distinta de la
formación moral, que busca cultivar la adhesión emocional y conductual a esas
normas \cite{hand2017moral}.

En la enseñanza de la indagación moral, es posible adoptar un enfoque
directivo, orientando al individuo hacia una conclusión particular sobre la
validez de una norma, o un enfoque no directivo, en el que se facilita el
análisis y la discusión sin influir en la opinión final. Ambos métodos
promueven la capacidad del individuo para pensar críticamente sobre las normas
morales y su justificación, complementando así la formación moral.
\cite{hand2017moral}

\section{Aprendizaje informal y brecha educativa}
El aprendizaje informal constituye una estrategia educativa que ocurre fuera de
los entornos formales, como escuelas o universidades, y se produce de manera
espontánea en la vida cotidiana. Este tipo de educación fomenta la autonomía
del aprendiz, la creatividad y la resolución de problemas; contribuyendo a
reducir la brecha educativa, especialmente cuando el acceso a la educación
formal es limitado \cite{coombs1968world, mills2014informal}.

\subsection{Educación informal}
La educación informal se refiere a las formas de aprendizaje que ocurren de
manera natural en la vida cotidiana, en una amplia variedad de contextos
geográficos e históricos. Este tipo de educación no se limita a entornos
específicos, sino que suele surgir en espacios donde las personas se sienten
cómodas y con la libertad de socializar entre sí. Aunque este concepto es
asociado tradicionalmente con actividades fuera de la escuela, hoy en día la
educación informal también puede darse dentro de escuelas convencionales o en
organizaciones como el voluntariado juvenil o el movimiento scout.
\cite{mills2014informal}

Este tipo de educación se basa en el diálogo y la conversación, fomentando la
confianza, el respeto y la empatía. No busca imponer resultados específicos,
sino que promueve el aprendizaje a partir de las preocupaciones reales y
cotidianas de las personas; generando cambios positivos y significativos en sus
vidas. Además, la educación informal puede tener un carácter político,
inspirándose en enfoques críticos que buscan que las personas tomen conciencia
de las injusticias sociales y encuentren formas de superarlas, conectando lo
personal con temas sociales y políticos más amplios \cite{mills2014informal}.

\subsection{Autoformación guiada}
La autoformación guiada es un proceso intencional en el que el sujeto
desarrolla su aprendizaje autónomo con el apoyo de una institución, un educador
o un colectivo social. Aunque el aprendiz asume responsabilidad sobre sus
objetivos, recursos, métodos y ritmos, recibe orientación y acompañamiento que
facilitan el desarrollo de su capacidad de aprendizaje y autorregulación
\cite{mills2014informal}.

En este enfoque, la autoformación deja de ser un esfuerzo completamente
solitario o espontáneo para convertirse en una práctica educativa estructurada,
donde el apoyo externo configura las condiciones que permiten que el individuo
desarrolle su propio proyecto formativo y consolide su agencia como aprendiz
activo en contextos educativos no formales \cite{mills2014informal}.

\subsection{Brecha educativa y tecnológica}
La brecha educativa y tecnológica se refiere a las diferencias en el acceso y
aprovechamiento de recursos educativos y tecnológicos entre distintos grupos
sociales. Estas desigualdades afectan la calidad del aprendizaje, limitan la
participación en entornos digitales y pueden amplificar la exclusión social.
Factores como el acceso desigual a internet, dispositivos digitales y
capacitación docente contribuyen a esta brecha, la cual requiere estrategias
integrales de inclusión digital y políticas educativas que promuevan la equidad
\cite{van2005digital, unesco2023monitoring}.

\subsection{Tecnología como herramienta de inclusión educativa}
La tecnología educativa se ha consolidado como una herramienta estratégica para
promover la inclusión educativa, al facilitar el acceso a contenidos y recursos
didácticos a estudiantes con diversidad de contextos, habilidades y
necesidades. Plataformas digitales, dispositivos móviles y herramientas de
aprendizaje asistidas por inteligencia artificial permiten superar barreras
geográficas, socioeconómicas y culturales, contribuyendo a mejorar la equidad
en la educación \cite{teras2022education, unesco2023monitoring}.

\section{Fundamentos de Inteligencia Artificial en Educación}
La inteligencia artificial (IA), aplicada a la educación, ofrece oportunidades
para diseñar entornos de aprendizaje interactivos y personalizados. Una de las
estrategias más prometedoras es la implementación de métodos socráticos
digitales, donde los sistemas de IA guían a los estudiantes mediante preguntas
y diálogos reflexivos, estimulando el pensamiento crítico y la autonomía en la
construcción del conocimiento \cite{holmes2019ai, woolf2010building}.

\subsection{Inteligencia Artificial aplicada a la educación}
La IA en la educación permite automatizar tareas administrativas, ofrecer
tutorías personalizadas, monitorear el progreso de los estudiantes y adaptar
los contenidos a sus necesidades individuales. Estas aplicaciones han
demostrado mejorar la motivación, la eficiencia del aprendizaje y la calidad de
la enseñanza, siempre que se acompañen de supervisión pedagógica y criterios
éticos claros \cite{elstad2024ai, frontiers2025education, carter2024ethics}.

\subsection{Modelos de Lenguaje de Gran Escala}
Los modelos de lenguaje de gran escala (LLMs, por sus siglas en inglés) son
sistemas de inteligencia artificial entrenados con enormes volúmenes de texto
para comprender y generar lenguaje natural. Estos modelos permiten ofrecer
respuestas contextualizadas, realizar tutorías personalizadas y asistir en la
construcción de conocimiento mediante diálogo interactivo. Su potencial
educativo radica en la capacidad de proporcionar retroalimentación inmediata,
adaptada al nivel del estudiante, fomentando la reflexión crítica y la
autoformación \cite{brown2020language, raffel2020exploring}.

\subsection{Arquitectura y funcionamiento de sistemas RAG}
La Recuperación Aumentada por Búsqueda (RAG, por sus siglas en inglés) combina
modelos de lenguaje (LLM, por sus siglas en inglés) con motores de recuperación
de información (IR, por sus siglas en inglés), con el fin de proporcionar
respuestas fundamentadas en el contenido específico deseado. Esta técnica,
introducida por primera vez en 2020 en el artículo \textit{"Retrieval-Augmented
    Generation for Knowledge-Intensive NLP Tasks"}, permite la obtención de
respuestas fundamentadas en fuentes confiables, que bajo el contexto del
sistema promueve la reflexión ética y la resolución de dilemas morales basados
en evidencia. RAG amplía las capacidades de tutoría digital al integrar
conocimiento externo con generación de lenguaje natural
\cite{lewis2020retrieval, khandelwal2020generalization}.

\subsubsection{Arquitectura de RAG}

Un sistema RAG opera en dos fases principales:

\begin{enumerate}
    \item \textbf{Fase de indexación:} Los documentos fuente se procesan mediante:
          \begin{itemize}
              \item Segmentación (chunking) en fragmentos semánticamente coherentes.
              \item Generación de embeddings vectoriales para cada fragmento.
              \item Almacenamiento en bases de datos vectoriales con metadatos.
          \end{itemize}

          Esta fase corresponde con la obtención de conocimiento externo a la
          implementación del sistema. En el artículo original, la figura
          \ref{fig:flujo-rag-teoria} muestra cómo este proceso se implementa mediante la
          combinación de \textit{"memoria paramétrica"} (el LLM utilizado para recibir
          preguntas y generar respuestas) y la \textit{"memoria no paramétrica"} (índice
          vectorial del cual se obtiene el contexto para fundamentar la respuesta), con
          lo cual la generación final obtiene la información requerida para brindar al
          usuario un resultado fundamentado.

          \begin{figure}[H]
              \centering
              \includegraphics[width=0.8\textwidth]{assets/rag\_original.png}
              \caption{Vista general del enfoque aplicado en el artículo de \texttt{Lewis et al}. \cite{lewis2020retrieval}}
              \label{fig:flujo-rag-teoria}
          \end{figure}

    \item \textbf{Fase de inferencia:} Al recibir una consulta:
          \begin{itemize}
              \item Se genera un embedding de la pregunta del usuario.
              \item Se recuperan los K fragmentos más relevantes (típicamente un valor K de entre 3
                    y 10 elementos), conformando el contexto de la consulta.
              \item Se construye un prompt contextualizado que combina la pregunta original con los
                    fragmentos recuperados.
              \item El LLM genera una respuesta basándose exclusivamente en el contexto dado o
                    tomándolo como guía (el enfoque depende de la estructura utilizada para
                    construir el \textit{prompt}) de manera que todo resultado se ve anclado a las
                    fuentes verificadas previamente seleccionadas para alimentar el sistema.

                    Según la implementación, se pueden utilizar distintas estrategias para la
                    combinación de los fragmentos con la generación final. En el documento original
                    se menciona la diferencia de utilizar \textit{RAG sequence} (el modelo
                    selecciona un único documento sobre el cual basará su respuesta) frente a
                    \textit{RAG token} (el modelo genera la respuesta por pasos, seleccionando la
                    fuente a utilizar para cada token independiente, lo que permite combinar más de
                    una fuente).
          \end{itemize}
\end{enumerate}

\subsubsection{Estrategias de segmentación}
El método de segmentación (\textit{chunking}) seleccionado puede llegar a
afectar directamente la calidad de las respuestas obtenidas por el sistema. Las
estrategias más comunes incluyen:
\begin{itemize}
    \item \textbf{Segmentación por tamaño fijo:} se divide el texto en fragmentos de longitud uniforme, independiente de su semántica. Presenta la ventaja de que es el tipo de segmentación más fácil de implementar, ya que basta solamente con establecer un límite de palabras y separar todo el texto en dicho límite. Sin embargo, al no evaluar el sentido semántico en cada fragmento, podría dar lugar a rupturas de contexto o pérdida de sentido. \cite{wang-etal-2025-document}
    \item \textbf{Segmentación semántica:} esta segmentación divide el texto respetando unidades de sentido, identificadas mediante signos de puntuación, saltos de línea, identificación de encabezados, secciones, listas, etc. El propósito de esta estrategia es preservar la continuidad de contexto entre cada fragmento. \cite{wang-etal-2025-document}
    \item \textbf{Segmentación recursiva con solapamiento:} esta técnica divide el texto utilizando alguna técnica anterior, con la peculiaridad de incluir al inicio o al final una parte del fragmento contiguo. Es decir, se define un porcentaje de solapamiento que se refiere a qué tanto del fragmento siguiente (o anterior) se incluirá como parte del nuevo fragmento, de manera que se controle de forma explícita la continuidad. \cite{wang-etal-2025-document}
\end{itemize}

La estrategia de segmentación dependerá de la implementación del sistema RAG
que se desea realizar. Se debe tomar en cuenta que la estrategia utilizada
afectará directamente la calidad de las respuestas, ya que esto define cómo el
LLM obtendrá el contexto del cual se basará para brindar la respuesta a la
consulta dada. \cite{wang-etal-2025-document}

\subsubsection{Ventajas de RAG en educación}
La implementación de sistemas RAG en enfoques educativos, brinda varias
ventajas orientadas al uso de modelos de inteligencia artificial:
\begin{itemize}
    \item \textbf{Reduce alucinaciones} al obligar al sistema a utilizar respuestas fundamentadas en el \textit{corpus} definido para el proyecto. \cite{gupta2024comprehensivesurveyretrievalaugmentedgeneration}
    \item \textbf{Permite actualización del conocimiento} sin reentrenar el modelo; basta solamente con modificar, añadir o eliminar el corpus del proyecto para que el sistema utilice esta nueva información. \cite{gupta2024comprehensivesurveyretrievalaugmentedgeneration}
    \item \textbf{Facilita trazabilidad} y citación de fuentes, lo cual es particularmente importante en contextos educativos en los que es necesario fundamentar de dónde se obtiene toda la información proporcionada. \cite{gupta2024comprehensivesurveyretrievalaugmentedgeneration}
    \item Es apropiado para \textbf{dominios especializados} o corpus limitados, como
          asignaturas o materiales didácticos que no están bien cubiertos en los datos de
          entrenamiento general del LLM.
          \cite{gupta2024comprehensivesurveyretrievalaugmentedgeneration}
\end{itemize}

\subsubsection{Desafíos conocidos}
A pesar de sus beneficios, la implementación de RAG también presenta ciertos
retos a cubrir en proyectos educativos de alto impacto:
\begin{itemize}
    \item \textbf{Dependencia crítica} de la calidad del corpus. La calidad de las respuestas depende estrictamente de la calidad del contenido educativo utilizado para alimentar el modelo, por lo que si los documentos están mal organizados, contienen errores o están desactualizados; la recuperación de contexto será débil. \cite{zheng2025knowshiftqarobustragsystems}
    \item \textbf{Riesgo de fragmentación} que rompa la coherencia contextual. Utilizar una estrategia o combinación de estrategias de segmentación inapropiada, puede llevar a que el contexto obtenido por el modelo pierda de sentido, o bien, que una consulta que sí está relacionada con el contenido del corpus no pueda ser respondida. \cite{zheng2025knowshiftqarobustragsystems}
    \item \textbf{Limitaciones de la ventana de contexto del LLM} a pesar de que el sistema de recuperación diseñado obtenga una base contextual amplia, según el modelo LLM utilizado, generalmente se tiene un límite de tokens permitido para cada consulta, por lo que el \text{prompt} construido también cuenta con limitaciones de longitud y, por lo tanto, del nivel de especificación y claridad exigido al modelo. \cite{zheng2025knowshiftqarobustragsystems}
    \item Es apropiado para \textbf{Dificultad para sintetizar información de múltiples
              fragmentos dispersos} Al combinar varios fragmentos en un solo \textit{prompt},
          si no se ha seleccionado el \textit{corpus} cuidadosamente, se puede incurrir
          en contradicciones o redundancia en las consultas.
          \cite{zheng2025knowshiftqarobustragsystems}
\end{itemize}

\subsection{\textit{Embeddings} y representación semántica del texto} Los \textit{embeddings} son
representaciones vectoriales de palabras, frases o documentos que capturan sus
significados semánticos. Esta técnica permite que los sistemas de IA comparen y
recuperen información de manera eficiente, lo que permite medir la similitud
entre conceptos y facilita búsquedas semánticas. En educación, los
\textit{embeddings} permiten vincular preguntas de los estudiantes con
contenidos relevantes, apoyando la personalización del aprendizaje
\cite{mikolov2013efficient, le2014distributed}.

\subsection{Bases de datos vectoriales y búsqueda semántica}
Las bases de datos vectoriales permiten almacenar y consultar
\textit{embeddings} de manera eficiente, habilitando la búsqueda semántica en
grandes volúmenes de información. Este enfoque supera las limitaciones de las
búsquedas basadas en palabras clave, lo que permite que los estudiantes y sistemas
educativos accedan a contenidos relevantes de manera más precisa y
contextualizada, facilitando la recuperación de conocimiento en entornos
digitales \cite{johnson2019billion, han2023comprehensive}.

\subsection{\textit{Prompt Engineering} y diseño de instrucciones} La disciplina de \textit{Prompt Engineering}
consiste en el diseño de instrucciones efectivas para guiar el comportamiento
de modelos de lenguaje hacia objetivos específicos.
\cite{white2023promptpatterncatalogenhance} El enfoque principal es brindar al
modelo directivas claras con el fin de obtener el resultado final esperado,
enfocados en ser tan específicos y directos como el modelo permita.

\subsubsection{Componentes de un \textit{prompt} efectivo}
\begin{itemize}
    \item \textbf{System prompt:} Define el rol que el asistente debe adoptar ante cada consulta que se le solicite, así como el tono de las respuestas generadas y los límites que debe cumplir (por ejemplo, indicar que debe responder basado solamente en el \textit{corpus} del proyecto e ignorar todo lo externo). \cite{zhou2023largelanguagemodelshumanlevel}
    \item \textbf{Contexto:} Aquí se especifica la base que debe fundamentar todas las respuestas del modelo. En el caso de un sistema RAG, es aquí donde se incluyen todos los fragmentos recuperados de la fuente de datos creada previamente. \cite{zhou2023largelanguagemodelshumanlevel}
    \item \textbf{Instrucción:} Define la tarea específica que se espera que deba cumplir el asistente. Aquí puede ir la pregunta, solicitud de información o generación de contenido multimedia (si aplica). \cite{zhou2023largelanguagemodelshumanlevel}
    \item \textbf{Formato de salida:} Se debe definir también cómo se espera que el modelo responda, ya sea porque se busca obtener una respuesta segmentada en separadores identificables, o bien, para especificar un formato específico. \cite{zhou2023largelanguagemodelshumanlevel}
    \item \textbf{Ejemplos (opcional):} Si se quiere ser aún más explícito sobre cómo se espera que el modelo se comporte, se pueden indicar ejemplos claros del comportamiento que debe tener el modelo a partir de las consultas recibidas. \cite{zhou2023largelanguagemodelshumanlevel}
\end{itemize}

\subsubsection{Estrategias en contextos educativos}
Con el objetivo de enfocar el diseño de \textit{prompts} al campo de la
educación, se enfatizan prácticas específicas que permiten obtener el flujo de
pensamiento del asistente, establecer un tipo de interacción específica con el
estudiante (por ejemplo, el uso del método socrático) o solicitar las fuentes
utilizadas.
\begin{itemize}
    \item \textbf{\textit{Chain-of-thoutht prompting}:} Solicita al modelo el razonamiento que utilizó para responder la pregunta, se exige el paso a paso de cómo llegó hasta la respuesta brindada. \cite{NEURIPS2022_9d560961}
    \item \textbf{\textit{Socratic prompting}:} Indica al modelo que se debe guiar por el método socrático, el cual consiste en incentivar al usuario a obtener una respuesta final por sí mismo, en lugar de brindar una respuesta directa a la consulta dada.\cite{NEURIPS2022_9d560961}
    \item \textbf{\textit{Constitutional AI}:} Incorpora principios éticos en las instrucciones, por ejemplo, la omisión de palabras o temas sensibles.\cite{NEURIPS2022_9d560961}
    \item \textbf{\textit{Retrieval-aware prompting}:} Exige al modelo citar fuentes en todas sus respuestas. Puede ser útil, aunque también vale la pena analizar si conviene más esta estrategia o simplemente almacenar en los metadatos de los fragmentos los documentos originales.\cite{NEURIPS2022_9d560961}
\end{itemize}

\subsection{Tutoría personalizada con IA}
La tutoría personalizada con IA permite adaptar los contenidos y las
estrategias de enseñanza al nivel, intereses y ritmo de cada estudiante. Los
sistemas inteligentes analizan patrones de aprendizaje y ofrecen
retroalimentación inmediata, identificando áreas de dificultad y recomendando
recursos específicos. Esta personalización mejora la motivación, la retención
de conocimiento y promueve la autonomía del aprendiz \cite{woolf2010building,
    zawacki-richter2019systematic}.

\subsection{Método socrático aplicado a entornos digitales}
Los entornos digitales permiten implementar el método socrático mediante
sistemas de IA que guían a los estudiantes a través de preguntas reflexivas y
secuencias de razonamiento. Esta estrategia fomenta el pensamiento crítico y la
autonomía, ya que los alumnos deben analizar, argumentar y evaluar sus propias
respuestas antes de recibir retroalimentación. El uso de chatbots y asistentes
inteligentes basados en este método facilita un aprendizaje personalizado y
continuo, replicando la interacción dialógica propia del enfoque socrático
tradicional \cite{favero2024socratic, woolf2010building}.

\subsection{Métricas de evaluación de chatbots educativos}
La evaluación de asistentes conversacionales educativos requiere métricas más
allá de la precisión técnica, incorporando dimensiones pedagógicas incluso si
no son herramientas planificadas para su utilización en entornos formales
tradicionales de educación. \cite{10.3389/frai.2021.654924}

\subsubsection{Métricas de calidad de respuesta}
A nivel semántico, es importante identificar aspectos que pueden definir una
respuesta dada como exitosa o no, enfocados en la pregunta inicial dada, el
\textit{corpus} utilizado para la alimentación del modelo y las instrucciones
de tono y comprensión indicadas al modelo. \cite{ADAMOPOULOU2020100006}
\begin{itemize}
    \item \textbf{Relevancia:} ¿La respuesta aborda la pregunta planteada?
    \item \textbf{Precisión:} ¿La información es correcta?
    \item \textbf{Completitud:} ¿Cubre todos los aspectos necesarios?
    \item \textbf{Claridad:} ¿Es comprensible para el público objetivo?
\end{itemize}

\subsubsection{Métricas de desempeño técnico}
Por otro lado, a nivel técnico también es útil identificar el desempeño del
sistema en cuanto a tiempos de respuesta, integridad de los componentes, coste
de operaciones y recursos utilizados.
\begin{itemize}
    \item \textbf{Tasa de éxito} (\% de preguntas respondidas apropiadamente)
    \item \textbf{Precisión y Recall} en detección de preguntas fuera de alcance
    \item \textbf{Latencia} (tiempos de respuesta)
    \item \textbf{Consumo de tokens}
\end{itemize}

\subsection{Evaluación de calidad de respuestas en sistemas RAG}
Por su parte, los sistemas que implementan RAG presentan sus propios desafíos
específicos de evaluación relacionados con la recuperación y síntesis de la
información almacenada y recuperada. Es necesario identificar si la
recuperación semántica fue exitosa, si el almacenamiento de los fragmentos y
metadatos se está haciendo correctamente y si el modelo está tomando de
referencia los documentos apropiados dentro del \textit{corpus}.

\subsubsection{Métricas de recuperación (\textit{Retrieval})}
Para poder obtener estas métricas, las pruebas deben consistir de un conjunto de preguntas previamente analizadas, así como la identificación de qué fragmentos son realmente relevantes para abordar cada pregunta.
\begin{itemize}
    \item \textbf{Precisión@K:} Se refiere al porcentaje de fragmentos obtenidos que corresponden a los fragmentos relevantes identificados para la consulta.
    \item \textbf{Recall@K:} Se refiere al porcentaje de fragmentos relevantes recuperados por el sistema. Es decir, de todos los fragmentos relevantes identificados para la consulta, qué porcentaje fue seleccionado por el modelo.
    \item \textbf{\textit{Mean Reciprocal Rank} (MRR):} Del top K obtenido por el modelo, en qué posición se sitúa el primer fragmento relevante identificado. Se guía bajo la siguiente ecuación:
          \begin{equation}
              \text{MRR} = \frac{1}{|N|} \sum_{i=1}^{|N|} \frac{1}{\text{K}_i}
          \end{equation}

          En donde \texttt{K} el número de fragmentos recuperados, \texttt{i} representa
          cada fragmento relevante y \texttt{N} la posición del fragmento dentro del
          top-K.
\end{itemize}

\subsubsection{Métricas de generación (\textit{Generation})}
A diferencia de los sistemas puramente extractivos, los sistemas RAG deben ser
evaluados tanto por la calidad de la información generada como por su fidelidad
al contexto recuperado. Para ello, se emplean métricas que permiten analizar
si la respuesta es coherente, pertinente y derivada correctamente del material
de referencia.

\begin{itemize}
    \item \textbf{Faithfulness (Fidelidad):} Evalúa si la respuesta se limita a
          la información contenida en los documentos recuperados, evitando
          alucinaciones o afirmaciones no sustentadas.
          \cite{shuster2021retrievalaugmentationreduceshallucination}

    \item \textbf{Answer Relevance (Relevancia de la respuesta):} Determina el
          grado en que la respuesta generada responde efectivamente a la pregunta
          formulada, sin desviarse hacia información tangencial o irrelevante.
          \cite{bang2023multitaskmultilingualmultimodalevaluation}

    \item \textbf{Context Utilization (Utilización del contexto):} Mide la
          dependencia efectiva del modelo respecto al contexto recuperado. Una
          respuesta es de alta calidad cuando demuestra haber utilizado
          fragmentos relevantes del corpus y no únicamente el conocimiento previo
          del modelo.
          \cite{lewis2020retrieval}

    \item \textbf{Coherencia y fluidez:} Se relaciona con la estructura interna
          de la respuesta, su legibilidad y consistencia discursiva. En el contexto
          educativo, la claridad también constituye un indicador pedagógico de
          calidad.
          \cite{ADAMOPOULOU2020100006}
\end{itemize}

\subsection{Validación sin usuarios finales: enfoques y limitaciones}
La validación técnica de sistemas educativos sin participación de usuarios
finales presenta limitaciones inherentes, pero permite establecer la viabilidad
funcional del sistema antes de su despliegue \cite{luckin2016intelligence}.
Estos procedimientos son fundamentales en etapas tempranas del desarrollo.

\subsubsection{Enfoques de validación técnica:}
\begin{enumerate}
    \item \textbf{Validación basada en corpus:} Verifica que las respuestas
          provengan de los materiales fuente y que no contengan información ajena
          al contenido educativo.
    \item \textbf{Validación por expertos:} Permite asegurar la pertinencia
          pedagógica y detectar errores conceptuales o éticos en las respuestas.
    \item \textbf{Pruebas de estrés:} Incluyen preguntas fuera de alcance,
          ambiguas o sensibles para evaluar la resiliencia del modelo.
    \item \textbf{Simulación de usuarios:} Reproduce escenarios conversacionales
          con perfiles variados, como estudiantes de diferentes edades o niveles
          educativos.
          \cite{luckin2016intelligence,holmes2019ai,zawacki-richter2019systematic}
\end{enumerate}

\subsubsection{Limitaciones reconocidas:}
\begin{itemize}
    \item No mide el impacto pedagógico real ni el aprendizaje logrado.
    \item La validación manual puede estar sujeta a sesgos del evaluador.
    \item Los escenarios simulados no reflejan completamente la complejidad del uso real.
    \item No existe retroalimentación iterativa basada en la experiencia de usuarios
          finales.
          \cite{luckin2016intelligence,holmes2019ai,zawacki-richter2019systematic}
\end{itemize}

\subsubsection{Implicaciones:}
Los resultados obtenidos de esta validación deben interpretarse como una
demostración de factibilidad técnica y coherencia funcional, pero no como
evidencia de efectividad educativa. Estudios posteriores con usuarios reales
serán necesarios para medir impacto pedagógico y aceptación.
\cite{holmes2019ai}

\subsection{Congruencia y fundamentación en respuestas educativas}
En contextos educativos, además de la precisión técnica, se espera que el
sistema responda de forma congruente con las expectativas del rol asignado
(alumno, tutor o asistente). La \textbf{congruencia conversacional} se define
como el grado en que el sistema actúa según lo esperado ante cada tipo de
pregunta: responder apropiadamente cuando debe hacerlo y abstenerse en
situaciones fuera de alcance \cite{Li2021ConsistencyChatbot}.

Este tipo de métrica puede calcularse empíricamente como el porcentaje de
interacciones en las que el modelo actúa conforme al comportamiento esperado.
En el presente proyecto, dicha métrica complementa las tradicionales de
precisión y fidelidad, proporcionando una visión integral de la calidad del
comportamiento conversacional \cite{Li2021ConsistencyChatbot}.

\section{Ética y Responsabilidad en IA Educativa}
La ética en IA educativa aborda la responsabilidad en el diseño, implementación
y uso de sistemas inteligentes en contextos de aprendizaje. Incluye
consideraciones sobre privacidad de los datos, equidad, transparencia,
inclusión e impacto social. Garantizar que los estudiantes sean tratados de
manera justa y que los sistemas no reproduzcan sesgos existentes es crucial
para la confianza y efectividad de la educación asistida por IA
\cite{selwyn2019should, williamson2023social}.

\subsection{Principios éticos fundamentales en IA}
Los principios éticos fundamentales en IA incluyen transparencia, justicia, no
discriminación, responsabilidad, privacidad y seguridad. En el ámbito
educativo, estos principios guían el desarrollo de sistemas que respeten la
dignidad del estudiante, promuevan equidad en el aprendizaje y faciliten la
rendición de cuentas por parte de desarrolladores y educadores. La aplicación
de estos principios permite aprovechar el potencial de la IA sin comprometer la
integridad pedagógica \cite{jobin2019global, floridi2018ai}.

\subsection{Sesgos algorítmicos y culturales en contextos latinoamericanos}
La prevención de sesgos algorítmicos se centra en garantizar que los sistemas
de IA no reproduzcan ni amplifiquen desigualdades existentes en la educación.
Esto implica analizar los datos de entrenamiento, identificar posibles sesgos y
aplicar técnicas de mitigación, como ajuste de ponderaciones, diversificación
de datasets y pruebas de equidad en los resultados. La prevención de sesgos
asegura que todos los estudiantes reciban oportunidades de aprendizaje justas y
equitativas \cite{mehrabi2019survey, binns2018fairness}.

\subsection{Transparencia y explicabilidad en sistemas inteligentes}
La transparencia y explicabilidad son fundamentales para que docentes,
estudiantes y desarrolladores comprendan cómo un sistema de IA toma decisiones.
Esto incluye técnicas de interpretabilidad que permitan visualizar la lógica de
los modelos y justificar las recomendaciones que generan. En educación, la
explicabilidad ayuda a confiar en las decisiones automatizadas, facilita la
supervisión pedagógica y permite detectar errores o sesgos
\cite{doshi2017towards, lipton2018mythos}.

\subsection{Responsabilidad ante respuestas incorrectas o inadecuadas}
La responsabilidad en sistemas de IA educativa implica definir con claridad los
mecanismos para abordar errores, recomendaciones inadecuadas o información
potencialmente nociva generada por los algoritmos. Cuando un sistema
automatizado produce contenidos incorrectos, los efectos pueden ser
especialmente sensibles en contextos educativos, ya que influyen directamente
en el aprendizaje, la motivación y las decisiones académicas de los estudiantes
\cite{jobin2019global, floridi2018ai}.

Esta responsabilidad recae tanto en los desarrolladores, quienes deben
implementar modelos seguros, mecanismos de verificación y pruebas continuas,
como en los docentes y las instituciones que integran la tecnología. Esto
incluye ofrecer rutas de corrección, permitir retroalimentación humana y
asegurar canales claros para reportar fallos. De esta manera, la IA se integra
como una herramienta asistiva bajo supervisión profesional, en lugar de delegar
completamente la evaluación y orientación pedagógica \cite{jobin2019global,
    floridi2018ai}.

\subsection{Privacidad y seguridad en aplicaciones educativas para menores}
El uso de aplicaciones educativas basadas en IA en contextos escolares requiere
un enfoque riguroso de protección de datos, especialmente cuando se trata de
menores de edad. La información académica, conductual y biométrica recopilada
por estos sistemas constituye un activo sensible que debe ser gestionado bajo
principios de minimización de datos, consentimiento informado y almacenamiento
seguro \cite{selwyn2019should, williamson2023social}.

Organismos internacionales han enfatizado la importancia de resguardar la
identidad digital de los estudiantes, evitar usos secundarios no autorizados y
garantizar que los datos no se utilicen para prácticas discriminatorias o
comerciales. Las instituciones tienen la responsabilidad de establecer
políticas claras de acceso, supervisar proveedores tecnológicos y aplicar
estándares robustos de ciberseguridad. La protección de los datos de menores no
solo es una obligación legal y ética, sino también una condición necesaria para
preservar la confianza en entornos educativos mediados por IA
\cite{selwyn2019should, williamson2023social}.

\subsection{Supervisión pedagógica en sistemas automatizados}
A pesar de la autonomía de los sistemas de IA, la supervisión pedagógica es
esencial para garantizar la calidad del aprendizaje. Docentes y tutores deben
monitorear el funcionamiento de los sistemas automatizados, evaluar la
relevancia y exactitud de las respuestas generadas, y ajustar los parámetros de
personalización según las necesidades de los estudiantes. Este enfoque mixto
asegura que la tecnología complemente, y no reemplace, la guía educativa
\cite{holmes2019ai, luckin2016intelligence}.

\section{Aprendizaje Móvil en Contextos de Recursos Limitados}
El aprendizaje móvil (\textit{mobile learning} o m-learning) se ha convertido
en un medio clave para ampliar el acceso a experiencias educativas,
especialmente en regiones donde las limitaciones tecnológicas, de
infraestructura o económicas dificultan el aprendizaje tradicional. En
contextos con recursos limitados, los dispositivos móviles permiten superar
barreras geográficas y temporales, democratizando oportunidades de acceso a
información, formación técnica y herramientas de apoyo educativo
\cite{traxler2007defining, unesco2013policy}.

Sin embargo, la implementación efectiva del aprendizaje móvil requiere
considerar retos como la disponibilidad de dispositivos, la alfabetización
digital de los usuarios, los costos de conectividad y las brechas de
infraestructura. El diseño de soluciones educativas móviles sostenibles debe
responder a estos factores para garantizar accesibilidad, pertinencia cultural
y equidad tecnológica. \cite{unesco2013policy}

\subsection{Panorama del aprendizaje móvil en Guatemala y Centroamérica}
El crecimiento del aprendizaje móvil en Guatemala y Centroamérica ha sido
gradual pero progresivo, impulsado por iniciativas de digitalización,
comunidades tecnológicas emergentes y el interés institucional por modernizar
procesos educativos y productivos. Según el BID, la región ha avanzado
significativamente en adopción tecnológica, pero aún enfrenta brechas en cuanto
a infraestructura digital, capacidad de investigación e inversión en innovación
\cite{worldbank2022revolution}.

\subsection{Diseño de experiencias móviles para usuarios con baja alfabetización digital}
El diseño de experiencias educativas móviles para usuarios con baja
alfabetización digital requiere estrategias centradas en usabilidad,
simplicidad y acompañamiento formativo. UNESCO destaca que las interfaces
visuales claras, los flujos guiados y los recursos multimedia accesibles pueden
favorecer la participación de usuarios principiantes
\cite{unesco2021reimagining}.

\subsection{Consideraciones de conectividad intermitente y consumo de datos}
En muchos contextos latinoamericanos, incluidos sectores rurales de Guatemala,
el acceso a Internet es costoso e intermitente. Por ello, las aplicaciones
educativas móviles deben optimizar el consumo de datos, ofrecer funcionalidad
fuera de línea y emplear técnicas de sincronización diferida para resguardar el
progreso del usuario cuando no haya conexión \cite{shrestha2010offline,
    unesco2013policy, android_offline_first}.

Prácticas recomendadas incluyen compresión de recursos multimedia,
almacenamiento local temporal, caching inteligente y utilización de formatos
eficientes. La capacidad de operar con conectividad limitada no solo reduce
barreras de acceso, sino que también mejora la adopción sostenida de
herramientas educativas en zonas marginadas \cite{col_offline_design_model}.

\subsection{Aplicaciones móviles en la educación}
Las aplicaciones móviles educativas permiten acceder a recursos y experiencias
de aprendizaje en cualquier momento y lugar. Integradas con IA, estas apps
pueden ofrecer tutorías personalizadas, seguimiento del progreso,
retroalimentación inmediata y gamificación del aprendizaje. Su portabilidad y
accesibilidad contribuyen a reducir la brecha educativa y facilitan la
inclusión digital \cite{traxler2009learning, crompton2018use}.

\section{Tecnologías de Implementación}
El uso responsable de IA en educación implica enseñar a los estudiantes a
utilizar herramientas inteligentes sin vulnerar normas éticas ni académicas.
Esto incluye fomentar la autoría propia, la citación adecuada de fuentes y el
desarrollo de habilidades de pensamiento crítico para interpretar la
información generada por la IA. La integridad académica asegura que la
tecnología complemente el aprendizaje sin reemplazar la reflexión y el esfuerzo
personal \cite{bretag2019academic, eaton2023postplagiarism}.

\subsection{Arquitectura cliente-servidor (\textit{frontend}/\textit{backend})}
La arquitectura cliente-servidor es un modelo de diseño de software en el que
el cliente (por ejemplo, una app móvil o navegador web) solicita servicios al
servidor, el cual procesa la información, ejecuta lógica de negocio y responde
con datos. En educación digital, esta arquitectura permite centralizar recursos
educativos, gestionar bases de datos y ofrecer aplicaciones interactivas
seguras y escalables. El \textit{frontend} se encarga de la interfaz y la experiencia de
usuario, mientras que el \textit{backend} gestiona la lógica, la seguridad y la
integración con IA y bases de datos \cite{tanenbaum2007distributed,
    hwang2011cloud}.

\subsubsection{Patrón de diseño MVC}
El patrón de diseño \textit{Model-View-Controller} (MVC) es una arquitectura
ampliamente utilizada para la construcción de interfaces de usuario. Su
objetivo principal es separar la representación de la información de la lógica
que la gestiona, promoviendo modularidad, reutilización y facilidad de
mantenimiento. \cite{reenskaug1979mvc}

En este patrón, el \textbf{Model} contiene los datos y la lógica de negocio; la
\textbf{View} es responsable de mostrar la información al usuario; y el
\textbf{Controller} actúa como intermediario, recibiendo entradas del usuario y
coordinando las actualizaciones entre el modelo y la vista. Esta separación
permite que cambios en la interfaz no afecten directamente a la lógica del
sistema y viceversa. \cite{reenskaug1979mvc}

El patrón MVC tuvo sus orígenes en el entorno \textit{Smalltalk-80} y ha sido
ampliamente adoptado en múltiples \textit{frameworks} modernos de desarrollo
tanto web como de escritorio debido a su capacidad de estructurar aplicaciones
complejas de forma eficiente \cite{reenskaug1979mvc}.

\subsection{Frameworks de desarrollo móvil: Kotlin y ecosistema Android}
Kotlin es un lenguaje de programación moderno y seguro que se utiliza para el
desarrollo de aplicaciones Android. Presenta características como tipado
estático, interoperabilidad con Java, sintaxis concisa y soporte nativo en
Android Studio. Su uso permite crear aplicaciones robustas, escalables y
fáciles de mantener, integrando librerías modernas y \textit{frameworks} de IA
para educación digital \cite{leiva2018kotlin, fedirchuk2018kotlin}.

\subsubsection{Patrón de diseño MVVM}
El patrón \textit{Model-View-ViewModel} (MVVM) es una arquitectura de software
que separa de forma clara la lógica de negocio de la interfaz de usuario,
promoviendo el desacoplamiento y facilitando la mantenibilidad del código.
\cite{fowler2015mvvm}

En este patrón, el \textbf{Model} encapsula los datos y reglas de negocio; la
\textbf{View} representa la interfaz de usuario; y el \textbf{ViewModel} actúa
como un intermediario que gestiona el estado de la vista, expone datos al
usuario y maneja la lógica de presentación. La comunicación suele realizarse
mediante mecanismos de enlace de datos (\textit{data binding}), lo que permite
que la interfaz se actualice automáticamente ante cambios en los datos.
\cite{fowler2015mvvm}

\subsection{Bases de datos vectoriales y su contraste con bases relacionales}
Las bases de datos vectoriales almacenan representaciones numéricas
(\textit{embeddings}) de información, permitiendo búsquedas semánticas rápidas
y precisas. En cambio, las bases de datos relacionales organizan información en
tablas con relaciones explícitas y consultas estructuradas. Para educación
digital basada en IA, las bases vectoriales permiten recuperar contenido
relevante según el significado, mientras que las relacionales son útiles para
gestión de usuarios, cursos y registros administrativos. Integrar ambos tipos
optimiza tanto la eficiencia semántica como la consistencia estructural de los
datos \cite{johnson2019billion, chaudhuri1995self}.

\subsection{APIs de IA: integración de modelos conversacionales}
Las APIs de IA, como \textit{OpenAI} y \textit{Gemini}, permiten integrar
modelos de lenguaje conversacionales en aplicaciones educativas. Estos
servicios ofrecen capacidades de generación de texto, comprensión de lenguaje
natural y tutoría personalizada, facilitando la interacción del estudiante con
sistemas de IA. La integración se realiza mediante solicitudes a la API, manejo
de \textit{tokens} y adaptación de respuestas al contexto educativo,
permitiendo desarrollar tutores digitales eficientes y éticos
\cite{openai2023api, google2024gemini}.