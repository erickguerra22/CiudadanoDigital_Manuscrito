La rápida expansión de la inteligencia artificial (IA) en el ámbito educativo
ha transformado las dinámicas de enseñanza y aprendizaje, lo que ofrece nuevas
posibilidades para la personalización, la tutoría automatizada y la
retroalimentación inmediata \cite{frontiers2024chatgpt}. Sin embargo, su
adopción sin una orientación ética y pedagógica clara conlleva riesgos
significativos, como la desinformación, la dependencia tecnológica o la
reproducción de sesgos culturales \cite{tulsiani2024chatgpt}. En este
escenario, se vuelve imprescindible diseñar propuestas que integren los
beneficios de la IA con una visión educativa centrada en el desarrollo humano y
la formación en valores.

En América Latina, y particularmente en Guatemala, los esfuerzos por incorporar
tecnologías digitales al sistema educativo han estado marcados por
desigualdades estructurales, carencias de infraestructura y una limitada
capacitación docente en el uso pedagógico de herramientas tecnológicas
\cite{worldbank2022revolution,rivas2023future}. Aunque el Programa Nacional de
Educación en Valores (Acuerdo Ministerial 2810-2023) reconoce la importancia de
fortalecer la formación ciudadana \cite{mineduc2023acuerdo}, estudios
nacionales han evidenciado que su implementación enfrenta obstáculos
relacionados con la falta de recursos, la baja cobertura y la escasa
integración de medios tecnológicos \cite{cien2019diagnostico}. Estas
condiciones reflejan la necesidad de propuestas educativas innovadoras,
accesibles y adaptadas al contexto local que contribuyan a mejorar la educación
cívica y moral.

Los modelos de lenguaje de gran escala (LLMs, por sus siglas en inglés)
representan una oportunidad para atender esta necesidad. Su capacidad de
mantener diálogos personalizados, ajustar el nivel de complejidad y ofrecer
acompañamiento continuo los convierte en herramientas idóneas para promover la
reflexión ética y el pensamiento crítico
\cite{qin2024transforming,cordova2025aiagents}. En particular, el enfoque
\textit{Retrieval-Augmented Generation} (RAG) permite que estos sistemas
fundamenten sus respuestas en documentos educativos verificados, garantizando
precisión y coherencia \cite{levonian2025safechats}. A través de esta técnica,
es posible crear un entorno conversacional confiable, orientado al aprendizaje
moral y al fortalecimiento de competencias ciudadanas.

No obstante, el desarrollo de este tipo de tecnologías exige una atención
especial a los principios de equidad, inclusión y transparencia algorítmica,
tal como lo señalan las recomendaciones éticas de la UNESCO
\cite{unesco2021ethics,unesco2021guidance}. La pertinencia de este proyecto
radica en su propósito de adaptar dichas innovaciones al contexto guatemalteco,
mediante una solución funcional que no solo aproveche los avances de la IA,
sino que también incorpore criterios específicos de la realidad nacional.

En este sentido, el proyecto \textit{Ciudadano Digital} se justifica por tres
razones fundamentales:
\begin{enumerate}
      \item \textbf{Educativa,} porque propone un modelo de acompañamiento informal que complementa
            la labor docente y promueve la reflexión crítica en los jóvenes.
      \item \textbf{Social,} porque busca reducir la brecha tecnológica mediante una aplicación
            accesible en dispositivos de bajo costo y adaptable a contextos con recursos
            limitados.
      \item \textbf{Ética,} porque prioriza la formación ciudadana y moral a través del uso responsable de la inteligencia artificial.
\end{enumerate}

Así, \textit{Ciudadano Digital} responde a una necesidad real del sistema
educativo guatemalteco: contar con herramientas tecnológicas que fortalezcan la
educación en valores, promuevan el pensamiento crítico y fomenten la
participación ciudadana activa desde una perspectiva inclusiva y humanista.