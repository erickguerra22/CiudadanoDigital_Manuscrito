La llegada de modelos conversacionales, como ChatGPT, ha revolucionado el aprendizaje, permitiendo tutorías personalizadas que adaptan el contenido al ritmo del estudiante y ofrecen retroalimentación inmediata, factores clave para mejorar la motivación y el rendimiento académico en contextos diversos \cite{frontiers2024chatgpt}. Sin embargo, investigaciones advierten sobre riesgos como la desinformación o el plagio si estos sistemas se emplean sin supervisión pedagógica, lo que subraya la necesidad de un diseño ético y guiado \cite{tulsiani2024chatgpt}. Además, la inteligencia artificial (IA) está transformando la gestión educativa al automatizar tareas administrativas como la calificación de exámenes y el seguimiento de asistencia, liberando tiempo para que los docentes se centren en procesos pedagógicos de mayor valor \cite{unesco2023monitoring}.

En América Latina, las plataformas de IA están extendiendo recursos digitales a zonas rurales, reduciendo brechas de cobertura; sin embargo, su escalabilidad sigue limitada por deficiencias en infraestructura y falta de formación técnica docente \cite{worldbank2022revolution,rivas2023future}. En el caso de Guatemala, el Acuerdo Ministerial 2810-2023 estableció el Programa Nacional de Educación en Valores para priorizar la formación ciudadana \cite{mineduc2023acuerdo}, aunque el Diagnóstico del CIEN, realizado en 2019 evidenció problemas persistentes en cobertura, eficiencia, calidad y ausencia de estrategias de tecnología educativa \cite{cien2019diagnostico}.

En este contexto, los modelos de lenguaje (LLMs) ofrecen una propuesta tecnológica altamente pertinente, en este caso, como tutores en valores y ciudadanía. Su capacidad para imitar patrones de diálogo humano, que incluyen desde el ajuste del tono al hablar hasta el nivel de complejidad, facilita interacciones conversacionales continuas que se acoplan al contexto que se esté tratando, similar a como lo haría un tutor humano para con el estudiante \cite{qin2024transforming}. Diversos estudios han demostrado que los LLMs pueden fomentar la auto-reflexión y el pensamiento crítico mediante estrategias como el método Socrático que, en algunos casos, alcanzan e incluso llegan a superar niveles de efectividad comparables a cuestionarios estructurados \cite{cordova2025aiagents}. En particular, la técnica de Retrieval-Augmented Generation (RAG), permite que los LLMs fundamenten sus respuestas en documentos específicos (por ejemplo, guías curriculares, casos de estudio resueltos), de manera que se reducen enormemente las respuestas imaginadas por la IA, a la vez que se mejora la confianza en la herramienta \cite{cordova2025aiagents,levonian2025safechats}.

Los LLMs, por tanto, se diferencian de otras herramientas de IA al ofrecer una tutoría que se adapta al estilo de aprendizaje y comprensión del estudiante, ofrece retroalimentación inmediata, y mantiene el diálogo necesario para estimular el juicio ético en casos de la vida cotidiana del estudiante; todo ello sin pretender sustituir la educación tradicional, sino complementando el aprendizaje teórico al permitir que el estudiante fortalezca su rol como ciudadano en el día a día. Para asegurar que este sistema contribuya a reducir en lugar de profundizar brechas, es necesario entrenarlo con materiales éticos y contextuales, y mantener supervisión continua por educadores y profesionales en el área \cite{unesco2021ethics,unesco2021guidance}. Estas condiciones permiten generar interacciones fundamentadas y culturalmente pertinentes, garantizando una tutoría ética y efectiva en valores y ciudadanía.