La inteligencia artificial (IA) ofrece un potencial transformador para la
educación, especialmente en contextos marcados por desigualdades sociales,
económicas y tecnológicas. En países como Guatemala, donde persiste una amplia
brecha educativa, la IA puede convertirse en una herramienta clave para
facilitar el acceso a aprendizajes significativos.

Uno de los ámbitos más desatendidos es la formación ciudadana y el desarrollo
de valores morales que, aunque incluidos en los programas educativos, suelen
abordarse de forma teórica y desvinculada de la realidad social. Es aquí donde
se plantea la importancia de una herramienta que brinde acompañamiento a los
estudiantes en la reflexión sobre su papel como ciudadanos y en la práctica de
valores como el respeto, la empatía y la responsabilidad. Este proyecto combina
tecnología accesible con un enfoque centrado en el usuario para fortalecer no
solo el aprendizaje, sino también la conciencia social y ética de los jóvenes.

La implementación de esta herramienta incluye el procesamiento de contenidos
educativos, el desarrollo de un asistente de IA adaptado a los contenidos
seleccionados, a través de una aplicación móvil interactiva y, finalmente, la
validación directa de la interacción con la herramienta para verificar que las
respuestas dadas coincidan con el material proporcionado. Como resultado, se
obtuvo una solución funcional, innovadora y escalable que demuestra cómo la IA
puede contribuir significativamente al fortalecimiento de la educación en
valores en entornos con recursos limitados.