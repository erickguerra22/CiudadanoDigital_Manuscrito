El desarrollo del proyecto \textit{Ciudadano Digital} se llevó a cabo bajo el marco de
trabajo SCRUM, un enfoque ágil ampliamente utilizado en ingeniería de software
que permite la entrega incremental de productos funcionales mediante ciclos
cortos de desarrollo denominados \textit{sprints}. Esta metodología fue
seleccionada debido a su flexibilidad, capacidad de adaptación a cambios en los
requerimientos y enfoque en la mejora continua, elementos clave en un proyecto
de innovación educativa como el presente.

A lo largo del proceso, se definieron seis \textit{sprints} principales, cada
uno con objetivos concretos y entregables verificables, orientados a la
obtención progresiva de un prototipo funcional y validado de la aplicación.
Cada \textit{sprint} tuvo una duración de entre tres y cuatro semanas,
ajustándose según la complejidad técnica y la carga académica del periodo
correspondiente.

Cada ciclo SCRUM siguió las fases de planificación, desarrollo, revisión y
retrospectiva, bajo los siguientes principios:

\begin{itemize}
      \item Planificación (\textit{Sprint Planning}): se definieron los objetivos y alcance
            del \textit{sprint}, así como las tareas específicas necesarias para cumplir la
            meta establecida.
      \item Desarrollo (\textit{Sprint Execution}): se ejecutaron las tareas asignadas con
            enfoque en la funcionalidad incremental, priorizando siempre la obtención de
            resultados medibles.
      \item Revisión (\textit{Sprint Review}): al cierre de cada \textit{sprint}, se evaluó
            el cumplimiento de los objetivos, la calidad del producto obtenido y la
            satisfacción de los criterios de aceptación definidos.
      \item Retrospectiva (\textit{Sprint Retrospective}): se analizaron los aprendizajes
            obtenidos, los obstáculos encontrados y las oportunidades de mejora para el
            siguiente \textit{sprint}.
\end{itemize}

El enfoque SCRUM permitió mantener un flujo de trabajo iterativo, controlado y
adaptable, asegurando que cada componente técnico se validara en función de la
experiencia real del usuario objetivo. En este caso, el usuario fue
representado a través de una Persona desarrollada con base en un proceso de
investigación y perfilamiento descrito en el primer \textit{sprint}.

A partir del segundo \textit{sprint}, los entregables se enfocaron en la
construcción progresiva del sistema técnico, desde la recopilación y
procesamiento de contenido educativo, hasta la implementación del
\textit{backend}, el desarrollo de la interfaz móvil y las fases finales de
validación y documentación.

El producto mínimo viable (MVP) obtenido al finalizar el último \textit{sprint}
constituye una versión funcional del asistente inteligente de educación
ciudadana, capaz de interactuar con el usuario, contextualizar sus preguntas y
generar respuestas basadas en la información previamente curada y vectorizada.

\section{Enfoque metodológico aplicado al contexto del proyecto}
A diferencia de proyectos puramente técnicos, \textit{Ciudadano Digital} combina
aspectos de ingeniería de software, inteligencia artificial y educación en
valores. Por ello, la aplicación de SCRUM fue adaptada a un enfoque
sociotécnico, que no solo prioriza la funcionalidad del sistema, sino también
la pertinencia ética y pedagógica del contenido.

En cada \textit{sprint}, se incluyeron tareas de análisis cualitativo y
cuantitativo relacionadas con el perfil del usuario objetivo: estudiantes de
nivel medio en Guatemala, de entre 15 y 18 años, con acceso limitado a
formación cívica más allá del aula formal. Este enfoque garantizó que las
decisiones técnicas (estructura del \textit{backend}, procesamiento de datos,
interfaz y validación) respondieran a necesidades reales detectadas en el
público meta.

Así, el proceso metodológico buscó alinear el desarrollo tecnológico con la
misión educativa del proyecto, entendiendo que la calidad del producto no se
mide solo por su rendimiento, sino también por su capacidad de promover la
reflexión moral y la ciudadanía responsable en contextos informales de
aprendizaje.

\subsection{Estructura de los \textit{Sprints}}
\begin{table}[H]
      \centering
      \renewcommand{\arraystretch}{1.2}
      \begin{tabular}{|l|p{8cm}|c|}
            \hline
            \textbf{\textit{Sprint}} & \textbf{Meta Principal}                                  & \textbf{Duración Estimada} \\ \hline
            \textit{Sprint} 1        & Identificación del perfil de usuario objetivo (Persona). & 3 semanas                  \\ \hline
            \textit{Sprint} 2        & Recolección y procesamiento del contenido educativo.     & 4 semanas                  \\ \hline
            \textit{Sprint} 3        & Construcción e implementación del \textit{backend}.      & 4 semanas                  \\ \hline
            \textit{Sprint} 4        & Desarrollo de la interfaz móvil en Kotlin.               & 4 semanas                  \\ \hline
            \textit{Sprint} 5        & Pruebas y validación funcional.                          & 3 semanas                  \\ \hline
            \textit{Sprint} 6        & Documentación, presentación y cierre del proyecto.       & 3 semanas                  \\ \hline
      \end{tabular}
      \caption[Estructura de los \textit{Sprints}]{Estructura de los \textit{sprints} del proyecto, incluyendo la meta principal y duración estimada de cada uno.}
      \label{tab:estructura-sprints}
\end{table}

Cada \textit{sprint} culminó con un entregable verificable que sirvió como
criterio de avance para el siguiente ciclo, asegurando así la trazabilidad y
coherencia entre la visión inicial del proyecto y el producto final obtenido.

\section{\textit{Sprint} 1: Identificación del perfil de usuario objetivo}
\textbf{Duración estimada:} 3 semanas

Este \textit{sprint} tuvo como objetivo desarrollar un perfil de usuario
(Persona) que sirviera como insumo accionable para orientar las decisiones de
diseño interactivo y priorización técnica del proyecto. Dado que no fue posible
realizar entrevistas ni trabajo de campo, el perfil se elaboró exclusivamente a
partir del análisis de fuentes documentales que reflejan la situación actual de
los estudiantes en el país, considerando aspectos demográficos, académicos y
sociales. Con base en esta información, se construyó una ficha de Persona
completa, acompañada de criterios de diseño alineados con las necesidades y
características identificadas.

\subsection{Objetivo}
Definir y validar un perfil de usuario objetivo representativo (Persona),
incluyendo sus características sociodemográficas, motivaciones, frustraciones,
competencias digitales y contextos de uso, que sirva como referencia para guiar
las decisiones de diseño UX (\textit{User Experience}), tono comunicativo,
prioridades de contenido y criterios de evaluación de usabilidad del asistente.

\subsection{Ejecución}
Para lograr el objetivo planteado, se llevaron a cabo las siguientes tareas:

\begin{enumerate}
      \item \textbf{Investigación documental}
            \begin{itemize}
                  \item Revisión de informes académicos y/o gubernamentales sobre educación ciudadana,
                        competencias cívicas y valores en jóvenes guatemaltecos.
                  \item Consulta de programas educativos oficiales, como el Currículo Nacional Base
                        (CNB) y materiales de formación en valores del Ministerio de Educación de
                        Guatemala, así como contenido internacional enfocado en brindar una educación
                        más completa.
                  \item Análisis de estudios internacionales de organismos como UNESCO (Organización de
                        las Naciones Unidas para la Educación, la Ciencia y la Cultura), CEPAL
                        (Comisión Económica para América Latina y el Caribe), CIEN (Centro de
                        Investigaciones Económicas Nacionales) y BID (Banco Interamericano de
                        Desarrollo) sobre hábitos digitales, desigualdad educativa y desarrollo de
                        competencias ciudadanas en adolescentes y jóvenes.
            \end{itemize}

      \item \textbf{Análisis e interpretación de la información}
            \begin{itemize}
                  \item Sistematización de datos demográficos, educativos y tecnológicos relevantes
                        para el contexto juvenil guatemalteco.
                  \item Identificación de patrones generales de comportamiento, motivaciones,
                        frustraciones y objetivos (enfocados en aspiraciones cívicas), a partir de
                        tendencias reportadas en las fuentes analizadas.
                  \item Construcción de categorías de análisis que permitieran traducir los hallazgos
                        documentales en insumos para el diseño centrado en el usuario.
            \end{itemize}

      \item \textbf{Definición del perfil Persona}
            \begin{itemize}
                  \item Elaboración de una ficha de usuario basada en la interpretación crítica de los
                        datos documentales, con los siguientes componentes:
                        \begin{itemize}
                              \item \textbf{Perfil base:} edad estimada, nivel educativo, ubicación, etnia, acceso tecnológico y contexto social.
                              \item \textbf{Motivaciones:} interés por la participación comunitaria y el aprendizaje de ciudadanía.
                              \item \textbf{Frustraciones:} barreras de acceso a recursos educativos y desconfianza en la calidad o adecuación de los materiales disponibles.
                              \item \textbf{Objetivos:} qué quisiera conseguir el usuario a través de sus motivaciones y frustraciones, bajo el contexto de educación en valores y formación ciudadana.
                              \item \textbf{Consideraciones especiales:} limitaciones de conectividad, recursos económicos y brechas culturales.
                        \end{itemize}
                  \item Producción de una ficha visual que sirviera como base para las decisiones de
                        diseño en \textit{sprints} posteriores.
            \end{itemize}

      \item \textbf{Documentación de criterios de diseño}
            \begin{itemize}
                  \item Derivación de recomendaciones de diseño UX basadas en el perfil construido:
                        tono comunicativo, estructura de funciones, rol a asumir por el asistente, y
                        adaptabilidad tecnológica.
                  \item Identificación de necesidades prioritarias que el asistente debe ser capaz de
                        abordar a través de la interacción pregunta-respuesta.
            \end{itemize}

\end{enumerate}

\subsection{Resultado final}
Como resultado de este primer \textit{sprint}, se construyó un perfil de
Persona detallado, basado en fuentes documentales, que permitió comprender las
necesidades, barreras y expectativas del usuario objetivo frente a una
herramienta de apoyo educativo.

\begin{itemize}
      \item Edad promedio: 18-22 años.
      \item Contexto educativo: estudiantes de nivel medio y universitario inicial.
      \item Motivaciones: aprender de forma práctica y reflexiva, mejorar su comprensión de
            ciudadanía y valores.
      \item Frustraciones: enseñanza teórica, falta de espacios de diálogo y escasez de
            herramientas interactivas.
      \item Competencias digitales: nivel bajo a medio en uso de aplicaciones y
            herramientas digitales.
      \item Contexto de uso de la aplicación: dispositivos móviles, principalmente Android,
            con sesiones cortas de interacción y preferencia por contenidos dinámicos y
            cercanos a su realidad.
\end{itemize}

Este perfil se utilizó como base para orientar el diseño conversacional, las
estrategias de análisis documental y los lineamientos pedagógicos que guiarán
las siguientes etapas del desarrollo del proyecto.

\section{\textit{Sprint} 2: Recolección y procesamiento del contenido educativo}
\textbf{Duración estimada:} 4 semanas

Este \textit{sprint} se centró en recopilar, procesar y estructurar el
contenido educativo que alimentará al asistente virtual de inteligencia
artificial, con la finalidad de garantizar que el sistema pueda generar
respuestas precisas y contextualizadas sobre formación ciudadana y valores
morales, basándose en información confiable y organizada de manera semántica.
Se combinó la selección documental, curación de contenido, digitalización,
segmentación temática y almacenamiento vectorial de manera sistemática,
asegurando la trazabilidad y calidad de los datos utilizados.

\subsection{Objetivo}
Obtener una base de datos documental organizada, curada y vectorizada, que
permita al asistente virtual ofrecer respuestas contextualizadas, precisas y
alineadas con principios de formación ciudadana y valores morales, considerando
casos prácticos y escenarios de consulta relevantes para el público objetivo
definido en el \textit{Sprint} 1.

\subsection{Ejecución}

Para cumplir el objetivo se desarrolló un proceso sistemático dividido en
cuatro etapas principales: selección documental, curación y digitalización,
segmentación temática, y vectorización (\textit{OpenAI}) con almacenamiento en
\textit{Pinecone}. Este flujo se diseñó de forma reproducible para permitir
futuras ampliaciones o actualizaciones del \textit{corpus} de información.

\begin{enumerate}

      \item \textbf{Selección documental}
            \begin{itemize}
                  \item \textbf{Identificación de fuentes oficiales y confiables:} se recopilaron documentos emitidos por el Ministerio de Educación de Guatemala, tales como el \textit{Currículo Nacional Base (CNB)} y el \textit{Programa Nacional de Educación en Valores} (Acuerdo Ministerial 2810-2023), obtenidos directamente desde los portales institucionales oficiales.
                  \item \textbf{Revisión de fuentes internacionales:} se incorporaron publicaciones y estudios de la UNESCO, CEPAL, BID y CIEN relacionados con educación en valores, ciudadanía digital y formación ética en jóvenes.
                  \item \textbf{Selección de estudios y casos prácticos:} se priorizaron textos académicos que presentaran dilemas morales o ejemplos reales de aplicación de valores en contextos educativos, para fortalecer la capacidad contextual del asistente.
                  \item \textbf{Registro de metadatos:} cada documento fue registrado en un archivo \texttt{sources.csv}, incluyendo:
                        \begin{itemize}
                              \item Identificador único (\texttt{ID})
                              \item Título del documento
                              \item Fuente o institución
                              \item Año de publicación
                              \item Categoría temática (ética, civismo, convivencia, etc.)
                              \item Tipo de documento (guía, informe, artículo, caso práctico)
                        \end{itemize}
                        Este registro garantiza trazabilidad desde la fuente original hasta el fragmento vectorizado.
            \end{itemize}

      \item \textbf{Curación y digitalización}
            \begin{itemize}
                  \item \textbf{Conversión de documentos:} los archivos se transformaron a texto plano (\texttt{.txt}) con codificación UTF-8 mediante herramientas como \texttt{pdftotext} o \texttt{Tesseract OCR}.
                  \item \textbf{Limpieza y normalización:} se eliminaron encabezados, saltos de línea innecesarios y caracteres especiales, para lo cual se aplicó el siguiente proceso:

                        \begin{algorithm}[H]
                              \caption{Proceso de limpieza y normalización de texto}
                              \label{alg:limpiar-texto}
                              \begin{algorithmic}[1]
                                    \Procedure{LimpiarTexto}{archivo\_entrada, archivo\_salida}
                                    \State texto $\gets$ LeerArchivo(archivo\_entrada, ``utf-8'')
                                    \State texto $\gets$ EliminarSaltosDeLínea(texto)
                                    \State texto $\gets$ EliminarEspaciosRepetidos(texto)
                                    \State texto $\gets$ EliminarCaracteresEspeciales(texto)
                                    \State EscribirArchivo(archivo\_salida, texto, ``utf-8'')
                                    \EndProcedure
                              \end{algorithmic}
                        \end{algorithm}

                  \item \textbf{Estandarización de formato:} se uniformaron títulos y subtítulos con reglas jerárquicas para facilitar la segmentación automática.
                  \item \textbf{Validación de integridad:} se verificó que los textos conservaran coherencia y completitud, eliminando duplicados o secciones ilegibles.
                  \item \textbf{Respaldo del \textit{corpus} curado:} los textos limpios se almacenaron en \texttt{/data/cleaned/}, junto con un índice de control que vincula cada documento con su \texttt{ID}.
            \end{itemize}

      \item \textbf{Segmentación temática}
            \begin{itemize}
                  \item \textbf{Diseño del esquema de categorías:} se definieron seis temas guía: \textit{ética y moral}, \textit{participación ciudadana}, \textit{derechos humanos}, \textit{convivencia y respeto}, \textit{responsabilidad social} y \textit{cultura digital}.
                  \item \textbf{División en fragmentos:} los textos fueron segmentados automáticamente en bloques de 300-500 palabras, conservando coherencia semántica.
                  \item \textbf{Etiquetado y registro:} cada fragmento se asoció a una categoría temática y se registró en \texttt{fragments.csv} con los campos:
                        \texttt{fragment\_id}, \texttt{doc\_id}, \texttt{category}, \texttt{text}, \texttt{source}.
                  \item \textbf{Control de calidad:} se revisó manualmente una muestra del 10\% de los fragmentos para validar su correcta asignación temática y coherencia contextual.
            \end{itemize}

      \item \textbf{Vectorización y almacenamiento en \textit{Pinecone}}
            \begin{itemize}
                  \item \textbf{Generación de embeddings:} cada fragmento fue procesado con el modelo \textit{text-embedding-3-small} de OpenAI, generando vectores de 1536 dimensiones:

                        \begin{algorithm}[H]
                              \caption{Generación de embeddings y registro de vectores}
                              \label{alg:vectorizar-fragmentos}
                              \begin{algorithmic}[1]
                                    \Procedure{VectorizarFragmentos}{fragmentos}
                                    \For{frag \textbf{en} fragmentos}
                                    \State vector $\gets$ GenerarEmbedding(modelo=``text-embedding-3-small'', texto=frag.texto)
                                    \State Registrar(frag.fragment\_id, vector, frag.metadatos)
                                    \EndFor
                                    \EndProcedure
                              \end{algorithmic}
                        \end{algorithm}

                  \item \textbf{Normalización final:} se verificó la unicidad de cada \texttt{fragment\_id} y la ausencia de duplicados o errores de codificación.
                  \item \textbf{Creación del índice vectorial:} se configuró un índice en Pinecone con los parámetros:
                        \begin{itemize}
                              \item \texttt{namespace = ``valores-ciudadania''}
                              \item \texttt{metric = ``cosine''}
                              \item \texttt{dimension = 1536}
                        \end{itemize}
                  \item \textbf{Inserción de vectores:} los embeddings se insertaron junto con sus metadatos (fuente, categoría, bloque, documento de origen) para permitir consultas semánticas eficientes.
                  \item \textbf{Validación de consultas:} se realizaron pruebas con preguntas como
                        \textit{“¿Qué es la empatía ciudadana?”} o
                        \textit{“Cómo practicar la responsabilidad social en la escuela”},
                        verificando la recuperación correcta de fragmentos temáticamente relacionados.
            \end{itemize}
\end{enumerate}

\subsection{Resultado final}
Al finalizar el \textit{Sprint} 2, se obtuvo:

\begin{itemize}
      \item Una base documental curada, digitalizada y segmentada en categorías temáticas.
      \item \textit{Embeddings} generados para cada fragmento de texto, con metadatos completos para
            garantizar trazabilidad.
      \item Un índice en \textit{Pinecone} listo para consultas semánticas, capaz de
            proporcionar contexto preciso al asistente virtual para cualquier pregunta del
            usuario.
      \item Establecimiento de un flujo reproducible de selección, curación, segmentación y
            vectorización de contenido para futuras actualizaciones del sistema.
\end{itemize}

Este \textit{sprint} permitió sentar las bases para un sistema de respuesta
contextualizada, alineado con los objetivos de formación ciudadana y valores
morales definidos en el proyecto, asegurando que el asistente virtual cuente
con información confiable, organizada y accesible para generar respuestas
pertinentes y fundamentadas.

\section{\textit{Sprint} 3: Construcción e implementación del \textit{backend}}
\textbf{Duración estimada:} 4 semanas

Este \textit{sprint} se enfocó en el diseño, construcción e implementación de
la arquitectura \textit{backend} del asistente virtual, garantizando la
integración de bases de datos relacionales y vectoriales, y estableciendo la
comunicación segura y eficiente con el modelo de lenguaje (LLM) mediante un
flujo RAG (\textit{Retrieval-Augmented Generation}). Se definieron módulos
claros bajo el patrón de diseño MVC, así como servicios complementarios
internos en \textit{Python} para procesar consultas y generar respuestas
contextualizadas basadas en los documentos previamente vectorizados.

\subsection{Objetivo}
Desarrollar un \textit{backend} robusto y seguro que permita la gestión de
usuarios, almacenamiento de documentos y chats, recuperación de información
semántica desde la base vectorial, y generación de respuestas contextualizadas
desde el LLM, garantizando un flujo eficiente de pregunta-respuesta para
cualquier interfaz de usuario.

\subsection{Ejecución}
\begin{enumerate}
      \item \textbf{Diseño de arquitectura}
            \begin{itemize}
                  \item Se adoptó el patrón de diseño \textbf{Modelo–Vista–Controlador (MVC)} para
                        asegurar modularidad, escalabilidad y mantenimiento del código.
                        \begin{itemize}
                              \item \textbf{Estructura general del proyecto:}
                                    \begin{itemize}
                                          \item /src/
                                                \begin{itemize}
                                                      \item models/ – interacción con la base de datos
                                                      \item controllers/ – lógica de negocio
                                                      \item routes/ – endpoints del API
                                                      \item middlewares/ – validaciones y seguridad
                                                      \item helpers/ – funciones reutilizables
                                                      \item config/ – configuración y variables de entorno
                                                      \item services/ – comunicación con Python (RAG)
                                                \end{itemize}
                                    \end{itemize}
                        \end{itemize}
                  \item \textbf{Modelos:}
                        \begin{itemize}
                              \item Representan entidades del sistema: usuarios, chats, mensajes, sesiones,
                                    documentos.
                              \item Cada modelo incluye operaciones CRUD (\textit{Create, Read, Update, Delete}):
                                    crear, consultar, actualizar y eliminar registros.
                              \item Mantienen integridad referencial entre entidades.
                        \end{itemize}
                  \item \textbf{Rutas (Vistas):} Endpoints REST para interacción con el usuario, como registro, login, listado de chats, envío de preguntas, entre otros.
                  \item \textbf{Controladores:} Gestionan la lógica de negocio: validación de datos, comunicación con modelos,
                        manejo de errores y generación de respuestas.
                        \begin{algorithm}[H]
                              \caption{Controlador de solicitudes}
                              \label{alg:controlador-solicitudes}
                              \begin{algorithmic}[1]
                                    \Procedure{Controlador}{request}
                                    \State Validar(request.datos)
                                    \State resultado $\gets$ modelo.operacion(request)
                                    \State \textbf{devolver}(resultado)
                                    \EndProcedure
                              \end{algorithmic}
                        \end{algorithm}
                  \item \textbf{Módulos auxiliares:}
                        \begin{itemize}
                              \item \textbf{Middlewares:} validan autenticación y seguridad antes de pasar al controlador.
                                    \begin{algorithm}[H]
                                          \caption{Validación de token}
                                          \label{alg:validar-token}
                                          \begin{algorithmic}[1]
                                                \Procedure{ValidarTokenRequest}{request}
                                                \If{not ValidarToken(request.token)}
                                                \State DevolverError(401, ``Token inválido'')
                                                \Else
                                                \State Continuar(request)
                                                \EndIf
                                                \EndProcedure
                                          \end{algorithmic}
                                    \end{algorithm}
                              \item \textbf{Helpers:} funciones reutilizables como:
                                    \begin{itemize}
                                          \item encriptarContraseña(contraseña)
                                          \item generarToken(usuarioID)
                                          \item formatearFecha(fecha)
                                    \end{itemize}
                        \end{itemize}
            \end{itemize}

      \item \textbf{Diseño y construcción de bases de datos}
            \subsubsection{Base de datos relacional}
            \begin{itemize}
                  \item Motor: PostgreSQL en AWS RDS.
                  \item Tablas principales: usuarios, chats, mensajes, sesiones, documentos.
                  \item Relaciones:
                        \begin{itemize}
                              \item Un usuario puede tener varios chats.
                              \item Cada chat contiene múltiples mensajes.
                              \item Cada sesión pertenece a un usuario.
                        \end{itemize}
                  \item Mantenimiento: restricciones de claves foráneas, eliminación en cascada,
                        índices para consultas frecuentes.
                        \begin{algorithm}[H]
                              \caption{Ejemplo: Almacenar mensaje del usuario}
                              \label{alg:guardar-mensaje}
                              \begin{algorithmic}[1]
                                    \Procedure{GuardarMensaje}{usuario, pregunta}
                                    \State chat $\gets$ ObtenerChatActivo(usuario)
                                    \State mensaje $\gets$ \{
                                    \Statex \quad ``chat\_id'': chat.id,
                                    \Statex \quad ``remitente'': ``usuario'',
                                    \Statex \quad ``contenido'': pregunta,
                                    \Statex \quad ``timestamp'': Ahora()
                                    \Statex \}
                                    \State Insertar(mensaje, tabla=``mensajes'')
                                    \EndProcedure
                              \end{algorithmic}
                        \end{algorithm}
            \end{itemize}

            \subsubsection{Base de datos vectorial}
            \begin{itemize}
                  \item Motor: Pinecone, con métrica de similitud \textit{cosine}.
                  \item Contenido: embeddings de fragmentos de documentos con metadatos (fuente,
                        categoría, documento, bloque, relevancia).
                        \begin{algorithm}[H]
                              \caption{Creación de índice de vectores}
                              \label{alg:crear-indice}
                              \begin{algorithmic}[1]
                                    \Procedure{CrearIndice}{}
                                    \State CrearIndice(
                                    \Statex \quad nombre=``ciudadano-digital'',
                                    \Statex \quad dimension=1536,
                                    \Statex \quad metrica=``cosine''
                                    \Statex )
                                    \EndProcedure
                              \end{algorithmic}
                        \end{algorithm}
                  \item Consultas top-K para recuperar los fragmentos más relevantes.
            \end{itemize}

      \item \textbf{Servicio complementario de \textit{Python} para comunicación con LLM}
            \begin{itemize}
                  \item \textbf{Función:} gestionar el flujo de \textit{Retrieval-Augmented Generation (RAG)}, integrando la base vectorial con el modelo de lenguaje para generar respuestas contextualizadas y fundamentadas.
                  \item \textbf{Flujo realizado:}
                        \begin{enumerate}
                              \item \textbf{Recepción de la pregunta en NodeJS:}
                                    \begin{enumerate}
                                          \item El usuario envía una pregunta en texto plano a través de la interfaz de la
                                                aplicación.
                                          \item NodeJS recibe la pregunta y prepara la solicitud para el microservicio Python.
                                    \end{enumerate}

                              \item \textbf{Procesamiento en Python:}
                                    \begin{enumerate}
                                          \item El microservicio Python recibe la pregunta enviada desde NodeJS.
                                          \item Genera un \textit{embedding} del texto de la pregunta utilizando el modelo
                                                \textit{text-embedding-3-small} de OpenAI, transformando la información textual
                                                en un vector semántico.
                                          \item Se realiza una consulta al índice de Pinecone con el \textit{embedding}
                                                generado, recuperando los cinco fragmentos más relevantes del \textit{corpus}
                                                vectorizado, que servirán como contexto para la respuesta.
                                          \item Combina la pregunta original con los fragmentos de contexto obtenidos,
                                                construyendo un \textit{prompt} en texto plano que resume la información
                                                relevante y plantea la consulta al modelo de lenguaje.
                                          \item Envía el \textit{prompt} al modelo LLM para generar la respuesta.
                                          \item Evalúa la respuesta generada, identificando si el contexto proporcionado fue
                                                suficiente para garantizar precisión y pertinencia.
                                          \item Genera un objeto JSON que incluye la pregunta original, la respuesta obtenida y
                                                el tiempo de procesamiento, garantizando trazabilidad de la interacción.
                                    \end{enumerate}

                              \item \textbf{Devolución de la respuesta a NodeJS:}
                                    \begin{enumerate}
                                          \item NodeJS recibe el JSON con la respuesta generada por el LLM.
                                          \item Formatea y entrega la respuesta al usuario final a través de la interfaz de la
                                                aplicación.
                                          \item Simultáneamente, guarda la pregunta y la respuesta en la base de datos
                                                relacional para mantener un historial de interacciones.
                                    \end{enumerate}
                        \end{enumerate}

                        \begin{algorithm}[H]
                              \caption{Flujo completo de procesamiento de preguntas}
                              \label{alg:flujo-completo-preguntas}
                              \begin{algorithmic}[1]
                                    \Procedure{ProcesarPregunta}{usuario}
                                    \State pregunta $\gets$ RecibirPregunta(usuario)
                                    \State EnviarPreguntaAPython(pregunta)
                                    \State embedding $\gets$ GenerarEmbedding(pregunta)
                                    \State contexto $\gets$ ConsultarPinecone(embedding, topK=5)
                                    \State prompt $\gets$ ConstruirPrompt(pregunta, contexto)
                                    \State respuesta $\gets$ ConsultarLLM(prompt)
                                    \State jsonRespuesta $\gets$ ArmarJSON(pregunta, respuesta, tiempoProcesamiento)
                                    \State EnviarAlCliente(jsonRespuesta)
                                    \State GuardarMensaje(usuario, respuesta)
                                    \EndProcedure
                              \end{algorithmic}
                        \end{algorithm}

                  \item \textbf{Validación:} se realizaron pruebas de 20 consultas aleatorias para evaluar:
                        \begin{itemize}
                              \item La pertinencia de los fragmentos de contexto recuperados desde Pinecone.
                              \item La coherencia semántica y precisión de las respuestas generadas por el LLM.
                              \item La correcta integración y comunicación entre NodeJS y el microservicio Python.
                        \end{itemize}
            \end{itemize}
\end{enumerate}

\subsection{Resultado final}
Al finalizar este \textit{sprint}, se obtuvo:

\begin{itemize}
      \item \textit{Backend} modular bajo MVC, con rutas, controladores y modelos independientes.
      \item Base de datos relacional (PostgreSQL) con integridad referencial y seguridad.
      \item Base vectorial en Pinecone, indexada y lista para búsquedas semánticas
            eficientes.
      \item Servicio Python que integra recuperación contextual y generación ética de
            respuestas mediante LLM.
      \item Flujo completo validado: desde envío de pregunta hasta devolución de respuesta
            fundamentada.
\end{itemize}

Este \textit{sprint} consolidó la infraestructura técnica del sistema,
asegurando operación confiable, trazabilidad de datos y escalabilidad futura
para el asistente educativo.

\section{\textit{Sprint} 4: Desarrollo de la interfaz móvil en Kotlin}
\textbf{Duración estimada:} 4 semanas

Este \textit{sprint} tuvo como objetivo diseñar e implementar la interfaz móvil
de la aplicación del asistente educativo, utilizando Kotlin para garantizar
integración nativa con Android y un flujo de interacción intuitivo para el
usuario. Se buscó crear módulos claros y escalables, organizar recursos y
establecer los patrones de navegación y comunicación con el \textit{backend},
asegurando que la aplicación fuera funcional, accesible y adaptable al perfil
de usuario definido en el \textit{Sprint} 1.

\subsection{Objetivo}
Desarrollar una aplicación móvil Android con una interfaz intuitiva y eficiente
que permita a los usuarios interactuar con el asistente virtual, enviar
preguntas, recibir respuestas contextualizadas y gestionar su historial de
interacción, manteniendo consistencia con el perfil de usuario objetivo y las
decisiones de diseño UX previamente definidas.

\subsection{Ejecución}
\begin{enumerate}
      \item \textbf{Definición de módulos y arquitectura de la aplicación}
            \begin{itemize}
                  \item \textbf{Data:} gestión de modelos de datos, repositorios y fuentes de información, incluyendo comunicación con el \textit{backend} vía API y almacenamiento local temporal.
                  \item \textbf{Dependency Injection:} configuración de Hilt para inyectar dependencias de forma eficiente y centralizada, facilitando la escalabilidad y pruebas.
                  \item \textbf{Helpers:} funciones y utilidades reutilizables, como manejo de errores, validación de entradas de usuario, formateo de fechas y gestión de sesiones.
                  \item \textbf{User Interface (UI):} construcción de pantallas, componentes visuales y navegación, siguiendo principios de Material Design y adaptabilidad a distintos tamaños de pantalla.
                  \item \textbf{Resources:} gestión de strings, colores, dimensiones, iconografía y estilos para mantener consistencia visual y facilitar traducciones o ajustes futuros.
            \end{itemize}

      \item \textbf{Diseño de flujo de interacción}
            \begin{itemize}
                  \item Mapeo de pantallas principales: inicio, login, chat con asistente, historial de
                        interacciones y visualización de documentos.
                  \item Implementación de navegación mediante \textit{Navigation Component},
                        garantizando consistencia y control del back stack.
                  \item Diseño de interacción de chat: envío de preguntas, visualización de respuestas
                        con formato enriquecido y mensajes de sistema para notificaciones o errores.
                  \item Integración de indicadores de carga y estado de conexión, ofreciendo
                        retroalimentación inmediata al usuario sobre la consulta al LLM.
            \end{itemize}

      \item \textbf{Integración con \textit{backend} y servicios de \textit{Python}}
            \begin{itemize}
                  \item Consumo de endpoints REST del \textit{backend} para autenticación, gestión de
                        sesiones, envío de preguntas y recuperación de respuestas.
                  \item Procesamiento de respuestas JSON, parseo y renderizado en la interfaz de
                        usuario de manera clara y comprensible.
                  \item Manejo de errores y reconexión ante fallos de red, asegurando robustez en la
                        experiencia de usuario.
            \end{itemize}

      \item \textbf{Pruebas de funcionalidad y usabilidad}
            \begin{itemize}
                  \item Pruebas unitarias y de integración en Kotlin para asegurar correcto
                        funcionamiento de los módulos y la comunicación con el \textit{backend}.
                  \item Pruebas de usabilidad simulando interacción de usuarios representativos según
                        la Persona definida en el \textit{Sprint} 1.
                  \item Ajustes en la interfaz y flujo de navegación basados en observaciones de uso,
                        priorizando claridad, accesibilidad y eficiencia en la interacción.
            \end{itemize}

      \item \textbf{Documentación y guías de uso}
            \begin{itemize}
                  \item Documentación de los módulos implementados, incluyendo dependencias, estructura
                        de carpetas, responsabilidades de cada componente y ejemplos de uso.
                  \item Guía de buenas prácticas para futuras iteraciones y ampliaciones de la
                        aplicación móvil.
            \end{itemize}
\end{enumerate}

\subsection{Resultado final}
Al finalizar este \textit{sprint}, se obtuvo:

\begin{itemize}
      \item Aplicación móvil funcional en Android, con integración nativa mediante Kotlin y
            comunicación estable con el \textit{backend}.
      \item Estructura modular clara (Data, Dependency Injection, Helpers, UI, Resources)
            que facilita mantenimiento y escalabilidad.
      \item Flujo de interacción optimizado para el usuario, incluyendo envío de preguntas,
            recepción de respuestas contextuales y visualización de documentos.
      \item Pruebas internas de usabilidad que validaron la intuitividad del diseño y la
            adecuación mediante simulaciones de uso por el perfil de usuario objetivo.
      \item Documentación completa del \textit{frontend}, con guías para futuras mejoras y
            desarrollo colaborativo.
\end{itemize}

Este \textit{sprint} permitió contar con una interfaz móvil operativa, lista
para el despliegue y pruebas piloto, estableciendo las bases para la fase final
de evaluación y refinamiento del proyecto.

\section{\textit{Sprint} 5: Pruebas y validación}
\textbf{Duración estimada:} 3 semanas

Este \textit{sprint} se centró en validar el funcionamiento integral del
sistema, asegurando la correcta interacción entre la aplicación móvil, el
\textit{backend}, la base de datos relacional y vectorial, y el modelo de
lenguaje (LLM). Además, se buscó evaluar la calidad, precisión y confiabilidad
de las respuestas generadas por el asistente virtual, así como recopilar
retroalimentación mediante pruebas internas para ajustar y mejorar la
herramienta antes de su entrega.

\subsection{Objetivo}
Realizar pruebas funcionales y de usabilidad del sistema, validar la precisión
y confiabilidad de las respuestas del asistente virtual en el contexto de
educación ciudadana y valores morales y aplicar mejoras continuas basadas en
observaciones internas y criterios expertos (con ayuda del asesor),
garantizando un producto final robusto y alineado con el perfil de usuario
objetivo.

\subsection{Ejecución}
\begin{enumerate}
      \item \textbf{Pruebas funcionales del sistema}
            \begin{itemize}
                  \item Verificación de la comunicación entre la aplicación móvil (Kotlin), el
                        \textit{backend} (NodeJS), la base de datos relacional (PostgreSQL) y la base
                        vectorial (\textit{Pinecone}).
                  \item Pruebas de endpoints para asegurar correcta autenticación de usuarios, envío de
                        preguntas, recuperación de respuestas y gestión de historial de chats.
                  \item Comprobación de la integridad de los datos entre los distintos módulos,
                        incluyendo creación, lectura, actualización y eliminación de información
                        (CRUD).
                  \item Simulación de sesiones múltiples para validar estabilidad y manejo de
                        concurrencia.
            \end{itemize}

      \item \textbf{Pruebas de calidad y confiabilidad de las respuestas}
            \begin{itemize}
                  \item Validación directa de las respuestas generadas por el asistente comparando con
                        el contenido base documentado utilizado para entrenar y alimentar al modelo.
                  \item Evaluación por parte del asesor, revisando pertinencia, claridad y adecuación
                        pedagógica de las respuestas.
                  \item Identificación de casos en los que el modelo no proporcione información
                        suficiente o presente inconsistencias, documentando hallazgos para ajuste de
                        contenido o configuración de \textit{embeddings}.
            \end{itemize}

      \item \textbf{Análisis de resultados y mejora continua}
            \begin{itemize}
                  \item Consolidación de los resultados obtenidos en las pruebas funcionales y de
                        calidad.
                  \item Priorización de ajustes según impacto en la experiencia del usuario y
                        relevancia educativa.
                  \item Aplicación de mejoras en la interfaz, flujo de interacción y lógica de
                        generación de prompts para aumentar precisión y contextualización de las
                        respuestas.
                  \item Documentación de lecciones aprendidas, recomendaciones de optimización y pautas
                        para futuros \textit{sprints} de mantenimiento o escalabilidad.
            \end{itemize}
\end{enumerate}

\subsection{Resultado final}
Al concluir este \textit{sprint}, se logró:

\begin{itemize}
      \item Validación completa de la integración entre \textit{frontend}, \textit{backend}
            y bases de datos, garantizando estabilidad y funcionalidad del sistema.
      \item Confirmación de la calidad y confiabilidad de las respuestas generadas por el
            asistente virtual, alineadas con el contenido educativo base y criterios
            expertos.
      \item Identificación y corrección de errores o inconsistencias en la interacción,
            flujo de navegación y procesamiento de información.
      \item Implementación de mejoras en la interfaz y en la lógica de generación de
            prompts para optimizar la experiencia de usuario y la pertinencia pedagógica.
      \item Documentación de resultados de prueba, retroalimentación de usuarios y
            expertos, y recomendaciones para mantenimiento futuro y escalabilidad del
            proyecto.
\end{itemize}

Este \textit{sprint} permitió asegurar que la aplicación estuviera lista para
su uso efectivo público, proporcionando respuestas precisas y contextualizadas
y estableciendo las bases para fases futuras de despliegue y monitoreo continuo
del asistente virtual.

\section{\textit{Sprint} 6: Documentación y presentación}
\textbf{Duración estimada:} 3 semanas

Este \textit{sprint} se centró en consolidar toda la documentación generada
durante el desarrollo del proyecto y preparar la presentación final del
asistente virtual de formación ciudadana y valores morales. El objetivo fue
garantizar que tanto los resultados como los procesos utilizados quedaran
claramente registrados, así como asegurar que el producto final estuviera
disponible para revisión, prueba y entrega formal al cliente.

\subsection{Objetivo}
Elaborar y organizar toda la documentación técnica, académica y operativa del
proyecto, incluyendo resultados, análisis, conclusiones y recomendaciones, y
realizar la presentación formal del desarrollo, asegurando la transferencia
completa de información a la Fundación de Scouts de Guatemala y dejando el
producto listo para pruebas finales y futuras mejoras.

\subsection{Ejecución}
\begin{enumerate}
      \item \textbf{Elaboración del informe final}
            \begin{itemize}
                  \item Integración de la información de todos los \textit{sprints} previos en un
                        documento único, estructurado y coherente.
                  \item Inclusión de resultados de cada \textit{sprint}, análisis de hallazgos,
                        decisiones de diseño y mejoras implementadas.
                  \item Redacción de conclusiones generales y recomendaciones para futuras iteraciones,
                        escalabilidad o mejoras del asistente virtual.
                  \item Formateo del documento en LaTeX, asegurando uniformidad, claridad y
                        cumplimiento de estándares académicos y de presentación profesional.
            \end{itemize}

      \item \textbf{Preparación de la presentación final}
            \begin{itemize}
                  \item Desarrollo de material visual que resuma el proyecto, incluyendo diagramas de
                        arquitectura, capturas de pantalla del prototipo móvil, flujo de interacción y
                        ejemplos de uso del asistente.
                  \item Elaboración de una presentación estructurada para explicar el proceso de
                        desarrollo, resultados obtenidos y funcionalidades del sistema.
                  \item Ensayo de la presentación y ajuste de contenido para garantizar claridad,
                        concisión y relevancia para el público objetivo.
            \end{itemize}

      \item \textbf{Traslado y entrega de documentación al cliente}
            \begin{itemize}
                  \item Consolidación de repositorios de código, documentos de investigación, fichas de
                        usuario, diagramas de arquitectura y demás materiales generados.
                  \item Entrega formal de toda la documentación y repositorios a la Fundación de Scouts
                        de Guatemala, asegurando que puedan acceder a todos los recursos para pruebas,
                        mantenimiento y futuras actualizaciones.
                  \item Registro de la entrega, incluyendo inventario de archivos, versión final de
                        documentación y evidencia de disponibilidad del producto para pruebas finales.
            \end{itemize}
\end{enumerate}

\subsection{Resultado final}
Como resultado de este \textit{sprint}, se logró:

\begin{itemize}
      \item Un informe final consolidado, claro y completo que documenta todo el proceso de
            desarrollo, análisis y resultados del proyecto.
      \item Material de presentación profesional listo para exponer ante el cliente y otros
            interesados.
      \item Entrega formal de toda la documentación y repositorios al cliente, asegurando
            disponibilidad total de recursos para pruebas, evaluación y futuras mejoras.
      \item Registro de la entrega y validación de que el producto final está operativo y
            listo para su uso y pruebas definitivas.
\end{itemize}

Este \textit{sprint} concluyó con la transferencia completa del conocimiento y
del producto, cerrando oficialmente el ciclo de desarrollo inicial del
asistente virtual y dejando una base sólida para el mantenimiento y
escalabilidad futura del proyecto.