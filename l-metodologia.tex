El desarrollo del proyecto Ciudadano Digital se llevó a cabo bajo el marco de
trabajo SCRUM, un enfoque ágil ampliamente utilizado en ingeniería de software
que permite la entrega incremental de productos funcionales mediante ciclos
cortos de desarrollo denominados sprints. Esta metodología fue seleccionada
debido a su flexibilidad, capacidad de adaptación a cambios en los
requerimientos y enfoque en la mejora continua, elementos clave en un proyecto
de innovación educativa como el presente.

A lo largo del proceso, se definieron seis sprints principales, cada uno con
objetivos concretos y entregables verificables, orientados a la obtención
progresiva de un prototipo funcional y validado de la aplicación. Cada sprint
tuvo una duración de entre tres y cuatro semanas, ajustándose según la
complejidad técnica y la carga académica del periodo correspondiente.

Cada ciclo SCRUM siguió las fases de planificación, desarrollo, revisión y
retrospectiva, bajo los siguientes principios:

\begin{itemize}
      \item Planificación (Sprint Planning): se definieron los objetivos y alcance del
            sprint, así como las tareas específicas necesarias para cumplir la meta
            establecida.
      \item Desarrollo (Sprint Execution): se ejecutaron las tareas asignadas con enfoque
            en la funcionalidad incremental, priorizando siempre la obtención de resultados
            medibles.
      \item Revisión (Sprint Review): al cierre de cada sprint, se evaluó el cumplimiento
            de los objetivos, la calidad del producto obtenido y la satisfacción de los
            criterios de aceptación definidos.
      \item Retrospectiva (Sprint Retrospective): se analizaron los aprendizajes obtenidos,
            los obstáculos encontrados y las oportunidades de mejora para el siguiente
            sprint.
\end{itemize}

El enfoque SCRUM permitió mantener un flujo de trabajo iterativo, controlado y
adaptable, asegurando que cada componente técnico se validara en función de la
experiencia real del usuario objetivo. En este caso, el usuario fue
representado a través de una Persona desarrollada con base en un proceso de
investigación y perfilamiento descrito en el primer sprint.

A partir del segundo sprint, los entregables se enfocaron en la construcción
progresiva del sistema técnico, desde la recopilación y procesamiento de
contenido educativo, hasta la implementación del backend, el desarrollo de la
interfaz móvil y las fases finales de validación y documentación.

El producto mínimo viable (MVP) obtenido al finalizar el último sprint
constituye una versión funcional del asistente inteligente de educación
ciudadana, capaz de interactuar con el usuario, contextualizar sus preguntas y
generar respuestas basadas en la información previamente curada y vectorizada.

\section{Enfoque metodológico aplicado al contexto del proyecto}
A diferencia de proyectos puramente técnicos, Ciudadano Digital combina
aspectos de ingeniería de software, inteligencia artificial y educación en
valores. Por ello, la aplicación de SCRUM fue adaptada a un enfoque
sociotécnico, que no solo prioriza la funcionalidad del sistema, sino también
la pertinencia ética y pedagógica del contenido.

En cada sprint, se incluyeron tareas de análisis cualitativo y cuantitativo
relacionadas con el perfil del usuario objetivo: estudiantes de nivel medio en
Guatemala, de entre 15 y 18 años, con acceso limitado a formación cívica más
allá del aula formal. Este enfoque garantizó que las decisiones técnicas
(estructura del backend, procesamiento de datos, interfaz y validación)
respondieran a necesidades reales detectadas en el público meta.

Así, el proceso metodológico buscó alinear el desarrollo tecnológico con la
misión educativa del proyecto, entendiendo que la calidad del producto no se
mide solo por su rendimiento, sino también por su capacidad de promover la
reflexión moral y la ciudadanía responsable en contextos informales de
aprendizaje.

\subsection{Estructura de los Sprints}
\begin{table}[H]
      \centering
      \renewcommand{\arraystretch}{1.2}
      \begin{tabular}{|l|p{8cm}|c|}
            \hline
            \textbf{Sprint} & \textbf{Meta Principal}                                  & \textbf{Duración Estimada} \\ \hline
            Sprint 1        & Identificación del perfil de usuario objetivo (Persona). & 3 semanas                  \\ \hline
            Sprint 2        & Recolección y procesamiento del contenido educativo.     & 4 semanas                  \\ \hline
            Sprint 3        & Construcción e implementación del backend.               & 4 semanas                  \\ \hline
            Sprint 4        & Desarrollo de la interfaz móvil en Kotlin.               & 4 semanas                  \\ \hline
            Sprint 5        & Pruebas y validación funcional.                          & 3 semanas                  \\ \hline
            Sprint 6        & Documentación, presentación y cierre del proyecto.       & 3 semanas                  \\ \hline
      \end{tabular}
      \caption[Estructura de los Sprints]{Estructura de los sprints del proyecto, incluyendo la meta principal y duración estimada de cada uno.}
      \label{tab:estructura-sprints}
\end{table}

Cada sprint culminó con un entregable verificable que sirvió como criterio de
avance para el siguiente ciclo, asegurando así la trazabilidad y coherencia
entre la visión inicial del proyecto y el producto final obtenido.

\section{Sprint 1: Identificación del perfil de usuario objetivo}
\textbf{Duración estimada:} 3 semanas

Este sprint tuvo como objetivo desarrollar un perfil de usuario (Persona) que
sirviera como insumo accionable para orientar las decisiones de diseño
interactivo y priorización técnica del proyecto. Dado que no fue posible
realizar entrevistas ni trabajo de campo, el perfil se elaboró exclusivamente a
partir del análisis de fuentes documentales que reflejan la situación actual de
los estudiantes en el país, considerando aspectos demográficos, académicos y
sociales. Con base en esta información, se construyó una ficha de Persona
completa, acompañada de criterios de diseño alineados con las necesidades y
características identificadas.

\subsection{Objetivo}
Definir y validar un perfil de usuario objetivo representativo (Persona),
incluyendo sus características sociodemográficas, motivaciones, frustraciones,
competencias digitales y contextos de uso, que sirva como referencia para guiar
las decisiones de diseño UX, tono comunicativo, prioridades de contenido y
criterios de evaluación de usabilidad del asistente.

\subsection{Ejecución}
Para lograr el objetivo planteado, se llevaron a cabo las siguientes tareas:

\begin{enumerate}
      \item \textbf{Investigación documental}
            \begin{itemize}
                  \item Revisión de informes académicos y/o gubernamentales sobre educación ciudadana,
                        competencias cívicas y valores en jóvenes guatemaltecos.
                  \item Consulta de programas educativos oficiales, como el Currículo Nacional Base
                        (CNB) y materiales de formación en valores del Ministerio de Educación de
                        Guatemala, así como contenido internacional enfocado en brindar una educación
                        más completa.
                  \item Análisis de estudios internacionales de organismos como UNESCO, CEPAL, CIEN y
                        BID sobre hábitos digitales, desigualdad educativa y desarrollo de competencias
                        ciudadanas en adolescentes y jóvenes.
            \end{itemize}

      \item \textbf{Análisis e interpretación de la información}
            \begin{itemize}
                  \item Sistematización de datos demográficos, educativos y tecnológicos relevantes
                        para el contexto juvenil guatemalteco.
                  \item Identificación de patrones generales de comportamiento, motivaciones,
                        frustraciones y aspiraciones cívicas, a partir de tendencias reportadas en las
                        fuentes analizadas.
                  \item Construcción de categorías de análisis que permitieran traducir los hallazgos
                        documentales en insumos para el diseño centrado en el usuario.
            \end{itemize}

      \item \textbf{Definición del perfil Persona}
            \begin{itemize}
                  \item Elaboración de una ficha de usuario basada en la interpretación crítica de los
                        datos documentales, con los siguientes componentes:
                        \begin{itemize}
                              \item \textbf{Perfil base:} edad estimada, nivel educativo, ubicación, etnia, acceso tecnológico y contexto social.
                              \item \textbf{Motivaciones:} interés por la participación comunitaria y el aprendizaje de ciudadanía.
                              \item \textbf{Frustraciones:} barreras de acceso a recursos educativos y desconfianza en la calidad o adecuación de los materiales disponibles.
                              \item \textbf{Objetivos:} qué quisiera conseguir el usuario a través de sus motivaciones y frustraciones, bajo el contexto de educación en valores y formación ciudadana.
                              \item \textbf{Consideraciones especiales:} limitaciones de conectividad, recursos económicos y brechas culturales.
                        \end{itemize}
                  \item Producción de una ficha visual que sirviera como base para las decisiones de
                        diseño en sprints posteriores.
            \end{itemize}

      \item \textbf{Documentación de criterios de diseño}
            \begin{itemize}
                  \item Derivación de recomendaciones de diseño UX basadas en el perfil construido:
                        tono comunicativo, estructura de funciones, rol a asumir por el asistente, y
                        adaptabilidad tecnológica.
                  \item Identificación de necesidades prioritarias que el asistente debe ser capaz de
                        abordar a través de la interacción pregunta-respuesta.
            \end{itemize}

\end{enumerate}

\subsection{Resultado final}
Como resultado de este primer sprint, se construyó un perfil de Persona
detallado, basado en fuentes documentales, que permitió comprender las
necesidades, barreras y expectativas del usuario objetivo frente a una
herramienta de apoyo educativo.

\begin{itemize}
      \item Edad promedio: 18-22 años.
      \item Contexto educativo: estudiantes de nivel medio y universitario inicial.
      \item Motivaciones: aprender de forma práctica y reflexiva, mejorar su comprensión de
            ciudadanía y valores.
      \item Frustraciones: enseñanza teórica, falta de espacios de diálogo y escasez de
            herramientas interactivas.
      \item Competencias digitales: nivel bajo a medio en uso de aplicaciones y
            herramientas digitales.
      \item Contexto de uso de la aplicación: dispositivos móviles, principalmente Android,
            con sesiones cortas de interacción y preferencia por contenidos dinámicos y
            cercanos a su realidad.
\end{itemize}

Este perfil se utilizó como base para orientar el diseño conversacional, las
estrategias de análisis documental y los lineamientos pedagógicos que guiarán
las siguientes etapas del desarrollo del proyecto.

\section{Sprint 2: Recolección y procesamiento del contenido educativo}
\textbf{Duración estimada:} 4 semanas

Este sprint se centró en recopilar, procesar y estructurar el contenido
educativo que alimentará al asistente virtual de inteligencia artificial, con
la finalidad de garantizar que el sistema pueda generar respuestas precisas y
contextualizadas sobre formación ciudadana y valores morales, basándose en
información confiable y organizada de manera semántica. Se combinó la selección
documental, curación de contenido, digitalización, segmentación temática y
almacenamiento vectorial de manera sistemática, asegurando la trazabilidad y
calidad de los datos utilizados.

\subsection{Objetivo}
Obtener una base de datos documental organizada, curada y vectorizada, que
permita al asistente virtual ofrecer respuestas contextualizadas, precisas y
alineadas con principios de formación ciudadana y valores morales, considerando
casos prácticos y escenarios de consulta relevantes para el público objetivo
definido en el Sprint 1.

\subsection{Ejecución}

Para cumplir el objetivo se realizaron las siguientes tareas:

\begin{enumerate}

      \item \textbf{Selección documental}
            \begin{itemize}
                  \item Identificación de fuentes oficiales y confiables: Currículo Nacional Base
                        (CNB), programas nacionales de educación en valores, documentos del Ministerio
                        de Educación de Guatemala.
                  \item Revisión de documentos internacionales: artículos de UNESCO, CEPAL, CIEN, BID y
                        estudios académicos sobre educación en valores, competencias ciudadanas y
                        hábitos digitales en jóvenes.
                  \item Inclusión de casos prácticos: se priorizaron documentos que incluyeran ejemplos
                        de aplicación de valores y formación cívica en escenarios reales, para
                        enriquecer la capacidad del asistente de generar respuestas contextualizadas.
                  \item Registro de metadatos iniciales: cada documento seleccionado se catalogó con
                        información sobre fuente, fecha, categoría temática y tipo de contenido.
            \end{itemize}

      \item \textbf{Curación y digitalización}
            \begin{itemize}
                  \item Conversión de documentos a formato digital estándar (UTF-8), eliminando
                        inconsistencias de formato.
                  \item Corrección de errores tipográficos y estandarización de encabezados, títulos y
                        numeraciones.
                  \item Estructuración de contenido en bloques temáticos y secciones claras para
                        facilitar el posterior procesamiento y la creación de embeddings.
            \end{itemize}

      \item \textbf{Segmentación temática}
            \begin{itemize}
                  \item Definición de categorías temáticas (esquemas) basadas en valores, principios
                        cívicos y competencias ciudadanas: ética, participación comunitaria, derechos
                        humanos, responsabilidad social, etc. Con la flexibilidad de añadir nuevos
                        esquemas en caso sea necesario en el futuro.
                  \item Asignación de cada fragmento de texto a una categoría específica, asegurando
                        coherencia y granularidad adecuada para la vectorización.
                  \item Creación de una estructura de referencia para vincular cada fragmento con su
                        documento de origen y su categoría, garantizando trazabilidad.
            \end{itemize}

      \item \textbf{Vectorización y almacenamiento en Pinecone}
            \begin{itemize}
                  \item \textbf{Fragmentación:} los documentos se dividieron en bloques de longitud controlada para preservar contexto semántico sin exceder los límites de entrada del modelo de embeddings.
                  \item \textbf{Normalización del texto:} limpieza de caracteres especiales, uniformidad de mayúsculas, puntuación y espacios, asegurando consistencia en la representación vectorial.
                  \item \textbf{Generación de embeddings:} cada bloque se envió al modelo \textit{text-embedding-3-small} de OpenAI, obteniendo vectores numéricos que representan semánticamente el contenido.
                  \item \textbf{Metadatos asociados:} cada vector incluyó información sobre su fuente, categoría temática, bloque dentro del documento y código de referencia para la base relacional.
                  \item \textbf{Almacenamiento en Pinecone:} los vectores se insertaron en un índice configurado con la métrica \textit{cosine similarity}, permitiendo consultas semánticas eficientes desde el backend del asistente.
                  \item \textbf{Estructura final de almacenamiento:}
                        \begin{itemize}
                              \item ID único del vector
                              \item Vector de embeddings
                              \item Metadatos: fuente, categoría, bloque, referencia documental
                        \end{itemize}
            \end{itemize}
\end{enumerate}

\subsection{Resultado final}
Al finalizar el Sprint 2, se obtuvo:

\begin{itemize}
      \item Una base documental curada, digitalizada y segmentada en categorías temáticas.
      \item Embeddings generados para cada fragmento de texto, con metadatos completos para
            garantizar trazabilidad.
      \item Un índice en Pinecone listo para consultas semánticas, capaz de proporcionar
            contexto preciso al asistente virtual para cualquier pregunta del usuario.
      \item Establecimiento de un flujo reproducible de selección, curación, segmentación y
            vectorización de contenido para futuras actualizaciones del sistema.
\end{itemize}

Este sprint permitió sentar las bases para un sistema de respuesta
contextualizada, alineado con los objetivos de formación ciudadana y valores
morales definidos en el proyecto, asegurando que el asistente virtual cuente
con información confiable, organizada y accesible para generar respuestas
pertinentes y fundamentadas.

\section{Sprint 3: Construcción e implementación del backend}
\textbf{Duración estimada:} 4 semanas

Este sprint se enfocó en el diseño, construcción e implementación de la
arquitectura backend del asistente virtual, garantizando la integración de
bases de datos relacionales y vectoriales, y estableciendo la comunicación
segura y eficiente con el modelo de lenguaje (LLM) mediante un flujo RAG
(\textit{Retrieval-Augmented Generation}). Se definieron módulos claros bajo el
patrón de diseño MVC, así como servicios complementarios internos en Python
para procesar consultas y generar respuestas contextualizadas basadas en los
documentos previamente vectorizados.

\subsection{Objetivo}
Desarrollar un backend robusto y seguro que permita la gestión de usuarios,
almacenamiento de documentos y chats, recuperación de información semántica
desde la base vectorial, y generación de respuestas contextualizadas desde el
LLM, garantizando un flujo eficiente de pregunta-respuesta para cualquier
interfaz de usuario.

\subsection{Ejecución}
\begin{enumerate}
      \item \textbf{Diseño de arquitectura}
            \begin{itemize}
                  \item Se adoptó el patrón MVC para organizar la aplicación en módulos separados:
                        modelos, rutas (vistas) y controladores.
                  \item \textbf{Modelos:} se definieron todas las interacciones con la base de datos relacional, incluyendo operaciones CRUD (Create, Read, Update, Delete) para usuarios, chats, mensajes, sesiones y documentos.
                  \item \textbf{Rutas (Vistas):} se crearon los endpoints accesibles desde la aplicación, permitiendo la consulta, envío de preguntas y gestión de sesiones.
                  \item \textbf{Controladores:} se implementó la lógica de validación de datos, seguridad, manejo de errores y verificación de roles, comunicándose directamente con los modelos.
                  \item \textbf{Módulos auxiliares:}
                        \begin{itemize}
                              \item \textbf{Middlewares:} acciones intermedias para validar autenticación, tokens y la integridad de las solicitudes antes de llegar a los controladores.
                              \item \textbf{Helpers:} funciones reutilizables en distintos módulos para operaciones frecuentes, como encriptación de contraseñas, generación de tokens y formateo de datos.
                        \end{itemize}
            \end{itemize}

      \item \textbf{Diseño y construcción de bases de datos}
            \subsubsection{Base de datos relacional}
            \begin{itemize}
                  \item Motor: PostgreSQL en AWS RDS.
                  \item Estructura:
                        \begin{itemize}
                              \item \textbf{Usuarios:} nombres, apellidos, correo electrónico, contraseña encriptada, fecha de nacimiento.
                              \item \textbf{Chats:} ID del chat, ID del usuario, título, estado (activo/inactivo).
                              \item \textbf{Mensajes:} ID del chat, contenido del mensaje, origen (usuario o asistente), fecha y hora.
                              \item \textbf{Sesiones:} token, ID del usuario, ID del dispositivo, fecha y hora de creación, estado de expiración.
                              \item \textbf{Documentos:} Identificador, fuente de origen, categoría y enlace a archivo S3.
                        \end{itemize}
            \end{itemize}

            \subsubsection{Base de datos vectorial}
            \begin{itemize}
                  \item Motor: Pinecone.
                  \item Contenido: embeddings de los fragmentos de documentos, asociados con metadatos
                        para identificar su fuente, categoría temática y bloque original.
                  \item Configuración: índice con métrica \textit{cosine similarity} para consultas
                        semánticas eficientes desde el backend.
            \end{itemize}

      \item \textbf{Servicio complementario de Python para comunicación con LLM}
            \begin{itemize}
                  \item Función: procesar preguntas del usuario, consultar la base vectorial y generar
                        prompts para el modelo LLM.
                  \item \textbf{Estructuración de pregunta desde NodeJS:}
                        \begin{itemize}
                              \item El endpoint recibe la pregunta en texto plano.
                              \item Se genera el embedding correspondiente con \textit{text-embedding-3-small}.
                              \item Se consulta Pinecone para obtener los 5 fragmentos más relevantes como
                                    contexto.
                              \item Se envía un JSON estructurado al servicio de Python con la pregunta y contexto.
                        \end{itemize}
                  \item \textbf{Servicio interno de Python:}
                        \begin{itemize}
                              \item Recibe el JSON con la pregunta y el contexto.
                              \item Construye el prompt y consulta al LLM de OpenAI.
                              \item Devuelve la respuesta generada al backend de NodeJS, indicando si el contexto
                                    fue suficiente o no para ofrecer una respuesta precisa.
                        \end{itemize}
                  \item \textbf{Respuesta final:} NodeJS formatea la respuesta final en JSON, que será entregada al usuario final a través del API.
            \end{itemize}

\end{enumerate}

\subsection{Resultado final}
Al finalizar este sprint, se obtuvo:

\begin{itemize}
      \item Un backend robusto y modular bajo MVC, con controladores, modelos y rutas
            claramente definidos.
      \item Base de datos relacional (PostgreSQL) para usuarios, chats, sesiones y
            documentos, asegurando seguridad y trazabilidad de la información.
      \item Base vectorial (Pinecone) con embeddings organizados y metadatos completos,
            lista para consultas semánticas.
      \item Servicio interno de Python funcionando como intermediario para la generación de
            prompts y la interacción con el LLM, estableciendo un flujo RAG completo.
      \item Capacidad de recibir preguntas de cualquier interfaz de usuario y generar
            respuestas contextualmente precisas, basadas en documentos previamente curados
            y vectorizados.
\end{itemize}

Este sprint estableció la infraestructura necesaria para que el asistente
educativo pueda operar de manera confiable, eficiente y escalable, sentando las
bases para el siguiente sprint centrado en el frontend y la experiencia del
usuario.

\section{Sprint 4: Desarrollo de la interfaz móvil en Kotlin}
\textbf{Duración estimada:} 4 semanas

Este sprint tuvo como objetivo diseñar e implementar la interfaz móvil de la
aplicación del asistente educativo, utilizando Kotlin para garantizar
integración nativa con Android y un flujo de interacción intuitivo para el
usuario. Se buscó crear módulos claros y escalables, organizar recursos y
establecer los patrones de navegación y comunicación con el backend, asegurando
que la aplicación fuera funcional, accesible y adaptable al perfil de usuario
definido en el Sprint 1.

\subsection{Objetivo}
Desarrollar una aplicación móvil Android con una interfaz intuitiva y eficiente
que permita a los usuarios interactuar con el asistente virtual, enviar
preguntas, recibir respuestas contextualizadas y gestionar su historial de
interacción, manteniendo consistencia con el perfil de usuario objetivo y las
decisiones de diseño UX previamente definidas.

\subsection{Ejecución}
\begin{enumerate}
      \item \textbf{Definición de módulos y arquitectura de la aplicación}
            \begin{itemize}
                  \item \textbf{Data:} gestión de modelos de datos, repositorios y fuentes de información, incluyendo comunicación con el backend vía API y almacenamiento local temporal.
                  \item \textbf{Dependency Injection:} configuración de Hilt para inyectar dependencias de forma eficiente y centralizada, facilitando la escalabilidad y pruebas.
                  \item \textbf{Helpers:} funciones y utilidades reutilizables, como manejo de errores, validación de entradas de usuario, formateo de fechas y gestión de sesiones.
                  \item \textbf{User Interface (UI):} construcción de pantallas, componentes visuales y navegación, siguiendo principios de Material Design y adaptabilidad a distintos tamaños de pantalla.
                  \item \textbf{Resources:} gestión de strings, colores, dimensiones, iconografía y estilos para mantener consistencia visual y facilitar traducciones o ajustes futuros.
            \end{itemize}

      \item \textbf{Diseño de flujo de interacción}
            \begin{itemize}
                  \item Mapeo de pantallas principales: inicio, login, chat con asistente, historial de
                        interacciones y visualización de documentos.
                  \item Implementación de navegación mediante \textit{Navigation Component},
                        garantizando consistencia y control del back stack.
                  \item Diseño de interacción de chat: envío de preguntas, visualización de respuestas
                        con formato enriquecido y mensajes de sistema para notificaciones o errores.
                  \item Integración de indicadores de carga y estado de conexión, ofreciendo
                        retroalimentación inmediata al usuario sobre la consulta al LLM.
            \end{itemize}

      \item \textbf{Integración con backend y servicios de Python}
            \begin{itemize}
                  \item Consumo de endpoints REST del backend para autenticación, gestión de sesiones,
                        envío de preguntas y recuperación de respuestas.
                  \item Procesamiento de respuestas JSON, parseo y renderizado en la interfaz de
                        usuario de manera clara y comprensible.
                  \item Manejo de errores y reconexión ante fallos de red, asegurando robustez en la
                        experiencia de usuario.
            \end{itemize}

      \item \textbf{Pruebas de funcionalidad y usabilidad}
            \begin{itemize}
                  \item Pruebas unitarias y de integración en Kotlin para asegurar correcto
                        funcionamiento de los módulos y la comunicación con el backend.
                  \item Pruebas de usabilidad simulando interacción de usuarios representativos según
                        la Persona definida en el Sprint 1.
                  \item Ajustes en la interfaz y flujo de navegación basados en observaciones de uso,
                        priorizando claridad, accesibilidad y eficiencia en la interacción.
            \end{itemize}

      \item \textbf{Documentación y guías de uso}
            \begin{itemize}
                  \item Documentación de los módulos implementados, incluyendo dependencias, estructura
                        de carpetas, responsabilidades de cada componente y ejemplos de uso.
                  \item Guía de buenas prácticas para futuras iteraciones y ampliaciones de la
                        aplicación móvil.
            \end{itemize}
\end{enumerate}

\subsection{Resultado final}
Al finalizar este sprint, se obtuvo:

\begin{itemize}
      \item Aplicación móvil funcional en Android, con integración nativa mediante Kotlin y
            comunicación estable con el backend.
      \item Estructura modular clara (Data, Dependency Injection, Helpers, UI, Resources)
            que facilita mantenimiento y escalabilidad.
      \item Flujo de interacción optimizado para el usuario, incluyendo envío de preguntas,
            recepción de respuestas contextuales y visualización de documentos.
      \item Pruebas internas de usabilidad que validaron la intuitividad del diseño y la
            adecuación mediante simulaciones de uso por el perfil de usuario objetivo.
      \item Documentación completa del frontend, con guías para futuras mejoras y
            desarrollo colaborativo.
\end{itemize}

Este sprint permitió contar con una interfaz móvil operativa, lista para el
despliegue y pruebas piloto, estableciendo las bases para la fase final de
evaluación y refinamiento del proyecto.

\section{Sprint 5: Pruebas y validación}
\textbf{Duración estimada:} 3 semanas

Este sprint se centró en validar el funcionamiento integral del sistema,
asegurando la correcta interacción entre la aplicación móvil, el backend, la
base de datos relacional y vectorial, y el modelo de lenguaje (LLM). Además, se
buscó evaluar la calidad, precisión y confiabilidad de las respuestas generadas
por el asistente virtual, así como recopilar retroalimentación mediante pruebas
internas para ajustar y mejorar la herramienta antes de su entrega.

\subsection{Objetivo}
Realizar pruebas funcionales y de usabilidad del sistema, validar la precisión
y confiabilidad de las respuestas del asistente virtual en el contexto de
educación ciudadana y valores morales y aplicar mejoras continuas basadas en
observaciones internas y criterios expertos (con ayuda del asesor),
garantizando un producto final robusto y alineado con el perfil de usuario
objetivo.

\subsection{Ejecución}
\begin{enumerate}
      \item \textbf{Pruebas funcionales del sistema}
            \begin{itemize}
                  \item Verificación de la comunicación entre la aplicación móvil (Kotlin), el backend
                        (NodeJS), la base de datos relacional (PostgreSQL) y la base vectorial
                        (Pinecone).
                  \item Pruebas de endpoints para asegurar correcta autenticación de usuarios, envío de
                        preguntas, recuperación de respuestas y gestión de historial de chats.
                  \item Comprobación de la integridad de los datos entre los distintos módulos,
                        incluyendo creación, lectura, actualización y eliminación de información
                        (CRUD).
                  \item Simulación de sesiones múltiples para validar estabilidad y manejo de
                        concurrencia.
            \end{itemize}

      \item \textbf{Pruebas de calidad y confiabilidad de las respuestas}
            \begin{itemize}
                  \item Validación directa de las respuestas generadas por el asistente comparando con
                        el contenido base documentado utilizado para entrenar y alimentar al modelo.
                  \item Evaluación por parte del asesor, revisando pertinencia, claridad y adecuación
                        pedagógica de las respuestas.
                  \item Identificación de casos en los que el modelo no proporcione información
                        suficiente o presente inconsistencias, documentando hallazgos para ajuste de
                        contenido o configuración de embeddings.
            \end{itemize}

      \item \textbf{Análisis de resultados y mejora continua}
            \begin{itemize}
                  \item Consolidación de los resultados obtenidos en las pruebas funcionales y de
                        calidad.
                  \item Priorización de ajustes según impacto en la experiencia del usuario y
                        relevancia educativa.
                  \item Aplicación de mejoras en la interfaz, flujo de interacción y lógica de
                        generación de prompts para aumentar precisión y contextualización de las
                        respuestas.
                  \item Documentación de lecciones aprendidas, recomendaciones de optimización y pautas
                        para futuros sprints de mantenimiento o escalabilidad.
            \end{itemize}
\end{enumerate}

\subsection{Resultado final}
Al concluir este sprint, se logró:

\begin{itemize}
      \item Validación completa de la integración entre frontend, backend y bases de datos,
            garantizando estabilidad y funcionalidad del sistema.
      \item Confirmación de la calidad y confiabilidad de las respuestas generadas por el
            asistente virtual, alineadas con el contenido educativo base y criterios
            expertos.
      \item Identificación y corrección de errores o inconsistencias en la interacción,
            flujo de navegación y procesamiento de información.
      \item Implementación de mejoras en la interfaz y en la lógica de generación de
            prompts para optimizar la experiencia de usuario y la pertinencia pedagógica.
      \item Documentación de resultados de prueba, retroalimentación de usuarios y
            expertos, y recomendaciones para mantenimiento futuro y escalabilidad del
            proyecto.
\end{itemize}

Este sprint permitió asegurar que la aplicación estuviera lista para su uso
efectivo público, proporcionando respuestas precisas y contextualizadas y
estableciendo las bases para fases futuras de despliegue y monitoreo continuo
del asistente virtual.

\section{Sprint 6: Documentación y presentación}
\textbf{Duración estimada:} 3 semanas

Este sprint se centró en consolidar toda la documentación generada durante el
desarrollo del proyecto y preparar la presentación final del asistente virtual
de formación ciudadana y valores morales. El objetivo fue garantizar que tanto
los resultados como los procesos utilizados quedaran claramente registrados,
así como asegurar que el producto final estuviera disponible para revisión,
prueba y entrega formal al cliente.

\subsection{Objetivo}
Elaborar y organizar toda la documentación técnica, académica y operativa del
proyecto, incluyendo resultados, análisis, conclusiones y recomendaciones, y
realizar la presentación formal del desarrollo, asegurando la transferencia
completa de información a la Fundación de Scouts de Guatemala y dejando el
producto listo para pruebas finales y futuras mejoras.

\subsection{Ejecución}
\begin{enumerate}
      \item \textbf{Elaboración del informe final}
            \begin{itemize}
                  \item Integración de la información de todos los sprints previos en un documento
                        único, estructurado y coherente.
                  \item Inclusión de resultados de cada sprint, análisis de hallazgos, decisiones de
                        diseño y mejoras implementadas.
                  \item Redacción de conclusiones generales y recomendaciones para futuras iteraciones,
                        escalabilidad o mejoras del asistente virtual.
                  \item Formateo del documento en LaTeX, asegurando uniformidad, claridad y
                        cumplimiento de estándares académicos y de presentación profesional.
            \end{itemize}

      \item \textbf{Preparación de la presentación final}
            \begin{itemize}
                  \item Desarrollo de material visual que resuma el proyecto, incluyendo diagramas de
                        arquitectura, capturas de pantalla del prototipo móvil, flujo de interacción y
                        ejemplos de uso del asistente.
                  \item Elaboración de una presentación estructurada para explicar el proceso de
                        desarrollo, resultados obtenidos y funcionalidades del sistema.
                  \item Ensayo de la presentación y ajuste de contenido para garantizar claridad,
                        concisión y relevancia para el público objetivo.
            \end{itemize}

      \item \textbf{Traslado y entrega de documentación al cliente}
            \begin{itemize}
                  \item Consolidación de repositorios de código, documentos de investigación, fichas de
                        usuario, diagramas de arquitectura y demás materiales generados.
                  \item Entrega formal de toda la documentación y repositorios a la Fundación de Scouts
                        de Guatemala, asegurando que puedan acceder a todos los recursos para pruebas,
                        mantenimiento y futuras actualizaciones.
                  \item Registro de la entrega, incluyendo inventario de archivos, versión final de
                        documentación y evidencia de disponibilidad del producto para pruebas finales.
            \end{itemize}
\end{enumerate}

\subsection{Resultado final}
Como resultado de este sprint, se logró:

\begin{itemize}
      \item Un informe final consolidado, claro y completo que documenta todo el proceso de
            desarrollo, análisis y resultados del proyecto.
      \item Material de presentación profesional listo para exponer ante el cliente y otros
            interesados.
      \item Entrega formal de toda la documentación y repositorios al cliente, asegurando
            disponibilidad total de recursos para pruebas, evaluación y futuras mejoras.
      \item Registro de la entrega y validación de que el producto final está operativo y
            listo para su uso y pruebas definitivas.
\end{itemize}

Este sprint concluyó con la transferencia completa del conocimiento y del
producto, cerrando oficialmente el ciclo de desarrollo inicial del asistente virtual y
dejando una base sólida para el mantenimiento y escalabilidad futura del
proyecto.