Artificial intelligence (AI) offers transformative potential for education, especially in contexts marked by social, economic, and technological inequalities. In countries like Guatemala, where a wide educational gap persists, AI can become a key tool to facilitate access to meaningful learning.

One of the most neglected areas is civic education and the development of moral values, which, although included in educational programs, are often addressed theoretically and detached from social reality. It is here that the importance of a tool that provides support to students in reflecting on their role as citizens and in the practice of values such as respect, empathy, and responsibility arises, a project that combines both accessible technology and a user-centered approach to strengthen not only learning, but also the social and ethical awareness of young people.

The implementation of this tool contemplates the digitization of educational content, the development of an AI model adapted to the local context, and finally, the direct validation of the interaction with the tool to verify that the answers given match the material provided. As a result, a functional, innovative, and scalable solution is sought that demonstrates how AI can significantly contribute to the strengthening of values education in environments with limited resources.