Artificial intelligence (AI) offers transformative potential for education,
especially in contexts marked by social, economic, and technological
inequalities. In countries like Guatemala, where a significant educational gap
persists, AI can become a key tool to facilitate access to meaningful learning.

One of the most neglected areas is civic education and the development of moral
values, which, although included in educational programs, are often addressed
theoretically and detached from social reality. This is where the importance of
a tool that supports students in reflecting on their role as citizens and in
practicing values such as respect, empathy, and responsibility arises. This
project combines both accessible technology and a user-centered approach to
strengthen not only learning but also the social and ethical awareness of young
people.

The implementation of this tool includes the digitization of educational
content, the development of an AI model adapted to the local context, and
finally, the direct validation of the interaction with the tool to ensure that
the answers provided align with the material given. As a result, a functional,
innovative, and scalable solution is sought that demonstrates how AI can
significantly contribute to strengthening values education in environments with
limited resources.