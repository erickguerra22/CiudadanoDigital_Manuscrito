Esta fase del proyecto \textit{Ciudadano Digital} culminó con la implementación
de un prototipo funcional de un asistente virtual para la educación informal en
ciudadanía y valores morales. El sistema integra de manera coherente
componentes de procesamiento de lenguaje natural (NLP, por sus siglas en
inglés), recuperación semántica de contexto y una interfaz nativa móvil para
Android. Si bien los resultados no constituyen una validación empírica del
impacto educativo, al no contar con pruebas de campo reales con usuarios
finales, esta primera versión demostró la viabilidad técnica y conceptual del
proyecto. Asimismo, se verificó su alineación con el marco teórico y los
objetivos específicos, destacando los logros técnicos y de diseño alcanzados
durante los sprints definidos.

\section{Definición de usuario objetivo (Persona)}
Durante el primer \textit{sprint} se elaboró una ficha técnica detallada del
\textbf{perfil Persona}, ilustrada en la Figura \ref{fig:persona}.

\begin{figure}[H]
      \centering
      \includegraphics[width=0.8\textwidth]{assets/persona.png}
      \caption{Ficha de \textbf{Perfil Persona} con datos básicos del usuario objetivo.}
      \label{fig:persona}
\end{figure}

Al identificar las características más comunes entre la población guatemalteca
en edad escolar, se definieron los parámetros necesarios para conformar un
ejemplo del usuario objetivo del proyecto. Estas características incluyen:

\begin{itemize}
      \item Edad promedio: 13-17 años.
      \item Contexto educativo: estudiantes de nivel medio y universitario inicial.
      \item Motivaciones: aprender de forma práctica y reflexiva, mejorar su comprensión de
            ciudadanía y valores.
      \item Frustraciones: enseñanza teórica, falta de espacios de diálogo y escasez de
            herramientas interactivas.
      \item Competencias digitales: nivel bajo a medio en uso de aplicaciones y
            herramientas digitales.
      \item Contexto de uso de la aplicación: dispositivos móviles, principalmente Android,
            con sesiones cortas de interacción y preferencia por contenidos dinámicos y
            cercanos a su realidad.
\end{itemize}

A partir de este perfil se definieron los siguientes aspectos centrales de la
aplicación:

\begin{itemize}
      \item \textbf{Plataforma:} En esta fase, el desarrollo se concentró en el sistema operativo Android. Se obtuvo compatibilidad para dispositivos de este sistema operativo a partir de la versión \textbf{Android 7}.
      \item \textbf{Enfoque:} La interacción pregunta-respuesta se diseñó bajo la filosofía del método socrático, lo que permitió que en cada interacción con el asistente se se obtenga también una lista de preguntas sugeridas.
      \item \textbf{Diseño:} La interfaz sigue los principios de \textit{Material Design}, esto al priorizar un diseño minimalista y enfocado en la funcionalidad del sistema
\end{itemize}

\section{Corpus vectorizado}
El sistema definido en los \textit{sprints} 2 y 3 permitió el procesamiento y
gestión de los documentos con los que se alimenta el modelo. Con ello, se
tomaron los documentos seleccionados al inicio del \textit{sprint} 2, se
sometieron a procesamiento, limpieza e indexación, todo a través de solicitudes
realizadas desde el cliente, validando la correcta integración de sistemas
internos y cliente-servidor.

Las pruebas realizadas verificaron la sincronización entre todas las fuentes de
datos a través del servidor, asegurando que las consultas se asociaran con el
contenido adecuado. La Figura \ref{fig:preview-pinecone} muestra el índice
vectorial en Pinecone después de cargar las representaciones numéricas
(\textit{embeddings}) de los fragmentos de texto.

\begin{figure}[H]
      \centering
      \includegraphics[width=0.8\textwidth]{assets/pinecone-preview.png}
      \caption{Vista del índice vectorial en Pinecone con las representaciones numéricas (\textit{embeddings}) cargadas.}
      \label{fig:preview-pinecone}
\end{figure}

A través de este proceso, se obtuvo:

\begin{itemize}
      \item Una base documental curada y segmentada en categorías temáticas.
      \item Representaciones numéricas (\textit{embeddings}) generadas para cada fragmento
            de texto, con metadatos completos para garantizar trazabilidad.
      \item Un índice en Pinecone listo para consultas semánticas, capaz de proporcionar
            contexto preciso al asistente virtual para cualquier pregunta del usuario.
      \item Establecimiento de un flujo reproducible de selección, curación, segmentación y
            vectorización de contenido para una continua actualización del corpus del
            proyecto.
\end{itemize}

\section{Interfaz móvil en Kotlin (Cliente)}

Durante el cuarto \textit{sprint} se desarrolló la aplicación móvil nativa para
Android utilizando Kotlin y el patrón de arquitectura MVVM (Modelo-Vista-Modelo
de Vista). Esta estructura permitió mantener una organización clara del código,
lo que facilita la escalabilidad y el mantenimiento. Como resultado, se obtuvo
una interfaz intuitiva y accesible para cualquier usuario, incluso sin
experiencia previa en herramientas similares.

\subsection{Flujo de interacción}

El flujo de interacción del usuario se compone de tres actividades principales
y un conjunto de fragmentos especializados que gestionan las distintas
funciones de la aplicación.

\begin{itemize}
      \item \textbf{Actividades}
            \begin{itemize}
                  \item \textbf{\textit{SplashActivity}:} Es la primera vista al abrir la aplicación. Aunque únicamente muestra el logotipo del proyecto, se encarga de verificar si existe una sesión activa mediante el token JWT allmacenado en las Preferencias Compartidas (\textit{SharedPreferences}). Si el token es válido, redirige directamente a la vista principal, evitando que el usuario deba iniciar sesión cada vez que abre la aplicación.
                  \item \textbf{\textit{UnloggedActivity}:} Agrupa los fragmentos relacionados con la autenticación (\textit{LoginFragment}, \textit{RegisterFragment} y \textit{RecoverPasswordFragment}). Cada fragmento aplica sus propias validaciones y coordina la comunicación con el servidor para garantizar un ingreso seguro al sistema.

                  \item \textbf{\textit{MainActivity}:} Es el núcleo de la aplicación una vez el usuario ha sido autenticado. Aloja el \textit{NavHostFragment}, componente encargado de gestionar la navegación entre los fragmentos funcionales; como chat, perfil de usuario y visualización de documentos. Además, incorpora un menú lateral que permite acceder rápidamente al historial de conversaciones.
            \end{itemize}

      \item \textbf{Fragmentos}
            \begin{itemize}
                  \item \textbf{\textit{LoginFragment}:} Permite al usuario ingresar sus credenciales y valida su formato y autenticidad.
                  \item \textbf{\textit{RegisterFragment}:} Permite crear una cuenta nueva con validaciones de integridad y unicidad. El correo electrónico es indispensable para recuperación de contraseña y notificaciones de ingreso o eliminación de documentos.
                  \item \textbf{\textit{SendRecoveryFragment}:} Solicita el envío del código de recuperación al correo registrado, el cual solo se envía si se encuentra una cuenta asociada al mismo.
                  \item \textbf{\textit{VerifyCodeFragment}:} Valida el código de recuperación ingresado.
                  \item \textbf{\textit{ResetPasswordFragment}:} Permite establecer una nueva contraseña tras validar el código enviado.
                  \item \textbf{\textit{ChatFragment}:} Facilita la interacción con el asistente virtual, ya que muestra el historial de mensajes y permite enviar nuevas consultas.
                  \item \textbf{\textit{ProfileFragment}:} Permite visualizar y editar la información personal del usuario, exceptuando la contraseña, que solo puede modificarse mediante el proceso de recuperación.
                  \item \textbf{\textit{DocumentsFragment}:} Muestra la lista de documentos cargados y permite su gestión.
            \end{itemize}

      \item La navegación entre fragmentos se implementó mediante el componente de
            navegación de Android (\textit{Navigation Component}), lo cual garantiza
            transiciones coherentes y control sobre la pila de navegación.
      \item Se integraron indicadores de carga y estado de conexión para brindar
            retroalimentación inmediata durante las consultas al LLM.
\end{itemize}

\subsection{Funcionalidades Comunes}

La aplicación distingue entre dos roles de usuario: \textbf{Usuario} y
\textbf{Administrador}, definidos en el campo
\textbf{\guillemetleft{}role\guillemetright{}} de la base de datos. Sin
embargo, ambos comparten un conjunto de funcionalidades esenciales:

\begin{itemize}
      \item \textbf{Conversaciones:} Permite crear chats, enviar mensajes y recibir respuestas del modelo.
      \item \textbf{Perfil de Usuario:} Los usuarios pueden visualizar y actualizar su información personal (nombre, apellidos, fecha de nacimiento, correo y teléfono).
      \item \textbf{Cerrar Sesión:} Se puede cerrar la sesión manualmente o esperar a que expire automáticamente tras una hora de inactividad.
\end{itemize}

La Figura \ref{fig:app-interfaces} muestra las pantallas principales
disponibles para todos los usuarios.

\begin{figure}[H]
      \centering
      \begin{minipage}{0.22\textwidth}
            \centering
            \includegraphics[width=\textwidth]{assets/app-chat.jpeg}
            \subcaption{Pantalla de chat con asistente virtual.}
      \end{minipage}
      \hfill
      \begin{minipage}{0.22\textwidth}
            \centering
            \includegraphics[width=\textwidth]{assets/app-messages.jpeg}
            \subcaption{Pantalla de chat con mensajes.}
      \end{minipage}
      \hfill
      \begin{minipage}{0.22\textwidth}
            \centering
            \includegraphics[width=\textwidth]{assets/profile.jpeg}
            \subcaption{Vista de perfil del usuario.}
      \end{minipage}
      \hfill
      \begin{minipage}{0.22\textwidth}
            \centering
            \includegraphics[width=\textwidth]{assets/side-panel.jpeg}
            \subcaption{Panel lateral general.}
      \end{minipage}
      \caption{Interfaz de la aplicación móvil, opciones generales.}
      \label{fig:app-interfaces}
\end{figure}

\subsection{Funcionalidades del Administrador}

Por otro lado, existen funciones exclusivas para el rol \textbf{Administrador},
orientadas al control del corpus documental usado por el modelo. Estas
funciones incluyen:

\begin{itemize}
      \item \textbf{Ver documentos:} Acceso al listado de archivos cargados y opción de abrir el documento original.
      \item \textbf{Añadir documentos:} Permite registrar nuevos archivos completando un formulario con información básica (Nombre, Autor, Año, Rango de edades). La carga y procesamiento del documento se realiza en segundo plano y se notifica por correo al finalizar.

      \item \textbf{Eliminar documentos:} Permite remover documentos del corpus. La eliminación también se procesa en segundo plano y el administrador recibe una notificación al completarse.
\end{itemize}

La Figura \ref{fig:app-interfaces-admin} muestra las interfaces destinadas
exclusivamente a administradores.

\begin{figure}[H]
      \centering
      \begin{minipage}{0.22\textwidth}
            \centering
            \includegraphics[width=\textwidth]{assets/side-panel-admin.jpeg}
            \subcaption{Panel lateral con acceso a documentos.}
      \end{minipage}
      \hfill
      \begin{minipage}{0.22\textwidth}
            \centering
            \includegraphics[width=\textwidth]{assets/app-documents.jpeg}
            \subcaption{Pantalla de gestión de documentos.}
      \end{minipage}
      \hfill
      \begin{minipage}{0.22\textwidth}
            \centering
            \includegraphics[width=\textwidth]{assets/new-document.jpeg}
            \subcaption{Formulario para ingreso de documentos.}
      \end{minipage}
      \caption{Interfaz de la aplicación móvil, opciones de administrador.}
      \label{fig:app-interfaces-admin}
\end{figure}

\section{Pruebas y Validación}
La implementación del prototipo concluyó con una evaluación exhaustiva del
sistema integrado. Se realizaron pruebas a los puntos de conexión mediante
Postman, los cuales cubren la autenticación, recuperación de contraseña, envío
de mensajes, historial de chats y administración de documentos.

En el Anexo \ref{tab:consultas-resultados}, se presenta una serie de 45
preguntas relacionadas al tema central del proyecto, abarcando conceptos de
ciudadanía, formación ciudadana y valores morales. Adicionalmente, se
realizaron 10 pregutnas de control (no relacionadas con el tema de civismo ni
con los documentos que conforman el corpus), diseñadas para evaluar la
\textbf{no alucinación} del modelo; es decir, que no responda cuando no deba
hacerlo.

Para cada pregunta, independientemente de si era de control o no, se tomó el
tiempo de respuesta, así como las referencias devueltas por el sistema (Anexo
\ref{tab:consultas-referencias}). La columna \texttt{Exitosa} se refiere a si
la pregunta fue o no fue respondida por el modelo, mientras que la columna
\texttt{Congruente} establece la congruencia según el tipo de consulta
realizada:

\begin{itemize}
      \item \textbf{Consulta:} Estas se refieren a preguntas que el modelo \textbf{sí debería responder} ya que estan basadas estrictamente en el contenido del corpus proporcionado.
      \item \textbf{Control:} Se trata de preguntas deliberadamente fuera del contexto del sistema, las cuales el modelo \textbf{no debería responder}. En este caso, la congruencia es positiva cuando el modelo efectivamente se niega a responder por estar fuera de su alcance y propósito.
\end{itemize}

Como se mencionó, el Anexo \ref{tab:consultas-referencias} muestra la relación
entre los documentos que se espera que el modelo seleccione para responder cada
pregunta, frente a las referencias que finalmente utilizó para extraer el
contexto necesario. Se obtienen los siguientes resultados para las 3
clasificaciones descritas previamente en la sección 7.5.1 de este documento:
\begin{itemize}
      \item \textbf{Coincidencia exacta:} Se evidencia en las filas 13, 16, 17, 22, 23, 34, 39, 42, 46, 48 y 53. Destaca el caso mostrado en la fila 13, puesto que si bien las referencias utilizadas son las mismas que las esperadas, difiere el orden de prioridad interpretado por el modelo.
      \item\textbf{Coincidencia parcial:} Se evidencia este comportamiento en las filas 1, 7, 11, 14, 18, 26, 28, 33, 36, 38, 41, 43, 44, 47, 49, 51, 52 y 55.
      \item \textbf{Coincidencia nula:} Se evidencia en las filas 3, 6, 8, 12, 19 y 31.
\end{itemize}

\subsection{Métricas Encontradas}

A partir de los resultados, se calcularon las siguientes métricas relacionadas
con el desempeño del sistema:

En primer lugar, el Anexo \ref{tab:consultas-resultados} muestra que, de 45
consultas reales (excluyendo las de control), 35 fueron respondidas
exitosamente, lo que representa una tasa de éxito global del 77.78\%. Para las
consultas de control, ninguna fue respondida, lo que resulta en una tasa de
éxito del 100\% para este tipo.

Respecto a la congruencia, de las 55 consultas totales, 45 mostraron un
comportamiento congruente con lo establecido, lo que equivale a una congruencia
fáctica global del 81.81\%.

Para la latencia, el tiempo de respuesta mínimo fue de 3.303 segundos y el
máximo de 19.121 segundos, con una desviación estándar aproximada de 2.96. El
tiempo de respuesta promedio fue de 7.8184 segundos. La comparación de estos
datos con modelos comerciales será abordada en la sección de discusión.

Analizando el Anexo \ref{tab:consultas-referencias}, de las 35 consultas
respondidas correctamente, la distribución por tipo de coincidencia fue:

\begin{itemize}
      \item Coincidencia Exacta: 11 (31.43\%)
      \item Coincidencia Parcial: 18 (51.43\%)
      \item Coincidencia Nula: 6 (17.14\%)
\end{itemize}

Las métricas calculadas se resumen en el Cuadro \ref{tab:metricas}.

\begin{table}[H]
      \centering
      \caption{Métricas de desempeño del sistema}
      \label{tab:metricas}
      \begin{tabular}{|l|c|}
            \hline
            \textbf{Métrica}     & \textbf{Valor} \\
            \hline
            Tasa de éxito global & 77.78\%        \\
            Congruencia fáctica  & 81.81\%        \\
            Latencia promedio    & 7.8184         \\
            \hline
      \end{tabular}
\end{table}

