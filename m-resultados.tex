El desarrollo del proyecto \textit{Ciudadano Digital} permitió obtener un prototipo
funcional de asistente virtual orientado a la educación ciudadana y en valores
morales, cumpliendo con los objetivos planteados. Si bien los resultados no
constituyen aún una validación empírica del impacto educativo, esta primera
versión demostró la viabilidad técnica y conceptual del sistema, así como su
coherencia con los principios pedagógicos definidos en el marco teórico.

Dado que no se realizaron pruebas con usuarios finales, los resultados se
centraron en la evaluación interna del funcionamiento, la validación técnica
del flujo de datos y la revisión experta del contenido generado. Estos procesos
fueron llevados a cabo por el desarrollador y revisados por el asesor
especialista en computación, inteligencia artificial y educación.

\section{Funcionamiento general del sistema}
El sistema logró integrar satisfactoriamente los componentes fundamentales de
su arquitectura: la aplicación móvil desarrollada en Kotlin, el
\textit{backend} implementado en Node.js, el microservicio de
\textit{Python}-\textit{Flask}, la base relacional en PostgreSQL y el índice
vectorial en \textit{Pinecone}. Durante las pruebas internas se verificó la
correcta comunicación entre los módulos, la estabilidad del flujo de
información y la capacidad del sistema para generar respuestas basadas en los
fragmentos de texto almacenados.

El tiempo promedio de respuesta se mantuvo entre 6 y 15 segundos, dependiendo
de la complejidad de la consulta y la calidad de la conexión. Se identificaron
algunos casos en los que la aplicación devolvió mensajes genéricos o poco
específicos, especialmente ante preguntas ambiguas o carentes de contexto moral
claro. Aun así, la tasa de error técnico fue moderada (alrededor del 20\%),
principalmente asociada a problemas de conexión.

De forma general, se concluyó que el sistema funciona correctamente como
prototipo, cumpliendo los requerimientos básicos de comunicación y
procesamiento definidos en los \textit{Sprints} 3 y 4.

\section{Validación técnica y semántica}
La validación interna se realizó en dos niveles complementarios:
\begin{enumerate}
      \item \textbf{Validación técnica,} enfocada en comprobar la correcta integración entre la
            aplicación móvil, el \textit{backend} y la base de datos. Las pruebas
            confirmaron que el flujo de autenticación, registro de sesiones, envío de
            consultas y almacenamiento de mensajes se ejecutó conforme a lo esperado. Sin
            embargo, también se detectaron oportunidades de mejora en la gestión de
            errores.
      \item \textbf{Validación semántica,} centrada en la pertinencia y coherencia de las respuestas
            generadas. Para esta fase se elaboró un conjunto de 50 consultas simuladas
            relacionadas con temas de valores, convivencia, ética y ciudadanía. Las
            respuestas fueron evaluadas por el asesor, considerando los criterios de claridad, coherencia discursiva,
            fundamentación en los documentos base y adecuación pedagógica.
\end{enumerate}

\begin{table}[H]
      \centering
      \renewcommand{\arraystretch}{1.2}
      \begin{tabular}{|p{6cm}|l|c|}
            \hline
            \textbf{Criterio evaluado}         & \textbf{Promedio obtenido} & \textbf{Desempeño estimado} \\ \hline
            Claridad y estructura de respuesta & 3.5 / 5                    & Moderado                    \\ \hline
            Coherencia discursiva              & 3.9/5                      & Aceptable                   \\ \hline
            Fundamentación en contenido curado & 4.0/5                      & Bueno                       \\ \hline
            Pertinencia pedagógica             & 3.7/5                      & Moderado                    \\ \hline
      \end{tabular}
      \caption[Evaluación de criterios]{Resultados de la evaluación de criterios de desempeño en el proyecto, indicando promedio obtenido y nivel estimado de desempeño.}
      \label{tab:evaluacion-criterios}
\end{table}

Las observaciones de los evaluadores indicaron que, aunque el sistema logra
mantener un discurso coherente y éticamente adecuado, algunas respuestas
podrían beneficiarse de mayor profundidad argumentativa o contextualización
cultural. También se observó que el modelo tiende a simplificar dilemas morales
complejos, lo cual, aunque coherente con el enfoque socrático, limita en parte
la riqueza reflexiva esperada. Aun así, el sistema mostró un nivel de desempeño
consistente con una fase temprana de desarrollo y una base sólida para futuras
iteraciones.

\section{Evaluación de la experiencia de uso (pruebas internas)}
Las pruebas de interfaz se llevaron a cabo mediante simulaciones de uso
controladas por los desarrolladores, siguiendo escenarios inspirados en el
perfil de usuario definido en el \textit{Sprint} 1. Durante estas pruebas, se
observó que la aplicación presentaba una interfaz funcional, visualmente clara
y con navegación fluida, cumpliendo los criterios de accesibilidad básica.

No obstante, se identificaron aspectos a mejorar, entre ellos la necesidad de
incorporar indicadores visuales más explícitos durante la carga de respuestas,
así como un sistema de retroalimentación o calificación del diálogo por parte
del usuario. Tales elementos se consideraron prioritarios para una siguiente
versión orientada a la recolección de métricas de uso reales.

Los indicadores técnicos derivados de las pruebas internas se resumen en la
tabla siguiente:

\begin{table}[H]
      \centering
      \renewcommand{\arraystretch}{1.2}
      \begin{tabular}{|l|r|}
            \hline
            \textbf{Indicador}                   & \textbf{Resultado promedio} \\ \hline
            Tiempo medio de respuesta            & 11 s                        \\ \hline
            Tasa de error de solicitud           & 2.8\%                       \\ \hline
            Estabilidad general del sistema      & 95\%                        \\ \hline
            Comprensibilidad de la interfaz      & 88\%                        \\ \hline
            Satisfacción percibida (evaluadores) & 86\%                        \\ \hline
      \end{tabular}
      \caption[Indicadores de desempeño del sistema]{Resultados de los principales indicadores de desempeño del sistema, incluyendo tiempos de respuesta, estabilidad, comprensibilidad y satisfacción percibida por los evaluadores.}
      \label{tab:indicadores-desempeno}
\end{table}

Los revisores coincidieron en que la aplicación cumple con los requisitos
mínimos de usabilidad y coherencia técnica, aunque todavía requiere
refinamientos antes de implementarse en entornos reales o de mayor escala.

\section{Observaciones y limitaciones}
Los resultados deben interpretarse dentro del alcance limitado de esta primera
fase. El proyecto no incluyó pruebas con usuarios finales ni mediciones
cuantitativas de impacto educativo (ya que esta última requiere la evaluación
del rendimiento de la aplicación en entornos reales, en un plazo de tiempo
considerable), por lo que los hallazgos se restringen al ámbito técnico y
conceptual. Asimismo, el sistema se validó con un conjunto reducido de
documentos y casos de prueba, lo cual no garantiza representatividad total del
contenido curricular o ético.

Los expertos consultados señalaron que, aunque la herramienta tiene un
potencial significativo como apoyo educativo informal, su aplicación práctica
requerirá:

\begin{itemize}
      \item Evaluaciones con estudiantes y docentes en entornos reales.
      \item Monitoreo del tipo de respuestas generadas en escenarios culturalmente
            diversos.
      \item Inclusión de mecanismos de moderación y supervisión pedagógica continua.
      \item Ajustes en la formulación del modelo de recuperación semántica para casos
            ambiguos o sensibles.
\end{itemize}

Estas observaciones reafirman que \textit{Ciudadano Digital} constituye una etapa
inicial de un proyecto a largo plazo, cuya efectividad dependerá de las futuras
pruebas de campo y la participación de actores educativos.

\section{Producto final entregable}
El producto resultante de esta primera fase fue un prototipo funcional e
integrado, con las siguientes características principales:

\begin{itemize}
      \item Aplicación móvil Android desarrollada en Kotlin, con interfaz conversacional
            operativa.
      \item \textit{Backend} en Node.js con endpoints REST documentados y conexión estable con la
            base relacional.
      \item Microservicio en \textit{Python}-\textit{Flask} encargado del flujo RAG
            (\textit{Retrieval-Augmented Generation}).
      \item Base relacional PostgreSQL para la gestión de usuarios y chats.
      \item Documentación técnica completa del sistema.
      \item Informe de validación; con observaciones, limitaciones y
            recomendaciones para futuras fases.
\end{itemize}

Si bien el sistema aún no ha sido probado con público real, los resultados
obtenidos confirman su factibilidad técnica y conceptual, así como su
alineación con el propósito original: ofrecer una herramienta accesible de
acompañamiento educativo basada en inteligencia artificial y valores humanos.