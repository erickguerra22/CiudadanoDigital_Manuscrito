Esta fase del desarrollo del proyecto \textit{Sapien - Ciudadano Digital}
concluyó con la implementación de un prototipo funcional de asistente virtual
orientado a la educación informal en ciudadananía y valores morales, el cual
integra de manera coherente los componentes de procesamiento de lenguaje
natural, recuperación semántico de contexto y el diseño y desarrollo de una
interfaz nativa móvil para dispositivos \textit{Android}. Aunque los resultados
obtenidos no constituyen una validación empírica del impacto educativo del
sistema, esta primera versión permitió demostrar la viabilidad técnica y
conceptual del proyecto, así como su alineación con los principios establecidos
en el marco teórico y los objetivos específicos planteados, enfatizando los
logros técnicos y visuales alcanzados durante los \textit{sprints} definidos
previamente.

\section{Implementación del modelo LLM y flujo RAG}
Durante el tercer \textit{sprint} del desarrollo, se implementó un flujo RAG
(\textit{Retrieval-Augmented Generation}), el cual permite al asistente virtual
generar respuestas fundamentadas en los contenidos educativos previamente
procesados y almacenados en la base de datos. Este flujo se compone de los
siguientes pasos. Este flujo integra los siguientes componentes principales:
\begin{enumerate}
      \item \textbf{REST API en NodeJS} como \textit{backend} del sistema, encargado de orquestar la
            comunicación entre la aplicación móvil y los servicios de procesamiento de
            lenguaje natural.
      \item \textbf{Microservicio en Python} para la interacción con el modelo LLM
            (\textit{Large Language Model}) de OpenAI, responsable de generar las respuestas, así como del procesamiento de documentos.
      \item \textbf{API de OpenAI} para la generación de embeddings y respuestas.
      \item \textbf{Base de datos vectorial en Pinecone} para el almacenamiento y recuperación de embeddings
            generados a partir de los contenidos educativos.
      \item \textbf{Base de datos relacional en PostgreSQL}, la cual se encarga de gestionar usuarios, mensajes, sesiones y la información relacionada con los documentos procesados para asegurar trazabilidad desde los vectores hasta el documento original almacenado en el sistema.
\end{enumerate}

El flujo completo del procesamiento RAG, desde que se recibe una respuesta a
través del cliente (aplicación móvil) hasta que se devuelve la respuesta
generada, se ilustra en la Figura \ref{fig:flujo-rag}

\begin{figure}[H]
      \centering
      \includegraphics[width=0.8\textwidth]{assets/RAG.png}
      \caption{Flujo de procesamiento de preguntas mediante RAG.}
      \label{fig:flujo-rag}
\end{figure}

\section{Integración de la base de datos vectorial y relacional}
El almacenamiento de información a lo largo del sistema se describe en 3
elementos principales:
\begin{itemize}
      \item \textbf{Base de datos relacional (PostgreSQL):} gestor principal de los datos del sistema; aquí se almacenan los datos de los usuarios, chats, mensajes, sesiones, códigos de recuperación de contraseñas y metadatos básicos de los documentos procesados.
      \item \textbf{Base de datos vectorial (Pinecone):} encargada de almacenar los embeddings generados a partir de los documentos seleccionados. Esta es la base del funcionamiento RAG, ya que permite recuperar el contexto necesario, según la pregunta realizada, para que el modelo LLM pueda generar respuestas fundamentadas.
      \item \textbf{Documentos originales (Contenedor AWS S3):} aquí se almacenan todos los documentos originales cargados al sistema, lo que permite su posterior consulta o descarga por parte de los usuarios con rol de \textbf{Administrador}. Cabe aclarar que se maneja un protocolo de trazabilidad basado en el identificador único generado para el contenedor, el cual se almacena como parte de los metadatos tanto en la base de datos vectorial como en la relacional; de esta forma, se mantiene la coherencia al momento de eliminar documentos o consultar su fuente original.
\end{itemize}

permitieron verificar la correcta sincronización entre todas las fuentes de
datos mediante los servicios expuestos por el \textit{backend}, lo que permitió que
las consultas del usuario se asocien con el contenido adecuado. La Figura
\ref{fig:preview-pinecone} muestra la vista desde el panel de pinecone, una vez
se han cargado vectores correctamente.

\begin{figure}[H]
      \centering
      \includegraphics[width=0.8\textwidth]{assets/pinecone-preview.png}
      \caption{Vista del índice vectorial en Pinecone con los embeddings cargados.}
      \label{fig:preview-pinecone}
\end{figure}

\section{Desarrollo del backend en NodeJS}
El \textit{backend} implementado durante el \textit{sprint} 3 se comprobó
mediante pruebas de endpoints \textit{REST}, a través de la herramienta
\textit{Postman}, lo cual demostró la existencia de comunicación estable con
los servicios de Python y la aplicación móvil. Los \textit{endpoints}
principales cubren autenticación, recuperación de contraseña, envío de
mensajes, historial de chats y administración de documentos.

Las pruebas realizadas a través de \textit{Postman} permitieron verificar la
correcta integración entre los módulos del sistema, la estabilidad del flujo de
información y la capacidad del sistema para generar respuestas basadas en los
fragmentos de texto almacenados. El Cuadro \ref{tab:consultas-resultados}
presenta los resultados de las consultas y controles realizados durante la fase
de validación técnica y semántica del sistema.

\textbf{NOTA:} Durante las pruebas realizadas, se incluyen solicitudes de \textit{control}, pensadas para validar que el modelo no sea capaz de responder a preguntas fuera del contexto establecido mediante la base de datos vectorial. Se espera que estas preguntas no sean respondidas, para evitar \textit{alucinaciones} del modelo y mantener la congruencia del sistema. Por otro lado, las preguntas de \textit{consulta} están relacionadas directamente con los contenidos educativos procesados y almacenados en el sistema, por lo que su congruencia se refiere a que el modelo haya sido capaz no solo de responderla, sino que el contenido retornado tenga relación con los contenidos y sepa responder apropiadamente a la pregunta planteada.

\renewcommand{\arraystretch}{1.15}

\begin{longtable}{|p{4cm}|c|c|c|c|}
      \caption[Resultados de consultas y controles]{Resultados de las consultas y controles con sus tiempos de respuesta, éxito y congruencia.}
      \label{tab:consultas-resultados}                                                                                                                                                                     \\

      \hline
      \textbf{\textit{Consulta}}                                           & \textbf{\textit{Tiempo de respuesta (s)}} & \textbf{\textit{Exitosa}} & \textbf{\textit{Congruente}} & \textbf{\textit{Tipo}} \\
      \hline
      \endfirsthead

      \hline
      \textbf{\textit{Consulta}}                                           & \textbf{\textit{Tiempo de respuesta (s)}} & \textbf{\textit{Exitosa}} & \textbf{\textit{Congruente}} & \textbf{\textit{Tipo}} \\
      \hline
      \endhead

      \hline
      \multicolumn{5}{r}{\textit{Continúa en la siguiente página}}                                                                                                                                         \\
      \endfoot

      \hline
      \endlastfoot

      Dime en una frase qué es el civismo                                  & 9.643                                     & Sí                        & Sí                           & Consulta               \\ \hline ¿Qué
      es la ciudadanía?                                                    & 12.285                                    & Sí                        & Sí                           & Consulta               \\ \hline ¿Cómo se pueden
      abordar los dilemas morales en la educación cívica?                  & 16.715                                    & Sí                        & Sí                           &
      Consulta                                                                                                                                                                                             \\ \hline ¿Qué es Formación Ciudadana? & 10.854 & Sí & Sí & Consulta
      \\ \hline Dame el listado de todos los departamentos de Guatemala & 10.329 & No
                                                                           & Sí                                        & Control                                                                           \\ \hline ¿Cuál es la relación entre Formación Ciudadana y
      Democracia?                                                          & 17.456                                    & Sí                        & Sí                           & Consulta               \\ \hline ¿Qué importancia crees que
      tiene la formación ciudadana en la educación actual?                 & 14.502                                    & Sí                        & Sí                           &
      Consulta                                                                                                                                                                                             \\ \hline ¿Cuál es el rol de la escuela en la educación en valores? &
      19.409                                                               & Sí                                        & Sí                        & Consulta                                              \\ \hline ¿Cómo puedo practicar el civismo en mi
      familia?                                                             & 20.086                                    & Sí                        & Sí                           & Consulta               \\ \hline ¡Es una emergencia! ¿Cómo
      aplico un torniquete?                                                & 8.12                                      & No                        & Sí                           & Control                \\ \hline Dame 5 tips
      prácticos para ser mejor ciudadano                                   & 13.652                                    & Sí                        & Sí                           & Consulta               \\ \hline
      ¿Cómo crees que la empatía puede influir en la convivencia escolar?  & 11.255                                    &
      Sí                                                                   & Sí                                        & Consulta                                                                          \\ \hline ¿Qué importancia tiene la discusión sobre ética en
      el aula para la formación ciudadana?                                 & 16.29                                     & Sí                        & Sí                           & Consulta               \\ \hline
      ¿Qué es la democracia?                                               & 12.53                                     & Sí                        & Sí                           & Consulta               \\ \hline ¿Cuál es el
      procedimiento de emergencia si se incendia un televisor?             & 11.418                                    & No                        & Sí                           &
      Control                                                                                                                                                                                              \\ \hline ¿Cómo practico la democracia si soy estudiante? & 16.49 & Sí
                                                                           & Sí                                        & Consulta                                                                          \\ \hline ¿Es posible ser corrupto sin ser político? & 7.252 &
      No                                                                   & No                                        & Consulta                                                                          \\ \hline Si soy buen ciudadano, ¿qué puedo esperar? &
      13.584                                                               & Sí                                        & Sí                        & Consulta                                              \\ \hline ¿Cómo se reconoce la formación ciudadana?
                                                                           & 16.203                                    & Sí                        & Sí                           & Consulta               \\ \hline Le di chocolate a mi perro, ¿cómo puedo
      salvarlo?                                                            & 8.183                                     & No                        & Sí                           & Control                \\ \hline ¿La democracia representativa
      es perfecta?                                                         & 10.79                                     & No                        & No                           & Consulta               \\ \hline ¿Cómo es un ciudadano
      ejemplar?                                                            & 21.064                                    & Sí                        & Sí                           & Consulta               \\ \hline ¿Una nación sin democracia
      tiene ciudadanía?                                                    & 12.842                                    & No                        & No                           & Consulta               \\ \hline ¿Votar es importante?
                                                                           & 10.135                                    & Sí                        & Sí                           & Consulta               \\ \hline ¿Cuáles son los componentes básicos de
      un motor?                                                            & 9.246                                     & No                        & Sí                           & Control                \\ \hline Lista 5 ideas para que mi
      círculo sea más democrático.                                         & 14.8                                      & Sí                        & Sí                           & Consulta               \\ \hline ¿Qué puedo
      hacer para mostrar civismo además de votar?                          & 18.837                                    & Sí                        & Sí                           & Consulta               \\
      \hline Si veo corrupción y no denuncio, ¿fallé como ciudadano?       & 12.974                                    & Sí                        &
      Sí                                                                   & Consulta                                                                                                                      \\ \hline ¿Qué caracteriza un país democrático? & 17.86                        & Sí & Sí
                                                                           & Consulta                                                                                                                      \\ \hline ¿Cuáles son los países más democráticos del mundo? &
      10.319                                                               & No                                        & Sí                        & Control                                               \\ \hline ¿Qué valores son esenciales para vivir en
      sociedad?                                                            & 13.914                                    & Sí                        & Sí                           & Consulta               \\ \hline ¿Qué acciones diarias me
      hacen buena persona?                                                 & 10.533                                    & No                        & No                           & Consulta               \\ \hline ¿Qué implica la
      democracia en el bienestar social?                                   & 14.528                                    & Sí                        & Sí                           & Consulta               \\ \hline
      ¿Cómo demuestro valores en mi día a día?                             & 18.311                                    & Sí                        & Sí                           & Consulta               \\
      \hline ¿Qué tan algo puede respirar un ser humano?                   & 10.846                                    & No                        & Sí                           & Control
      \\ \hline Además de la escuela y la familia, ¿dónde más se obtienen los
      valores?                                                             & 9.213                                     & No                        & No                           & Consulta               \\ \hline ¿Una persona en el tráfico casi
      me choca, cómo puedo actuar de forma ética?                          & 8.178                                     & No                        & No                           & Consulta               \\
      \hline ¿Qué papel representa la tecnología en la ciudadanía moderna? & 15.59                                     &
      Sí                                                                   & Sí                                        & Consulta                                                                          \\ \hline ¿Por qué me debería preocupar por mis valores y
      ciudadanía?                                                          & 13.73                                     & Sí                        & Sí                           & Consulta               \\ \hline ¿Cuántos vasos de agua al
      día bebe un buen ciudadano?                                          & 8.943                                     & No                        & Sí                           & Control                \\ \hline ¿La
      democracia es un modelo de gobierno ideal?                           & 10.591                                    & Sí                        & Sí                           & Consulta               \\
      \hline ¿Cómo puedo ser fiel a mis valores en un mal día?             & 7.045                                     & No                        & No                           &
      Consulta                                                                                                                                                                                             \\ \hline ¿Por qué es importante la ética? & 14.326 & Sí & Sí &
      Consulta                                                                                                                                                                                             \\ \hline ¿Ser buen ciudadano es ser alguien ético? & 10.728                      & Sí & Sí
                                                                           & Consulta                                                                                                                      \\ \hline ¿Cuáles son los componentes de un motor por combustión? &
      9.004                                                                & No                                        & Sí                        & Control                                               \\ \hline ¿Cuál es la diferencia entre moral y ética?
                                                                           & 8.176                                     & No                        & No                           & Consulta               \\ \hline ¿Qué acciones demuestran ética en mi
      conducta?                                                            & 16.241                                    & Sí                        & Sí                           & Consulta               \\ \hline ¿Cuál es la forma ética de
      actuar ante alguien que me ha hecho un mal?                          & 9.471                                     & No                        & No                           & Consulta               \\
      \hline Una persona en el tráfico casi me choca, ¿cómo puedo actuar de forma
      ética?                                                               & 16.114                                    & Sí                        & Sí                           & Consulta               \\ \hline ¿Qué animales guatemaltecos
      están en peligro de extinción?                                       & 8.531                                     & No                        & Sí                           & Control                \\ \hline
      \multicolumn{1}{|r|}{\textbf{Promedio / Totales}}                    & 12.783                                    & 77.5\%                    & 82\%                         &                        \\
      \hline

\end{longtable}

Como se puede observar en el Cuadro \ref{tab:consultas-resultados}, el sistema
logró responder correctamente el 77.5\% de las consultas realizadas, con un
nivel de congruencia del 82\%. Estos resultados, aunque preliminares, indican
un buen desempeño del sistema en la generación de respuestas fundamentadas en
los contenidos educativos procesados.

\section{Interfaz móvil en Kotlin (Frontend)}
Durante el cuarto \textit{sprint}, se desarrolló la aplicación móvil nativa
para dispositivos \textit{Android} utilizando el lenguaje de programación
Kotlin, implementando el patrón de diseño MVVM (Modelo-Vista-ViewModel) para
asegurar una arquitectura modular y mantenible. Como resultado, se obtuvo una
interfaz funcional, amigable con el usuario y cuyo uso no requiere estar
familizarizado previamente con herramientas parecidas. A través de esta
herramienta, se permite a los usuarios acceder al sistema mediante
autenticación, mantener conversaciones con el asistente, visualizar el
historial de chats, administrar el perfil del usuario y, para usuarios con rol
de \textbf{Administrador}, gestionar los documentos cargados en el sistema.

% Quiero cargar 3 o 4 capturas de pantalla que se vean una al lado de la otra (ya que son capturas móviles son más altas que anchas)
La Figura \ref{fig:app-interfaces} muestra algunas de las pantallas principales
de la aplicación móvil desarrollada.

\begin{figure}[H]
      \centering
      \begin{minipage}{0.22\textwidth}
            \centering
            \includegraphics[width=\textwidth]{assets/app-login.jpeg}
            \subcaption{Pantalla de inicio de sesión.}
      \end{minipage}
      \hfill
      \begin{minipage}{0.22\textwidth}
            \centering
            \includegraphics[width=\textwidth]{assets/app-chat.jpeg}
            \subcaption{Pantalla de chat con asistente virtual.}
      \end{minipage}
      \hfill
      \begin{minipage}{0.22\textwidth}
            \centering
            \includegraphics[width=\textwidth]{assets/app-documents.jpeg}
            \subcaption{Pantalla de gestión de documentos.}
      \end{minipage}
      \hfill
      \begin{minipage}{0.22\textwidth}
            \centering
            \includegraphics[width=\textwidth]{assets/app-messages.jpeg}
            \subcaption{Pantalla de chat con mensajes.}
      \end{minipage}
      \caption{Interfaz de la aplicación móvil desarrollada.}
      \label{fig:app-interfaces}
\end{figure}

El diseño de la interfaz emplea principios y componentes de \textit{Material
      Design}, así como el uso de navegación por componentes (\textit{Navigation
      Component}) y \textit{Hilt}, para la inyección de dependencias.

