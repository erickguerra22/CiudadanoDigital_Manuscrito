El proyecto \textit{Compañero Digital: Ocho a Dieciocho} es una iniciativa que
busca crear una herramienta de educación cívica basada en inteligencia
artificial (IA) dirigida a la juventud guatemalteca. El propósito principal
consiste en la construcción de una herramienta conversacional
(\textit{chatbot}) especializada, capaz de brindar orientación, información y
acompañamiento sobre temas de educación cívica informal, accesible desde una
aplicación móvil en cualquier momento y lugar. Como megaproyecto, el objetivo
final busca \textbf{revitalizar la educación informal cívica en Guatemala}
mediante tecnología moderna.

Bajo las espectativas funcionales del megaproyecto principal, el proyecto
\textit{Ciudadano Digital} constituye una base técnica inicial,centrada en
demostrar la viabilidad de las herramientas tecnológicas contempladas para el
desarrollo del producto final. Esta iteración busca definir la arquitectura
base que permita el procesamiento del contenido educativo, la interacción con
el usuario y un sistema independiente que pueda ser consumido por una
aplicación móvil, pero a su vez esté disponible para cualquier otro tipo de
implementación (por ejemplo, una página web). A su vez, se realiza el
perfilamiento base del usuario final objetivo, con el fin de mantener el
enfoque principal del megaproyecto original.

Al tratarse meramente de una validación técnica, este proyecto no contempla
pruebas de campo con usuarios reales. Las respuestas brindadas por el modelo se
validan únicamente en comparación con los contenidos utilizados para la
alimentación del conocimiento del mismo. Sin embargo, sí se valida la
integración adecuada entre los distintos componentes tecnológicos que
constituyen el producto final: modelo de lenguaje de gran escala (LLM, por sus
siglas en inglés), base de datos vectorial, base de datos relacional y
aplicación móvil final.

\textit{Ciudadano Digital} se centra únicamente en la construcción del Producto Mínimo Viable (MVP, por sus siglas en inglés) necesario para comprobar que la arquitectura propuesta puede operar de manera funcional y coherente. Se evalúa el procesamiento del corpus, el flujo de generación aumentada por recuperación (RAG, por sus siglas en inglés) y la conexión entre la aplicación móvil y el servidor desarrollado, sin abordar aún la implementación completa de los componentes pedagógicos, escalabilidad institucional ni las funciones avanzadas previstas para el proyecto final.

