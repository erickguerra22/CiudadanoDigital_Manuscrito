El desarrollo del proyecto \textit{Sapien – Ciudadano Digital} permitió cumplir
satisfactoriamente la creación de una herramienta tecnológica de educación
informal orientada al acompañamiento en la adquisición de aprendizajes sobre
formación ciudadana y valores morales; puesto que el prototipo construido
demuestra la factibilidad técnica y conceptual de emplear modelos de lenguaje
de gran escala (LLM) como apoyo en procesos educativos no formales, con especial atención en la
coherencia discursiva, pertinencia temática y alineación con los principios
establecidos en el marco teórico.

Asimismo, se logró alcanzar la implementación de un modelo LLM preentrenado y
optimizado mediante la integración de un flujo RAG (\textit{Retrieval-Augmented
    Generation}), el cual permite generar respuestas coherentes y contextualizadas
con base en los documentos educativos procesados. Los resultados obtenidos en
la validación técnica y semántica evidencian un comportamiento consistente y un
adecuado control de respuestas no relacionadas con el \textit{corpus} del
proyecto, evitando así la alucinación. Esto respalda el cumplimiento del primer
objetivo específico planteado y respalda la viabilidad del enfoque adoptado,
asíc omo su escalabilidad para futuras iteraciones del proyecto.

En cuanto a la integración de una base de datos vectorial, esto fue alcanzado
mediante la implementación de \textit{Pinecone} como índice semántico, el cual
permitió el almacenamiento y recuperación eficiente de fragmentos relevantes a
partir de los contenidos educativos almacenados, garantizando la trazabilidad
entre los vectores, los metadatos y los documentos originales mediante la
conexión con la base de datos relacional en \textit{PostgreSQL} y el uso de
contenedores de tipo S3. Esta integración facilitó la optimización del proceso
de recuperación de información y permitió el desarrollo de un flujo RAG
completo y funcional.

Por consiguiente, se obtuvo como producto final una aplicación móvil nativa
para dispositivos \textit{Android} a través de \textit{Kotlin} y el patrón de
diseño MVVM, logrando alcanzar una interfaz clara, funcional y visualmente
coherente con los principios de \textit{Material Design}. Esta herramienta
facilita la interacción del usuario con todos los elementos que componen el
sistema principal, lo que da paso a una portabilidad y accesibilidad adecuada
para el público objetivo planteado, ya que permite el acceso al flujo RAG desde
cualquier lugar y en cualquier momento.

Cabe resaltar también que, al haber desarrollado el flujo RAG mediante un
sistema modular y desacoplado, el producto final presenta una arquitectura
flexible y escalable, encapsulado en una API REST capaz de ser utilizada por
cualquier otro cliente, por lo que este proyecto no está limitado únicamente a
la aplicación móvil desarrollada, sino que queda abierto a su uso tanto en
plataformas web como en otros entornos tecnológicos.

De manera general, se concluye que el prototipo constituye una base sólida para
futuras fases del proyecto, demostrando la viabilidad de integrar inteligencia
artificial generativa en entornos educativos informales. Si bien se reconoce
que los resultados se limitan al ámbito técnico y conceptual, los logros
alcanzados hasta esta fase actúan como incentivo para avanzar hacia etapas de
validación empírica con usuarios reales para evaluar el impacto pedagógico, la
pertinencia cultural de las respuestas y la efectividad del acompañamiento en
la formación ciudadana.

En síntesis, \textit{Ciudadano Digital} representa un avance significativo en
la aplicación de tecnologías emergentes al fortalecimiento de la educación en
valores y ciudadanía. Su diseño modular, su arquitectura funcional y su enfoque
ético ofrecen una base prometedora para continuar el desarrollo de soluciones
educativas innovadoras, centradas en la reflexión moral, el pensamiento crítico
y la participación ciudadana responsable.
