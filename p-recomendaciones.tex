Se plantean las siguientes recomendaciones para continuar el desarrollo y
validación de \textit{Ciudadano Digital}:

\begin{itemize}
      \item Optimizar el flujo de obtención de respuestas por parte del modelo. En esta
            fase, la interacción con los servicios secundarios de Python se llevó a cabo
            mediante ejecución directa de los scripts. Esto puede ser optimizado a partir
            de la creación de una API interna que ofrezca mayor eficiencia y escalabilidad
            en la comunicación entre los componentes del sistema. Esto puede lograrse a
            través de herramientas como \textbf{Flask}.
      \item Realizar pruebas de campo con usuarios finales (estudiantes), así como la
            validación pertinente de los objetivos, respuestas e impacto a largo plazo con
            expertos en el área educativa. De esta forma se podrá evaluar la efectividad
            del prototipo en entornos educativos reales al medir el impacto real en la
            comprensión de valores ciudadanos y formación moral.
      \item Ampliación, diversificación y validación continua del conjunto de documentos
            utilizado para armar el corpus de la aplicación, con el fin de mejorar la
            pertinencia y profundidad de las respuestas generadas por el modelo y mantener
            un enfoque actualizado a partir del conocimiento que surja a lo largo del
            tiempo.
      \item Incorporar mecanismos de retroalimentación dentro de la aplicación que permitan
            a los usuarios evaluar la utilidad de las respuestas. Esto permitirá que el
            equipo técnico pueda ajustar continuamente el modelo según las necesidades de
            los usuarios.
      \item Integrar funcionalidades adicionales en el prototipo, como la capacidad de
            generar actividades interactivas o evaluaciones formativas basadas en los
            contenidos presentados, con el fin de fomentar un aprendizaje activo y
            participativo entre los estudiantes.
      \item Incluir la posibilidad de compartir la aplicación con otros usuarios, por medio
            de foros o espacios de discusión apoyados por el propio asistente. De esta
            manera se podría promover el diálogo y la reflexión colectiva sobre los temas
            de ciudadanía digital y valores cívicos, permitiendo alcanzar avances como
            sociedad.
\end{itemize}