Con base en los resultados obtenidos en esta primera fase del proyecto, se
plantean las siguientes recomendaciones para continuar el desarrollo y
validación de \textit{Ciudadano Digital}:

\begin{itemize}
    \item Optimizar el flujo de obtención de respuestas por parte del modelo. En esta
          fase, la interacción con los servicios secundarios de \textit{Python} se llevó
          a cabo mediante ejecución directa de los scripts, lo cual puede ser optimizado
          a partir de la creación de un \textit{API} interna que ofrezca mayor eficiencia
          y escalabilidad en la comunicación entre los componentes del sistema, esto
          puede lograrse a través de herramientas como \textit{Flask}.
    \item Realizar pruebas de campo con usuarios finales (estudiantes), así como la
          validación pertinente de los objetivos, respuestas e impacto a largo plazo con
          expertos en el área educativa y educadores, de manera que se permita evaluar la
          efectividad del prototipo en entornos educativos reales al medir el impacto
          real en la comprensión de valores ciudadanos y formación moral.
    \item Ampliación, diversificación y validación continua del conjunto de documentos
          utilizado para armar el \textit{corpus} de la aplicación, esto con el fin de
          mejorar la pertinencia y profundidad de las respuestas generadas por el modelo,
          así como mantener un enfoque actualizado a partir del conocimiento que surja a lo largo del tiempo.
    \item Incorporar mecanismos de retroalimentación dentro de la aplicación que permitan
          a los usuarios evaluar la utilidad de las respuestas, de manera que el equipo
          técnico pueda ajustar continuamente el modelo según las necesidades de los
          usuarios.
    \item Integrar funcionalidades adicionales en el prototipo, como la capacidad de
          generar actividades interactivas o evaluaciones formativas basadas en los
          contenidos presentados, con el fin de fomentar un aprendizaje activo y
          participativo entre los estudiantes.
    \item Incluir la posibilidad de compartir con otros usuarios de la aplicación, a
          partir de foros o espacios de discusión apoyados por el propio asistente, de
          manera que se promueva el diálogo y la reflexión colectiva sobre los temas de
          ciudadanía digital y valores cívicos, ya que es esta interaccion interpersonal
          la que nos permite alcanzar avances como sociedad.
\end{itemize}

Estas recomendaciones buscan consolidar el prototipo actual y sentar las bases
para futuras versiones del sistema, orientadas a la validación educativa, la
mejora continua y la escalabilidad de la herramienta en distintos contextos
socioeducativos.
